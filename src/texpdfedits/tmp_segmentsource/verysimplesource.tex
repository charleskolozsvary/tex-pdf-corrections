\documentclass{amsart}
\begin{document}
Here is just some straight up text, no math or equations or anything. I just want to see if I can visualize the information that is output to {\tt boxpositions\_\string{filename\string}.txt} without writing a huge file and drawing a bunch of rectangles for all the different words. I'll also need to think a little more clearly on how exactly I intend to use all the information in that {\tt boxpositions\_\string{filename\string}.txt} file. Let's do that now---why not. So in theory the information this gives me is
\begin{enumerate}
\item the width, height, and depth of words in the source that are not inside forbidden environments or a naked block and are surrounded by just space, no punctuation.
\item the start and end x and y coordinate of the word in the PDF: I should be able to to use this to get the word's x0 and x1 rectangle in the PDF
\item the start and end page of the word in the PDF
\end{enumerate}
I believe the start and end y coordinates are always the same unless the word spans a newline or pagebreak, which might be worth using? Or maybe I'll just ignore those? Ultimately I want some funciton $f$ where $f(\text{rectangle in PDF} = \text{the source LaTeX which created and/or is very near that rectangle}$. So\dots.

Okay! This might be overkill, but we could do something like this:
Given an arbitrary page and rectangle specified by x0, y0, x1, y1 see which individual word rectangles on that page the larger rectangle intersects.

Then identify the intersecting word rectangle that is earliest in the page (smallest y0 then x0---I believe a larger y coordinate means further down the page; it's not the usual cartesian coordinate system) and the one which is last in the page (largest y1 and then largest x1) and then just grab all of the \LaTeX{} source which is between the first and last and call it a day---with the exception of figure source.

If a rectangle is given which selects the caption of a figure that \textit{shouldn't}\footnote{(fingers crossed)} intersect any word rectangles because only words which are in the body or enunciations or proofs or the bibliography are marked. So \dots nice.

Right now the next thing I need to do is, though, is actually create those word (and actually also inline math, now that I think of it) rectangles and draw them and see if they're accurate.
\end{document}
