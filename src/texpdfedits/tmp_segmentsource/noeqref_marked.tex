\documentclass{amsart}
\usepackage{amsmath}
\usepackage{amssymb}
\usepackage[mathscr]{eucal}

% --- Other utilities ---
\usepackage[shortlabels]{enumitem}
\usepackage{graphicx}
\usepackage[all]{xy}
\usepackage[dvipsnames]{xcolor}
\usepackage{tikz-cd}

% --- Hyperref should be loaded last ---
\usepackage[colorlinks=true,citecolor=red,urlcolor=blue,linkcolor=blue,bookmarksopen=true]{hyperref}

\renewcommand{\eqref}[1]{(\ref{#1})}

% --- Theorem environments ---
\newtheorem{theorem}{Theorem}[section]
\newtheorem{proposition}[theorem]{Proposition}
\newtheorem{lemma}[theorem]{Lemma}
\newtheorem{remark}[theorem]{Remark}
\newtheorem{definition}[theorem]{Definition}
\newtheorem{example}[theorem]{Example}
\newtheorem{corollary}[theorem]{Corollary}

% --- Commands ---
\newcommand{\thmref}[1]{Theorem~\ref{#1}}
\newcommand{\thmmref}[1]{Theorem$^\prime$~\ref{#1}}
\newcommand{\secref}[1]{Section~\ref{#1}}
\newcommand{\lemref}[1]{Lemma~\ref{#1}}
\newcommand{\propref}[1]{Proposition~\ref{#1}}
\newcommand{\propnref}[1]{Proposition$'$~\ref{#1}}
\newcommand{\corref}[1]{Corollary~\ref{#1}}
\newcommand{\remref}[1]{Remark~\ref{#1}}
\newcommand{\defref}[1]{Definition~\ref{#1}}
\newcommand{\subsecref}[1]{Subsection~\ref{#1}}
\newcommand{\homeo}{\mathrm{Homeo}}
\newcommand{\gal}{\mathrm{Gal}}
\newcommand{\bz}{\mathbb{Q}}
\newcommand{\bn}{\mathbb{N}}
\newcommand{\bq}{\mathbb{Q}}
\newcommand{\br}{\mathbb{R}}
\newcommand{\bc}{\mathbb{C}}
\newcommand{\bp}{\mathbb{P}}
\newcommand{\bs}{\mathbb{S}}
\newcommand{\co}{\mathcal{O}}
\newcommand{\rank}{\mathrm{Rank}}
\newcommand{\cl}{\mathcal{L}}
\newcommand{\lr}{\longrightarrow}
\renewcommand{\hom}{\mathrm{Hom}}
\newcommand{\wt}{\widetilde}
\newcommand{\im}{\mathrm{Im}}
\newcommand{\re}{\mathrm{Re}}
\newcommand{\Span}{\mathrm{span}}
\newcommand{\tor}{\mathrm{Tor}}
\newcommand{\pic}{\mathrm{Pic}}
\newcommand{\flag}{\mathrm{Flag}}
\newcommand{\ul}{\underline}
\newcommand{\fix}{\mathrm{Fix}}

\title[Graded Endomorphisms of $H^*(\mathbb{S}^m \times \mathbb{C}G_{n,k};\mathbb Q)$]{Rational Cohomology Endomorphisms of product of Sphere with Grassmannian and coincidence theory}

\author[M. Mandal]{Manas Mandal}
\address{Indian Institute of Technology Kanpur, Kanpur 208016, India}
	\email{manasm.imsc@gmail.com}
	
	\author[D. Setia]{Divya Setia}
	\address{Institute of Mathematics, Polish Academy of Sciences, Krak{\'o}w 31-027, Poland}
	\email{divyasetia01@gmail.com}
	
	\subjclass[2020]{Primary: 55S37, 08A35,  55M20; Secondary: 14M15}
	\keywords{cohomology endomorphisms, complex Grassmann manifolds, generalized Dold spaces,  fixed point theory, coincidence theory}


\newwrite\markfile
\immediate\openout\markfile=boxpositions_noeqref.txt

\newcommand{\markbox}[2]{%
  \setbox0=\hbox{#2}%
  \immediate\write\markfile{#1:whd{}:\the\value{page}:\the\wd0:\the\ht0:\the\dp0}%
  \pdfsavepos
  \write\markfile{#1:start:\the\value{page}:\the\pdflastxpos:\the\pdflastypos}%
  #2% 
  \pdfsavepos
  \write\markfile{#1:end{}:\the\value{page}:\the\pdflastxpos:\the\pdflastypos}%
}

\begin{document}
\begin{abstract}
We classified graded endomorphisms of the rational cohomology algebra of the product of a sphere and a complex Grassmannian, whose images are nonzero  in the second cohomology of the Grassmannian.

We also derive necessary conditions for the generalized Dold spaces to satisfy the coincidence property, in particular the fixed-point property. As an application of our results, we obtain several sufficient conditions for the existence of a point of coincidence between a pair of continuous functions on certain generalized Dold spaces. 
\end{abstract}
	
\maketitle
	
	\section{Introduction} \label{intro}
	The \markbox{0}{classification} \markbox{1}{of} \markbox{2}{endomorphisms} \markbox{3}{of} \markbox{4}{the} \markbox{5}{rational} \markbox{6}{cohomology} \markbox{7}{algebra} \markbox{8}{of} \markbox{9}{formal} \markbox{10}{spaces} \markbox{11}{was} \markbox{12}{greatly} \markbox{13}{motivated} \markbox{14}{by} Sullivan's \markbox{15}{theory} \markbox{16}{where} \markbox{17}{it} \markbox{18}{was}  \markbox{19}{proved} \markbox{20}{that} \markbox{21}{rational} \markbox{22}{homotopy} \markbox{23}{class} \markbox{24}{of} self-maps \markbox{25}{are} \markbox{26}{completely} \markbox{27}{determined} \markbox{28}{by} \markbox{29}{the} \markbox{30}{induced} \markbox{31}{graded} \markbox{32}{endomorphisms} \markbox{33}{of} \markbox{34}{their} \markbox{35}{rational} \markbox{36}{cohomology} algebras. 
	
	In \cite{brewster}, \markbox{37}{the} \markbox{38}{authors} \markbox{39}{developed} \markbox{40}{the} \markbox{41}{foundational} \markbox{42}{work} \markbox{43}{by} \markbox{44}{classifying} \markbox{45}{automorphisms} \markbox{46}{of} \markbox{47}{the} \markbox{48}{rational} \markbox{49}{cohomology} \markbox{50}{algebra} \markbox{51}{of} \markbox{52}{complex} Grassmannian. \markbox{53}{Their} \markbox{54}{results} \markbox{55}{were} \markbox{56}{generalized} \markbox{57}{in} \cite{hoffman}, \markbox{58}{where} \markbox{59}{the} \markbox{60}{author} \markbox{61}{classified} \markbox{62}{graded} \markbox{63}{endomorphisms} \markbox{64}{of} \markbox{65}{the} \markbox{66}{rational} \markbox{67}{cohomology} \markbox{68}{algebra} \markbox{69}{of} \markbox{70}{complex} \markbox{71}{Grassmannian} \markbox{72}{which} \markbox{73}{are} \markbox{74}{nonzero} \markbox{75}{on} \markbox{76}{dimension} \markbox{m77}{$2$}. Further, \markbox{78}{he} \markbox{79}{conjectured} \markbox{80}{that} \markbox{81}{every} \markbox{82}{graded} \markbox{83}{endomorphism}  \markbox{84}{vanishing} \markbox{85}{on} \markbox{86}{dimension} \markbox{87}{two} \markbox{88}{is} \markbox{89}{necessarily} trivial. \markbox{90}{This} \markbox{91}{conjecture} \markbox{92}{was} \markbox{93}{proved} \markbox{94}{in} \cite{glover-homer} \markbox{95}{for} \markbox{96}{several} cases.
	
	The \markbox{97}{cohomology} \markbox{98}{endomorphisms} \markbox{99}{are} \markbox{100}{also} \markbox{101}{studied} \markbox{102}{for} \markbox{103}{a} \markbox{104}{variety} \markbox{105}{of} \markbox{106}{homogeneous} \markbox{107}{spaces} \markbox{m108}{$G/H$}, \markbox{109}{where} \markbox{m110}{$G$} \markbox{111}{is} \markbox{112}{a} \markbox{113}{compact} \markbox{114}{connected} \markbox{115}{Lie} \markbox{116}{group} \markbox{117}{and} \markbox{m118}{$H$} \markbox{119}{is} \markbox{120}{a} closed
	subgroup \markbox{121}{of} \markbox{122}{maximal} rank. \markbox{123}{This} \markbox{124}{is} \markbox{125}{a} \markbox{126}{topic} \markbox{127}{of} \markbox{128}{interest} \markbox{129}{since} \markbox{130}{past} \markbox{131}{fifty} \markbox{132}{years} \markbox{133}{and} \markbox{134}{are} \markbox{135}{studied} \markbox{136}{in} \markbox{137}{several} \markbox{138}{papers} \cite{shiga-tezuka2, brewster-homer, hoffman-homer, Papadima, duan, duan-fang, duan-zhao, lin, goswami-sarkar}. 
	
	%However, we are interested in product spaces because a little is known about how cohomology endomorphisms behave under products.
    However, \markbox{139}{the} \markbox{140}{behavior} \markbox{141}{of} \markbox{142}{cohomology} \markbox{143}{endomorphisms} \markbox{144}{for} \markbox{145}{the} \markbox{146}{product} \markbox{147}{spaces} \markbox{148}{is} \markbox{149}{comparatively} \markbox{150}{less} explored, \markbox{151}{and} \markbox{152}{this} \markbox{153}{provides} \markbox{154}{a} \markbox{155}{direction} \markbox{156}{for} study.
    \markbox{157}{We} \markbox{158}{are} \markbox{159}{mainly} \markbox{160}{interested} \markbox{161}{in} \markbox{162}{the} \markbox{163}{product} \markbox{164}{of} \markbox{165}{a} sphere
	with \markbox{166}{complex} \markbox{167}{Grassmannian} \markbox{m168}{$\mathbb S^m\times \mathbb CG_{n,k}$} \markbox{169}{because} \markbox{170}{the} \markbox{171}{sphere} \markbox{m172}{$\mathbb{S}^m$} \markbox{173}{has} \markbox{174}{a} simple, \markbox{175}{singly} \markbox{176}{generated} \markbox{177}{cohomology} algebra, \markbox{178}{while} \markbox{179}{the} \markbox{180}{Grassmannian} \markbox{m181}{$\mathbb{C}G_{n,k}$} (of \markbox{m182}{$k$}-planes \markbox{183}{in} \markbox{m184}{$\mathbb C^n$})
	carries \markbox{185}{the} \markbox{186}{rich} \markbox{187}{structure} \markbox{188}{arising} \markbox{189}{from} \markbox{190}{Schubert} calculus. \markbox{191}{We} \markbox{192}{have} \markbox{193}{classified} \markbox{194}{graded} \markbox{195}{endomorphisms} \markbox{196}{of} \markbox{197}{the} \markbox{198}{rational} \markbox{199}{cohomology} \markbox{200}{algebra} \markbox{m201}{$H^*(\mathbb{S}^m \times \mathbb{C}G_{n,k};\mathbb{Q})$}, \markbox{202}{whose} \markbox{203}{image} \markbox{204}{has} \markbox{205}{a} \markbox{206}{nonzero} \markbox{207}{component} \markbox{208}{in} \markbox{m209}{$H^2(\mathbb CG_{n,k},\mathbb Q)$}.  \markbox{210}{As} \markbox{211}{an} \markbox{212}{application} \markbox{213}{we} \markbox{214}{obtain} \markbox{215}{useful} \markbox{216}{results} \markbox{217}{in} \markbox{218}{coincidence} \markbox{219}{theory} \markbox{220}{and} \markbox{221}{in} particular, fixed-point theory.  
	
	The \markbox{222}{rational} \markbox{223}{cohomology} \markbox{224}{algebra} \markbox{225}{of} \markbox{226}{the} product
	\[
	H^*(\mathbb S^m \times \mathbb CG_{n,k};\mathbb Q)\cong H^*(\mathbb S^m,\mathbb Q)\otimes H^*(\mathbb CG_{n,k};\mathbb Q)
	\]
	is \markbox{227}{generated} \markbox{228}{by} \markbox{m229}{$u,c_1,c_2,\dots,c_k$}, \markbox{230}{where} \markbox{m231}{$H^*(\mathbb{C}G_{n,k};\mathbb{Q})$} (resp. \markbox{m232}{$H^*(\mathbb S^m;\mathbb Q)$}) \markbox{233}{is} \markbox{234}{generated} \markbox{235}{by} \markbox{236}{certain} \markbox{237}{Chern} \markbox{238}{classes} \markbox{m239}{$c_1, \dots, c_k$} (resp. \markbox{m240}{$u$}). Now, \markbox{241}{we} \markbox{242}{are} \markbox{243}{ready} \markbox{244}{to} \markbox{245}{state} \markbox{246}{one} \markbox{247}{of} \markbox{248}{the} \markbox{249}{main} \markbox{250}{results} \markbox{251}{of} \markbox{252}{our} \markbox{253}{paper} \markbox{254}{which} \markbox{255}{proves} \markbox{256}{that} \markbox{257}{the} \markbox{258}{rigidity} \markbox{259}{of} \markbox{m260}{$H^*(\mathbb CG_{n,k};\mathbb Q)$} \markbox{261}{persists} \markbox{262}{in} \markbox{m263}{$H^*(\mathbb S^m \times \mathbb CG_{n,k};\mathbb Q)$} \markbox{264}{even} \markbox{265}{in} \markbox{266}{the} \markbox{267}{presence} \markbox{268}{of} \markbox{269}{spherical} \markbox{270}{cohomology} classes.
	\begin{theorem}
		Let \markbox{m271}{$\phi$} \markbox{272}{be} \markbox{273}{a} \markbox{274}{graded} \markbox{275}{endomorphism} \markbox{276}{of} 
		\markbox{m277}{$H^*(\mathbb S^{m}\times\mathbb C G_{n,k};\mathbb Q)$} 
		satisfying \markbox{m278}{$\phi(c_1)\neq a u, \, a\in\mathbb Q$}.  
		Then \markbox{279}{there} \markbox{280}{exists} \markbox{281}{a} \markbox{282}{nonzero} \markbox{283}{rational} \markbox{m284}{$\lambda$} \markbox{285}{such} \markbox{286}{that} \markbox{287}{the} \markbox{288}{following} holds.
		\begin{enumerate}\label{result main}
			\item If \markbox{m289}{$k < n - k,$} 
			$$ \phi(c_i) = \lambda^i c_i, \forall i \in \{1,2,\dots,k\}$$
			If \markbox{m290}{$k = n - k$}, \markbox{291}{there} \markbox{292}{is} \markbox{293}{an} \markbox{294}{additional} \markbox{295}{possibility} \markbox{296}{of} \markbox{m297}{$\phi$} \markbox{298}{that} \markbox{299}{it} \markbox{300}{is} \markbox{301}{induced} \markbox{302}{by} \markbox{303}{the} \markbox{304}{homeomorphism} 
			\[
			\mathbb{C}G_{2k,k} \longrightarrow \mathbb{C}G_{2k,k}, 
			\quad L \longmapsto L^{\perp},
			\]
			where \markbox{m305}{$L^{\perp}$} \markbox{306}{denotes} \markbox{307}{the} \markbox{308}{orthogonal} \markbox{309}{complement} \markbox{310}{of} \markbox{311}{the} \markbox{m312}{$k$}-plane \markbox{m313}{$L$} \markbox{314}{in} \markbox{m315}{$\mathbb{C}^{2k}$}.
			
			\item 
			The \markbox{316}{image} \markbox{317}{of} \markbox{m318}{$H^*(\mathbb{S}^m;\mathbb{Q})$} \markbox{319}{under} \markbox{m320}{$\phi$} \markbox{321}{lies} \markbox{322}{in} 
			\markbox{m323}{$H^*(\mathbb{S}^m;\mathbb{Q})$} \markbox{324}{or} \markbox{325}{in} \markbox{m326}{$H^*(\mathbb{C}G_{n,k};\mathbb{Q})$} i.e. $$\phi(u) = \mu u,\, \mu \in \mathbb{Q}, \text{ or } \phi(u) \in H^*(\mathbb{C}G_{n,k};\mathbb{Q}), \text{ with } (\phi(u))^2 =0.$$
			
		\end{enumerate}
	\end{theorem}
	Unlike \markbox{327}{the} \markbox{328}{case} \markbox{329}{of} \markbox{330}{the} \markbox{331}{complex} Grassmannian, \markbox{332}{we} \markbox{333}{cannot} \markbox{334}{expect} \markbox{335}{a} graded
	endomorphism \markbox{336}{of} \markbox{m337}{$H^*(\mathbb S^m \times \mathbb CG_{n,k};\mathbb Q)$} \markbox{338}{to} be
	trivial \markbox{339}{merely} \markbox{340}{because} \markbox{341}{it} \markbox{342}{vanishes} \markbox{343}{in} \markbox{m344}{$H^2(\mathbb{C}G_{n,k}; \mathbb{Q})$}. 
	In fact, \markbox{345}{we} \markbox{346}{proved} \markbox{347}{that} \markbox{348}{for} \markbox{349}{any} \markbox{350}{choice} \markbox{351}{of} \markbox{m352}{$P_i \in H^{2i-m}(\mathbb{C}G_{n,k};\mathbb{Q})$} \markbox{353}{and} \markbox{m354}{$Q\in \mathbb Qu\cup  H^*(\mathbb CG_{n,k};\mathbb Q)$} \markbox{355}{with} \markbox{m356}{$Q^2=0$}, \markbox{357}{there} \markbox{358}{exist} \markbox{359}{a} \markbox{360}{graded} \markbox{361}{endomorphism} \markbox{m362}{$\phi$} \markbox{363}{on} \markbox{m364}{$H^*(\mathbb S^m \times \mathbb CG_{n,k};\mathbb Q)$} \markbox{365}{such} \markbox{366}{that} \markbox{m367}{$\phi(c_i) = uP_i, \, \forall i$} \markbox{368}{and} \markbox{m369}{$\phi(u)=Q$}.
	%In fact, we proved that for any choice of $P_i \in H^{2i-m}(\mathbb{C}G;\mathbb{Q})$ and $Q =au, a\in \mathbb{Q}$, there exist a graded endomorphism $\phi$ on $H^*(\mathbb S^m \times \mathbb CG_{n,k};\mathbb Q)$ such that $\phi(c_i) = uP_i, \, \forall i$ and $\phi(u)=Q$.
	We \markbox{370}{also} \markbox{371}{proved} \markbox{372}{that} \markbox{373}{if} \markbox{374}{a} \markbox{375}{continuous} \markbox{376}{function} \markbox{377}{on} \markbox{m378}{$\mathbb S^m \times \mathbb CG_{n,k}$} \markbox{379}{stabilizes} \markbox{380}{a} \markbox{381}{copy} \markbox{382}{of} \markbox{383}{Grassmannian} \markbox{384}{then} \markbox{385}{the} \markbox{386}{induced} \markbox{387}{cohomology} \markbox{388}{endomorphism} \markbox{389}{stabilizes} \markbox{390}{the} \markbox{391}{subalgebra} \markbox{m392}{$H^*(\mathbb S^m ;\mathbb Q)$}.
	
	Our \markbox{393}{study} \markbox{394}{is} \markbox{395}{also} \markbox{396}{motivated} \markbox{397}{by} \markbox{398}{the} \markbox{399}{theory} \markbox{400}{of} \markbox{401}{generalized} \markbox{402}{Dold} \markbox{403}{spaces} \markbox{404}{because} \markbox{405}{the} \markbox{406}{product} \markbox{407}{space} \markbox{m408}{$\mathbb S^m \times \mathbb CG_{n,k}$} \markbox{409}{is} \markbox{410}{a} \markbox{411}{double} \markbox{412}{cover} \markbox{413}{of} \markbox{414}{certain}  \markbox{415}{generalized} \markbox{416}{Dold} \markbox{417}{spaces} (GDS). %denoted by $P(m,n,k)$.
	The \markbox{418}{classical} \markbox{419}{Dold} \markbox{420}{manifolds} \markbox{421}{were} \markbox{422}{introduced} \markbox{423}{in} \cite{dold} \markbox{424}{to} \markbox{425}{construct} odd-dimensional \markbox{426}{generators} \markbox{427}{for} Thom's \markbox{428}{unoriented} \markbox{429}{cobordism} ring. \markbox{430}{In} \markbox{431}{this} paper, \markbox{432}{we} \markbox{433}{are} \markbox{434}{interested} \markbox{435}{in} \markbox{436}{the} \markbox{437}{GDS} \markbox{438}{introduced} \markbox{439}{in} \cite{nath-sankaran} \markbox{440}{and} \markbox{441}{defined} \markbox{442}{as}  %These spaces have been extensively studied from various aspects (see \cite{ucci, khare, korbas, mukerjee}). 
	\[
	P(m,n,k):=\mathbb S^m\times \mathbb CG_{n,k}/\!\!\sim, \text { where } (s,L)\sim (-s,\bar L).
	\]
	As \markbox{443}{an} \markbox{444}{application} \markbox{445}{of} \thmref{result main}, \markbox{446}{we} \markbox{447}{describe} \markbox{448}{endomorphisms} \markbox{449}{of} \markbox{m450}{$H^*(P(m,n,k);\mathbb{Q})$} \markbox{451}{induced} \markbox{452}{by} \markbox{453}{continuous} \markbox{454}{functions} \markbox{455}{on} \markbox{m456}{$P(m,n,k)$}. \markbox{457}{Using} \markbox{458}{this} description, \markbox{459}{we} \markbox{460}{prove} \markbox{461}{that} \markbox{462}{every} \markbox{463}{automorphism} \markbox{464}{of} \markbox{m465}{$H^*(P(m,n,k);\mathbb{Q})$} \markbox{466}{induced} \markbox{467}{by} \markbox{468}{a} \markbox{469}{continuous} \markbox{470}{function} \markbox{471}{on} \markbox{m472}{$P(m,n,k)$} \markbox{473}{lifts} \markbox{474}{to} \markbox{475}{an} \markbox{476}{automorphism} \markbox{477}{of} \markbox{m478}{$H^*(\mathbb S^m\times \mathbb CG_{n,k}; \mathbb{Q})$} \markbox{479}{if} \markbox{m480}{$n>2$}. 
	
	Our \markbox{481}{broader} \markbox{482}{aim} \markbox{483}{is} \markbox{484}{to} \markbox{485}{apply} \thmref{result main} \markbox{486}{to} \markbox{487}{obtain} \markbox{488}{results} \markbox{489}{in} \markbox{490}{coincidence} theory. \markbox{491}{Coincidence} \markbox{492}{theory} \markbox{493}{has} \markbox{494}{been} \markbox{495}{extensively} \markbox{496}{studied} \markbox{497}{in} \cite{hoffman-noncoin, glover-homer coin, wong}. %and fixed point theory.
	A \markbox{498}{pair} \markbox{m499}{$(X,g)$} \markbox{500}{where} \markbox{m501}{$g$} \markbox{502}{is} \markbox{503}{a} \markbox{504}{continuous} \markbox{505}{function} \markbox{506}{on} \markbox{m507}{$X$} \markbox{508}{is} \markbox{509}{said} \markbox{510}{to} \markbox{511}{have} \markbox{512}{the} \markbox{513}{coincidence} \markbox{514}{property} \markbox{515}{if} \markbox{m516}{$g$} \markbox{517}{has} \markbox{518}{a} \markbox{519}{point} \markbox{520}{of} \markbox{521}{coincidence} \markbox{522}{with} \markbox{523}{every} \markbox{524}{continuous} \markbox{525}{function} \markbox{526}{on} \markbox{m527}{$X$}. \markbox{528}{In} particular, \markbox{529}{if} \markbox{m530}{$g$} \markbox{531}{is} \markbox{532}{the} \markbox{533}{identity} map, \markbox{534}{then} \markbox{535}{the} \markbox{536}{coincidence} \markbox{537}{property} \markbox{538}{is} \markbox{539}{same} \markbox{540}{as} \markbox{541}{the} fixed-point \markbox{542}{property} \markbox{543}{of} \markbox{m544}{$X$}.
	%We are providing necessary conditions for certain generalized Dold spaces to satisfy coincidence property.
	
	We \markbox{545}{have} \markbox{546}{generalized} \markbox{547}{Theorem} \markbox{m548}{$2$} \markbox{549}{of} \cite{glover-homer} \markbox{550}{to} \markbox{551}{the} \markbox{552}{setting} \markbox{553}{of} \markbox{554}{coincidence} \markbox{555}{theory} \markbox{556}{and} \markbox{557}{proved} \markbox{558}{that} \markbox{559}{the} \markbox{560}{pair} \markbox{m561}{$(\mathbb{C}G_{n,k},g)$} \markbox{562}{satisfies} \markbox{563}{the} \markbox{564}{coincidence} \markbox{565}{property} \markbox{566}{if} \markbox{m567}{$k(n-k)$} \markbox{568}{is} \markbox{569}{even} \markbox{570}{and} \markbox{m571}{$g$} \markbox{572}{has} \markbox{573}{nonzero} \markbox{574}{Brouwer} degree.
	To conclude, \markbox{575}{using} \markbox{576}{the} \markbox{577}{Lefschetz} \markbox{578}{Coincidence} \markbox{579}{Theorem} \markbox{580}{and} \thmref{result main}, \markbox{581}{we} \markbox{582}{obtained} \markbox{583}{multiple} \markbox{584}{situations} \markbox{585}{when} \markbox{586}{two} \markbox{587}{continuous} \markbox{588}{functions} \markbox{589}{on} \markbox{590}{the} \markbox{591}{generalized} \markbox{592}{Dold} \markbox{593}{space} \markbox{m594}{$P(m,n,k)$} \markbox{595}{are} \markbox{596}{guaranteed} \markbox{597}{to} \markbox{598}{have} \markbox{599}{a} \markbox{600}{point} \markbox{601}{of} coincidence, \markbox{602}{and} \markbox{603}{found} \markbox{604}{certain} \markbox{605}{pairs} \markbox{m606}{$(P(m,n,k),g)$} \markbox{607}{satisfying} \markbox{608}{the} \markbox{609}{coincidence} property.
	%Using Lefschetz Coincidence Theorem and \thmref{result main}, we succeed in obtaining sufficient conditions for the existence of a point of coincidence between any two continuous functions on $P(m,n,k)$ when $k(n-k)$ is even.
	%At last, we explored when two continuous functions on the generalized Dold space $P(m,n,k)$ has a point of coincidence. Using Lefschetz Coincidence Theorem and \thmref{result main}, we succeed in obtaining sufficient conditions for the existence of a point of coincidence between any two continuous functions on $P(m,n,k)$ when $k(n-k)$ is even.
	
	The \markbox{610}{paper} \markbox{611}{is} \markbox{612}{organized} \markbox{613}{as} follows: \\
	In \secref{section 2} \markbox{614}{we} \markbox{615}{develop} \markbox{616}{the} \markbox{617}{necessary} \markbox{618}{background} \markbox{619}{and} \markbox{620}{recall} \markbox{621}{some} \markbox{622}{relevant} results. \secref{section 3} \markbox{623}{is} \markbox{624}{devoted} \markbox{625}{to} \markbox{626}{the} \markbox{627}{study} \markbox{628}{of} \markbox{629}{graded} \markbox{630}{endomorphisms} \markbox{631}{of} \markbox{632}{the} \markbox{633}{rational} \markbox{634}{cohomology} \markbox{635}{algebra} \markbox{636}{of} \markbox{m637}{$\mathbb S^{m}\times \mathbb C G_{n,k}$}, \markbox{638}{from} \markbox{639}{which} \markbox{640}{we} \markbox{641}{extract} \markbox{642}{several} consequences. \markbox{643}{These} \markbox{644}{are} \markbox{645}{applied} \markbox{646}{in} \secref{section 4} \markbox{647}{to} \markbox{648}{obtain} \markbox{649}{the} coincidence-theoretic results.
	
	\iffalse For \markbox{650}{formal} spaces, Sullivan's \markbox{651}{theory} \markbox{652}{shows} \markbox{653}{that} \markbox{654}{the} \markbox{655}{rational} \markbox{656}{homotopy} \markbox{657}{class} \markbox{658}{of} \markbox{659}{a} self-maps \markbox{660}{are} \markbox{661}{completely} \markbox{662}{determined} \markbox{663}{by} \markbox{664}{the} \markbox{665}{induced} \markbox{666}{graded} \markbox{667}{endomorphisms} \markbox{668}{of} \markbox{669}{their} \markbox{670}{rational} \markbox{671}{cohomology} algebras. \markbox{672}{This} \markbox{673}{viewpoint} \markbox{674}{has} \markbox{675}{motivated} \markbox{676}{extensive} \markbox{677}{work} \markbox{678}{on} \markbox{679}{cohomology} \markbox{680}{endomorphisms} \markbox{681}{of} \markbox{682}{various} \markbox{683}{classical}  spaces.
	%, particularly homogeneous spaces of compact Lie groups.
	%Initial results for complex Grassmannians were obtained by Brewster \cite{}, Glover-Homer \cite{}, and Hoffman \cite{}, followed by contributions of  Shiga-Tezuka \cite{}, Papadima \cite{}, and Duan et al. (\cite{},\cite{},\cite{}) for broader classes of homogeneous spaces.
	% $G/H$, where $G$ is complact connected Lie group and $H$ is a closed subgroup of maximal rank.
	
	
	Early \markbox{684}{foundational} \markbox{685}{work} \markbox{686}{on} \markbox{687}{cohomology} \markbox{688}{endomorphisms} \markbox{689}{of} \markbox{690}{complex} Grassmannians
	was \markbox{691}{carried} \markbox{692}{out} \markbox{693}{in} 1978 \markbox{694}{by} Brewster, \markbox{695}{in} \markbox{696}{his} Ph.D.\ thesis \cite{brewster} \markbox{697}{on} cohomology
	automorphisms, \markbox{698}{and} \markbox{699}{by} \markbox{700}{Glover} \markbox{701}{and} \markbox{702}{Homer} \cite{glover-homer}.
	This \markbox{703}{was} \markbox{704}{followed} \markbox{705}{by} Hoffman's 1984 work
	\cite{hoffman}, \markbox{706}{which} \markbox{707}{classified} \markbox{708}{all} \markbox{709}{rational} \markbox{710}{cohomology} \markbox{711}{endomorphisms} \markbox{712}{that} are
	non-trivial \markbox{713}{in} \markbox{714}{degree} two. \markbox{715}{He} \markbox{716}{showed} \markbox{717}{that} \markbox{718}{for} \markbox{719}{the} \markbox{720}{Grassmannians} \markbox{m721}{$\mathbb CG_{n,k}$} (of \markbox{m722}{$k$}-planes \markbox{723}{in} \markbox{m724}{$\mathbb{C}^n$}), \markbox{725}{such} \markbox{726}{endomorphisms} \markbox{727}{are} \markbox{728}{precisely} \emph{Adams maps}: \markbox{729}{each} \markbox{730}{cohomology} \markbox{731}{class} \markbox{m732}{$x$} \markbox{733}{of} \markbox{734}{degree} \markbox{m735}{$2i$} \markbox{736}{maps} \markbox{737}{to} \markbox{738}{a} \markbox{739}{scalar} \markbox{740}{multiple} \markbox{m741}{$\lambda^i x$} \markbox{742}{for} \markbox{743}{some} \markbox{744}{fixed} \markbox{745}{nonzero} \markbox{m746}{$\lambda \in \mathbb{Q}$}, \markbox{747}{except} \markbox{748}{in} \markbox{749}{special} \markbox{750}{symmetric} \markbox{751}{cases} \markbox{m752}{$n=2k$}, \markbox{753}{where} \markbox{754}{additional} \markbox{755}{involutive} \markbox{756}{automorphisms} appear. \markbox{757}{Hoffman} \markbox{758}{further} \markbox{759}{conjectured} \markbox{760}{that} \markbox{761}{every} \markbox{762}{endomorphism} \markbox{763}{vanishing} \markbox{764}{in} \markbox{765}{degree} \markbox{766}{two} \markbox{767}{is} \markbox{768}{necessarily} trivial. \markbox{769}{The} \markbox{770}{results} \markbox{771}{of} \markbox{772}{Glover} \markbox{773}{and} \markbox{774}{Homer} (see \cite{glover-homer}) \markbox{775}{support} \markbox{776}{this} \markbox{777}{conjecture} \markbox{778}{in} \markbox{779}{several} cases, \markbox{780}{under} \markbox{781}{a} \markbox{782}{certain} hypothesis, \markbox{783}{say} (GH).
	
	
	Later, \markbox{784}{cohomology} \markbox{785}{endomorphisms} \markbox{786}{were} \markbox{787}{studied} \markbox{788}{for} \markbox{789}{a} \markbox{790}{variety} \markbox{791}{of} homogeneous
	spaces \markbox{m792}{$G/H$}, \markbox{793}{where} \markbox{m794}{$G$} \markbox{795}{is} \markbox{796}{a} \markbox{797}{compact} \markbox{798}{connected} \markbox{799}{Lie} \markbox{800}{group} \markbox{801}{and} \markbox{m802}{$H$} \markbox{803}{is} \markbox{804}{a} closed
	subgroup \markbox{805}{of} \markbox{806}{maximal} rank. \markbox{807}{For} examples,
	Shiga \markbox{808}{and} \markbox{809}{Tezuka} \markbox{810}{examined} cohomology
	automorphisms \markbox{811}{of} \markbox{812}{homogeneous} \markbox{813}{spaces} \markbox{814}{of} \markbox{815}{simple} \markbox{816}{Lie} \markbox{817}{groups} \cite{shiga-tezuka2}, \markbox{818}{and} \markbox{819}{Papadima} \markbox{820}{described} \markbox{821}{all} \markbox{822}{cohomology} \markbox{823}{automorphisms} \markbox{824}{when} \markbox{825}{the} \markbox{826}{subgroup} \markbox{827}{is} \markbox{828}{a} \markbox{829}{maximal} torus,
	working \markbox{830}{over} \markbox{831}{both} \markbox{m832}{$\mathbb Q$} \markbox{833}{and} \markbox{m834}{$\mathbb R$} \cite{Papadima}. \markbox{835}{Hoffman} \markbox{836}{and} \markbox{837}{Homer} also
	proposed \markbox{838}{a} \markbox{839}{general} \markbox{840}{classification} \markbox{841}{for} \markbox{842}{homogeneous} \markbox{843}{spaces} \markbox{844}{arising} \markbox{845}{from} unitary
	groups, \markbox{846}{together} \markbox{847}{with} \markbox{848}{partial} \markbox{849}{results} \markbox{850}{supporting} \markbox{851}{their} conjecture. Further
	progress \markbox{852}{includes} \markbox{853}{the} \markbox{854}{study} \markbox{855}{of} \markbox{856}{integral} \markbox{857}{cohomology} \markbox{858}{endomorphisms} \markbox{859}{of} \markbox{860}{the} \markbox{861}{Grassmannian} \markbox{862}{of} \markbox{863}{complex} structures
	\markbox{m864}{$SO(2n)/U(n)$} \cite{duan}, \markbox{865}{and} \markbox{866}{the} \markbox{867}{work} \markbox{868}{of} \markbox{869}{Duan} \markbox{870}{and} \markbox{871}{Fang} \markbox{872}{on} \markbox{873}{the} \markbox{874}{eight} exceptional
	Grassmannians \cite{duan-fang}. \markbox{875}{Related} \markbox{876}{developments} \markbox{877}{appear} \markbox{878}{in} \cite{duan-zhao}, \cite{lin}, \markbox{879}{and} \cite{goswami-sarkar}.
	
	
	Most \markbox{880}{of} \markbox{881}{these} results, however, \markbox{882}{concern} \markbox{883}{specific} \markbox{884}{classes} \markbox{885}{of} \markbox{886}{homogeneous} spaces.
	Comparatively \markbox{887}{little} \markbox{888}{is} \markbox{889}{known} \markbox{890}{about} \markbox{891}{how} \markbox{892}{cohomology} \markbox{893}{endomorphisms} \markbox{894}{behave} for
	products. \markbox{895}{The} \markbox{896}{product} \markbox{897}{of} \markbox{898}{a} sphere
	with \markbox{899}{a} \markbox{900}{complex} \markbox{901}{Grassmannian} \markbox{m902}{$\mathbb S^m\times \mathbb CG_{n,k}$} \markbox{903}{offers} \markbox{904}{a} \markbox{905}{particularly} \markbox{906}{useful} \markbox{907}{setting} \markbox{908}{for} \markbox{909}{such} questions: the
	sphere \markbox{910}{has} \markbox{911}{a} simple, \markbox{912}{singly} \markbox{913}{generated} \markbox{914}{cohomology} algebra, \markbox{915}{while} \markbox{916}{the} Grassmannian
	carries \markbox{917}{the} \markbox{918}{rich} \markbox{919}{structure} \markbox{920}{arising} \markbox{921}{from} \markbox{922}{Schubert} calculus.  
	Although \markbox{923}{the} \markbox{924}{garded} \markbox{925}{endomorphisms} \markbox{926}{of} \markbox{927}{the} \markbox{928}{rational} \markbox{929}{cohomology} \markbox{930}{algebra} \markbox{931}{of} \markbox{932}{the} product, \markbox{m933}{$\mathrm{End}\big(H^*(\mathbb S^m\times \mathbb CG_{n,k};\mathbb Q)\big)$}, \markbox{934}{are} \markbox{935}{interesting} \markbox{936}{in} \markbox{937}{their} \markbox{938}{own} right, \markbox{939}{our} \markbox{940}{broader} \markbox{941}{aim} \markbox{942}{is} \markbox{943}{to} \markbox{944}{apply} \markbox{945}{them} \markbox{946}{to} \markbox{947}{the} \textit{coincidence theory} \markbox{948}{of} \markbox{949}{certain} \textit{generalized Dold space}s.
	Coincidence \markbox{950}{theory} \markbox{951}{studies} \markbox{952}{the} \markbox{953}{set} \markbox{954}{of} \markbox{955}{points} \markbox{m956}{$x$} \markbox{957}{for} \markbox{958}{which} \markbox{959}{two} \markbox{960}{maps} 
	\markbox{m961}{$f,g : X \to X$} \markbox{962}{satisfy} \markbox{m963}{$f(x)=g(x)$}, \markbox{964}{and} \markbox{965}{seeks} \markbox{966}{criteria} \markbox{967}{for} \markbox{968}{the} 
	existence \markbox{969}{or} \markbox{970}{elimination} \markbox{971}{of} \markbox{972}{such} points. 
	
	
	%For a continuous map $g$ on a space $X$, the pair $(X,g)$ is said to have the \emph{coincidence property} (abbreviated as CP) if every continuous map $f:X\to X$ has a coincidence point with $g$; that is, $f(x)=g(x)$ for some $x\in X$.
	The \markbox{973}{classical} \markbox{974}{Dold} \markbox{975}{manifolds} \markbox{m976}{$P(m,n):=\mathbb S^{m}\times\mathbb CP^{n}/\!\sim$}, 
	where \markbox{m977}{$(s,L)\sim(-s,\bar L)$}, \markbox{978}{were} \markbox{979}{introduced} \markbox{980}{by} \markbox{981}{Dold} \cite{dold} \markbox{982}{to} \markbox{983}{construct} odd-dimensional \markbox{984}{generators} \markbox{985}{for} Thom's \markbox{986}{unoriented} \markbox{987}{cobordism} ring. \markbox{988}{These} \markbox{989}{spaces} \markbox{990}{have} \markbox{991}{been} \markbox{992}{extensively} \markbox{993}{studied} \markbox{994}{and} generalized. \markbox{995}{Nath} \markbox{996}{and} \markbox{997}{Sankaran} \cite{nath-sankaran} \markbox{998}{extended} \markbox{999}{the} \markbox{1000}{construction} \markbox{1001}{by} \markbox{1002}{replacing} \markbox{m1003}{$\mathbb CP^{n}$} \markbox{1004}{with} \markbox{1005}{a} \markbox{1006}{Hausdorff} \markbox{1007}{space} \markbox{m1008}{$X$} \markbox{1009}{equipped} \markbox{1010}{with} \markbox{1011}{an} \markbox{1012}{involution} \markbox{m1013}{$\sigma$} \markbox{1014}{with} \markbox{1015}{nonempty} \markbox{1016}{fixed} points. \markbox{1017}{A} \markbox{1018}{further} \markbox{1019}{generalization} \markbox{1020}{in} \cite{mandal-sankaran} \markbox{1021}{replaces} \markbox{1022}{the} \markbox{1023}{sphere} \markbox{m1024}{$\mathbb S^{m}$} \markbox{1025}{by} \markbox{1026}{a} \markbox{1027}{space} \markbox{m1028}{$S$} \markbox{1029}{equipped} \markbox{1030}{with} \markbox{1031}{a} \markbox{1032}{free} \markbox{1033}{involution} \markbox{m1034}{$\alpha$}, \markbox{1035}{leading} to
	\[
	P(S,\alpha,X,\sigma):=S\times X/\!\sim,\text{ where } (s,x)\sim(\alpha(s),\sigma(x)),
	\]
	which \markbox{1036}{is} \markbox{1037}{called} \markbox{1038}{a} \emph{generalized Dold space} (GDS).
	Here \markbox{1039}{the} \markbox{1040}{quotient} \markbox{1041}{map} \markbox{m1042}{$\pi: S\times X\to P(S,\alpha,X,\sigma)$} \markbox{1043}{is} \markbox{1044}{a} \markbox{1045}{double} covering.
	%The space $P(S,\alpha,X,\sigma)$ has a $X$-bundle structure over $Y:=S/\!\sim,$ where $s\sim \alpha(s)$.
	
	
	The \markbox{1046}{class} \markbox{1047}{of} \markbox{1048}{GDS} \markbox{1049}{which} \markbox{1050}{we} \markbox{1051}{are} \markbox{1052}{interested} \markbox{1053}{in} \markbox{1054}{is} \markbox{m1055}{$P(m,n,k):=P(\mathbb S^m,\alpha,\mathbb CG_{n,k},\sigma)$}, \markbox{1056}{where} \markbox{m1057}{$\alpha $} \markbox{1058}{is} \markbox{1059}{the} \markbox{1060}{antipodal} \markbox{1061}{map} \markbox{1062}{on} \markbox{1063}{the} \markbox{1064}{sphere} \markbox{1065}{and} \markbox{1066}{the} \markbox{1067}{involution} \markbox{m1068}{$\sigma$} \markbox{1069}{on} \markbox{m1070}{$\mathbb CG_{n,k}$} \markbox{1071}{is} \markbox{1072}{induced} \markbox{1073}{from} \markbox{1074}{the} \markbox{1075}{standard} \markbox{1076}{complex} \markbox{1077}{conjugation} \markbox{1078}{on} \markbox{m1079}{$\mathbb C^n$}. \markbox{1080}{One} \markbox{1081}{observes} \markbox{1082}{that} \markbox{1083}{for} \markbox{1084}{any} \markbox{1085}{map} \markbox{m1086}{$f$} \markbox{1087}{on} \markbox{m1088}{$P(m,n,k)$}, \markbox{1089}{there} \markbox{1090}{exist} \markbox{1091}{a} \markbox{m1092}{$\alpha\times \sigma$}-equivariant \markbox{1093}{map} \markbox{m1094}{$\tilde f$} \markbox{1095}{on} \markbox{m1096}{$\mathbb S^m\times\mathbb CG_{n,k}$} \markbox{1097}{such} \markbox{1098}{that} \markbox{m1099}{$\pi\circ \tilde f=f\circ \pi.$}
	
	
	Let \markbox{1100}{us} \markbox{1101}{regard} \markbox{m1102}{$H^*(\mathbb S^m;\mathbb Q)$} \markbox{1103}{and} \markbox{m1104}{$H^*(\mathbb CG_{n,k};\mathbb Q)$} as
	subalgebras \markbox{1105}{of} \markbox{m1106}{$H^*(\mathbb S^m \times \mathbb CG_{n,k};\mathbb Q)$}.
	Since \markbox{m1107}{$H^*(\mathbb CG_{n,k};\mathbb Q)$} \markbox{1108}{is} \markbox{1109}{generated} \markbox{1110}{by} \markbox{1111}{certain} \markbox{1112}{Chern} classes
	\markbox{m1113}{$c_1,c_2,\dots,c_k$}, \markbox{1114}{it} \markbox{1115}{follows} \markbox{1116}{that} 
	\[
	H^*(\mathbb S^m \times \mathbb CG_{n,k};\mathbb Q)\cong H^*(\mathbb S^m,\mathbb Q)\otimes H^*(\mathbb CG_{n,k};\mathbb Q)
	\]
	is \markbox{1117}{generated} \markbox{1118}{by} \markbox{m1119}{$u,c_1,c_2,\dots,c_k$}, \markbox{1120}{where} \markbox{m1121}{$u$} \markbox{1122}{generates} \markbox{m1123}{$H^m(\mathbb S^m;\mathbb Q)$}.
	We \markbox{1124}{are} \markbox{1125}{now} \markbox{1126}{ready} \markbox{1127}{to} \markbox{1128}{present} \markbox{1129}{our} \markbox{1130}{main} results; \markbox{1131}{the} \markbox{1132}{following} \markbox{1133}{is} \markbox{1134}{one} \markbox{1135}{of} them, \markbox{1136}{which} \markbox{1137}{shows} \markbox{1138}{that} \markbox{1139}{the} \markbox{1140}{rigidity} of
	\markbox{m1141}{$H^*(\mathbb CG_{n,k};\mathbb Q)$} \markbox{1142}{persists} \markbox{1143}{in} \markbox{1144}{the} \markbox{1145}{cohomology} \markbox{1146}{of} \markbox{1147}{the}  product: \markbox{1148}{even} \markbox{1149}{in} \markbox{1150}{the} presence
	of \markbox{1151}{spherical} \markbox{1152}{cohomology} classes, \markbox{1153}{any} \markbox{1154}{graded} \markbox{1155}{endomorphism} \markbox{1156}{whose} \markbox{1157}{image} \markbox{1158}{has} \markbox{1159}{a} \markbox{1160}{nonzero} \markbox{1161}{component} \markbox{1162}{in} \markbox{m1163}{$H^{2}(\mathbb C G_{n,k};\mathbb Q)$} \markbox{1164}{forces}  \markbox{m1165}{$H^*(\mathbb CG_{n,k};\mathbb Q)$} \markbox{1166}{to} \markbox{1167}{behave} \markbox{1168}{exactly} \markbox{1169}{as} \markbox{1170}{it} \markbox{1171}{does} \markbox{1172}{on} \markbox{1173}{its} own.
	More precisely, \markbox{1174}{we} obtain: %that any graded endomorphism which is nontrivial on $H^2(\mathbb CG_{n,k};\mathbb Q)$ necessarily preserves  $H^*(\mathbb CG_{n,k};\mathbb Q)$.  Furthermore, the image of $H^*(\mathbb S^m;\mathbb Q)$ under such an endomorphism lies either entirely in $H^*(\mathbb S^m;\mathbb Q)$ or entirely in $H^*(\mathbb CG_{n,k};\mathbb Q)$.
	\begin{theorem}
		Let \markbox{m1175}{$\phi$} \markbox{1176}{be} \markbox{1177}{a} \markbox{1178}{graded} \markbox{1179}{endomorphism} \markbox{1180}{of} 
		\markbox{m1181}{$H^*(\mathbb S^{m}\times\mathbb C G_{n,k};\mathbb Q)$} 
		satisfying \markbox{m1182}{$\phi(c_1)\neq a u$} \markbox{1183}{for} \markbox{1184}{any} \markbox{m1185}{$a\in\mathbb Q$}.  
		Then \markbox{1186}{the} \markbox{1187}{following} \markbox{1188}{statements} hold:
		\begin{enumerate}
			\item If \markbox{m1189}{$k < n - k$}, \markbox{1190}{there} \markbox{1191}{exists} \markbox{1192}{a} \markbox{1193}{nonzero} \markbox{1194}{rational} \markbox{1195}{number} \markbox{m1196}{$\lambda$} \markbox{1197}{such} that
			\[
			\phi(x) = \lambda^i x 
			\quad \text{for all } x \in H^{2i}(\mathbb{C}G_{n,k}; \mathbb{Q}).
			\]
			If \markbox{m1198}{$k = n - k$}, \markbox{1199}{there} \markbox{1200}{is} \markbox{1201}{an} \markbox{1202}{additional} \markbox{1203}{possibility} \markbox{1204}{of} \markbox{m1205}{$\phi$} \markbox{1206}{induced} \markbox{1207}{by} \markbox{1208}{the} \markbox{1209}{homeomorphism} 
			\[
			\mathbb{C}G_{2k,k} \longrightarrow \mathbb{C}G_{2k,k}, 
			\quad L \longmapsto L^{\perp},
			\]
			where \markbox{m1210}{$L^{\perp}$} \markbox{1211}{denotes} \markbox{1212}{the} \markbox{1213}{orthogonal} \markbox{1214}{complement} \markbox{1215}{of} \markbox{1216}{the} \markbox{m1217}{$k$}-plane \markbox{m1218}{$L$} \markbox{1219}{in} \markbox{m1220}{$\mathbb{C}^{2k}$}.
			
			\item 
			The \markbox{1221}{image} \markbox{1222}{of} \markbox{m1223}{$H^*(\mathbb{S}^m;\mathbb{Q})$} \markbox{1224}{under} \markbox{m1225}{$\phi$} \markbox{1226}{lies} \markbox{1227}{either} \markbox{1228}{in} 
			\markbox{m1229}{$H^*(\mathbb{S}^m;\mathbb{Q})$} \markbox{1230}{or} \markbox{1231}{in} \markbox{m1232}{$H^*(\mathbb{C}G_{n,k};\mathbb{Q})$}.
			
		\end{enumerate}
	\end{theorem}
	
	Unlike \markbox{1233}{the} \markbox{1234}{case} \markbox{1235}{of} \markbox{1236}{the} \markbox{1237}{complex} Grassmannian, \markbox{1238}{we} \markbox{1239}{cannot} \markbox{1240}{expect} \markbox{1241}{a} graded
	endomorphism \markbox{1242}{of} \markbox{m1243}{$H^*(\mathbb S^m \times \mathbb CG_{n,k};\mathbb Q)$} \markbox{1244}{to} be
	trivial \markbox{1245}{merely} \markbox{1246}{because} \markbox{1247}{it} \markbox{1248}{vanishes} \markbox{1249}{in} \markbox{1250}{degree} two. Indeed, \markbox{1251}{for} \markbox{1252}{any} \markbox{1253}{choice} of
	elements \markbox{m1254}{$P_i \in H^{2i-m}(\mathbb S^m \times \mathbb CG_{n,k};\mathbb Q)$},
	\markbox{m1255}{$i=1,\dots,k$}, \markbox{1256}{one} \markbox{1257}{obtains} \markbox{1258}{a} well-defined \markbox{1259}{graded} \markbox{1260}{endomorphism} \markbox{m1261}{$\phi$} of
	\markbox{m1262}{$H^*(\mathbb S^m \times \mathbb CG_{n,k};\mathbb Q)$} \markbox{1263}{by} setting
	\markbox{m1264}{$\phi(c_i) = uP_i$} \markbox{1265}{for} \markbox{m1266}{$i=1,\dots,k$} \markbox{1267}{and} \markbox{m1268}{$\phi(u)=u$}, \markbox{1269}{when} \markbox{1270}{the} \markbox{1271}{hypothesis} (GH) holds. Hence, \markbox{1272}{even} \markbox{1273}{in} \markbox{1274}{the} \markbox{1275}{case} \markbox{m1276}{$P_1 = 0$}, \markbox{1277}{a} \markbox{1278}{large} \markbox{1279}{family} \markbox{1280}{of} \markbox{1281}{nontrivial} \markbox{1282}{graded} \markbox{1283}{endomorphisms} \markbox{1284}{still} exists. \markbox{1285}{See} \markbox{1286}{Proposition} \ref{main thm 2}.
	We \markbox{1287}{show} \markbox{1288}{that} \markbox{1289}{if} \markbox{1290}{a} \markbox{1291}{map} \markbox{1292}{on} \markbox{m1293}{$\mathbb S^{m}\times \mathbb C G_{n,k}$} \markbox{1294}{stabilizes} \markbox{1295}{a} \markbox{1296}{Grassmannian} factor, \markbox{1297}{then} \markbox{1298}{the} \markbox{1299}{induced} \markbox{1300}{cohomology} \markbox{1301}{endomorphism} \markbox{1302}{stabilizes} \markbox{1303}{the} \markbox{1304}{subalgebra} \markbox{m1305}{$H^*(\mathbb S^{m};\mathbb Q)$}. \markbox{1306}{See} \thmref{ind from top}.
	
	%(see Theorem \ref{ind from top}). 
	%\textcolor{gray}{If the map on $\mathbb S^m \times \mathbb CG_{n,k}$ has nonzero  Brouwer degree, the induced map in cohomology splits as a tensor product of two maps: one acting on the cohomology of the sphere and the other on the cohomology of the Grassmannian. (see  \ref{}).}
	
	\iffalse
	Using \markbox{1307}{the} \markbox{1308}{study} \markbox{1309}{of} \markbox{m1310}{$\mathrm{End}(H^*(\mathbb S^{m}\times \mathbb C G_{n,k};\mathbb Q))$}, \markbox{1311}{we} \markbox{1312}{obtain} \markbox{1313}{several} coincidence-theoretic consequences. \markbox{1314}{The} \markbox{1315}{next} \markbox{1316}{result} \markbox{1317}{provides} \markbox{1318}{a} \markbox{1319}{necessary} \markbox{1320}{criterion} \markbox{1321}{for} \markbox{1322}{determining} \markbox{1323}{when} \markbox{1324}{the} \markbox{1325}{pair} \markbox{m1326}{$\big(P(S,\alpha,X,\sigma),g\big)$} \markbox{1327}{have} \markbox{1328}{the} \markbox{1329}{coincidence} property, \markbox{1330}{for} \markbox{1331}{certain} \markbox{1332}{maps} \markbox{m1333}{$g$} \markbox{1334}{on} \markbox{1335}{the} GDS, \markbox{1336}{expressed} \markbox{1337}{in} \markbox{1338}{terms} \markbox{1339}{of} \markbox{1340}{its} \markbox{1341}{fibre} \markbox{m1342}{$X$} \markbox{1343}{and} \markbox{1344}{its} \markbox{1345}{base} \markbox{m1346}{$Y:=S/\!\sim_{\alpha}$}.
	
	
	
	
	\begin{proposition}
		For  \markbox{1347}{a} \markbox{1348}{continuous} \markbox{1349}{map} \markbox{m1350}{$g$} \markbox{1351}{on} \markbox{1352}{the} \markbox{1353}{generalized} \markbox{1354}{Dold} \markbox{1355}{space} \markbox{m1356}{$P(S,\alpha, X,\sigma)$}, \markbox{1357}{the} \markbox{1358}{pair} \markbox{m1359}{$\big(P(S,\alpha,X,\sigma),g\big)$} \markbox{1360}{does} \markbox{1361}{not} \markbox{1362}{have} \markbox{1363}{the} \markbox{1364}{CP} \markbox{1365}{if} \markbox{1366}{one} \markbox{1367}{of} \markbox{1368}{the} \markbox{1369}{following} hold:
		\begin{enumerate}
			\item The \markbox{1370}{pair} \markbox{m1371}{$\big(Y,p \circ g \circ s\big)$} \markbox{1372}{does} \markbox{1373}{not} \markbox{1374}{have} \markbox{1375}{the} CP, 
			where \markbox{m1376}{$s$} \markbox{1377}{denotes} \markbox{1378}{a} \markbox{1379}{global} \markbox{1380}{section} \markbox{1381}{of} \markbox{1382}{the} \markbox{m1383}{$X$}-bundle  
			\markbox{m1384}{$p: P(S, X) \to Y$}, \markbox{1385}{and} \markbox{m1386}{$g$} \markbox{1387}{is} \markbox{1388}{a} \markbox{1389}{fiber} \markbox{1390}{bundle} map.
			\item  There \markbox{1391}{exists} \markbox{1392}{a} \markbox{m1393}{$\sigma$}-equivariant \markbox{1394}{map} \markbox{m1395}{$f$}  \markbox{1396}{on} \markbox{m1397}{$X$} \markbox{1398}{and} \markbox{1399}{a}  \markbox{m1400}{$\alpha \times \sigma$}-equivariant \markbox{1401}{map} \markbox{m1402}{$\tilde g$} \markbox{1403}{on} \markbox{m1404}{$S\times X$} \markbox{1405}{inducing} \markbox{m1406}{$g$} \markbox{1407}{such} \markbox{1408}{that} \markbox{m1409}{$\mathrm{id}_S \times f$} \markbox{1410}{coincides} \markbox{1411}{with} \markbox{1412}{neither} \markbox{m1413}{$\tilde{g}$} \markbox{1414}{nor} \markbox{m1415}{$(\alpha \times \sigma) \circ \tilde{g}$}.
		\end{enumerate}
	\end{proposition}
	\fi
	
	Using \markbox{1416}{the} \markbox{1417}{study} \markbox{1418}{of} \markbox{m1419}{$\mathrm{End}\bigl(H^*(\mathbb S^{m}\times \mathbb C G_{n,k};\mathbb Q)\bigr),$}
	we \markbox{1420}{obtain} \markbox{1421}{several} coincidence-theoretic consequences. First, \markbox{1422}{we} \markbox{1423}{extend} Theorem~2 \markbox{1424}{of} \cite{glover-homer} \markbox{1425}{to} \markbox{1426}{the} \markbox{1427}{setting} \markbox{1428}{of} \markbox{1429}{coincidence} \markbox{1430}{theory} \markbox{1431}{and} \markbox{1432}{show} \markbox{1433}{that} \markbox{1434}{for} \markbox{1435}{a} \markbox{1436}{continuous} self-map \markbox{m1437}{$g$} \markbox{1438}{with} \markbox{1439}{nonzero} \markbox{1440}{Brouwer} \markbox{1441}{degree} \markbox{1442}{on} \markbox{m1443}{$\mathbb C G_{n,k}$} \markbox{1444}{with} \markbox{m1445}{$k(n-k)$} even, \markbox{1446}{any} \markbox{1447}{continuous} \markbox{1448}{map} \markbox{m1449}{$f$} \markbox{1450}{on} \markbox{m1451}{$\mathbb CG_{n,k}$} \markbox{1452}{has} \markbox{1453}{a} \markbox{1454}{coincidence} \markbox{1455}{with} \markbox{m1456}{$g$}. \markbox{1457}{See} Proposition~\ref{CP of CGnk}.
	
	We \markbox{1458}{then} \markbox{1459}{investigate} \markbox{1460}{the} \markbox{1461}{coincidence} \markbox{1462}{behaviour} \markbox{1463}{of} \markbox{1464}{the} \markbox{1465}{generalized} \markbox{1466}{Dold} \markbox{1467}{spaces} \markbox{m1468}{$P(m,n,k)$}. \markbox{1469}{For} \markbox{1470}{any} \markbox{1471}{continuous} self-map \markbox{m1472}{$g$} \markbox{1473}{with} \markbox{1474}{nonzero} \markbox{1475}{Brouwer} \markbox{1476}{degree} \markbox{1477}{on} \markbox{m1478}{$P(m,n,k)$}  \markbox{1479}{with} \markbox{m1480}{$k(n-k)$} \markbox{1481}{even} \markbox{1482}{and} \markbox{m1483}{$m$} even, \markbox{1484}{every} \markbox{1485}{map} \markbox{m1486}{$f$} \markbox{1487}{satisfying} \markbox{m1488}{$\tilde f^{*}(c_1)\neq a u$} \markbox{1489}{for} \markbox{1490}{any} \markbox{m1491}{$a\in\mathbb Q$} \markbox{1492}{has} \markbox{1493}{a} \markbox{1494}{coincidence} \markbox{1495}{with} \markbox{m1496}{$g$}; \markbox{1497}{when} \markbox{m1498}{$m$} \markbox{1499}{is} odd, \markbox{1500}{one} \markbox{1501}{additionally} \markbox{1502}{requires} \markbox{m1503}{$\tilde f^{*}(u)\neq -\tilde g^{*}(u)$}. \markbox{1504}{See} Theorem~\ref{coincidence thm}.
	
	Under \markbox{1505}{hypothesis} (GH), \markbox{1506}{if} \markbox{m1507}{$g$} \markbox{1508}{is} \markbox{1509}{a} \markbox{1510}{homotopy} \markbox{1511}{equivalence} \markbox{1512}{on} \markbox{m1513}{$P(m,n,k)$} \markbox{1514}{with} \markbox{m1515}{$k(n-k)$} even, \markbox{1516}{then} \markbox{m1517}{$g$} \markbox{1518}{has} \markbox{1519}{a} \markbox{1520}{coincidence} \markbox{1521}{with} \markbox{1522}{any} \markbox{1523}{map} \markbox{m1524}{$f$} \markbox{1525}{on} \markbox{m1526}{$P(m,n,k)$} \markbox{1527}{such} that: \\ 
	-- \markbox{1528}{when} \markbox{m1529}{$m$} \markbox{1530}{is} even,  \markbox{m1531}{$\tilde f^{*}(c_1)=a u, a\in \mathbb Q$} \markbox{1532}{implies} \markbox{m1533}{$\tilde f^{*}(u)=b u$} \markbox{1534}{for} \markbox{1535}{some} \markbox{m1536}{$b\in\mathbb Q$};  \\
	-- \markbox{1537}{when} \markbox{m1538}{$m$} \markbox{1539}{is} odd,  \markbox{m1540}{$\tilde f^{*}(u)\neq -\tilde g^{*}(u)$} holds. \markbox{1541}{See} Theorem~\ref{coincidence thm under hom}.  \\
	Moreover, \markbox{1542}{if} \markbox{m1543}{$m>2k$}, \markbox{1544}{no} \markbox{1545}{additional} \markbox{1546}{assumptions} \markbox{1547}{on} \markbox{m1548}{$f$} \markbox{1549}{are} needed.
	
	
	
	
	The \markbox{1550}{paper} \markbox{1551}{is} \markbox{1552}{organized} \markbox{1553}{as} follows: \markbox{1554}{In} Section~2 \markbox{1555}{we} \markbox{1556}{develop} \markbox{1557}{the} \markbox{1558}{necessary} \markbox{1559}{background} \markbox{1560}{and} \markbox{1561}{recall} \markbox{1562}{some} \markbox{1563}{relevant} results. Section~3 \markbox{1564}{is} \markbox{1565}{devoted} \markbox{1566}{to} \markbox{1567}{the} \markbox{1568}{study} \markbox{1569}{of} \markbox{1570}{graded} \markbox{1571}{endomorphisms} \markbox{1572}{of} \markbox{1573}{the} \markbox{1574}{rational} \markbox{1575}{cohomology} \markbox{1576}{algebra} \markbox{1577}{of} \markbox{m1578}{$\mathbb S^{m}\times \mathbb C G_{n,k}$}, \markbox{1579}{from} \markbox{1580}{which} \markbox{1581}{we} \markbox{1582}{extract} \markbox{1583}{several} consequences. \markbox{1584}{These} \markbox{1585}{are} \markbox{1586}{applied} \markbox{1587}{in} Section~4 \markbox{1588}{to} \markbox{1589}{obtain} \markbox{1590}{the} coincidence-theoretic results.\fi
	
	
	
	
	
	
	%%%%%%%%%%%%%%%%%%%%%%%%%%%%%%%
	\section{Preliminaries}  \label{section 2}
	
	In \markbox{1591}{this} section, \markbox{1592}{we} \markbox{1593}{discuss} \markbox{1594}{some} \markbox{1595}{preliminaries} \markbox{1596}{and} \markbox{1597}{recall} \markbox{1598}{some} \markbox{1599}{results} \markbox{1600}{that} \markbox{1601}{will} \markbox{1602}{be} \markbox{1603}{required} \markbox{1604}{to} \markbox{1605}{proceed} \markbox{1606}{with} \markbox{1607}{our} study.
	
	
	
	
	
	
	
	\subsection{Cohomology of complex Grassmannians}
	
	
	Let \markbox{m1608}{$\mathbb{C}G_{n,k}$} \markbox{1609}{denote} \markbox{1610}{the} \markbox{1611}{complex} \markbox{1612}{Grassmannian} \markbox{1613}{consisting} \markbox{1614}{of} \markbox{1615}{complex} \markbox{m1616}{$k$}-planes \markbox{1617}{in} \markbox{m1618}{$\mathbb{C}^n$}. 
	%As a homogeneous space, it is diffeomorphic to $U(n)/U(k) \times U(n-k)$, where $U(r)$ denotes the group of unitary matrices of order $r$. 
	Let \markbox{m1619}{$\gamma_{n,k}$} \markbox{1620}{and} \markbox{m1621}{$\beta_{n,k}$} \markbox{1622}{denote} \markbox{1623}{the} \markbox{1624}{canonical} \markbox{1625}{complex} \markbox{m1626}{$k$}-plane \markbox{1627}{and} \markbox{m1628}{$(n-k)$}-plane bundles, respectively, \markbox{1629}{over} \markbox{m1630}{$\mathbb{C}G_{n,k}$}.
	Let \markbox{1631}{the} \markbox{1632}{total} \markbox{1633}{Chern} \markbox{1634}{classes} \markbox{1635}{of} \markbox{1636}{the} \markbox{1637}{vector} \markbox{1638}{bundles} \markbox{m1639}{$\gamma_{n,k}$} \markbox{1640}{and} \markbox{m1641}{$\beta_{n,k}$} \markbox{1642}{be} \markbox{1643}{denoted} \markbox{1644}{by} \markbox{m1645}{$c(\gamma_{n,k}) = c$} \markbox{1646}{and} \markbox{m1647}{$c(\beta_{n,k}) = \bar{c}$}, respectively. Thus,
	$$c = 1 + c_1 + c_2 + \cdots + c_k, \quad \bar{c} = 1 + \bar{c}_1 + \bar{c}_2 + \cdots + \bar{c}_{n-k},$$
	where \markbox{m1648}{$c_i$} \markbox{1649}{and} \markbox{m1650}{$\bar{c}_i$} \markbox{1651}{denote} \markbox{1652}{the} \markbox{m1653}{$i$}-th \markbox{1654}{Chern} \markbox{1655}{classes} \markbox{1656}{of} \markbox{m1657}{$\gamma_{n,k}$} \markbox{1658}{and} \markbox{m1659}{$\beta_{n,k}$}, respectively.
	Since \markbox{m1660}{$\gamma_{n,k} \oplus \beta_{n,k} \cong \varepsilon_{\mathbb{C}}^n$}, \markbox{1661}{it} \markbox{1662}{follows} \markbox{1663}{that}  \markbox{m1664}{$c \cdot \bar{c} = 1$}.
	The  \markbox{1665}{cohomology} \markbox{1666}{ring} \markbox{1667}{of} \markbox{1668}{the} \markbox{1669}{complex} \markbox{1670}{Grassmannian} \markbox{1671}{is} \markbox{1672}{well} \markbox{1673}{known} \markbox{1674}{and} \markbox{1675}{given} \markbox{1676}{by}  
	$$
	H^*_{\mathbb{C}G}:=H^*(\mathbb{C}G_{n,k};\mathbb{Q}) \cong \mathbb{Q}[c_1, c_2, \dots, c_k, \bar{c}_1, \bar{c}_2, \dots, \bar{c}_{n-k}]/\langle h_r: 1\leq r\leq n\rangle,$$  
	where  \markbox{1677}{the} \markbox{1678}{relations} \markbox{m1679}{$h_r$} \markbox{1680}{for} \markbox{m1681}{$r = 1, 2, \dots, n$} \markbox{1682}{are} \markbox{1683}{induced} \markbox{1684}{from} \markbox{1685}{the} \markbox{1686}{homogeneous} \markbox{1687}{parts} \markbox{1688}{of} \markbox{1689}{the} \markbox{1690}{equation} \markbox{m1691}{$c\cdot \bar c=1$} \markbox{1692}{and} \markbox{1693}{given} \markbox{1694}{by}  
	\[
	h_r := \sum_{i+j=r} c_i \bar{c}_j.
	\]  
	Using \markbox{1695}{the} \markbox{1696}{relations} \markbox{m1697}{$h_r, r=1,2,...,n-k$}, \markbox{1698}{the} \markbox{1699}{generators} \markbox{m1700}{$\bar{c}_i$} \markbox{1701}{for} \markbox{m1702}{$i = 1, 2, \dots, n-k$} \markbox{1703}{can} \markbox{1704}{be} \markbox{1705}{expressed} \markbox{1706}{inductively} \markbox{1707}{in} \markbox{1708}{terms} \markbox{1709}{of} \markbox{m1710}{$c_i$} \markbox{1711}{for} \markbox{m1712}{$i = 1, 2, \dots, k$}.  Consequently, \markbox{1713}{the} \markbox{1714}{relations} \markbox{m1715}{$h_r$} \markbox{1716}{for} \markbox{m1717}{$r = n-k+1, \dots, n$} \markbox{1718}{become} \markbox{1719}{homogeneous} \markbox{1720}{polynomials} \markbox{1721}{in} \markbox{m1722}{$c_i$} \markbox{1723}{of} \markbox{1724}{degree} \markbox{m1725}{$2r$}, \markbox{1726}{where} \markbox{1727}{the} \markbox{1728}{degree} \markbox{1729}{of} \markbox{1730}{each} \markbox{m1731}{$c_i$} \markbox{1732}{is} \markbox{m1733}{$2i$}. \markbox{1734}{Then} \markbox{1735}{the} \markbox{1736}{cohomology} \markbox{1737}{ring} \markbox{m1738}{$H^*_{\mathbb{C}G}$} \markbox{1739}{can} \markbox{1740}{be} \markbox{1741}{rewritten} \markbox{1742}{as}  
	\begin{equation}\label{cohomo of grass}
		\mathbb{Q}[c_1, c_2, \dots, c_k]/\langle h_{n-k+1}, h_{n-k+2}, \dots, h_n \rangle.
	\end{equation}
	Since \markbox{1743}{there} \markbox{1744}{are} \markbox{1745}{no} \markbox{1746}{relations} \markbox{1747}{among} \markbox{1748}{the} \markbox{1749}{generators} \markbox{m1750}{$c_i$} \markbox{1751}{for} \markbox{m1752}{$i = 1, 2, \dots, k$} \markbox{1753}{up} \markbox{1754}{to} \markbox{1755}{degree} \markbox{m1756}{$2(n-k)$}, \markbox{1757}{the} \markbox{1758}{set} \markbox{1759}{of} \markbox{1760}{all} \markbox{1761}{monomials} \markbox{1762}{of} \markbox{1763}{degree} \markbox{m1764}{$2r$} \markbox{1765}{in} \markbox{1766}{terms} \markbox{1767}{of} \markbox{m1768}{$c_1, c_2, \ldots, c_k$} \markbox{1769}{forms} \markbox{1770}{a} \markbox{m1771}{$\mathbb{Q}$}-basis \markbox{1772}{of} \markbox{m1773}{$H^{2r}(\mathbb{C}G_{n,k};\mathbb{Q})$} \markbox{1774}{for} \markbox{m1775}{$r \leq n-k$}.
	
	From \markbox{1776}{now} on, \markbox{1777}{we} \markbox{1778}{denote} \markbox{1779}{the} \markbox{1780}{indexing} \markbox{1781}{set} \markbox{m1782}{$\{1,2,\dots, k\}$} \markbox{1783}{by} \markbox{m1784}{$I$}.
	\begin{remark}
		We \markbox{1785}{can} \markbox{1786}{assume}  \markbox{m1787}{$k\leq n-k$} \markbox{1788}{for}  \markbox{m1789}{$\mathbb{C}G_{n,k}$} \markbox{1790}{as} \markbox{m1791}{$\mathbb{C}G_{n,k}$} \markbox{1792}{is} \markbox{1793}{homeomorphic} \markbox{1794}{to} \markbox{m1795}{$\mathbb{C}G_{n,n-k}$} \markbox{1796}{by} \markbox{1797}{using} \markbox{1798}{orthogonal} complementation.
	\end{remark}
	The \markbox{1799}{complex} \markbox{1800}{Grassmannian} \markbox{m1801}{$\mathbb{C}G_{n,k}$} \markbox{1802}{is} \markbox{1803}{a} \markbox{1804}{homogeneous} \markbox{1805}{space} \markbox{1806}{and} \markbox{1807}{can} \markbox{1808}{be} \markbox{1809}{represented} \markbox{1810}{as} \markbox{1811}{the} \markbox{1812}{quotient} \markbox{1813}{of} \markbox{1814}{the} \markbox{1815}{unitary} \markbox{1816}{group} \markbox{m1817}{$U(n)$} \markbox{1818}{by} \markbox{1819}{the} \markbox{1820}{stabilizer} \markbox{1821}{subgroup} \markbox{m1822}{$U(k)\times U(n-k)$} \markbox{1823}{that} \markbox{1824}{is} 
	\begin{equation}\label{cgn as hom}
		\mathbb{C}G_{n,k} = U(n)/ (U(k)\times U(n-k)). 
	\end{equation} \markbox{1825}{Now} \markbox{1826}{we} \markbox{1827}{recall} \markbox{1828}{a} \markbox{1829}{result} \markbox{1830}{given} \markbox{1831}{in} \cite{shiga-tezuka}.    
	\begin{theorem}[\cite{shiga-tezuka}, \markbox{1832}{Theorem} \markbox{m1833}{\(A^{'}\)}]\label{Tezuka}
		Let \markbox{m1834}{$D_i(H^*(G/H;\mathbb{Q}))$} \markbox{1835}{be} \markbox{1836}{the} \markbox{m1837}{$\mathbb{Q}$}-vector \markbox{1838}{space} \markbox{1839}{of} \markbox{m1840}{$\mathbb{Q}$}-derivations \markbox{1841}{of} \markbox{m1842}{$H^*(G/H;\mathbb{Q})$} \markbox{1843}{which} \markbox{1844}{decreases} \markbox{1845}{the} \markbox{1846}{degree} \markbox{1847}{by} \markbox{m1848}{$i>0$} \markbox{1849}{where} \markbox{m1850}{$G$} \markbox{1851}{is} \markbox{1852}{a} connected, \markbox{1853}{compact} \markbox{1854}{Lie} \markbox{1855}{group} \markbox{1856}{and} \markbox{m1857}{$H$} \markbox{1858}{is} \markbox{1859}{a} \markbox{1860}{closed} \markbox{1861}{subgroup} \markbox{1862}{of} \markbox{1863}{maximal} rank.
		%such that $\rank (H) = \rank (G)$.
		Then, \markbox{1864}{for} \markbox{1865}{all} \markbox{m1866}{$i$}, $$D_i(H^*(G/H;\mathbb{Q})) =0.$$  
	\end{theorem}
	\subsection{Graded endomorphisms on $\mathbf{H^*_{\mathbb{C}G}}$}
	It \markbox{1867}{was} \markbox{1868}{conjectured} \markbox{1869}{in} \cite{O} \markbox{1870}{that} \markbox{1871}{any} \markbox{1872}{graded} \markbox{1873}{endomorphism} \markbox{m1874}{$\phi$} \markbox{1875}{of} \markbox{1876}{the} \markbox{1877}{cohomology} \markbox{1878}{algebra} \markbox{m1879}{$H^*_{\mathbb{C}G}$} \markbox{1880}{is} an\textit{ Adams }map \markbox{1881}{when} \markbox{m1882}{$k < n - k$}; \markbox{1883}{that} is, \markbox{1884}{there} \markbox{1885}{exists} \markbox{1886}{a} \markbox{1887}{rational} \markbox{m1888}{$\lambda$} \markbox{1889}{such} that
	\markbox{m1890}{$\phi(c_i) = \lambda^i c_i$}, \markbox{1891}{for} \markbox{1892}{all} \markbox{m1893}{$ i \in I.$} \markbox{1894}{Glover} \markbox{1895}{and} \markbox{1896}{Homer} (see \cite{glover-homer}) \markbox{1897}{and} \markbox{1898}{Hoffman} (see \cite{hoffman}) \markbox{1899}{proved} \markbox{1900}{the} \markbox{1901}{conjecture} \markbox{1902}{under} \markbox{1903}{the} \markbox{1904}{following} \markbox{1905}{hypothesis} respectively:
	\begin{align}
		\text{Either } k \leq 3 \text{ and } n > 2k \text{, or } k>3 \text{ and } n>2k^2 -1. \label{Homer}\\
		\text{ The graded endomorphism } \varphi  \text{ of } H^*_{\mathbb{C}G} \text{ satisfies } \varphi(c_1) = \lambda c_1, \lambda\neq 0.  \label{Hoff}
	\end{align}
	%In this context, Glover and Homer \ref{Hoffman} (see \cite{glover-homer}) proved the following result.
	Let \markbox{1906}{us} \markbox{1907}{recall} \markbox{1908}{those} \markbox{1909}{results} \markbox{1910}{proved} \markbox{1911}{in} \cite{glover-homer, hoffman} \markbox{1912}{that} \markbox{1913}{will} \markbox{1914}{be} \markbox{1915}{used} \markbox{1916}{in} \markbox{1917}{the} \markbox{1918}{rest} \markbox{1919}{of} \markbox{1920}{this} paper.  
	\begin{theorem}[\cite{glover-homer}, \markbox{1921}{Theorem} 1, \cite{hoffman}, \markbox{1922}{Theorem} 1.1]\label{hom and hof}
		(i) \markbox{1923}{Assume} \markbox{1924}{that} \markbox{1925}{the} \markbox{1926}{hypothesis} \eqref{Homer} \markbox{1927}{is} satisfied. \markbox{1928}{Then} \markbox{1929}{for} \markbox{1930}{every} \markbox{1931}{graded} \markbox{1932}{endomorphism} \markbox{m1933}{$\varphi$} \markbox{1934}{on} \markbox{m1935}{$ H^*(\mathbb{C}G_{n,k}; \mathbb{Q})$}, \markbox{1936}{there} \markbox{1937}{exists} \markbox{1938}{a} \markbox{1939}{rational} \markbox{m1940}{$\lambda$} \markbox{1941}{such} that
		\[\varphi(c_i) = \lambda^i c_i,  \quad \forall i \in I.\]
		%where $c_i$ denotes the $i$-th Chern class of the canonical complex $k$-plane bundle $\gamma_{n,k}$ over the complex Grassmannian $\mathbb{C}G_{n,k}$.
		(ii) \markbox{1942}{Assume} \markbox{1943}{that} \markbox{1944}{the} \markbox{1945}{hypothesis} \eqref{Hoff} \markbox{1946}{is} satisfied. Then, \markbox{1947}{we} have
		$$\varphi(c_i) = \begin{cases}
			\lambda^i c_i,   \forall i \in I& \text{ if } k<n-k,\\
			\lambda^i c_i,  \forall i \in I \quad \text{ or } \quad(-\lambda)^i (c^{-1})_i,    \forall i \in I & \text{ if } k= n-k,
		\end{cases}$$
		where \markbox{m1948}{$ (c^{-1})_i $} \markbox{1949}{is} \markbox{1950}{the} \markbox{m1951}{$ 2i $}-dimensional \markbox{1952}{part} \markbox{1953}{of} \markbox{1954}{the} \markbox{1955}{inverse} \markbox{1956}{of} \markbox{m1957}{$ c = 1 + c_1 + \cdots + c_k $} \markbox{1958}{in} \markbox{m1959}{$ H^*(\mathbb CG_{n,k}; \mathbb{Q}) $}.
	\end{theorem}
	\iffalse Hoffman \markbox{1960}{provided} \markbox{1961}{a} \markbox{1962}{classification} \markbox{1963}{of} \markbox{1964}{graded} \markbox{1965}{endomorphisms} \markbox{1966}{of} \markbox{m1967}{$H^*(\mathbb{C}G_{n,k}; \mathbb{Q})$} \markbox{1968}{that} \markbox{1969}{are} \markbox{1970}{nonvanishing} \markbox{1971}{on} \markbox{m1972}{$H^2(\mathbb{C}G_{n,k}; \mathbb{Q})$} (see \cite{hoffman}). \markbox{1973}{We} \markbox{1974}{recall} \markbox{1975}{his} \markbox{1976}{result} \markbox{1977}{in} \markbox{1978}{the} \markbox{1979}{following} theorem.
	\begin{theorem}[]\label{hoffman}
		Let \markbox{m1980}{$ \varphi $} \markbox{1981}{be} \markbox{1982}{an} \markbox{1983}{endomorphism} \markbox{1984}{of} \markbox{m1985}{$ H^*(\mathbb CG_{n,k}; \mathbb{Q}) $} \markbox{1986}{with} \markbox{m1987}{$ \varphi(c_1) = \lambda c_1 $}, \markbox{m1988}{$ \lambda\neq 0 $}. \markbox{1989}{Then} \markbox{1990}{if} \markbox{m1991}{$ k < n-k $},
		\[
		\varphi(c_i) = \lambda^i c_i, \quad 1 \le i \le k.
		\]
		
		If \markbox{m1992}{$ k = n $}, \markbox{1993}{there} \markbox{1994}{is} \markbox{1995}{the} \markbox{1996}{additional} possibility
		\[
		\varphi(c_i) = (-\lambda)^i (c^{-1})_i, \quad 1 \le i \le k,
		\]
		
		
	\end{theorem}\fi
	
	\subsection{Generalized Dold spaces}  \label{gen dold}
	In \cite{dold}, \markbox{1997}{the} \markbox{1998}{author} \markbox{1999}{introduced} \markbox{2000}{the} \markbox{2001}{notion} \markbox{2002}{of} \textit{classical Dold manifolds}  
	\markbox{m2003}{$P(m,n) := \mathbb{S}^m \times \mathbb{C}P^n / \!\! \sim$}  
	where \markbox{m2004}{$ (s, L) \sim (-s, \bar{L}) $}, \markbox{2005}{where} \markbox{2006}{the} \markbox{2007}{involution} \markbox{m2008}{$L\mapsto\bar L$} \markbox{2009}{on} \markbox{m2010}{$\mathbb CG_{n,k}$} \markbox{2011}{is} \markbox{2012}{induced} \markbox{2013}{from} \markbox{2014}{the} \markbox{2015}{standard} \markbox{2016}{conjugation} \markbox{2017}{on} \markbox{m2018}{$\mathbb C^n$}, \markbox{2019}{to} \markbox{2020}{construct} \markbox{2021}{generators} \markbox{2022}{in} \markbox{2023}{odd} \markbox{2024}{dimensions} \markbox{2025}{for} Ren{\'e} Thom's \markbox{2026}{unoriented} \markbox{2027}{cobordism} ring. %These spaces are of great interest, and many generalizations and studies have been done on them in the literature.
	
	\iffalse (Put \markbox{2028}{this} \markbox{2029}{in} introduction)In \cite{nath-sankaran}, \markbox{2030}{authors} \markbox{2031}{generalized} \markbox{2032}{the} \markbox{2033}{notion} \markbox{2034}{of} \markbox{2035}{Dold} \markbox{2036}{manifolds} \markbox{2037}{by} \markbox{2038}{replacing} \markbox{2039}{the} \markbox{2040}{complex} \markbox{2041}{projective} \markbox{2042}{space} \markbox{m2043}{$\mathbb{C}P^n$} \markbox{2044}{with} \markbox{2045}{an} \markbox{2046}{almost} \markbox{2047}{complex} \markbox{2048}{manifold} \markbox{m2049}{$X$} \markbox{2050}{equipped} \markbox{2051}{with} \markbox{2052}{a} \markbox{2053}{complex} conjugation, i.e., \markbox{2054}{an} \markbox{2055}{involution} \markbox{m2056}{$\sigma: X \to X$} \markbox{2057}{with} \markbox{2058}{nonempty} fixed-point \markbox{2059}{set} \markbox{2060}{such} \markbox{2061}{that} \markbox{2062}{the} differential
	\markbox{m2063}{$d\sigma|_p: T_pX \to T_{\sigma(p)}X$}
	satisfies
	\markbox{m2064}{$J_{\sigma(p)} \circ d\sigma_p = -d\sigma_p \circ J_p,$}
	where \markbox{m2065}{$J$} \markbox{2066}{denotes} \markbox{2067}{the} \markbox{2068}{almost} \markbox{2069}{complex} \markbox{2070}{structure} \markbox{2071}{on} \markbox{m2072}{$X$}, \markbox{2073}{and} \markbox{2074}{called} \markbox{2075}{them} \textit{generalized Dold manifolds} \markbox{2076}{to} \markbox{2077}{investigate} \markbox{2078}{their} \markbox{2079}{manifold} \markbox{2080}{properties} \markbox{2081}{such} \markbox{2082}{as} \markbox{2083}{tangent} bundles, Stiefel--Whitney classes, (stable) parallelizability, \markbox{2084}{cobordism} classes, \markbox{2085}{and} \markbox{2086}{related} aspects.
	\fi
	
	In \cite{nath-sankaran, mandal-sankaran}, \markbox{2087}{the} \markbox{2088}{authors} \markbox{2089}{generalized} \markbox{2090}{the} \markbox{2091}{notion} \markbox{2092}{of} \markbox{2093}{classical} \markbox{2094}{Dold} \markbox{2095}{manifolds} \markbox{2096}{by} \markbox{2097}{replacing} \markbox{2098}{the} \markbox{2099}{sphere} \markbox{m2100}{$ \mathbb{S}^m $} \markbox{2101}{with} \markbox{2102}{an} \markbox{2103}{arbitrary} \markbox{2104}{topological} \markbox{2105}{space} \markbox{m2106}{$ S $} \markbox{2107}{equipped} \markbox{2108}{with} \markbox{2109}{a} \markbox{2110}{free} \markbox{2111}{involution} \markbox{m2112}{$ \alpha $}, \markbox{2113}{analogous} \markbox{2114}{to} \markbox{2115}{the} \markbox{2116}{antipodal} \markbox{2117}{map} \markbox{2118}{on} \markbox{m2119}{$ \mathbb{S}^m $}, \markbox{2120}{and} \markbox{m2121}{$ \mathbb CP^n $} \markbox{2122}{with} \markbox{2123}{an} \markbox{2124}{arbitrary} \markbox{2125}{topological} \markbox{2126}{space} \markbox{m2127}{$ X $} \markbox{2128}{with} \markbox{2129}{an} \markbox{2130}{involution} \markbox{m2131}{$ \sigma: X \to X $} \markbox{2132}{having} \markbox{2133}{a} \markbox{2134}{nonempty} fixed-point set, \markbox{2135}{analogously} \markbox{2136}{to} \markbox{2137}{complex} \markbox{2138}{conjugation} \markbox{2139}{on} \markbox{m2140}{$\mathbb CP^n$}. \markbox{2141}{Then} \markbox{2142}{the} \markbox{2143}{quotient} \markbox{2144}{space}  
	\begin{equation}\label{gen dold space}
		P(S, \alpha, X, \sigma) := S \times X / \!\! \sim, \quad \text{where } (s, x) \sim (\alpha(s), \sigma(x)), 
	\end{equation}  
	is \markbox{2145}{called} \textit{generalized Dold space} (in \markbox{2146}{short} GDS), \markbox{2147}{often} \markbox{2148}{denoted} \markbox{2149}{simply} \markbox{2150}{as} \markbox{m2151}{$ P(S, X) $}. Moreover, \markbox{2152}{the} \markbox{2153}{quotient} \markbox{2154}{map}  
	\markbox{m2155}{$ S \times X \to P(S,X) $}  
	is \markbox{2156}{a} \markbox{2157}{double} \markbox{2158}{covering} map. %The focus of their study were used to study cohomology and complex $K$-theory of various GDS.
	
	Let \markbox{2159}{us} \markbox{2160}{fix} \markbox{2161}{a} \markbox{2162}{notation} \markbox{m2163}{$ Y $} \markbox{2164}{for} \markbox{m2165}{$ S/\!\!\sim_{\alpha} $}, \markbox{2166}{where} \markbox{m2167}{$ s\sim_{\alpha} \alpha(s), \forall s\in S $}. Then, \markbox{2168}{a} \markbox{2169}{GDS} \markbox{m2170}{$ P(S,X) $} \markbox{2171}{is} \markbox{2172}{the} \markbox{2173}{total} \markbox{2174}{space} \markbox{2175}{of} \markbox{2176}{a} \markbox{2177}{fiber} \markbox{2178}{bundle}  
	\markbox{m2179}{$X\hookrightarrow P(S,X) \twoheadrightarrow Y $}, \markbox{2180}{where} \markbox{2181}{the} \markbox{2182}{fiber} \markbox{2183}{bundle} \markbox{2184}{projection} \markbox{2185}{is} \begin{equation}\label{proj}
		p: P(S,X) \twoheadrightarrow Y, \quad [s,x]\mapsto [s].
	\end{equation}  \markbox{2186}{Choosing} \markbox{2187}{a} fixed-point \markbox{2188}{of} \markbox{m2189}{$\sigma$}, \markbox{2190}{say} \markbox{m2191}{$x_0\in \text{Fix}(\sigma)\neq \emptyset$}, \markbox{2192}{we} \markbox{2193}{can} \markbox{2194}{construct} \markbox{2195}{a} \markbox{2196}{section} \markbox{2197}{of} \markbox{2198}{the} \markbox{2199}{fiber} \markbox{2200}{bundle} \begin{equation}\label{sectio}
		s: Y \hookrightarrow P(S,X), \quad [s]\mapsto [s,x_0].
	\end{equation}    
	In fact, \markbox{2201}{we} \markbox{2202}{have} \markbox{2203}{an} \markbox{2204}{embedding}  
	\markbox{m2205}{$Y\times \text{Fix}(\sigma)\hookrightarrow P(S,X),$}  
	where \markbox{m2206}{$ \text{Fix}(\sigma) \subseteq X $} \markbox{2207}{has} \markbox{2208}{the} \markbox{2209}{subspace} \markbox{2210}{topology} \markbox{2211}{induced} \markbox{2212}{from} \markbox{m2213}{$ X $}.
	
	
	
	
	
	
	\subsection{Rational cohomology of $\mathbf{P(\mathbb S^m,\mathbb CG_{n,k})}$} \label{gds}
	%The rational cohomology ring of \textit{generalized Dold space} $ P(\mathbb{S}^m, \mathbb{C}G(\nu)) $, which are fibered by a partial complex flag manifold $ \mathbb{C}G(\nu) $ of type $ \nu = (\nu_1 \leq \nu_2 \leq \dots \leq \nu_s) $ over a real projective space $ \mathbb{R}P^m $, is computed in \cite{mandal-sankaran2}. In particular, and we recall some relevant facts here.
	
	
	The \markbox{2214}{GDS} \markbox{m2215}{$P(\mathbb S^m,\mathbb CG_{n,k})$} \markbox{2216}{is} \markbox{2217}{defined} as
	\[
	\mathbb S^m\times \mathbb CG_{n,k}/\!\!\sim, \text { where } (s,L)\sim (-s,\bar L),
	\]
	for which, \markbox{m2218}{$\mathbb S^m$} \markbox{2219}{is} \markbox{2220}{equipped} \markbox{2221}{with} \markbox{2222}{the} \markbox{2223}{free} \markbox{2224}{action} \markbox{2225}{generated} \markbox{2226}{by} \markbox{2227}{the} \markbox{2228}{antipodal} \markbox{2229}{map} \markbox{m2230}{$\alpha$} \markbox{2231}{and} \markbox{2232}{the} \markbox{2233}{involution} \markbox{m2234}{$\sigma: L \mapsto \bar L$} \markbox{2235}{on} \markbox{m2236}{$\mathbb CG_{n,k}$} \markbox{2237}{is} \markbox{2238}{induced} \markbox{2239}{from} \markbox{2240}{the} \markbox{2241}{standard} \markbox{2242}{complex} \markbox{2243}{conjugation} \markbox{2244}{on} \markbox{m2245}{$\mathbb C^n.$} \markbox{2246}{We} \markbox{2247}{denote} \markbox{m2248}{$P(\mathbb S^m,\mathbb CG_{n,k})$} \markbox{2249}{simply} \markbox{2250}{by} \markbox{m2251}{$P(m,n,k).$}
	By \markbox{2252}{the} K\"unneth formula, \markbox{2253}{we} have
	\begin{equation}\label{Cohomology of H_times}
		H_{\times}^* := H^*(\mathbb{S}^m \times \mathbb{CG}_{n,k}; \mathbb{Q}) \cong H^*(\mathbb{S}^m; \mathbb{Q}) \otimes H^*(\mathbb{CG}_{n,k}; \mathbb{Q}) \cong \frac{\mathbb{Q}[u, c_1, \dots, c_k]}{\langle u^2, h_{n-k+1}, \dots, h_n \rangle}
	\end{equation}
	where \markbox{m2254}{$u \in H^m(\mathbb S^m; \mathbb Q)$} \markbox{2255}{denotes} \markbox{2256}{the} \markbox{2257}{generator} \markbox{2258}{corresponding} \markbox{2259}{to} \markbox{2260}{the} \markbox{2261}{fundamental} \markbox{2262}{class} \markbox{2263}{of} \markbox{m2264}{$\mathbb S^m$}. \markbox{2265}{Note} \markbox{2266}{that} \begin{equation}\label{H in terms of u}
		H^*_{\times} \cong H^*_{\mathbb{C}G}[u]/\langle u^2 \rangle \cong H^*_{\mathbb{C}G} \oplus u H^*_{\mathbb{C}G},
	\end{equation} \markbox{2267}{where} \markbox{2268}{the} \markbox{2269}{latter} \markbox{2270}{isomorphism} \markbox{2271}{is} \markbox{2272}{a} \markbox{m2273}{$\mathbb Q$}-module isomorphism. \markbox{2274}{We} \markbox{2275}{have} \markbox{2276}{that} \markbox{m2277}{$H^*_{\mathbb{C}G}$} \markbox{2278}{is} \markbox{2279}{a} \markbox{2280}{subring} \markbox{2281}{of} \markbox{m2282}{$H^*_\times$}.
	%For simplicity, denote $H^*(\mathbb{S}^m \times \mathbb{CG}_{n,k}; \mathbb{Q}) $ by $H_{\times}^*$
	The \markbox{2283}{product} \markbox{2284}{involution} \markbox{m2285}{$\theta:= \alpha\times\sigma$} \markbox{2286}{on} \markbox{m2287}{$\mathbb S^m\times \mathbb CG_{n,k}$} \markbox{2288}{induces} \markbox{2289}{an} \markbox{2290}{involution} \markbox{m2291}{$\theta^*$} \markbox{2292}{on} \markbox{m2293}{$H^*_\times$} \markbox{2294}{given} by
	\begin{equation}\label{defn of theta*}
		\theta^*(c_i) = (-1)^i c_i,i\in I, \quad \theta^*(u) =  
		\begin{cases}  
			u, & \text{if } m \text{ is odd}, \\  
			-u, & \text{if } m \text{ is even}.  
		\end{cases}  
	\end{equation}
	%where $ H^*(\mathbb{S}^m; \mathbb{Q}) \cong \mathbb{Q}[u]/\langle u^2 \rangle $.
	%The induced involution $\theta^*$ is given by
	%t is proved in \cite{mandal-sankaran2} that $ H^*(P(m,n,k); \mathbb{Q}) $ is isomorphic to the fixed subring $ \text{Fix}(H^*(\theta; \mathbb{Q})) $ under $ \theta^* $ of $ H^*(\mathbb{S}^m \times \mathbb{C}G_{n,k}; \mathbb{Q}) $ (Proposition 3.13 of \cite{mandal-sankaran2}). We recall the result below.
	The \markbox{2295}{cohomology} \markbox{2296}{ring} \markbox{m2297}{$ H^*(P(m, n,k);\mathbb Q) $} \markbox{2298}{was} \markbox{2299}{computed} \markbox{2300}{in} \cite{mandal-sankaran2} \markbox{2301}{and} \markbox{2302}{the} \markbox{2303}{following} \markbox{2304}{result} \markbox{2305}{was} proved.
	\begin{theorem}[{\cite[Theorem 3.13]{mandal-sankaran2}}]\label{cohomology of P(m,n,k)}
		The \markbox{2306}{cohomology} \markbox{2307}{algebra} \markbox{m2308}{\( H^*(P(m,n,k); \mathbb{Q}) \)} \markbox{2309}{is} \markbox{2310}{isomorphic} \markbox{2311}{to} \markbox{2312}{the} subalgebra
		\markbox{m2313}{$\mathrm{Fix}(\theta^*) \subseteq H^*(\mathbb{S}^m \times \mathbb{CG}_{n,k}; \mathbb{Q}),$}
		generated \markbox{2314}{by} \markbox{2315}{the} \markbox{2316}{following} elements:
		\begin{align*}
			u \, c_{2p-1},\quad c_{2j},\quad c_{2p-1} \, c_{2q-1},\; \forall 2p-1, 2q-1, 2j \in I, \text{ if } m \text{ is even};\\
			u,\quad c_{2j},\quad c_{2p-1} \, c_{2q-1},\; \forall 2p-1, 2q-1, 2j \in I, \text{ if } m \text{ is odd}.
		\end{align*}
	\end{theorem}
	A \markbox{2317}{description} \markbox{2318}{of} \markbox{2319}{the} \markbox{2320}{cohomology} \markbox{2321}{algebra} \markbox{m2322}{$ H^*(P(m,n,k); \mathbb{Q}) $}, \markbox{2323}{as} \markbox{2324}{a} \markbox{2325}{quotient} \markbox{2326}{of} \markbox{2327}{a} \markbox{2328}{polynomial} algebra, \markbox{2329}{can} \markbox{2330}{be} \markbox{2331}{deduced} \markbox{2332}{as} \markbox{2333}{a} \markbox{2334}{particular} \markbox{2335}{case} \markbox{2336}{in} \markbox{2337}{Theorem} 3.14 \markbox{2338}{of} \cite{mandal-sankaran2}.
	%Since the description of the cohomology  algebra $H^*(P(m,n,k);\mathbb Q)$ in abstarct variables, is quite complicated in notations, instead we shall present the $\fix (H^*_\times)$ as $\mathcal R/$
	
	
	
	
	
	
	
	
	
	%%%%%%%%%%%%%%%%%%%%%%%%%%%5%%%%%%
	
	\section{Graded endomorphisms of $H^*(\mathbb S^m\times \mathbb CG_{n,k};\mathbb Q)$} \label{section 3}
	
	In \markbox{2339}{this} section, \markbox{2340}{we} \markbox{2341}{classify} \markbox{2342}{graded} \markbox{2343}{endomorphisms} \markbox{2344}{of} \markbox{2345}{the} \markbox{2346}{rational} \markbox{2347}{cohomology} \markbox{2348}{algebra} \markbox{m2349}{$ H^*(\mathbb S^m\times \mathbb CG_{n,k}; \mathbb{Q}) $} \markbox{2350}{whose} \markbox{2351}{images} \markbox{2352}{are} \markbox{2353}{nonzero} \markbox{2354}{in} \markbox{m2355}{$H^2(\mathbb CG_{n,k};\mathbb Q)$}. \markbox{2356}{Our} \markbox{2357}{approach} \markbox{2358}{relies} \markbox{2359}{on} \markbox{2360}{the} \markbox{2361}{study} \markbox{2362}{of} \markbox{2363}{graded} \markbox{2364}{endomorphisms} \markbox{2365}{of} \markbox{m2366}{$ H^*(\mathbb{C}G_{n,k}; \mathbb{Q}) $} \markbox{2367}{from} \cite{glover-homer} \markbox{2368}{and} \cite{hoffman}. \markbox{2369}{Assume} \markbox{m2370}{$m>0$} \markbox{2371}{for} \markbox{2372}{the} \markbox{2373}{rest} \markbox{2374}{of} \markbox{2375}{this} paper.
	\subsection{} \markbox{2376}{The} \markbox{2377}{cohomology} \markbox{2378}{ring} \markbox{2379}{of} \markbox{2380}{the} \markbox{2381}{complex} \markbox{2382}{Grassmannian} \markbox{m2383}{$ \mathbb{C}G_{n,k} $} \markbox{2384}{is} \markbox{2385}{generated} \markbox{2386}{by} \markbox{2387}{the} \markbox{2388}{Chern} \markbox{2389}{classes} \markbox{m2390}{$c_i,\forall i \in I $} \markbox{2391}{as} \markbox{2392}{given} \markbox{2393}{in} \eqref{cohomo of grass}. \markbox{2394}{In} \eqref{Cohomology of H_times}, \markbox{2395}{we} \markbox{2396}{see} \markbox{2397}{that} \markbox{2398}{the} \markbox{2399}{cohomology} \markbox{2400}{ring} \markbox{2401}{of} \markbox{m2402}{$S^m \times \mathbb{C}G_{n,k}$} \markbox{2403}{is} \markbox{2404}{generated} \markbox{2405}{by} \markbox{m2406}{$u, c_i, \forall i \in I$}. Therefore, \markbox{2407}{it} \markbox{2408}{is} \markbox{2409}{sufficient} \markbox{2410}{to} \markbox{2411}{describe} \markbox{2412}{the} \markbox{2413}{images} \markbox{2414}{of} \markbox{2415}{the} \markbox{2416}{generators} \markbox{2417}{to} \markbox{2418}{classify} \markbox{2419}{graded} \markbox{2420}{endomorphisms} \markbox{2421}{of} \markbox{m2422}{$H_{\times}^*$}. \markbox{2423}{The} \markbox{2424}{following} \markbox{2425}{is} \markbox{2426}{the} \markbox{2427}{main} \markbox{2428}{result} \markbox{2429}{of} \markbox{2430}{this} section.
	
	
	
	
	
	
	
	% We now state the following result concerning the classification of graded endomorphisms of $H^*_\times$ that are non-vanishing on  $H^2_{\mathbb{C}G} \subseteq H^*_\times$.
	\begin{theorem}\label{main thm}
		Let \markbox{m2431}{$\phi$} \markbox{2432}{be} \markbox{2433}{a} \markbox{2434}{graded} \markbox{2435}{endomorphism} \markbox{2436}{of} \markbox{m2437}{$H^*_{\times}$} \markbox{2438}{satisfying} \markbox{m2439}{$\phi(c_1) \neq \mu u,\, \mu \in \mathbb{Q}$}.  
		Then \markbox{2440}{the} \markbox{2441}{following} holds, \begin{enumerate}
			\item Either \markbox{m2442}{$\phi(u)=au$} \markbox{2443}{for} \markbox{2444}{some} \markbox{m2445}{$a \in \mathbb{Q}$}, \markbox{2446}{or} \markbox{m2447}{$\phi(u) \in H^*_{\mathbb{C}G} \subseteq H^*_{\times}$} \markbox{2448}{with} \markbox{m2449}{$\phi(u)^2=0$} \markbox{2450}{in} \markbox{m2451}{$H^*_{\times}$}.
			\item There \markbox{2452}{exists} \markbox{m2453}{$\lambda \in \mathbb Q\backslash\{0\}$} \markbox{2454}{such} that
			$$\phi(c_i) = \begin{cases}
				\lambda^i c_i,   \forall i \in I& \text{ if } k<n-k,\\
				\lambda^i c_i,  \forall i \in I \quad \text{ or } \quad(-\lambda)^i (c^{-1})_i,    \forall i \in I & \text{ if } k= n-k,
			\end{cases}$$
		\end{enumerate} \markbox{2455}{where} \markbox{m2456}{$ (c^{-1})_i $} \markbox{2457}{is} \markbox{2458}{the} \markbox{m2459}{$ 2i $}-dimensional \markbox{2460}{part} \markbox{2461}{of} \markbox{2462}{the} \markbox{2463}{inverse} \markbox{2464}{of} \markbox{m2465}{$ c = 1 + c_1 + \cdots + c_k $} \markbox{2466}{in} \markbox{m2467}{$ H^*_{ \mathbb CG} $}.
	\end{theorem}
	
	% \textcolor{red}{Maybe we can write this theorem in full generality when $\mathbb CG_{n,k}$ is replaced by $G/H$ as in Theorem $A'$ in \cite{shiga-tezuka}, and deduce this result as corollary.}
	
	\begin{proof}
		%Since $\phi$ is graded and non-vanishing on $H^2_{\mathbb CG}$, the image $\phi(c_1) = \lambda c_1+uP_1$ for some nonzero $\lambda \in \mathbb{Q}$ and $P_1\in H^{2-m}_{\mathbb CG}$.
		From \markbox{2468}{equation} \eqref{Cohomology of H_times} \markbox{2469}{and} \eqref{H in terms of u}, \markbox{2470}{we} \markbox{2471}{have} \markbox{m2472}{$H^*_{\times}\cong \mathcal R/\mathcal I \cong H^*_{\mathbb{C}G} \oplus u H^*_{\mathbb{C}G}$}, \markbox{2473}{where} \markbox{m2474}{$\mathcal R:=\mathbb Q[u,c_1,\ldots,c_k]$} \markbox{2475}{and} \markbox{m2476}{$\mathcal I:=\langle u^2, h_{n-k+1},\ldots,h_n\rangle$}.
		%, we may regard $H^*_{\mathbb{C}G}$ as a subring of $H^*_{\times}$.
		%Therefore, the elements in $H^*_\times$ are of the $P+uQ$ where $P$ and $Q$ are in $H^*_{\mathbb{C}G}$.
        %that are not multiple of $u$, can be written purely in terms of $c_1,c_2,\ldots,c_k.$
		
		%We begin with \textit{(1)} for $k < n - k$.  
		Let \markbox{m2477}{$p_1: H^*_{\times} = H^*_{\mathbb{C}G} \oplus u H^*_{\mathbb{C}G} \to H^*_{\mathbb{C}G}$} \markbox{2478}{be} \markbox{2479}{the} \markbox{2480}{projection} \markbox{2481}{onto} \markbox{2482}{the} \markbox{2483}{first} \markbox{2484}{summand} \markbox{2485}{and} \markbox{m2486}{$i_1: H^*_{\mathbb{C}G} \hookrightarrow H^*_{\mathbb{C}G} \oplus u H^*_{\mathbb{C}G}$} \markbox{2487}{be} \markbox{2488}{the} \markbox{2489}{inclusion} \markbox{2490}{into} \markbox{2491}{the} \markbox{2492}{first} summand. \markbox{2493}{The} \markbox{2494}{composite} \markbox{m2495}{$\phi_1:= p_1 \circ \phi \circ i_1$} \markbox{2496}{is} \markbox{2497}{a} degree-preserving \markbox{2498}{endomorphism} \markbox{2499}{of} \markbox{m2500}{$H^*_{\mathbb{C}G}$}. \markbox{2501}{We} \markbox{2502}{have} \markbox{2503}{the} \markbox{2504}{following} diagram:
		% https://q.uiver.app/#q=WzAsNCxbMCwwLCJIXipfe1xcbWF0aGJiIENHfVxcb3BsdXMgdSBIXipfe1xcbWF0aGJiIENHfSJdLFsxLDAsIkheKl97XFxtYXRoYmIgQ0d9IFxcb3BsdXMgdSBIXipfe1xcbWF0aGJiIENHfSJdLFswLDEsIkheKl97XFxtYXRoYmIgQ0d9Il0sWzEsMSwiSF4qX3tcXG1hdGhiYiBDR30iXSxbMCwxLCJcXHBoaSJdLFsyLDAsImlfMSIsMCx7InN0eWxlIjp7InRhaWwiOnsibmFtZSI6Imhvb2siLCJzaWRlIjoidG9wIn19fV0sWzEsMywicF8xIiwwLHsic3R5bGUiOnsiaGVhZCI6eyJuYW1lIjoiZXBpIn19fV0sWzIsMywiXFx0aWxkZVxccGhpIl1d
		\begin{equation}\label{comm diagram}
			\begin{tikzcd}
				{H^*_{\mathbb CG}\oplus u H^*_{\mathbb CG}} & {H^*_{\mathbb CG} \oplus u H^*_{\mathbb CG}} \\
				{H^*_{\mathbb CG}} & {H^*_{\mathbb CG}}
				\arrow["\phi", from=1-1, to=1-2]
				\arrow["{p_1}", two heads, from=1-2, to=2-2]
				\arrow["{i_1}", hook, from=2-1, to=1-1]
				\arrow["{\phi_1}", from=2-1, to=2-2]
			\end{tikzcd}
		\end{equation}
		Thus, \markbox{2505}{for}  \markbox{m2506}{$x \in H^*_{\mathbb{C}G} \subset H^*_\times$}, \markbox{2507}{one} \markbox{2508}{can} write
		\markbox{m2509}{$\phi(x) = \phi_1(x) + u P_x$}
		for \markbox{2510}{some} \markbox{m2511}{$P_x \in H^*_{\mathbb{C}G} \subset H^*_\times$} \markbox{2512}{because} \markbox{2513}{the} \markbox{2514}{kernal} \markbox{2515}{of} \markbox{m2516}{$p_1$}, \markbox{m2517}{$\ker(p_1) = u H^*_{\mathbb{C}G}$}.
		\iffalse we have
		\begin{equation} 
			\phi(x) = \phi_1(x) + u P_x, \quad \text{where  } P_x\in H^*_{\mathbb CG}.
		\end{equation}\fi
		This \markbox{2518}{implies} \markbox{2519}{that} \begin{equation}\label{defn of phi}
			\phi(c_i) = \phi_1(c_i) + u P_{c_i},\, \forall i \in I.
		\end{equation} \markbox{2520}{For} simplicity, \markbox{2521}{denote} \markbox{m2522}{$P_{c_i}$} \markbox{2523}{by} \markbox{m2524}{$P_i\in H^{2i-m}_{\mathbb CG}$} \markbox{2525}{which} \markbox{2526}{is} \markbox{2527}{a} \markbox{2528}{polynomial} \markbox{2529}{in} \markbox{m2530}{$c_1, \dots, c_k$} \markbox{2531}{of} \markbox{2532}{degree} \markbox{m2533}{$2i - m$} \markbox{2534}{as} \markbox{m2535}{$\deg c_i = 2i$} \markbox{2536}{and} \markbox{m2537}{$\deg u = m$}.
		
		
		
		Since \markbox{m2538}{$\phi(c_1)\neq \mu u, \, \mu \in \mathbb{Q}$}, \markbox{2539}{that} \markbox{2540}{implies} \markbox{m2541}{$\phi(c_1)$} \markbox{2542}{is} \markbox{2543}{of} \markbox{2544}{the} \markbox{2545}{form} \markbox{m2546}{$\lambda c_1+\mu u,\, \lambda,\mu \in \mathbb{Q}, \, \lambda \neq 0$}. \markbox{2547}{Then} \markbox{2548}{we} \markbox{2549}{have} \markbox{m2550}{$\phi_1(c_1) = \lambda c_1,\, \lambda\neq 0$} \markbox{2551}{on} \markbox{m2552}{$H^*_{\mathbb{C}G}$}.  \markbox{2553}{By} \thmref{hom and hof} \markbox{2554}{part} \textit{(ii)}, \markbox{2555}{we} have, 
		\begin{equation}\label{phi_1}
			\phi_1(c_i) = \begin{cases}
				\lambda^i c_i,   \forall i \in I& \text{ if } k<n-k,\\
				\lambda^i c_i,  \forall i \in I \quad \text{ or } \quad(-\lambda)^i (c^{-1})_i,    \forall i \in I & \text{ if } k= n-k,
			\end{cases}
		\end{equation} \markbox{2556}{where} \markbox{m2557}{$ (c^{-1})_i $} \markbox{2558}{is} \markbox{2559}{the} \markbox{m2560}{$ 2i $}-dimensional \markbox{2561}{part} \markbox{2562}{of} \markbox{2563}{the} \markbox{2564}{inverse} \markbox{2565}{of} \markbox{m2566}{$ c = 1 + c_1 + \cdots + c_k $} \markbox{2567}{in} \markbox{m2568}{$ H^*_{ CG} $}. \markbox{2569}{Using} \markbox{2570}{the} \markbox{2571}{observations} \markbox{2572}{given} \markbox{2573}{above} \markbox{2574}{it} \markbox{2575}{is} \markbox{2576}{convenient} \markbox{2577}{to} \markbox{2578}{prove} \markbox{2579}{part} \textit{(2)} first. \\
		
		
		\textit{proof of part (2):} \markbox{2580}{Using} \eqref{defn of phi} \markbox{2581}{and} \eqref{phi_1}, \markbox{2582}{it} \markbox{2583}{is} \markbox{2584}{sufficient} \markbox{2585}{to} \markbox{2586}{prove} \markbox{2587}{that} \markbox{m2588}{$P_i =0, \, \forall i \in I$}. \markbox{2589}{By} \eqref{phi_1}, \markbox{2590}{we} \markbox{2591}{have} \markbox{2592}{that} \markbox{m2593}{$\phi_1$} \markbox{2594}{is} \markbox{2595}{an} \markbox{2596}{automorphism} \markbox{2597}{of} \markbox{m2598}{$H^*_{\mathbb CG}$}. \markbox{2599}{Using} \markbox{2600}{the} \markbox{2601}{invertibility} \markbox{2602}{of} \markbox{m2603}{$\phi_1$} \markbox{2604}{and} \eqref{defn of phi}, \markbox{2605}{let} \markbox{m2606}{$D: H^*_{\mathbb{C}G} \rightarrow H^*_{\mathbb{C}G}$} \markbox{2607}{be} \markbox{2608}{defined} \markbox{2609}{by} $$D(x) = P_{\phi_1^{-1}(x)},\, \forall x \in H^*_{\mathbb{C}G}.$$
		%By degree comparison, $P_i =0, \, \forall i \in I$ if $m$ is odd or $m > 2k$. Hence, we assume that $m$ is even, i.e. $m = 2s$ with $s\in I$.
		%whenever $\phi(x) = \phi_1(x) + uP_x$.  
		Equivalently, \markbox{2610}{we} \markbox{2611}{have} \markbox{m2612}{$D(\phi_1(x)) = P_x$}. Now, \markbox{2613}{we} \markbox{2614}{prove} \markbox{2615}{that} \markbox{m2616}{$D$} \markbox{2617}{is} \markbox{2618}{a} \markbox{m2619}{$\mathbb Q$}-linear \markbox{2620}{transformation} \markbox{2621}{and} \markbox{2622}{satisfies} \markbox{2623}{the} \markbox{2624}{Leibniz} rule.
		%Linearity over $\mathbb{Q}$ is immediate, since for $t \in \mathbb{Q}$ one has
		\begin{equation}\label{D is linear}
			\begin{split}
				uP_{tx} &= \phi(tx) - \phi_1(tx) = t(\phi(x) - \phi_1(x)) = utP_{x},\, \forall t \in \mathbb{Q},\\
				uP_{x + y} &= \phi(x+y) - \phi_1(x+y) = \phi(x) - \phi_1(x)+ \phi(y) - \phi_1(y)\\ &= u(P_{x}+P_{y}),\\
				uP_{x y} &= \phi(x y) - \phi_1(xy) = \phi(x)\phi(y)-\phi_1(x)\phi_1(y) \\
				&= (\phi_1(x)+uP_{x})(\phi_1(y)+uP_y)- \phi_1(x)\phi_1(y)\\
				&= u(P_x\phi_1(y)+\phi_1(x)P_y).
			\end{split}
		\end{equation}
		Using \eqref{H in terms of u} \markbox{2625}{and} \eqref{D is linear}, \markbox{2626}{we} \markbox{2627}{get}  \begin{align*}
			& D(t\phi_1(x)) = tD(\phi_1(x)),\quad D(\phi_1(x)+\phi_1(y)) = D(\phi_1(x))+D(\phi_1(y)),\\
			&D(\phi_1(x)\phi_1(y)) = D(\phi_1(x))\phi_1(y)+\phi_1(x)D(\phi_1(y)).
		\end{align*}
		\iffalse \textcolor{red}{(Do we need to emphasise that $D(h_i)\in \langle h_{n-k+1},\ldots,h_n\rangle$, or it is obvious from linearlity?)}
		
		To \markbox{2628}{verify} \markbox{2629}{that} \markbox{m2630}{$D$} \markbox{2631}{satisfies} \markbox{2632}{the} \markbox{2633}{Leibniz} \markbox{2634}{condition} \markbox{m2635}{$D(xy)=D(x)y+xD(y)$},
		it \markbox{2636}{suffices} \markbox{2637}{to} \markbox{2638}{check} that
		\markbox{m2639}{$D(\phi_1(x)\phi_1(y))=D(\phi_1(x))\phi_1(y)+\phi_1(x)D(\phi_1(y))$}
		for \markbox{2640}{all} \markbox{m2641}{$x,y \in H^*_{\mathbb{C}G}$}, \markbox{2642}{since} \markbox{m2643}{$\phi_1$} \markbox{2644}{is} \markbox{2645}{an} \markbox{2646}{automorphism} \markbox{2647}{of} \markbox{m2648}{$H^*_{\mathbb{C}G}$}.
		Indeed, from
		\markbox{m2649}{$\phi(xy)=\phi(x)\phi(y)=(\phi_1(x)+uP_x)(\phi_1(y)+uP_y)$} \markbox{2650}{where} \markbox{m2651}{$x,y\in H^*_{\mathbb CG}$}, \markbox{2652}{we} obtain
		\markbox{m2653}{$\phi(xy)=\phi_1(xy)+u(P_x\phi_1(y)+\phi_1(x)P_y)$}.
		Comparing \markbox{2654}{with} \markbox{m2655}{$\phi(xy)=\phi_1(xy)+uP_{xy}$} gives
		\markbox{m2656}{$P_{xy}=P_x\phi_1(y)+\phi_1(x)P_y$}, \markbox{2657}{and} therefore
		\markbox{m2658}{$D(\phi_1(x)\phi_1(y))=D(\phi_1(x))\phi_1(y)+\phi_1(x)D(\phi_1(y))$}. \fi
		This \markbox{2659}{proves} \markbox{2660}{that} \markbox{m2661}{$D$} \markbox{2662}{is} \markbox{2663}{a} derivation. \markbox{2664}{For} \markbox{m2665}{$x \in H^i_{\mathbb{C}G}$}, \markbox{2666}{we} \markbox{2667}{have} \markbox{m2668}{$D(x) \in H^{i-m}_{\mathbb{C}G}$} \markbox{2669}{which} \markbox{2670}{implies} \markbox{2671}{that} \markbox{2672}{the} \markbox{2673}{derivation} \markbox{m2674}{$D$} \markbox{2675}{decreases} \markbox{2676}{the} \markbox{2677}{degree} \markbox{2678}{by} \markbox{m2679}{$\deg(u)=m>0$}. \markbox{2680}{By} \eqref{cgn as hom} \markbox{2681}{and} \thmref{Tezuka}, \markbox{2682}{we} \markbox{2683}{get} \markbox{2684}{that} \markbox{m2685}{$D$} \markbox{2686}{is} \markbox{2687}{a} \markbox{2688}{zero} derivation. \markbox{2689}{In} \markbox{2690}{particular} $$D(\phi_1(c_i))=P_i=0, \, \forall i \in I.$$
		%This completes the proof of \textit{(1)} in both cases $k<n-k$ and $k=n-k$.
		
		\textit{proof of part (1):} \markbox{2691}{Since} \markbox{m2692}{$\phi$} \markbox{2693}{is} \markbox{2694}{a} \markbox{2695}{graded} \markbox{2696}{endomorphism} \markbox{2697}{on} \markbox{m2698}{$H^*_{\times}$}, \markbox{2699}{therefore} $$\phi(u) = a u + P, \, a \in \mathbb{Q}, \text{ satisfying } (a u + P)^2 =0,$$ \markbox{2700}{where} \markbox{m2701}{$P$} \markbox{2702}{is} \markbox{2703}{a} \markbox{2704}{homogeneous} \markbox{2705}{polynomial} \markbox{2706}{in} \markbox{m2707}{$c_1, \dots, c_k$} \markbox{2708}{of} \markbox{2709}{degree} \markbox{m2710}{$m$}. \markbox{2711}{We} \markbox{2712}{have} \markbox{m2713}{$P^2 + 2 a u P =0$} \markbox{2714}{in} \markbox{m2715}{$H^*_{\times}$}. \markbox{2716}{Using} \eqref{H in terms of u}, \markbox{2717}{we} \markbox{2718}{get} \markbox{2719}{that} \markbox{m2720}{$2aP =0$} \markbox{2721}{in} \markbox{m2722}{$H^*_{\times}=\mathcal{R}/\mathcal{I}$}. Hence, \markbox{2723}{either} \markbox{m2724}{$a=0$} \markbox{2725}{or} \markbox{m2726}{$P\in \mathcal{I}$}.
		%for some $f, f_i\in \mathcal R$ and generators $h_i$ of $\mathcal{I}\subset \mathcal R$. Write $f_i = g_i + u l_i$ with $g_i, l_i$ in $\mathbb{Q}[c_1, \dots, c_k]$, then comparing terms not in $u^2$ gives:\[P^2 + 2 a u P = \sum g_i h_i + u \sum l_i h_i.\]Thus, $2 a u P = u \sum l_i h_i \in \mathcal{I}$, so if $a \neq 0$, then $P \in \mathcal{I}$ and hence $\phi(u) = a u$ in $H^*_{\times}$. If $a = 0$, then $\phi(u) = P(c_1, \dots, c_k)$ with $P^2 \in \mathcal{I}$.This completes the proof.
	\end{proof}
	
	
		\begin{remark}
			\thmref{main thm} \markbox{2727}{classifies} \markbox{2728}{all} \markbox{2729}{graded} \markbox{2730}{endomorphisms} \markbox{m2731}{$\phi$} \markbox{2732}{of} \markbox{m2733}{$H^*_\times$} \markbox{2734}{whose} \markbox{2735}{image} \markbox{2736}{is} \markbox{2737}{nonzero} \markbox{2738}{in} \markbox{m2739}{$H^2_{\mathbb CG}$} \markbox{2740}{if} \markbox{m2741}{$n>2$}. \markbox{2742}{In} fact, \markbox{m2743}{$n>2$} \markbox{2744}{implies} \markbox{m2745}{$c_1^2\neq 0$} \markbox{2746}{and} \markbox{m2747}{$\phi(u) \neq ac_1,\, a \in \mathbb{Q}\setminus\{0\}$} \markbox{2748}{as} \markbox{m2749}{$\phi(u)^2=0$}. Therefore, \markbox{2750}{the} \markbox{2751}{only} \markbox{2752}{remaining} \markbox{2753}{possibility} \markbox{2754}{is} \markbox{m2755}{$\phi(c_1)\neq \mu u,\, \mu \in \mathbb Q.$} 
			
			On \markbox{2756}{the} \markbox{2757}{other} hand, \markbox{2758}{when} \markbox{m2759}{$n=2,$} \markbox{m2760}{$\mathbb CG_{n,k}$} \markbox{2761}{is} \markbox{2762}{either} \markbox{2763}{a} \markbox{2764}{point} \markbox{2765}{or} \markbox{m2766}{$\mathbb S^2$} \markbox{2767}{and} \markbox{2768}{the} \markbox{2769}{classification} \markbox{2770}{of} \markbox{2771}{graded} \markbox{2772}{endomorphisms} \markbox{2773}{of} \markbox{m2774}{$H^*_\times$} \markbox{2775}{is} easy.
		\end{remark}
	
	
	
	\subsection{} \markbox{2776}{In} \thmref{main thm}, \markbox{2777}{we} \markbox{2778}{assume} \markbox{2779}{that} \markbox{m2780}{$\phi(c_1) \neq \mu u$}. \markbox{2781}{Let} \markbox{2782}{us} \markbox{2783}{try} \markbox{2784}{to} \markbox{2785}{look} \markbox{2786}{at} \markbox{2787}{the} \markbox{2788}{other} \markbox{2789}{case} \markbox{2790}{where} \markbox{m2791}{$\phi(c_1) = \mu u$}. \markbox{2792}{To} \markbox{2793}{address} this, \markbox{2794}{we} \markbox{2795}{use} \markbox{2796}{part} (i) \markbox{2797}{of} \thmref{hom and hof} \markbox{2798}{which} \markbox{2799}{leads} \markbox{2800}{to} \markbox{2801}{the} \markbox{2802}{following} proposition.
	
	
	\begin{proposition}\label{main thm 2}
		Assume \markbox{2803}{that} \markbox{2804}{hypothesis} \eqref{Homer} \markbox{2805}{is} satisfied.
		Let \markbox{m2806}{$\phi$} \markbox{2807}{be} \markbox{2808}{a} \markbox{2809}{graded} \markbox{2810}{endomorphism} \markbox{2811}{such} \markbox{2812}{that} \markbox{m2813}{$\phi(c_1)=\mu u,\, \mu \in \mathbb{Q}$} \markbox{2814}{in} \markbox{m2815}{$H^*_{\times}$}. Then
		\begin{enumerate}
			\item Either \markbox{m2816}{$\phi(u)=a u$} \markbox{2817}{for} \markbox{2818}{some} \markbox{m2819}{$a \in \mathbb{Q}$}, \markbox{2820}{or} \markbox{m2821}{$\phi(u) \in H^*_{\mathbb{C}G} \subseteq H^*_{\times}$} \markbox{2822}{with} \markbox{m2823}{$\phi(u)^2=0$} \markbox{2824}{in} \markbox{m2825}{$H^*_{\times}$}.
			\item  \markbox{m2826}{$\phi(c_i) = uP_i, \, \forall i >1,$} \markbox{2827}{where} \markbox{m2828}{$P_i \in H^{2i-m}_{\mathbb CG}\subseteq H^*_\times$}. 
		\end{enumerate}
	\end{proposition}
	\begin{proof} \textit{(1):} \markbox{2829}{The} \markbox{2830}{proof} \markbox{2831}{of} \markbox{2832}{part} \textit{(1)} \markbox{2833}{is} \markbox{2834}{exactly} \markbox{2835}{the} \markbox{2836}{same} \markbox{2837}{as} \markbox{2838}{the} \markbox{2839}{proof} \markbox{2840}{of} \markbox{2841}{part} \textit{(1)} \markbox{2842}{of} \thmref{main thm}. Therefore, \markbox{2843}{we} \markbox{2844}{omit} \markbox{2845}{the} details.
		
		\textit{(2):} \markbox{2846}{Using} \eqref{comm diagram}, \markbox{2847}{we} \markbox{2848}{have} \markbox{2849}{that} \markbox{2850}{the} \markbox{2851}{map} \markbox{m2852}{$\phi_1$} \markbox{2853}{is} \markbox{2854}{a} \markbox{2855}{graded} \markbox{2856}{endomorphism} \markbox{2857}{on} \markbox{m2858}{$H^*_{\mathbb{C}G}$} \markbox{2859}{such} \markbox{2860}{that} \markbox{m2861}{$\phi_1(c_1) =0$}. \markbox{2862}{By} \thmref{hom and hof}, \markbox{m2863}{$\phi_1(c_i) =0, \, \forall i\in I$}, \markbox{2864}{then} \markbox{2865}{by} \eqref{defn of phi}, \markbox{2866}{we} \markbox{2867}{get} \markbox{m2868}{$\phi(c_i) = uP_i$} \markbox{2869}{for} \markbox{2870}{some} \markbox{m2871}{$P_i \in H^*_{\mathbb{C}G}$}, \markbox{2872}{with} \markbox{m2873}{$\deg(P_i) = 2i - m$}. \end{proof}
	%  When $\lambda \neq 0$, the proof follows directly from Theorem~\ref{main thm}.
	\begin{remark}
		In \thmref{main thm} \markbox{2874}{and} \propref{main thm 2}, \markbox{2875}{if} \markbox{2876}{we} \markbox{2877}{assume} \markbox{m2878}{$2m \leq n-k$}  \markbox{2879}{then} \markbox{m2880}{$\phi(u)=0$} \markbox{2881}{whenever} \markbox{m2882}{$\phi(u) \in H^*_{\mathbb{C}G}$}. \markbox{2883}{This} \markbox{2884}{is} \markbox{2885}{because} \markbox{m2886}{$H^*_{\mathbb{C}G}$} \markbox{2887}{has} \markbox{2888}{no} \markbox{2889}{nontrivial} \markbox{2890}{relations} \markbox{2891}{up} \markbox{2892}{to} \markbox{2893}{degree} \markbox{m2894}{$2(n-k)$} \markbox{2895}{and} \markbox{m2896}{$u^2=0$} \markbox{2897}{implies} \markbox{2898}{that} \markbox{m2899}{$\phi(u)^2=0$} \markbox{2900}{forcing} \markbox{m2901}{$\phi(u)=0$}.
		
	\end{remark}
	%Now suppose $\lambda = 0$, i.e., $\phi(c_1) = 0$ in $H^*_\times$.
	%Unlike the case of Grassmannian, the following proposition guarantees the existence of non-trivial graded endomorphisms $\phi$ on $H^*_\times$ even if $\phi(c_1) =0$.
   A \markbox{2902}{graded} \markbox{2903}{endomorphism} \markbox{2904}{of} \markbox{m2905}{$H^*_{\mathbb CG}$} \markbox{2906}{that} \markbox{2907}{vanishes} on
   \markbox{m2908}{$H^2_{\mathbb CG}$} \markbox{2909}{is} \markbox{2910}{expected} \markbox{2911}{to} \markbox{2912}{be} trivial, \markbox{2913}{in} \markbox{2914}{view} \markbox{2915}{of} Hoffman's \markbox{2916}{conjecture} \cite{hoffman}. However, \markbox{2917}{unlike} \markbox{2918}{the} \markbox{2919}{case} \markbox{2920}{of} \markbox{2921}{the} \markbox{2922}{complex} Grassmannian, \markbox{2923}{there} \markbox{2924}{exist} \markbox{2925}{many} non-trivial \markbox{2926}{graded} \markbox{2927}{endomorphisms} \markbox{2928}{of} \markbox{m2929}{$H^*_\times$} \markbox{2930}{that} \markbox{2931}{vanish} \markbox{2932}{on} \markbox{m2933}{$H^2_{\mathbb CG}$}. \markbox{2934}{The} \markbox{2935}{following} \markbox{2936}{proposition} \markbox{2937}{provides}   \markbox{2938}{such} \markbox{2939}{examples} \markbox{2940}{when} \markbox{m2941}{$m$} \markbox{2942}{is} \markbox{2943}{even} \markbox{2944}{and}   \markbox{m2945}{$1\le m \le 2k$}.



	
	
	\begin{proposition}
		For \markbox{2946}{each} \markbox{m2947}{$i\in I$}, \markbox{2948}{choose} \markbox{m2949}{$P_i \in H^{2i-m}_{\mathbb C G} \subseteq H^*_\times$} \markbox{2950}{and} \markbox{2951}{either} \markbox{m2952}{$Q = au,\, a\in \mathbb Q$}, \markbox{2953}{or}  \markbox{m2954}{$Q\in H^*_{\mathbb CG}\subseteq H^*_{\times}$} \markbox{2955}{with} \markbox{m2956}{$Q^2=0$}  \markbox{2957}{in} \markbox{m2958}{$ H^*_{\times}$}. \markbox{2959}{Then} \markbox{2960}{there} \markbox{2961}{exist} \markbox{2962}{a} \markbox{2963}{graded} \markbox{2964}{endomorphism} \markbox{m2965}{$\phi$} \markbox{2966}{on} \markbox{m2967}{$H^*_\times$} \markbox{2968}{such} \markbox{2969}{that} 
		\[
		\phi(c_i)=uP_i, \; \forall i\in I, \text{ and  } \quad \phi(u)=Q.
		\]
	\end{proposition}
	\begin{proof}
		Define \markbox{m2970}{$\phi$} \markbox{2971}{on} \markbox{m2972}{$H^*_{\times} = \mathcal{R}/\mathcal{I}$} \markbox{2973}{by} \markbox{m2974}{$\phi(c_i)=uP_i, \; \forall i\in I, \text{ and } \phi(u)=Q.$} \markbox{2975}{It} \markbox{2976}{is} \markbox{2977}{sufficient} \markbox{2978}{to} \markbox{2979}{prove} \markbox{2980}{that} \markbox{m2981}{$\phi$} \markbox{2982}{is} \markbox{2983}{well} defined, \markbox{2984}{that} is, \markbox{m2985}{$\mathcal{I}\subseteq \ker (\phi)$}. \markbox{2986}{Observe} \markbox{2987}{that} \markbox{m2988}{$u^2 =0$} \markbox{2989}{in} \markbox{m2990}{$H^*_{\times}$} \markbox{2991}{which} \markbox{2992}{implies} \markbox{2993}{that} \begin{equation}\label{ideal cicj}
			\phi(c_i c_j) = \phi(c_i) \phi(c_j) = uP_i \cdot uP_j = u^2 P_i P_j = 0.
		\end{equation} \markbox{2994}{Using} \eqref{ideal cicj} \markbox{2995}{and} \markbox{m2996}{$\phi(u^2)=Q^2 =0$}, \markbox{2997}{we} \markbox{2998}{have} \markbox{m2999}{$\mathcal{I}\subseteq \langle u^2, c_i c_j \,|\, i,j \in I \rangle \subseteq \ker(\phi).$} 
	\end{proof}
	
	
	\subsection{} \markbox{3000}{In} \markbox{3001}{this} subsection, \markbox{3002}{we} \markbox{3003}{derive} \markbox{3004}{some} \markbox{3005}{immediate} \markbox{3006}{applications} \markbox{3007}{of} \thmref{main thm}.
	\begin{corollary}
		%It is natural to consider a more general situation with several spheres instead of a single one.  
		Let \markbox{3008}{us} \markbox{3009}{consider} \markbox{m3010}{$X = \mathbb{S}^{2m_1} \times \cdots \times \mathbb{S}^{2m_r} \times \mathbb{C}G_{n,k}$} \markbox{3011}{and} \markbox{3012}{denote} \markbox{3013}{by} \markbox{m3014}{$u_j$} \markbox{3015}{the} \markbox{3016}{generator} \markbox{3017}{of} \markbox{m3018}{$H^{2m_j}(\mathbb{S}^{2m_j}; \mathbb{Q})$} \markbox{3019}{corresponding} \markbox{3020}{to} \markbox{3021}{the} \markbox{3022}{fundamental} \markbox{3023}{class} \markbox{3024}{of} \markbox{m3025}{$\mathbb S^{2m_j}$} \markbox{3026}{for} \markbox{3027}{all} \markbox{m3028}{$1\leq j \leq r.$} \markbox{3029}{Define}  
		\[
		H^*_{\mathbf{m}, \mathbb{C}G} := H^*(\mathbb{S}^{2m_1} \times \cdots \times \mathbb{S}^{2m_r} \times \mathbb{C}G_{n,k}; \mathbb{Q})
		\;\cong\; H^*_{\mathbb{C}G}[u_1,\ldots,u_r] \big/ \langle u_1^2,\ldots,u_r^2 \rangle,
		\] \markbox{3030}{where} \markbox{m3031}{$\mathbf{m} = (m_1, \ldots, m_r)$}.
		%where each $u_j$ denotes the degree-$m_j$ generator of $H^*(\mathbb{S}^{m_j}; \mathbb{Q})$.  
		Suppose \markbox{m3032}{$\phi: H^*_{\mathbf{m}, \mathbb{C}G} \to H^*_{\mathbf{m}, \mathbb{C}G}$} \markbox{3033}{is} \markbox{3034}{a} \markbox{3035}{graded} \markbox{3036}{endomorphism} \markbox{3037}{satisfying} \markbox{m3038}{$\phi(c_1)=\lambda c_1,\, \lambda \neq 0$}. \markbox{3039}{Then} $$\phi(c_i) = \begin{cases}
			\lambda^i c_i,   \forall i \in I& \text{ if } k<n-k,\\
			\lambda^i c_i,  \forall i \in I \quad \text{ or } \quad(-\lambda)^i (c^{-1})_i,    \forall i \in I & \text{ if } k= n-k,
		\end{cases}$$
		where \markbox{m3040}{$ (c^{-1})_i $} \markbox{3041}{is} \markbox{3042}{the} \markbox{m3043}{$ 2i $}-dimensional \markbox{3044}{part} \markbox{3045}{of} \markbox{3046}{the} \markbox{3047}{inverse} \markbox{3048}{of} \markbox{m3049}{$ c = 1 + c_1 + \cdots + c_k $} \markbox{3050}{in} \markbox{m3051}{$ H^*_{ \mathbb CG} $}. 
	\end{corollary}
	\begin{proof}
		The \markbox{3052}{proof} \markbox{3053}{of} \markbox{3054}{this} \markbox{3055}{corollary} \markbox{3056}{is} \markbox{3057}{similar} \markbox{3058}{to} \markbox{3059}{the} \markbox{3060}{proof} \markbox{3061}{of} \markbox{3062}{part} \textit{2} \markbox{3063}{of} \thmref{main thm}. \markbox{3064}{Apply} \markbox{3065}{induction} \markbox{3066}{on} \markbox{m3067}{$r$} \markbox{3068}{and} \markbox{3069}{replace} \markbox{m3070}{$\mathbb{C}G_{n,k}$} \markbox{3071}{with} \markbox{m3072}{$\hat{X} := \mathbb{S}^{2m_1}\times\cdots\times \mathbb{S}^{2m_{i-1}}\times \mathbb{S}^{2m_{i+1}}\times\cdots\times \mathbb{S}^{2m_r}\times \mathbb{C}G_{n,k},$} \markbox{3073}{and} \markbox{3074}{the} \markbox{3075}{sphere} \markbox{m3076}{$\mathbb{S}^m$} \markbox{3077}{with} \markbox{m3078}{$\mathbb{S}^{2m_i}$} \markbox{3079}{in} \thmref{main thm}. \markbox{3080}{Since} \begin{equation}\label{Sm as hom}
			\mathbb S^{2m_j}=SO(2m_j+1)/SO(2m_j)
		\end{equation}
		where \markbox{3081}{the} \markbox{3082}{orthogonal} \markbox{3083}{groups} \markbox{m3084}{$SO(2m_j+1)$} \markbox{3085}{and} \markbox{m3086}{$SO(2m_j)$} \markbox{3087}{have} \markbox{3088}{the} \markbox{3089}{same} \markbox{3090}{rank} \markbox{m3091}{$m_j$}. \markbox{3092}{Using} \eqref{Sm as hom} \markbox{3093}{and} \eqref{cgn as hom}, \markbox{m3094}{$\hat{X}$} \markbox{3095}{satisfies} \markbox{3096}{the} \markbox{3097}{hypothesis} \markbox{3098}{of} \thmref{Tezuka}. Therefore, \markbox{3099}{every} \markbox{m3100}{$\mathbb{Q}$}-linear \markbox{3101}{derivation} \markbox{3102}{of} \markbox{m3103}{$H^*(\hat{X};\mathbb{Q})$} \markbox{3104}{that} \markbox{3105}{decreases} \markbox{3106}{the} \markbox{3107}{degree} \markbox{3108}{by} \markbox{m3109}{$2m_i$} \markbox{3110}{is} trivial.
		%Since the $m_i$ are even, the    and $\mathbb CG_{n,k}=U(n)/U(k)\times U(n-k)$, the space $X$ can be realized as a homogeneous space $G/H$, where $G$ is a connected compact Lie group and $H$ a closed subgroup of maximal rank.  This satisfies the hypotheses of Shiga--Tezuka's Theorem~$A'$ \cite{sigha-tezuka}, which then implies that  The argument then proceeds as in Theorem~\ref{main thm}, which completes the proof.  
	\end{proof}
	Let \markbox{3111}{us} \markbox{3112}{turn} \markbox{3113}{our} \markbox{3114}{attention} \markbox{3115}{to} \markbox{3116}{the} \markbox{3117}{generalized} \markbox{3118}{Dold} \markbox{3119}{spaces} \markbox{m3120}{$P(m,n,k)$} \markbox{3121}{defined} \markbox{3122}{in} \subsecref{gds}. \markbox{3123}{The} \markbox{3124}{following} \markbox{3125}{remark} \markbox{3126}{helps} \markbox{3127}{us} \markbox{3128}{to} \markbox{3129}{describe} \markbox{3130}{endomorphisms} \markbox{3131}{of} \markbox{m3132}{$H^*(P(m,n,k);\mathbb{Q})$} \markbox{3133}{induced} \markbox{3134}{by} \markbox{3135}{continuous} \markbox{3136}{functions} \markbox{3137}{on} \markbox{m3138}{$P(m,n,k)$}. \markbox{3139}{These} \markbox{3140}{observations} \markbox{3141}{will} \markbox{3142}{be} \markbox{3143}{used} \markbox{3144}{in} \secref{section 4}.
	%to understand how continuous self-maps of $P(m,n,k)$ induce homomorphisms on the cohomology ring $H^*(P(m,n,k);\mathbb{Q})$. 
	%Let us record a few useful observations.
	\begin{remark}\label{lift}
		For \markbox{3145}{a} \markbox{3146}{continuous} \markbox{3147}{map} \markbox{m3148}{$f$} \markbox{3149}{on} \markbox{m3150}{$P(m,n,k)$}, \markbox{3151}{we} \markbox{3152}{have}  
		\begin{equation} \label{lift of f}
			f_*\circ \pi_*\big(\pi_1(\mathbb{S}^m\times \mathbb{C}G_{n,k})\big)
			\subseteq \pi_*\big(\pi_1(\mathbb{S}^m\times \mathbb{C}G_{n,k})\big),
		\end{equation}
		where \markbox{m3153}{$\pi_1(X)$} \markbox{3154}{denotes} \markbox{3155}{the} \markbox{3156}{fundamental} \markbox{3157}{group} \markbox{3158}{of} \markbox{3159}{a} \markbox{3160}{topological} \markbox{3161}{space} \markbox{m3162}{$X$}. Hence, \markbox{3163}{the} \markbox{3164}{composite} \markbox{m3165}{$f\circ \pi$} \markbox{3166}{admits} \markbox{3167}{a} \markbox{3168}{lift} \markbox{m3169}{$\tilde f$} on
		\markbox{m3170}{$\mathbb{S}^m\times \mathbb{C}G_{n,k}$} \markbox{3171}{for} \markbox{3172}{the} \markbox{3173}{double} \markbox{3174}{covering}  
		\markbox{m3175}{$\pi:\mathbb{S}^m\times \mathbb{C}G_{n,k}\to P(m,n,k)$}.
	\end{remark}
	Using \remref{lift}, \markbox{3176}{we} \markbox{3177}{get} \markbox{3178}{the} \markbox{3179}{following} \markbox{3180}{commutative} diagram,
	
	\begin{equation}\label{comm diag on H}
		\begin{tikzcd}
			{H^*(P(m,n,k);\mathbb Q)}  & {H^*(\mathbb S^m\times \mathbb CG_{n,k};\mathbb Q)} \\
			{H^*(P(m,n,k);\mathbb Q)} & {H^*(\mathbb S^m\times \mathbb CG_{n,k};\mathbb Q).}
			\arrow["{\pi^*}", from=1-1, to=1-2]
			\arrow["{f^*}"', from=1-1, to=2-1]
			%\arrow[hook, from=1-2, to=1-3]
			\arrow["{\bar f^*}", from=1-2, to=2-2]
			%\arrow["{\bar f^*}", from=1-3, to=2-3]
			\arrow["{\pi^*}", from=2-1, to=2-2]
			%\arrow[hook, from=2-2, to=2-3]
		\end{tikzcd}
	\end{equation}
	where \markbox{m3181}{$\pi^*$} \markbox{3182}{is} \markbox{3183}{an} \markbox{3184}{injective} map. \markbox{3185}{Using} \thmref{cohomology of P(m,n,k)} \markbox{3186}{and} \eqref{comm diag on H} \markbox{3187}{we} \markbox{3188}{obtain} \markbox{3189}{the} \markbox{3190}{following} \markbox{3191}{two} corollaries. 
	%as an immediate application of respectively. 	
	\begin{corollary}\label{cor3}
		Let \markbox{m3192}{$f^*$} \markbox{3193}{be} \markbox{3194}{an} \markbox{3195}{endomorphism} \markbox{3196}{of} \markbox{m3197}{$H^*(P(m,n,k); \mathbb{Q})$} \markbox{3198}{induced} \markbox{3199}{by} \markbox{3200}{a} \markbox{3201}{continuous} \markbox{3202}{function} \markbox{m3203}{$f$} \markbox{3204}{on} \markbox{m3205}{$P(m,n,k)$} \markbox{3206}{satisfying} \markbox{m3207}{$f^*(c_1^2) \ne 0$}. Then
		\markbox{m3208}{$f^*$} \markbox{3209}{is} \markbox{3210}{the} \markbox{3211}{restriction} \markbox{3212}{of} \markbox{3213}{a} \markbox{3214}{graded} \markbox{3215}{endomorphism} \markbox{m3216}{$\tilde{f}^*$} \markbox{3217}{on}  \markbox{m3218}{$H^*_\times$} \markbox{3219}{satisfying} \markbox{m3220}{$\tilde{f}^*(c_1) = \lambda c_1, \lambda \neq 0$},  \markbox{3221}{to} \markbox{3222}{the} \markbox{3223}{fixed} \markbox{3224}{subring} \markbox{m3225}{$\mathrm{Fix}(\theta^*)$} \markbox{3226}{of} \markbox{m3227}{$H^*_\times$} \markbox{3228}{where} \markbox{m3229}{$\theta = \alpha\times \sigma$}.
	\end{corollary}
	\begin{corollary}\label{cor4}
		Let \markbox{m3230}{$f^*$} \markbox{3231}{be} \markbox{3232}{an} \markbox{3233}{endomorphism} \markbox{3234}{of} \markbox{m3235}{$H^*(P(m,n,k); \mathbb{Q})$} \markbox{3236}{induced} \markbox{3237}{by} \markbox{3238}{a} \markbox{3239}{continuous} \markbox{3240}{function} \markbox{m3241}{$f$} \markbox{3242}{on} \markbox{m3243}{$P(m,n,k)$} \markbox{3244}{satisfying} \markbox{m3245}{$f^*(c_1^2) =0$} \markbox{3246}{and} \markbox{m3247}{$n>2$}. Then
		\markbox{m3248}{$f^*$} \markbox{3249}{is} \markbox{3250}{the} \markbox{3251}{restriction} \markbox{3252}{of} \markbox{3253}{a} \markbox{3254}{graded} \markbox{3255}{endomorphism} \markbox{m3256}{$\tilde{f}^*$} \markbox{3257}{on}  \markbox{m3258}{$H^*_\times$} \markbox{3259}{satisfying} \markbox{m3260}{$\tilde{f}^*(c_1) = au, a \in \mathbb{Q}$},  \markbox{3261}{to} \markbox{3262}{the} \markbox{3263}{fixed} \markbox{3264}{subring} \markbox{m3265}{$\mathrm{Fix}(\theta^*)$} \markbox{3266}{of} \markbox{m3267}{$H^*_\times$} \markbox{3268}{where} \markbox{m3269}{$\theta = \alpha\times \sigma$}.
	\end{corollary}
	Using \thmref{main thm} \markbox{3270}{in} \corref{cor3}, \markbox{3271}{and} \propref{main thm 2} \markbox{3272}{in} \corref{cor4} \markbox{3273}{along} \markbox{3274}{with} \markbox{3275}{hypothesis} \eqref{Homer}, \markbox{3276}{we} \markbox{3277}{can} \markbox{3278}{determine} \markbox{m3279}{$f^*$}.
    %if we add the assumption that the hypothesis \eqref{Homer} is satisfied in \corref{cor4}.
	
	Moreover, \markbox{3280}{there} \markbox{3281}{exist} \markbox{3282}{graded} \markbox{3283}{endomorphisms} \markbox{3284}{of} \markbox{m3285}{$H^*(P(m,n,k))$} \markbox{3286}{that} \markbox{3287}{are} \markbox{3288}{not} \markbox{3289}{induced} \markbox{3290}{by} \markbox{3291}{any} \markbox{3292}{continuous} self-map \markbox{3293}{of} \markbox{m3294}{$P(m,n,k)$}, \markbox{3295}{and} \markbox{3296}{cannot} \markbox{3297}{be} \markbox{3298}{realized} \markbox{3299}{as} \markbox{3300}{restrictions} \markbox{3301}{of} \markbox{3302}{graded} \markbox{3303}{endomorphisms} \markbox{3304}{of} \markbox{m3305}{$H^*_{\times}$}. \markbox{3306}{Let} \markbox{3307}{us} \markbox{3308}{see} \markbox{3309}{an} \markbox{3310}{example} \markbox{3311}{of} \markbox{3312}{such} \markbox{3313}{graded} endomorphism.
	
	
	%Also, there exist graded endomorphisms of $H^*(P(m,n,k); \mathbb{Q})$ which are not restriction of any graded endomorphism of $H^*_\times$, if the graded endomorphism is not induced from a continuous function on $P(m,n,k)$. Let us see an example of such graded endomorphism of $H^*(P(m,n,k);\mathbb{Q})$.
	\begin{example}
		If \markbox{m3314}{$m$} odd, \markbox{m3315}{$n>2$} \markbox{3316}{and} \markbox{m3317}{$k = 1$}, \markbox{3318}{then} \markbox{m3319}{$P(m,n,1)$} \markbox{3320}{is} \markbox{3321}{fibered} \markbox{3322}{by} \markbox{3323}{the} \markbox{3324}{complex} \markbox{3325}{projective} \markbox{3326}{space}  \markbox{m3327}{$\mathbb{C} P^{n-1}$} \markbox{3328}{over} \markbox{3329}{the} \markbox{3330}{real} \markbox{3331}{projective} \markbox{3332}{space} \markbox{m3333}{$\mathbb{R} P^m$}. \markbox{3334}{In} \markbox{3335}{this} case,  \markbox{m3336}{$H^*_\times\cong \mathbb Q[u,c_1]/\langle u^2, c_1^n\rangle$} \markbox{3337}{and} \markbox{3338}{using} \eqref{defn of theta*}\ and \thmref{cohomology of P(m,n,k)}, \markbox{3339}{the} \markbox{3340}{rational} \markbox{3341}{cohomology} \markbox{3342}{ring} $$H^*(P(m,n,1); \mathbb{Q}) \cong \mathbb{Q}[u, b] / \langle u^2, b^{\lfloor (n+1)/2 \rfloor} \rangle,$$ \markbox{3343}{where} \markbox{m3344}{$u$} \markbox{3345}{is} \markbox{3346}{a} \markbox{3347}{generator} \markbox{3348}{of} \markbox{m3349}{$H^m(\mathbb{R}P^m;\mathbb Q)$} \markbox{3350}{and} \markbox{m3351}{$b$} \markbox{3352}{restricts} \markbox{3353}{to} \markbox{m3354}{$c_1^2\in H^2(\mathbb CP^{n-1};\mathbb Q)$} \markbox{3355}{under} \markbox{3356}{the} \markbox{3357}{fiber} inclusion. 
		%Here, $\lfloor x\rfloor$ denotes the greatest integer less than or equal to $x$.
		
		Consider \markbox{3358}{the} endomorphism
		\[
		\phi \colon H^*(P(m,n,1); \mathbb{Q}) \to H^*(P(m,n,1); \mathbb{Q}), \quad \text{defined by }\quad u \mapsto u,  b \mapsto -b.
		\]
		Then \markbox{m3359}{$\phi$} \markbox{3360}{is} \markbox{3361}{a} well-defined \markbox{3362}{graded} \markbox{3363}{endomorphism} \markbox{3364}{but} \markbox{3365}{it} \markbox{3366}{cannot} \markbox{3367}{be} \markbox{3368}{a} \markbox{3369}{restriction} \markbox{3370}{of} \markbox{3371}{a} \markbox{3372}{graded} \markbox{3373}{endomorphism} \markbox{3374}{of} \markbox{m3375}{$H^*_\times$} \markbox{3376}{because} \markbox{3377}{any} \markbox{3378}{such} \markbox{3379}{map} \markbox{3380}{induces} \markbox{m3381}{$c_1^2 \mapsto \lambda^2 c_1^2$} \markbox{3382}{for} \markbox{3383}{some} \markbox{m3384}{$\lambda \in \mathbb{Q}$}, \markbox{3385}{and} \markbox{m3386}{$\lambda^2 \neq -1$}.
	\end{example}
	
	
	\iffalse
	\begin{corollary}
		Assume \markbox{3387}{that} \markbox{m3388}{$P(m,n,k)$} \markbox{3389}{be} \markbox{3390}{an} \markbox{3391}{orientable} \markbox{3392}{manifold} \markbox{3393}{and} \markbox{m3394}{$n>2$}. \markbox{3395}{Let} \markbox{m3396}{$f$} \markbox{3397}{be} \markbox{3398}{a} \markbox{3399}{continuous} \markbox{3400}{map} \markbox{3401}{on} \markbox{m3402}{$P(m,n,k)$} \markbox{3403}{with} \markbox{3404}{nonzero} \markbox{3405}{Brouwer} degree. \markbox{3406}{Then}  
		the \markbox{3407}{induced} \markbox{3408}{map} \markbox{m3409}{$f^*$} \markbox{3410}{is} \markbox{3411}{an} \markbox{3412}{automorphism} \markbox{3413}{on} \markbox{m3414}{$H^*(P(m,n,k);\mathbb Q)$}.
	\end{corollary}
	\begin{proof}
		Using \remref{lift}, \markbox{3415}{there} \markbox{3416}{exists} \markbox{3417}{a} \markbox{3418}{lift} \markbox{m3419}{$\tilde f: \mathbb{S}^m\times \mathbb{C}G_{n,k}\to \mathbb{S}^m\times \mathbb{C}G_{n,k}$} 
		satisfying \markbox{m3420}{$\pi\circ \tilde f = f\circ \pi$}. \markbox{3421}{Using} \corref{cor3} \markbox{3422}{and} \corref{cor4}, \markbox{3423}{we} \markbox{3424}{have} \markbox{m3425}{$f^*$} \markbox{3426}{is} \markbox{3427}{the} \markbox{3428}{restriction} \markbox{3429}{of} \markbox{3430}{a} \markbox{3431}{graded} \markbox{3432}{endomorphism} \markbox{m3433}{$\tilde{f}^*$} \markbox{3434}{on}  \markbox{m3435}{$H^*_\times$} \markbox{3436}{satisfying} \markbox{3437}{either} $$\tilde{f}^*(c_1) = \lambda c_1, \lambda \neq 0, \text{ if } f^*(c_1^2) \neq 0, \text{ or } \tilde f^*(c_1) =au, a\in \mathbb{Q}, \text{ if } f^*(c_1^2) =0,$$  \markbox{3438}{to} \markbox{3439}{the} \markbox{3440}{fixed} \markbox{3441}{subring} \markbox{m3442}{$\mathrm{Fix}(\theta^*)$} \markbox{3443}{of} \markbox{m3444}{$H^*_\times$}.
		
		Let \markbox{m3445}{$d=k(n-k)$}.  \markbox{3446}{The} \markbox{3447}{top} \markbox{3448}{cohomology} \markbox{m3449}{$H^{2d+m}_\times\cong\mathbb{Q}$} \markbox{3450}{is} \markbox{3451}{generated} \markbox{3452}{by} 
		\markbox{m3453}{$uc_1^d$}.  \markbox{3454}{The} \markbox{3455}{nonzero} \markbox{3456}{Brouwer} \markbox{3457}{degree} \markbox{3458}{of} \markbox{m3459}{$f$} \markbox{3460}{implies} \markbox{3461}{nonzero} \markbox{3462}{Brouwer} \markbox{3463}{degree} \markbox{3464}{of} \markbox{m3465}{$\tilde f$} i.e. \begin{equation}\label{brouwer}
			\tilde f^*(uc_1^d)=\nu\,uc_1^d \in uH^{*}_{\mathbb{C}G}, \, \nu \neq 0.
		\end{equation}
		
		Let \markbox{3466}{us} \markbox{3467}{consider} \markbox{3468}{the} \markbox{3469}{first} \markbox{3470}{case} \markbox{3471}{where} \markbox{m3472}{$f^*(c_1^2) \neq 0$} \markbox{3473}{then} \markbox{m3474}{$\tilde f^*(c_1) = \lambda c_1, \lambda \neq 0$}. \markbox{3475}{Using} \thmref{main thm}, \markbox{3476}{we} \markbox{3477}{have} \markbox{m3478}{$\tilde f^*(c_i) = \lambda^i c_i$}. Also, $$\tilde f^*(u) = \mu u, \mu \in \mathbb{Q}, \text{ or } \tilde f^*(u)\in H^*_{\mathbb{C}G}.$$ \markbox{3479}{If} \markbox{m3480}{$\tilde f^*(u)\in H^*_{\mathbb{C}G}$}, \markbox{3481}{then} \markbox{m3482}{$\tilde f^*(uc_1^d) \in H^*_{\mathbb{C}G}$}
		which \markbox{3483}{is} \markbox{3484}{a} \markbox{3485}{contradiction} \markbox{3486}{to} \eqref{brouwer}. Therefore, \markbox{m3487}{$\tilde f^*(u) = \mu u $} \markbox{3488}{where} \markbox{m3489}{$\mu \neq 0$} \markbox{3490}{because} \markbox{m3491}{$\nu \neq 0$}. So, \markbox{m3492}{$\tilde f^*$} \markbox{3493}{is} \markbox{3494}{an} automorphism.
		
		Let \markbox{3495}{us} \markbox{3496}{consider} \markbox{3497}{the} \markbox{3498}{other} \markbox{3499}{case} \markbox{3500}{when} \markbox{m3501}{$f^*(c_1^2) = 0$} \markbox{3502}{then} \markbox{m3503}{$\tilde f^*(c_1) = au, a \in \mathbb{Q}$}. Since, \markbox{m3504}{$u^2 =0$} \markbox{3505}{and} \markbox{m3506}{$d\geq 2$} \markbox{3507}{we} \markbox{3508}{have} $$\tilde f^*(uc_1^d) = \tilde f^*(u) (\tilde f^*(c_1))^d = \tilde f^*(u) a^d u^d =0$$ \markbox{3509}{which} \markbox{3510}{is} \markbox{3511}{a} \markbox{3512}{contradiction} \markbox{3513}{to} \eqref{brouwer}. Hence, \markbox{3514}{this} \markbox{3515}{case} \markbox{3516}{would} \markbox{3517}{not} arise. 
	\end{proof}
	\fi
	
	The \markbox{3518}{following} \markbox{3519}{corollary} \markbox{3520}{helps} \markbox{3521}{us} \markbox{3522}{to} \markbox{3523}{understand} \markbox{3524}{the} \markbox{3525}{relationship} \markbox{3526}{between} \markbox{3527}{the} \markbox{3528}{automorphisms} \markbox{3529}{of} \markbox{m3530}{$H^*(P(m,n,k))$} \markbox{3531}{with} \markbox{3532}{the} \markbox{3533}{automorphisms} \markbox{3534}{of} \markbox{m3535}{$H^*_{\times}$}.
	\begin{corollary}\label{automor}
		Let \markbox{m3536}{$f^*$} \markbox{3537}{be} \markbox{3538}{an} \markbox{3539}{automorphism} \markbox{3540}{of} \markbox{m3541}{$H^*(P(m,n,k);\mathbb{Q})$} \markbox{3542}{induced} \markbox{3543}{by} \markbox{3544}{a} \markbox{3545}{continuous} \markbox{3546}{function} \markbox{m3547}{$f$} \markbox{3548}{on} \markbox{m3549}{$P(m,n,k)$} \markbox{3550}{and} \markbox{3551}{assume} \markbox{3552}{that} \markbox{m3553}{$n> 2$}. \markbox{3554}{Then} \markbox{m3555}{$\tilde{f}^*$} \markbox{3556}{is} \markbox{3557}{an} \markbox{3558}{automorphism} \markbox{3559}{of} \markbox{m3560}{$H^*_{\times}$}, \markbox{3561}{where} \markbox{m3562}{$\tilde{f}$} \markbox{3563}{is} \markbox{3564}{as} \markbox{3565}{in} \remref{lift}. \\ \markbox{3566}{Moreover} \markbox{3567}{there} \markbox{3568}{exist} \markbox{m3569}{$\lambda, \mu \in \mathbb{Q}\backslash \{0\}$} \markbox{3570}{such} \markbox{3571}{that} \markbox{m3572}{$\tilde{f}^*(u) = \mu u$} \markbox{3573}{and} \markbox{m3574}{$\tilde{f}^*(c_i)$} \markbox{3575}{is} \markbox{3576}{of} \markbox{3577}{the} \markbox{3578}{form} \markbox{3579}{given} \markbox{3580}{in} \textit{(2)} \markbox{3581}{of} \thmref{main thm}.
	\end{corollary}
	\begin{proof}
		Using \remref{lift}, \markbox{3582}{we} \markbox{3583}{have} \markbox{m3584}{$\tilde{f}^*$} \markbox{3585}{is} \markbox{3586}{a} \markbox{3587}{graded} \markbox{3588}{endomorphism} \markbox{3589}{of} \markbox{m3590}{$H^*_{\times}$}. \markbox{3591}{When} \markbox{m3592}{$n>2$}, \markbox{3593}{we} \markbox{3594}{have} \markbox{m3595}{$c_1^2\neq 0$} \markbox{3596}{in} \markbox{m3597}{$\fix (\theta^*)\subseteq H^*_\times$}, \markbox{3598}{where} Fix\markbox{m3599}{$(\theta^*)$} \markbox{3600}{is} \markbox{3601}{the} \markbox{3602}{fixed} \markbox{3603}{subring} \markbox{3604}{under} \markbox{m3605}{$\theta^*$} \markbox{3606}{defined} \markbox{3607}{in} \eqref{defn of theta*}. \markbox{3608}{Since} \markbox{m3609}{$f^*$} \markbox{3610}{is} \markbox{3611}{an} automorphism, \markbox{3612}{we} \markbox{3613}{have} \markbox{m3614}{$f^*(c_1^2) \neq 0$}. \markbox{3615}{Using} \corref{cor3}, \markbox{3616}{there} \markbox{3617}{exist} \markbox{m3618}{$\lambda \in \mathbb{Q}$} \markbox{3619}{such} \markbox{3620}{that} \markbox{m3621}{$\tilde{f}^*(c_1) = \lambda c_1, \lambda \neq 0$}.\\
		By \thmref{main thm}, \markbox{m3622}{$\tilde{f}^*(c_i)$} \markbox{3623}{is} \markbox{3624}{of} \markbox{3625}{the} \markbox{3626}{form} \markbox{3627}{given} \markbox{3628}{in} \textit{(2)} \markbox{3629}{of} \thmref{main thm}. Also, $$\tilde{f}^*(u) = \mu u, \, \mu \in \mathbb{Q} \quad \text{ or } \quad \tilde{f}^*(u) = Q$$ \markbox{3630}{where} \markbox{m3631}{$Q$} \markbox{3632}{is} \markbox{3633}{a} \markbox{3634}{polynomial} \markbox{3635}{of} \markbox{3636}{degree} \markbox{m3637}{$m$} \markbox{3638}{in} \markbox{m3639}{$H^*_{\mathbb{C}G}$} \markbox{3640}{with} \markbox{m3641}{$Q^2 =0$}. \markbox{3642}{To} \markbox{3643}{conclude} \markbox{3644}{the} result, \markbox{3645}{we} \markbox{3646}{need} \markbox{3647}{to} \markbox{3648}{prove} \markbox{3649}{that} \markbox{m3650}{$\tilde{f}^*(u) = \mu u$} \markbox{3651}{where} \markbox{m3652}{$\mu \neq 0$}. \\
		Suppose \markbox{3653}{that} \markbox{m3654}{$\tilde{f}^*(u) = Q$}, \markbox{3655}{then} \markbox{3656}{the} \markbox{3657}{image} \markbox{3658}{set} \markbox{m3659}{$\im \tilde{f}^*\subseteq H^*_{\mathbb{C}G}$}. \markbox{3660}{Using} \corref{cor3}, \markbox{3661}{we} \markbox{3662}{get} $$\im f^*  \cong \im \tilde{f^*}|_{Fix (\theta^*)}\subseteq H^*_{\mathbb{C}G}.$$ \markbox{3663}{This} \markbox{3664}{is} \markbox{3665}{a} \markbox{3666}{contradiction} \markbox{3667}{to} \markbox{3668}{the} \markbox{3669}{assumption} \markbox{3670}{that} \markbox{m3671}{$f^*$} \markbox{3672}{is} \markbox{3673}{an} \markbox{3674}{automorphism} \markbox{3675}{because} \markbox{3676}{using} \thmref{cohomology of P(m,n,k)}, \markbox{3677}{either} \markbox{m3678}{$u$} \markbox{3679}{or} \markbox{m3680}{$uc_{1}$} (depending \markbox{3681}{on} \markbox{3682}{the} \markbox{3683}{parity} \markbox{3684}{of} \markbox{m3685}{$m$}) \markbox{3686}{is} \markbox{3687}{in} \markbox{m3688}{$\im f^*=\fix (\theta^*) \cong H^*(P(m,n,k);\mathbb{Q})$}. Therefore, \markbox{m3689}{$\tilde{f}^*(u) = \mu u, \mu \in \mathbb{Q}$} \markbox{3690}{and} \markbox{m3691}{$\mu \neq 0$} \markbox{3692}{because} \markbox{m3693}{$f^*$} \markbox{3694}{is} \markbox{3695}{an} automorphism.
	\end{proof}
	\subsection{} \markbox{3696}{The} \markbox{3697}{following} \markbox{3698}{theorem} \markbox{3699}{provides} \markbox{3700}{a} \markbox{3701}{criterion} \markbox{3702}{for} \markbox{3703}{the} \markbox{3704}{image} \markbox{3705}{of} \markbox{3706}{the} \markbox{3707}{spherical} \markbox{3708}{cohomology} \markbox{3709}{class} \markbox{3710}{mapped} \markbox{3711}{to} \markbox{3712}{a} \markbox{3713}{scaler} \markbox{3714}{multiple} \markbox{3715}{of} \markbox{3716}{itself} \markbox{3717}{under} \markbox{3718}{the} \markbox{3719}{graded} \markbox{3720}{endomorphism} on
	\markbox{m3721}{$H^*(\mathbb{S}^m \times \mathbb{C}G_{n,k};\mathbb Z)$} \markbox{3722}{induced} \markbox{3723}{from} \markbox{3724}{a} \markbox{3725}{continuous} map.
	
	
	
	\begin{theorem}\label{ind from top}
		Let \markbox{m3726}{$f$} \markbox{3727}{be} \markbox{3728}{a} \markbox{3729}{continuous} \markbox{3730}{map} \markbox{3731}{on} \markbox{m3732}{$\mathbb S^m\times \mathbb CG_{n,k}$} \markbox{3733}{such} \markbox{3734}{that} \markbox{3735}{it} \markbox{3736}{stabilizes} \markbox{3737}{a} \markbox{3738}{copy} \markbox{3739}{of} \markbox{3740}{Grassmannian} \markbox{m3741}{$\{{x_0}\}\times \mathbb CG_{n,k}$} \markbox{3742}{for} \markbox{3743}{some} \markbox{m3744}{$x_0\in \mathbb S^m.$} \markbox{3745}{Then} \markbox{3746}{the} \markbox{3747}{induced} \markbox{3748}{endomorphism} \markbox{3749}{in} \markbox{3750}{cohomology} \markbox{3751}{satisfies} \markbox{m3752}{$f^*(u) = \mu u$} \markbox{3753}{for} \markbox{3754}{some} \markbox{m3755}{$\mu \in \mathbb Z$}.
	\end{theorem}
	
	
	%\begin{theorem}\label{ind from top}
	%Let $x_0 \in \mathbb{S}^m$ and consider a graded endomorphism $f^*$ on $H^*_{\times}$  induced from a topological map $f$ on $\mathbb S^m\times \mathbb CG_{n,k}$ such that $f (\{x_0 \}\times\mathbb CG_{n,k})\subseteq \{x_0\}\times \mathbb{C}G_{n,k}$.  Then $f^*(u)=\mu u$ for some $\mu\in \mathbb Q.$
	%\end{theorem}
	
	
	\begin{proof}
		Let \markbox{m3756}{$\mathbb T^m$} \markbox{3757}{be} \markbox{3758}{the} \markbox{3759}{torus} \markbox{m3760}{$(\mathbb S^1)^m$} and
		\markbox{m3761}{$q:\mathbb T^m\to \mathbb S^m$} \markbox{3762}{be} \markbox{3763}{the} \markbox{3764}{quotient} \markbox{3765}{map} \markbox{3766}{that} \markbox{3767}{collapses} \markbox{3768}{the} \markbox{3769}{complement} \markbox{m3770}{$C$} \markbox{3771}{of} \markbox{3772}{an} \markbox{3773}{open} \markbox{3774}{disk} \markbox{m3775}{$D\subset \mathbb T^m$} \markbox{3776}{to} \markbox{3777}{the} \markbox{3778}{point} \markbox{m3779}{$x_0$} \markbox{3780}{in} \markbox{m3781}{$\mathbb{S}^m$}.  \markbox{3782}{Denote} \markbox{m3783}{$p_i$} \markbox{3784}{the} \markbox{m3785}{$i$}-th \markbox{3786}{projection} \markbox{3787}{map} \markbox{3788}{on} \markbox{m3789}{$\mathbb S^m\times \mathbb CG_{n,k}$} \markbox{3790}{for} \markbox{m3791}{$i=1,2$} \markbox{3792}{and} \markbox{m3793}{$s:\mathbb S^m\setminus\{x_0\}\to D$} \markbox{3794}{is} \markbox{3795}{the} \markbox{3796}{inverse} \markbox{3797}{of} \markbox{3798}{the} \markbox{3799}{restriction} \markbox{m3800}{$q|_D$}.
		Since \markbox{m3801}{$f$} \markbox{3802}{stabilizes} \markbox{m3803}{$\{x_0\}\times \mathbb{C}G_{n,k}$}, \markbox{3804}{define} \markbox{3805}{continuous} maps
		\markbox{m3806}{$g:\mathbb{C}G_{n,k}\to \mathbb{C}G_{n,k}$} \markbox{3807}{by} \markbox{m3808}{$(x_0,g(y)) = f(x_0,y)$} \markbox{3809}{and} \markbox{m3810}{$\tilde f:\mathbb T^m\times \mathbb{C}G_{n,k}\to \mathbb T^m\times \mathbb{C}G_{n,k}$} by
		\[
		\tilde f(x,y)=
		\begin{cases}
			\big(s\circ p_1\circ f(q(x),y),\,\,p_2\circ f(q(x),y)\big), & x\in D,\\[2pt]
			\big(x,g(y)\big), & x\in C.
		\end{cases}
		\]
		%where and $p_i$ denotes the $i$-th projection.  
		Then \markbox{3811}{it} \markbox{3812}{is} \markbox{3813}{easy} \markbox{3814}{to} \markbox{3815}{check} \markbox{3816}{that} \markbox{3817}{the} \markbox{3818}{following} \markbox{3819}{diagram} commutes:
		\[ \begin{tikzcd}
			{\mathbb T^m\times \mathbb{C}G_{n,k}} & {\mathbb T^m\times \mathbb{C}G_{n,k}} \\
			{\mathbb S^m\times \mathbb{C}G_{n,k}} & {\mathbb S^m\times \mathbb{C}G_{n,k}}
			\arrow["{\tilde f}", from=1-1, to=1-2]
			\arrow["{q\times \mathrm{id}}"', two heads, from=1-1, to=2-1]
			\arrow["{q\times \mathrm{id}}", two heads, from=1-2, to=2-2]
			\arrow["f", from=2-1, to=2-2]
		\end{tikzcd}
		\]
		
		Since, \markbox{3820}{the} \markbox{3821}{quotient} \markbox{3822}{map} \markbox{m3823}{$q$} \markbox{3824}{has} \markbox{3825}{Brouwer} \markbox{3826}{degree} 1, \markbox{3827}{the} \markbox{3828}{induced} \markbox{3829}{map} \markbox{3830}{on} \markbox{3831}{rational} \markbox{3832}{cohomology} \markbox{m3833}{$q^*: H^*(\mathbb{S}^m; \mathbb{Z}) \rightarrow H^*(\mathbb{T}^m;\mathbb{Z})$} \markbox{3834}{sends} \markbox{m3835}{$u\mapsto 1\cdot u_1 u_2 \dots u_m$} \markbox{3836}{where} \markbox{m3837}{$u_i$} \markbox{3838}{denote} \markbox{3839}{the} \markbox{3840}{one} \markbox{3841}{dimensional} \markbox{3842}{cohomology} \markbox{3843}{class} \markbox{3844}{corresponding} \markbox{3845}{to} \markbox{3846}{the} \markbox{3847}{fundamental} \markbox{3848}{class} \markbox{3849}{of} \markbox{3850}{the} \markbox{m3851}{$i$}-th \markbox{3852}{circle} \markbox{3853}{factor} \markbox{3854}{of} \markbox{m3855}{$\mathbb T^m$} \markbox{3856}{for} \markbox{m3857}{$i\in \{1,2,\ldots,m\}$} \markbox{3858}{with} \markbox{3859}{appropriate} orientation. 
		%Let $u$ be a generator of $H^m(\mathbb S^m;\mathbb Z)$ so that $q^*(u)=u_1u_2\cdots u_m$.  
		Since \markbox{m3860}{$H^{\mathrm{odd}}(\mathbb{C}G_{n,k};\mathbb Z)=0$}, \markbox{3861}{the} \markbox{3862}{induced} \markbox{3863}{map} \markbox{m3864}{$\tilde f^*$} \markbox{3865}{sends} \markbox{3866}{each} \markbox{m3867}{$u_i$} \markbox{3868}{to} \markbox{3869}{a} \markbox{3870}{polynomial} \markbox{m3871}{$P_i(u_1,\dots,u_m)$}.  \markbox{3872}{We} \markbox{3873}{slightly} \markbox{3874}{abuse} \markbox{3875}{notation} \markbox{3876}{by} \markbox{3877}{using} \markbox{3878}{the} \markbox{3879}{same} \markbox{3880}{symbols} \markbox{3881}{for} \markbox{3882}{the} \markbox{3883}{cohomology} \markbox{3884}{classes} \markbox{3885}{of} \markbox{m3886}{$H^*(\mathbb S^m;\mathbb Z)$} \markbox{3887}{and} \markbox{m3888}{$H^*(\mathbb{C}G_{n,k};\mathbb Z)$} \markbox{3889}{when} \markbox{3890}{viewed} \markbox{3891}{in} \markbox{m3892}{$H^*(\mathbb S^m\times \mathbb CG_{n,k};\mathbb Z)$}.
		The \markbox{3893}{induced} \markbox{3894}{diagram} \markbox{3895}{in} \markbox{3896}{cohomology} \markbox{3897}{implies} \markbox{3898}{the} \markbox{3899}{following} \markbox{3900}{commutative} diagram.
		\[
		\begin{tikzcd}
			{\prod_{i=1}^m u_i} & {\prod_{i=1}^m P_i(u_1,\ldots,u_m)} \\
			u & {f^*(u)}
			\arrow["{\tilde f^*}", maps to, from=1-1, to=1-2]
			\arrow["{(q\times \mathrm{id})^*}"', maps to, from=2-1, to=1-1]
			\arrow["{f^*}", maps to, from=2-1, to=2-2]
			\arrow["{(q\times \mathrm{id})^*}"', maps to, from=2-2, to=1-2]
		\end{tikzcd}
		\]
		This \markbox{3901}{implies} \markbox{3902}{that} \markbox{m3903}{$f^*(u)$} \markbox{3904}{does} \markbox{3905}{not} \markbox{3906}{contain} \markbox{3907}{any} \markbox{3908}{nonzero} \markbox{3909}{element} \markbox{3910}{from} \markbox{m3911}{$H^*(\mathbb{C}G_{n,k};\mathbb Z)$}.  
		Thus, \markbox{m3912}{$f^*(u)=\mu u$} \markbox{3913}{for} \markbox{3914}{some} \markbox{m3915}{$\mu\in \mathbb Z$}.
	\end{proof}
	%\begin{remark}
		%To remain consistent with notations, we have written the proof of \thmref{ind from top} over $\mathbb{Q}$, but the same proof also works over $\mathbb{Z}$.
	%\end{remark}
	
	
	
	
	
	
	
	
	
	
	
	
	\section{Coincidence theory of $P(m,n,k)$} \label{section 4}
	In \markbox{3916}{this} section, \markbox{3917}{we} \markbox{3918}{study} \markbox{3919}{the} \textit{coincidence theory} \markbox{3920}{of} \markbox{3921}{generalized} \markbox{3922}{Dold} \markbox{3923}{spaces} \markbox{m3924}{$P(m,n,k)$} \markbox{3925}{defined} \markbox{3926}{in} \subsecref{gds}. \markbox{3927}{We} \markbox{3928}{establish} \markbox{3929}{the} \markbox{3930}{necessary} \markbox{3931}{conditions} \markbox{3932}{for} \markbox{3933}{a} \markbox{3934}{generalized} \markbox{3935}{Dold} \markbox{3936}{space} \markbox{m3937}{$P(S,X)$} \markbox{3938}{defined} \markbox{3939}{in} \eqref{gen dold space} \markbox{3940}{to} \markbox{3941}{satisfy} \markbox{3942}{the} \markbox{3943}{coincidence} property. 
	%Our study builds upon previous results on the \emph{graded endomorphisms} of the rational cohomology ring $H^*_\times$ in section~\ref{graded endomorphism} and the \emph{rational cohomology ring of generalized Dold spaces} as established in \cite{mandal-sankaran2}.
	
	\subsection{} \markbox{3944}{Let} \markbox{3945}{us} \markbox{3946}{recall} \markbox{3947}{certain} \markbox{3948}{definitions} \markbox{3949}{that} \markbox{3950}{will} \markbox{3951}{be} \markbox{3952}{required} \markbox{3953}{in} \markbox{3954}{the} \markbox{3955}{rest} \markbox{3956}{of} \markbox{3957}{this} section.
	
	\begin{definition}
		Let \markbox{m3958}{$(X,g)$} \markbox{3959}{be} \markbox{3960}{a} pair, \markbox{3961}{where} \markbox{m3962}{$g$} \markbox{3963}{is} \markbox{3964}{a} \markbox{3965}{continuous} \markbox{3966}{map} \markbox{3967}{on} \markbox{3968}{a} \markbox{3969}{topological} \markbox{3970}{space} \markbox{m3971}{$X$}. \markbox{3972}{The} \markbox{3973}{pair} \markbox{m3974}{$(X,g)$} \markbox{3975}{is} \markbox{3976}{said} \markbox{3977}{to} \markbox{3978}{have} \markbox{3979}{the} \textbf{coincidence property} (in short, CP) if, \markbox{3980}{for} \markbox{3981}{every} \markbox{3982}{continuous} \markbox{3983}{map} \markbox{m3984}{$f : X \to X$}, \markbox{3985}{there} \markbox{3986}{exists} \markbox{3987}{a} \markbox{3988}{point} \markbox{m3989}{$x \in X$} \markbox{3990}{such} \markbox{3991}{that} \markbox{m3992}{$f(x) = g(x)$}.
	\end{definition}
	
	If \markbox{3993}{we} \markbox{3994}{consider} \markbox{m3995}{$g$} \markbox{3996}{to} \markbox{3997}{be} \markbox{3998}{the} \markbox{3999}{identity} \markbox{4000}{map} \markbox{4001}{on} \markbox{m4002}{$X$}, \markbox{4003}{then} \markbox{4004}{the} \markbox{4005}{notion} \markbox{4006}{of} \markbox{4007}{coincidence} \markbox{4008}{reduces} \markbox{4009}{to} \markbox{4010}{that} \markbox{4011}{of} \markbox{4012}{a} \markbox{4013}{fixed} point, \markbox{4014}{resulting} \markbox{4015}{in} \markbox{4016}{the} \markbox{4017}{following} definition.
	
	
	
	\begin{definition}
		A \markbox{4018}{topological} \markbox{4019}{space} \markbox{m4020}{$X$} \markbox{4021}{is} \markbox{4022}{said} \markbox{4023}{to} \markbox{4024}{have} \textbf{fixed-point property} (FPP) \markbox{4025}{if} \markbox{4026}{every} \markbox{4027}{continuous} \markbox{4028}{map} \markbox{m4029}{$f : X \to X$} \markbox{4030}{admits} \markbox{4031}{a} fixed-point; \markbox{4032}{that} is, \markbox{4033}{there} \markbox{4034}{exists} \markbox{m4035}{$x \in X$} \markbox{4036}{such} \markbox{4037}{that} \markbox{m4038}{$f(x) = x$}.
	\end{definition}
	
	
	%Our aim is to understand the situations when two continuous maps on the generalized Dold spaces $P(S,X)$ defined in \subsecref{gen dold}, have coincidence.
	 The \markbox{4039}{following} \markbox{4040}{proposition} \markbox{4041}{provides} \markbox{4042}{a} \markbox{4043}{criteria} \markbox{4044}{in} \markbox{4045}{terms} \markbox{4046}{of} \markbox{4047}{the} \markbox{4048}{fiber} \markbox{m4049}{$X$} \markbox{4050}{and} \markbox{4051}{the} \markbox{4052}{base} \markbox{4053}{space} \markbox{m4054}{$Y := S/\!\!\sim_\alpha$}, \markbox{4055}{allowing} \markbox{4056}{one} \markbox{4057}{to} \markbox{4058}{infer} \markbox{4059}{the} \markbox{4060}{coincidence} \markbox{4061}{properties} \markbox{4062}{of} \markbox{4063}{the} \markbox{4064}{total} \markbox{4065}{space} \markbox{m4066}{$P(S,X)$}.
	
	%As a first step, we establish the following proposition, which provides necessary conditions for a generalized Dold space to exhibit certain coincidence properties. 
	
	
	
	
	\iffalse
	\begin{proposition}
		\  \markbox{4067}{A} \markbox{4068}{generalized} \markbox{4069}{Dold} \markbox{4070}{space} \markbox{m4071}{$P(S,\alpha,X,\sigma)$} \markbox{4072}{does} \markbox{4073}{not} \markbox{4074}{have} \markbox{4075}{fixed} \markbox{4076}{point} \markbox{4077}{property} \markbox{4078}{if} \markbox{4079}{any} \markbox{4080}{of} \markbox{4081}{the} \markbox{4082}{following} holds:\\
		(i) \markbox{m4083}{$Y=S/\!\!\sim _{\alpha}$} \markbox{4084}{does} \markbox{4085}{not} \markbox{4086}{have} \markbox{4087}{the} \markbox{4088}{fixed} \markbox{4089}{point} property.\\
		(ii) \markbox{4090}{There} \markbox{4091}{exists} \markbox{4092}{a} \markbox{4093}{map} \markbox{m4094}{$f:X\to X$} \markbox{4095}{having} \markbox{4096}{no} fixed-point \markbox{4097}{and} \markbox{m4098}{$f\circ \sigma=\sigma \circ f$}.
	\end{proposition}
	\begin{proof}
		(i) \markbox{4099}{Let} \markbox{m4100}{$g:Y\to Y$} \markbox{4101}{be} \markbox{4102}{fixed} \markbox{4103}{point} free. \markbox{4104}{Then} \markbox{m4105}{$\phi:=s\circ g\circ p$} \markbox{4106}{on} \markbox{m4107}{$P(S,X)$} \markbox{4108}{also} \markbox{4109}{has} \markbox{4110}{no} \markbox{4111}{fixed} points, \markbox{4112}{where} \markbox{m4113}{$p$} \markbox{4114}{is} \markbox{4115}{the} \markbox{m4116}{$X$}-bundle \markbox{4117}{projection} \markbox{4118}{and} \markbox{m4119}{$s$} \markbox{4120}{is} \markbox{4121}{a} section.  
		(ii).   \markbox{4122}{Suppose} \markbox{4123}{that} \markbox{m4124}{$f:X\to X$} \markbox{4125}{has} \markbox{4126}{no} \markbox{4127}{fixed} \markbox{4128}{points} \markbox{4129}{and} \markbox{4130}{that} \markbox{m4131}{$f\circ \sigma=\sigma \circ f$}.
		Now \markbox{4132}{define} \markbox{m4133}{$\psi:P(S,X)\to P(S,X)$} \markbox{4134}{as} \markbox{m4135}{$[s,x]\mapsto [s,f(x)]$}. \markbox{4136}{The} well-definedness \markbox{4137}{follows} \markbox{4138}{because} \markbox{m4139}{$\psi([\alpha(s),\sigma(x)])=[\alpha(s),f\circ\sigma(x)]=[\alpha(s),\sigma \circ f(x)]=[s,f(x)]=\psi ([s,x])$}. Clearly, \markbox{m4140}{$\psi$} \markbox{4141}{has} \markbox{4142}{no} \markbox{4143}{fixed} points. \markbox{4144}{This} \markbox{4145}{completes} \markbox{4146}{the} proof.
	\end{proof}
	\fi
	
	
	%Recall that $P(S,X)$ has the fiber bundle structure $X\hookrightarrow P(SX)\twoheadrightarrow Y:=S/\!\! \sim_\alpha$, where $p$ is the $X$-bundle projection and $s$ is a section.  
	\begin{proposition}\label{necessary condition}
		Let \markbox{m4147}{$(P(S,X),g)$} \markbox{4148}{be} \markbox{4149}{a} pair, \markbox{4150}{where} \markbox{m4151}{$g$} \markbox{4152}{is} \markbox{4153}{a} \markbox{4154}{continuous} \markbox{4155}{map} \markbox{4156}{on} \markbox{4157}{the} \markbox{4158}{generalized} \markbox{4159}{Dold} \markbox{4160}{space} \markbox{m4161}{$P(S,X)$}. \markbox{4162}{Then} \markbox{m4163}{$(P(S,X),g)$} \markbox{4164}{does} \markbox{4165}{not} \markbox{4166}{have} \markbox{4167}{the} \markbox{4168}{CP} \markbox{4169}{if} \markbox{4170}{one} \markbox{4171}{of} \markbox{4172}{the} \markbox{4173}{following} hold:
		\begin{enumerate}
			\item The \markbox{4174}{continuous} \markbox{4175}{map} \markbox{m4176}{$g$} \markbox{4177}{is} \markbox{4178}{a} \markbox{4179}{fiber} \markbox{4180}{bundle} \markbox{4181}{map} \markbox{4182}{and} \markbox{4183}{the} \markbox{4184}{pair} \markbox{m4185}{$(Y,p \circ g \circ s)$} \markbox{4186}{does} \markbox{4187}{not} \markbox{4188}{have} \markbox{4189}{the} CP, 
			where \markbox{m4190}{$Y = S/\!\!\sim_\alpha$} \markbox{4191}{and} \markbox{m4192}{$s$} \markbox{4193}{denotes} \markbox{4194}{a} \markbox{4195}{section} \markbox{4196}{of} \markbox{4197}{the} \markbox{m4198}{$X$}-bundle \markbox{4199}{projection} \markbox{m4200}{$p$} \markbox{4201}{defined} \markbox{4202}{in} \eqref{sectio} \markbox{4203}{and} \eqref{proj}.
			
			\item  
			There \markbox{4204}{exists} \markbox{4205}{a} \markbox{m4206}{$\sigma$}-equivariant \markbox{4207}{map} \markbox{m4208}{$f$} (i.e. \markbox{m4209}{$f\circ\sigma =\sigma\circ f$}) \markbox{4210}{on} \markbox{m4211}{$X$} \markbox{4212}{and} \markbox{4213}{a} \markbox{m4214}{$\alpha \times \sigma$}-equivariant \markbox{4215}{map} \markbox{m4216}{$\tilde g$} \markbox{4217}{on} \markbox{m4218}{$S\times X$} \markbox{4219}{inducing} \markbox{m4220}{$g$} \markbox{4221}{such} that
			\markbox{m4222}{$\mathrm{id}_S \times f$} \markbox{4223}{coincides} \markbox{4224}{with} \markbox{4225}{neither} \markbox{m4226}{$\tilde{g}$} nor
			\markbox{m4227}{$(\alpha \times \sigma) \circ \tilde{g}$}.
		\end{enumerate}
		
		
		
	\end{proposition}
	\begin{proof} \textit{(1)}
		Suppose \markbox{4228}{that} \markbox{4229}{the} \markbox{4230}{pair} \markbox{m4231}{$(Y,p \circ g \circ s)$} \markbox{4232}{does} \markbox{4233}{not} \markbox{4234}{have} \markbox{4235}{the} CP. 
		Then \markbox{4236}{there} \markbox{4237}{exists} \markbox{4238}{a} \markbox{4239}{continuous} \markbox{4240}{map} \markbox{m4241}{$f : Y \to Y$} \markbox{4242}{such} \markbox{4243}{that} \begin{equation} \label{pogos}
			f(x) \neq p \circ g \circ s(x), \, \forall x \in Y.
		\end{equation}
		We \markbox{4244}{are} \markbox{4245}{given} \markbox{4246}{that} \markbox{m4247}{$g$} \markbox{4248}{is} \markbox{4249}{a} \markbox{4250}{fiber} \markbox{4251}{bundle} map, \markbox{4252}{which} \markbox{4253}{implies} \markbox{4254}{that} \markbox{4255}{there} \markbox{4256}{exist}  \markbox{m4257}{$g_1:Y\to Y$}, \markbox{4258}{satisfying}  \markbox{m4259}{$p \circ g = g_1 \circ p.$}
		Consider \markbox{m4260}{$p\circ g\circ s=g_1\circ p\circ s=g_1 $}.
		Thus, \markbox{m4261}{$p \circ g = g_1 \circ p$} implies
		\[
		p \circ g(x) = p \circ g \circ s \circ p(x),\, \forall x \in P(S, X).
		\]
		Define \markbox{4262}{the} \markbox{4263}{map} \markbox{m4264}{$\phi := s \circ f \circ p$} \markbox{4265}{on} \markbox{m4266}{$P(S, X)$}. \markbox{4267}{We} \markbox{4268}{claim} \markbox{4269}{that} \markbox{m4270}{$\phi(y)\neq g(y), \, \forall y \in P(S,X)$}. \\ \markbox{4271}{Suppose} \markbox{4272}{there} \markbox{4273}{exist} \markbox{m4274}{$y \in P(S, X)$} \markbox{4275}{such} \markbox{4276}{that} \markbox{m4277}{$\phi(y) = g(y)$}, \markbox{4278}{then} 
		\[
		p \circ g \circ s(p(y)) = p \circ g(y)
		= p \circ s \circ f \circ p (y) 
		= f(p(y)),
		\]
		which \markbox{4279}{contradicts} \eqref{pogos}.
		
		
		\textit{(2)} %Let $G$ be a group of order $2$ generated by $\alpha \times \sigma$ acting on the topological space $S\times X$ by composition.
		Let \markbox{m4280}{$G$} \markbox{4281}{denote} \markbox{4282}{the} \markbox{4283}{group} \markbox{4284}{of} \markbox{4285}{deck} \markbox{4286}{transformations} \markbox{4287}{of} \markbox{4288}{the} \markbox{4289}{double} \markbox{4290}{covering} \markbox{m4291}{$\pi : S \times X \to P(S, X)$}, \markbox{4292}{generated} \markbox{4293}{by} \markbox{4294}{the} \markbox{4295}{free} \markbox{4296}{involution} \markbox{m4297}{$\alpha \times \sigma.$}
		The \markbox{4298}{proof} \markbox{4299}{then} \markbox{4300}{follows} \markbox{4301}{from} \markbox{4302}{a} \markbox{4303}{general} \markbox{4304}{observation} \markbox{4305}{that} \markbox{4306}{if} \markbox{4307}{for} \markbox{4308}{two} \markbox{m4309}{$G$}-equivariant \markbox{4310}{maps} \markbox{m4311}{$\tilde\phi, \tilde\psi$} \markbox{4312}{on} \markbox{m4313}{$S\times X$}, \markbox{4314}{the} \markbox{4315}{maps} \markbox{m4316}{$\tilde \phi$} \markbox{4317}{and} \markbox{m4318}{$t \cdot \tilde\psi$} \markbox{4319}{have} \markbox{4320}{no} \markbox{4321}{point} \markbox{4322}{of} coincidence, \markbox{4323}{for} \markbox{4324}{any} \markbox{m4325}{$t \in G$}; \markbox{4326}{then} \markbox{4327}{the} \markbox{4328}{maps} \markbox{4329}{they} \markbox{4330}{induce} \markbox{4331}{on} \markbox{4332}{the} \markbox{4333}{orbit} \markbox{4334}{space} \markbox{m4335}{$P(S,X)$}, \markbox{4336}{namely} \markbox{m4337}{$\phi, \psi$}, \markbox{4338}{are} \markbox{4339}{also} coincidence-free.
		%$id_{S}\times f$ coincides with neither $\tilde g$ nor $(\alpha \times \sigma)\circ \tilde g$ then the two $G$-equivariant maps $\tilde\phi, \tilde\psi: M \to M$ on a topological space $M$, the maps $\tilde \phi$ and $t \cdot \tilde\psi$ have no point of coincidence, for any $t \in G$; then the maps they induce on the orbit space $M/G$, namely $\phi, \psi: M/G \to M/G$, are coincidence-free.
	\end{proof}
	
	\iffalse
	\begin{proposition}
		
		
		Let \markbox{m4340}{$g : P(S,X) \to P(S,X)$} \markbox{4341}{be} \markbox{4342}{a} \markbox{4343}{map} \markbox{4344}{induced} \markbox{4345}{by} \markbox{4346}{an} \markbox{m4347}{$\alpha \times \sigma$}-equivariant \markbox{4348}{map} \markbox{m4349}{$\tilde{g} : S \times X \to S \times X$}. 
		Suppose \markbox{4350}{there} \markbox{4351}{exists} \markbox{4352}{a} \markbox{m4353}{$\sigma$}-equivariant \markbox{4354}{map} \markbox{m4355}{$f : X \to X$} \markbox{4356}{such} \markbox{4357}{that} \markbox{4358}{both} \markbox{4359}{the} \markbox{4360}{pairs} \markbox{m4361}{$id_S\times f, \tilde g$} \markbox{4362}{and} \markbox{m4363}{$id_S\times f,\alpha\times\sigma\circ \tilde g$} \markbox{4364}{dont} \markbox{4365}{have} \markbox{4366}{any} \markbox{4367}{point} \markbox{4368}{of} coincidence.
		Then \markbox{4369}{the} \markbox{4370}{space} \markbox{m4371}{$P(S,X)$} \markbox{4372}{does} \markbox{4373}{not} \markbox{4374}{have} \markbox{4375}{the} \markbox{m4376}{$g$}-coincidence property.
	\end{proposition}
	\begin{proof}
		Define \markbox{m4377}{$\phi : P(S,X) \to P(S,X)$} \markbox{4378}{by} \markbox{m4379}{$\phi([s,x]) = [s, f(x)]$}. 
		This \markbox{4380}{map} \markbox{4381}{is} \markbox{4382}{well} \markbox{4383}{defined} because
		\markbox{m4384}{$\phi([\alpha(s), \sigma(x)]) = [\alpha(s), f(\sigma(x))] = [\alpha(s), \sigma(f(x))] = [s, f(x)]$}, \markbox{4385}{using} \markbox{4386}{the} \markbox{4387}{fact} \markbox{4388}{that} \markbox{m4389}{$f\circ \sigma=\sigma\circ f$}.
		
		Similarly, \markbox{4390}{define} \markbox{m4391}{$\psi : P(S,X) \to P(S,X)$} \markbox{4392}{by} \markbox{m4393}{$\psi([s,x]) = [g_1(s), g_2(x)]$}, 
		where \markbox{m4394}{$g_j = p_j \circ \tilde{g} \circ i_j$}, \markbox{4395}{and} \markbox{m4396}{$p_j$} \markbox{4397}{and} \markbox{m4398}{$i_j$} \markbox{4399}{denote} \markbox{4400}{the} \markbox{4401}{projection} \markbox{4402}{and} \markbox{4403}{inclusion} \markbox{4404}{maps} \markbox{4405}{on} \markbox{4406}{the} \markbox{m4407}{$j$}-th \markbox{4408}{factor} (\markbox{m4409}{$j=1,2$}). 
		The \markbox{4410}{map} \markbox{m4411}{$\psi$} \markbox{4412}{is} \markbox{4413}{well} \markbox{4414}{defined} \markbox{4415}{since} \markbox{m4416}{$\tilde{g}$} \markbox{4417}{is} \markbox{m4418}{$\alpha \times \sigma$}-equivariant.
		
		Since, \markbox{4419}{the} \markbox{4420}{maps} \markbox{m4421}{$f$} \markbox{4422}{and} \markbox{m4423}{$p_2 \circ \tilde{g} \circ i_2 = g_2$} \markbox{4424}{have} \markbox{4425}{no} \markbox{4426}{coincidence} points. 
		Hence, \markbox{4427}{for} \markbox{4428}{any} \markbox{m4429}{$[s,x] \in P(S,X)$}, \markbox{4430}{the} \markbox{4431}{second} \markbox{4432}{coordinates} \markbox{4433}{in} \markbox{m4434}{$\phi([s,x])$} \markbox{4435}{and} \markbox{m4436}{$\psi([s,x])$} \markbox{4437}{are} different. 
		Therefore, \markbox{m4438}{$\phi$} \markbox{4439}{and} \markbox{m4440}{$\psi$} \markbox{4441}{have} \markbox{4442}{no} \markbox{4443}{points} \markbox{4444}{of} coincidence, \markbox{4445}{and} \markbox{m4446}{$P(S,X)$} \markbox{4447}{does} \markbox{4448}{not} \markbox{4449}{possess} \markbox{4450}{the} \markbox{m4451}{$g$}-CP.
	\end{proof}
	\fi
	
		In \markbox{4452}{particular} \markbox{4453}{if} \markbox{4454}{we} \markbox{4455}{take} \markbox{m4456}{$g$} \markbox{4457}{to} \markbox{4458}{be} \markbox{4459}{the} \markbox{4460}{identity} \markbox{4461}{map} \markbox{4462}{in} \propref{necessary condition}, \markbox{4463}{we} \markbox{4464}{recover} Proposition~7.2.1 \markbox{4465}{of} \cite{mandal}, \markbox{4466}{which} \markbox{4467}{proves} \markbox{4468}{that} \markbox{4469}{if} \markbox{4470}{the} \markbox{4471}{base} \markbox{m4472}{$Y$} \markbox{4473}{does} \markbox{4474}{not} \markbox{4475}{have} \markbox{4476}{the} FPP, \markbox{4477}{or} \markbox{4478}{there} \markbox{4479}{exist} \markbox{4480}{a} \markbox{m4481}{$\sigma$}-equivarinat \markbox{4482}{map} \markbox{4483}{on} \markbox{4484}{the} \markbox{4485}{fibre} \markbox{m4486}{$X$} \markbox{4487}{with} \markbox{4488}{no} \markbox{4489}{fixed} \markbox{4490}{point} \markbox{4491}{then} \markbox{m4492}{$P(S,X)$} \markbox{4493}{does} \markbox{4494}{not} \markbox{4495}{have} \markbox{4496}{the} FPP. \markbox{4497}{As} \markbox{4498}{a} consequence, \markbox{m4499}{$P(m,n)$} \markbox{4500}{does} \markbox{4501}{not} \markbox{4502}{have} \markbox{4503}{the} \markbox{4504}{FPP} \markbox{4505}{if} \markbox{4506}{either} \markbox{m4507}{$m$} \markbox{4508}{or} \markbox{m4509}{$n$} \markbox{4510}{is} odd.
	\iffalse
	\begin{remark}\label{necessary criterion for FFP}
		A \markbox{4511}{generalized} \markbox{4512}{Dold} \markbox{4513}{space} \markbox{m4514}{$P(S,X)$} \markbox{4515}{does} \markbox{4516}{not} \markbox{4517}{have} \markbox{4518}{the} \markbox{4519}{FPP} \markbox{4520}{if} \markbox{4521}{any} \markbox{4522}{of} \markbox{4523}{the} \markbox{4524}{following} holds:
		
		\begin{enumerate}
			\item \markbox{m4525}{$Y=S/\!\!\sim _{\alpha}$} \markbox{4526}{does} \markbox{4527}{not} \markbox{4528}{have} \markbox{4529}{the} FPP.
			\item There \markbox{4530}{exists} \markbox{4531}{a} \markbox{m4532}{$\sigma$}-equivariant \markbox{4533}{map} \markbox{m4534}{$f$} \markbox{4535}{on} \markbox{m4536}{$X$} \markbox{4537}{that} \markbox{4538}{has} \markbox{4539}{no} \markbox{4540}{fixed} point.
		\end{enumerate}
	\end{remark}
	
	
	
	
	
	
	
	
	
	
	We \markbox{4541}{have} \markbox{4542}{the} \markbox{4543}{following} \markbox{4544}{observation} \markbox{4545}{as} \markbox{4546}{an} \markbox{4547}{immediate} \markbox{4548}{consequence} \markbox{4549}{of} \markbox{4550}{Remark} \ref{necessary criterion for FFP}.
	
	\begin{remark}
		Let \markbox{m4551}{$P(m,n)$} \markbox{4552}{be} \markbox{4553}{a} \markbox{4554}{classical} \markbox{4555}{Dold} \markbox{4556}{manifold} \markbox{4557}{such} \markbox{4558}{that} \markbox{4559}{either} \markbox{m4560}{$m$} \markbox{4561}{or} \markbox{m4562}{$n$} \markbox{4563}{is} odd. \markbox{4564}{Then}  
		%and the existence of a fixed point free map on $\mathbb CP^n$ which commutes with the conjugation, 
		\markbox{m4565}{$P(m,n)$} \markbox{4566}{does} \markbox{4567}{not} \markbox{4568}{have} \markbox{4569}{the} FPP.
	\end{remark}
	\begin{proof}
		When \markbox{m4570}{$m$} \markbox{4571}{is} odd, \markbox{4572}{the} \markbox{4573}{proof} \markbox{4574}{follows} \markbox{4575}{from} \markbox{4576}{part} \textit{(1)} \markbox{4577}{of} \remref{necessary criterion for FFP}. \\ \markbox{4578}{When} \markbox{m4579}{$n$} \markbox{4580}{is} odd, \markbox{4581}{we} \markbox{4582}{have} \markbox{4583}{a} \markbox{4584}{continuous} \markbox{m4585}{$\sigma$}-equivariant \markbox{4586}{map}  \markbox{m4587}{$f:\mathbb CP^n\to \mathbb CP^n$} \markbox{4588}{defined} \markbox{4589}{by} 
		\[
		[x_1:x_2:\cdots:x_n:x_{n+1}]\mapsto [ x_2:- x_1:\cdots: x_{n+1}:- x_{n}]
		\] \markbox{4590}{which} \markbox{4591}{does} \markbox{4592}{not} \markbox{4593}{have} \markbox{4594}{any} \markbox{4595}{fixed} point.
		%Clearly, $f\circ \sigma=\sigma\circ f$, where  
		Thus, \markbox{4596}{using} \markbox{4597}{part} \textit{(2)} \markbox{4598}{of} \remref{necessary criterion for FFP}, \markbox{m4599}{$P(m,n)$} \markbox{4600}{does} \markbox{4601}{not} \markbox{4602}{have} fixed-point property.
	\end{proof}
	%\textcolor{red}{Converse of the above remark seems to be true.}
	\fi
	
	
	\iffalse
	We \markbox{4603}{need} \markbox{4604}{the} \markbox{4605}{following} \markbox{4606}{theorems} \markbox{4607}{in} \markbox{4608}{the} sequal.
	\begin{theorem}[Lefschetz Fixed-Point Theorem]\label{Lefschetz theorem}
		Let \markbox{m4609}{$X$} \markbox{4610}{be} \markbox{4611}{a} \markbox{4612}{finite} \markbox{4613}{simplicial} complex, \markbox{4614}{and} \markbox{4615}{let} \markbox{m4616}{$f : X \to X$} \markbox{4617}{be} \markbox{4618}{a} \markbox{4619}{continuous} map.  
		Define \markbox{4620}{the} number
		\[
		\tau(f) = \sum_{n} (-1)^n \operatorname{tr}\!\big(f_* : H_n(X) \to H_n(X)\big),
		\]
		called \markbox{4621}{the} \emph{Lefschetz number} \markbox{4622}{of} \markbox{m4623}{$f$}.  
		If \markbox{m4624}{$\tau(f) \ne 0$}, \markbox{4625}{then} \markbox{m4626}{$f$} \markbox{4627}{has} \markbox{4628}{at} \markbox{4629}{least} \markbox{4630}{one} \markbox{4631}{fixed} point.
	\end{theorem}
	\fi
	
	
	%\begin{theorem}[\cite{glover-homer}, Theorem 2]\label{GH2} For $k,n$ as in Theorem \ref{GH1}, the complex Grassmannian $\mathbb CG_{n,k}$ has the fixed-point property if and only if $k(n-k)$ is even. \end{theorem}
	
	
	\iffalse
	In (Theorem 2, \cite{glover-homer}), \markbox{4632}{it} \markbox{4633}{is} \markbox{4634}{proved} \markbox{4635}{that} \markbox{m4636}{$\mathbb CG_{n,k}$} \markbox{4637}{has} fixed-point \markbox{4638}{property} \markbox{4639}{if} \markbox{4640}{and} \markbox{4641}{only} \markbox{4642}{if} \markbox{m4643}{$k(n-k)$} \markbox{4644}{is} even, \markbox{4645}{provided} \markbox{4646}{either} (i) \markbox{m4647}{$k\leq 3$} \markbox{4648}{and} \markbox{m4649}{$n>2k$} \markbox{4650}{or} (ii) \markbox{m4651}{$k>3$} \markbox{4652}{and} \markbox{m4653}{$n> 2k^2-1$}.
	It \markbox{4654}{is} \markbox{4655}{well} \markbox{4656}{known} \markbox{4657}{that} \markbox{4658}{there} \markbox{4659}{exists} \markbox{4660}{a} fixed-point \markbox{4661}{free} \markbox{4662}{map}   \markbox{m4663}{$ L \mapsto L^\perp   $} \markbox{4664}{on}   \markbox{m4665}{$ \mathbb{C}G_{n,k}   $} \markbox{4666}{when}   \markbox{m4667}{$ k = n - k   $}.
	
	The \markbox{4668}{following} \markbox{4669}{remark} \markbox{4670}{follows} \markbox{4671}{easily} \markbox{4672}{from} \markbox{4673}{Theorem} \ref{hoffman} \cite{hoffman} \markbox{4674}{and} \markbox{4675}{Theorem} \ref{GH2} \cite{glover-homer}.
	\begin{remark}Let \markbox{m4676}{$f$} \markbox{4677}{be} \markbox{4678}{continuous} self-map \markbox{4679}{on} \markbox{m4680}{$\mathbb CG_{n,k}$} \markbox{4681}{such} that
		\markbox{m4682}{$ f^*(c_1) = \lambda c_1 $}, \markbox{m4683}{$ \lambda\neq 0 $}. \markbox{4684}{Assume} \markbox{4685}{that} \markbox{m4686}{$k(n-k)$} \markbox{4687}{is} \markbox{4688}{even} \markbox{4689}{and} \markbox{m4690}{$k\neq n-k$}. \markbox{4691}{Then} \markbox{m4692}{$f$} \markbox{4693}{has} \markbox{4694}{a} \markbox{4695}{fixed} point. 
	\end{remark}
	\begin{proof}
		By Theorem~\ref{hoffman} \cite{hoffman}, \markbox{4696}{the} \markbox{4697}{map} \markbox{m4698}{$f^*$} \markbox{4699}{is} \markbox{4700}{an} \markbox{4701}{Adams} map. \markbox{4702}{Using} \markbox{4703}{the} \markbox{4704}{computation} \markbox{4705}{of} \markbox{4706}{the} \markbox{4707}{Lefschetz} \markbox{4708}{number} \markbox{m4709}{$L(f)$} \markbox{4710}{from} \markbox{4711}{the} \markbox{4712}{proof} \markbox{4713}{of} Theorem~\ref{GH1}, \markbox{4714}{we} \markbox{4715}{observe} \markbox{4716}{that} \markbox{m4717}{$L(f) \ne 0$} \markbox{4718}{because} \markbox{m4719}{$k(n-k)$} \markbox{4720}{is} even. Thus, \markbox{m4721}{$f$} \markbox{4722}{must} \markbox{4723}{have} \markbox{4724}{a} \markbox{4725}{fixed} point. \markbox{4726}{This} \markbox{4727}{completes} \markbox{4728}{the} proof.
	\end{proof}
	\fi
	
	\subsection{} \markbox{4729}{Let} \markbox{4730}{us} \markbox{4731}{recall} \markbox{4732}{a} \markbox{4733}{well} \markbox{4734}{known} \markbox{4735}{result} \markbox{4736}{in} \markbox{4737}{coincidence} theory, \markbox{4738}{the} \markbox{4739}{Lefschetz} \markbox{4740}{Coincidence} Theorem, \markbox{4741}{which} \markbox{4742}{will} \markbox{4743}{be} \markbox{4744}{used} \markbox{4745}{to} \markbox{4746}{prove} \markbox{4747}{results} \markbox{4748}{in} \markbox{4749}{the} \markbox{4750}{rest} \markbox{4751}{of} \markbox{4752}{this} paper.\\
	
	For \markbox{4753}{a} \markbox{4754}{closed} \markbox{4755}{oriented} \markbox{4756}{manifold} \markbox{m4757}{$M$} \markbox{4758}{of} \markbox{4759}{dimension} \markbox{m4760}{$n$}, \markbox{4761}{let} \markbox{m4762}{$[M]\in H^n(M;\mathbb Q)$} \markbox{4763}{denote} \markbox{4764}{a} \markbox{4765}{chosen} \markbox{4766}{fundamental} class. \markbox{4767}{Then} \markbox{4768}{we} \markbox{4769}{have} \markbox{4770}{the} Poincar\'e  \markbox{4771}{duality} \markbox{4772}{isomorphism} \markbox{m4773}{$D_M:H^k(M;\mathbb Q)\to H_{n-k}(M;\mathbb Q)$}, \markbox{4774}{defined} \markbox{4775}{by} 
	\begin{equation}\label{poinc dua}
		D_M(\alpha)=[M]\frown \alpha, \, \forall \alpha\in H^k(M;\mathbb Z).
	\end{equation}

	
	\begin{theorem}[Lefschetz \markbox{4776}{Coincidence} Theorem]\label{LCT}
		Let \markbox{m4777}{$f,g$} \markbox{4778}{be} \markbox{4779}{two} \markbox{4780}{continuous} \markbox{4781}{maps} \markbox{4782}{on} \markbox{4783}{a} compact, connected, \markbox{4784}{oriented} \markbox{4785}{manifold} \markbox{m4786}{$M$} \markbox{4787}{of}  \markbox{4788}{dimension} \markbox{m4789}{$n$}. \markbox{4790}{The} \markbox{4791}{Lefschetz} \markbox{4792}{coincidence} \markbox{4793}{number} \markbox{4794}{is} \markbox{4795}{defined} \markbox{4796}{as} 
		$$L(f,g):= \sum_{i=0}^{n} (-1)^i \mathrm{tr}\big(D_M \circ g^* \circ D_M^{-1} \circ f_* \; : \; H_i(M;\mathbb{Q}) \longrightarrow H_i(M;\mathbb{Q})\big).
		$$ \markbox{4797}{If} \markbox{m4798}{$L(f,g) \neq 0$}, \markbox{4799}{then} \markbox{4800}{there} \markbox{4801}{exists} \markbox{m4802}{$x \in M$} \markbox{4803}{such} \markbox{4804}{that} \markbox{m4805}{$f(x) = g(x)$}.
	\end{theorem}
	When  \markbox{m4806}{$g=\operatorname{id}_M$}, \markbox{4807}{the} \markbox{4808}{theorem} \markbox{4809}{reduces} \markbox{4810}{to} \markbox{4811}{the} \markbox{4812}{Lefschetz} Fixed-Point \markbox{4813}{Theorem} \markbox{4814}{for} \markbox{m4815}{$M$}.\\
	
	To \markbox{4816}{study} \markbox{4817}{the} \markbox{4818}{coincidence} \markbox{4819}{theory} \markbox{4820}{of} \markbox{4821}{generalized} \markbox{4822}{Dold} \markbox{4823}{spaces} \markbox{4824}{fibred} \markbox{4825}{by} \markbox{4826}{complex} \markbox{4827}{Grassmannians} \markbox{4828}{over} \markbox{4829}{real} \markbox{4830}{projective} spaces, \markbox{4831}{it} \markbox{4832}{is} \markbox{4833}{helpful} \markbox{4834}{to} \markbox{4835}{first} \markbox{4836}{understand} \markbox{4837}{the} \markbox{4838}{coincidence} \markbox{4839}{theory} \markbox{4840}{of} \markbox{4841}{complex} Grassmannians, \markbox{4842}{a} \markbox{4843}{topic} \markbox{4844}{of} \markbox{4845}{independent} interest. \markbox{4846}{We} \markbox{4847}{now} \markbox{4848}{prove} \markbox{4849}{the} \markbox{4850}{following} \markbox{4851}{lemma} (cf. \markbox{4852}{Theorem} 2, \cite{glover-homer}) \markbox{4853}{to} \markbox{4854}{prove} \propref{CP of CGnk}.
	\begin{lemma}\label{sum neq 0}
		Let \markbox{m4855}{$d_{2i}$} \markbox{4856}{be} \markbox{4857}{the} \markbox{m4858}{$2i$}-th \markbox{4859}{Betti} \markbox{4860}{number} \markbox{4861}{of} \markbox{4862}{a} \markbox{4863}{complex} \markbox{4864}{Grassmannian} \markbox{m4865}{$\mathbb CG_{n,k}$} \markbox{4866}{with} \markbox{m4867}{$d=k(n-k)$} even. \markbox{4868}{Then} \markbox{4869}{the} \markbox{4870}{sum} \markbox{m4871}{\(\sum _{i=0}^d d_{2i}\lambda^i\neq 0,\, \forall \lambda \in \mathbb Q.\)}
	\end{lemma}
	\begin{proof}
		Let \markbox{4872}{us} \markbox{4873}{consider} \markbox{4874}{the} \markbox{4875}{sum} \markbox{m4876}{$\sum_{i=0}^d d_{2i}\lambda^i$}, \markbox{4877}{when} \markbox{m4878}{$\lambda$} \markbox{4879}{is} \markbox{4880}{an} integer. 
		Clearly, $$\sum_{i=0}^d d_{2i}\lambda^i \equiv 1 \pmod{\lambda}.$$ %(see Theorem 2 in \cite{glover-homer}). 
		Hence, \markbox{m4881}{$\sum_{i=0}^d d_{2i}\lambda^i \neq 0$}, \markbox{4882}{if} \markbox{m4883}{$\lambda \neq \pm 1$}. \markbox{4884}{When} \markbox{m4885}{$\lambda = 1$}, \markbox{4886}{the} \markbox{4887}{sum} \markbox{4888}{is} \markbox{4889}{also} \markbox{4890}{positive} \markbox{4891}{and} \markbox{4892}{therefore} nonzero. \markbox{4893}{It} \markbox{4894}{remains} \markbox{4895}{to} \markbox{4896}{consider} \markbox{4897}{the} \markbox{4898}{case} \markbox{4899}{where} \markbox{m4900}{$\lambda = -1$}.
		Let \markbox{m4901}{$\chi(\mathbb{R}G_{n,k})$} \markbox{4902}{denote} \markbox{4903}{the} Euler-Poincar{\'e} \markbox{4904}{characteristic} \markbox{4905}{of} \markbox{m4906}{$\mathbb{R}G_{n,k}$} \markbox{4907}{and} \markbox{4908}{be} \markbox{4909}{defined} \markbox{4910}{by} $$\chi(X):=\sum_{i\ge0}\dim H^i(\mathbb{R}G_{n,k};\mathbb{Z}_2)$$ \markbox{4911}{where} \markbox{m4912}{$\mathbb{R}G_{n,k}$} \markbox{4913}{denotes} \markbox{4914}{the} \markbox{4915}{Grassmannian} \markbox{4916}{of} \markbox{4917}{real} \markbox{m4918}{$k$}-planes \markbox{4919}{in} \markbox{m4920}{$\mathbb{R}^n$}.
		Now \markbox{4921}{we} \markbox{4922}{observe} \markbox{4923}{that} 
		\markbox{m4924}{$\sum_{i=0}^d d_{2i}(-1)^i = \chi(\mathbb{R}G_{n,k})$} \markbox{4925}{where} \markbox{m4926}{$d_{2i} = \operatorname{dim} H^{2i}(\mathbb{C}G_{n,k}; \mathbb{Q}) = \dim H^{i}(\mathbb{R}G_{n,k}; \mathbb{Z}_2)$}. 
		It \markbox{4927}{is} \markbox{4928}{a} \markbox{4929}{well} \markbox{4930}{known} \markbox{4931}{fact} \markbox{4932}{that} \markbox{m4933}{$\chi(\mathbb{R}G_{n,k}) \neq 0$} \markbox{4934}{if} \markbox{m4935}{$k(n-k)$} \markbox{4936}{is} even. 
		
		Let \markbox{4937}{us} \markbox{4938}{move} \markbox{4939}{to} \markbox{4940}{the} \markbox{4941}{other} \markbox{4942}{case} \markbox{4943}{where} \markbox{m4944}{$\lambda\in \mathbb{Q}\backslash \mathbb{Z}$}. \markbox{4945}{Suppose} \markbox{m4946}{$\sum_{i=0}^{d} d_{2i}\lambda^i =0$} \markbox{4947}{for} \markbox{4948}{some} \markbox{m4949}{$\lambda = \frac{p}{q}$} \markbox{4950}{where} \markbox{m4951}{$p$} \markbox{4952}{and} \markbox{m4953}{$q$} \markbox{4954}{are} \markbox{4955}{coprime} integers. \markbox{4956}{Since} \markbox{m4957}{$d_0 =d_d =1$}, \markbox{4958}{using} \markbox{4959}{the} \markbox{4960}{rational} \markbox{4961}{root} \markbox{4962}{theorem} \markbox{m4963}{$p|1$} \markbox{4964}{and} \markbox{m4965}{$q|1$}. Hence, \markbox{m4966}{$\lambda = \pm 1$}, \markbox{4967}{which} \markbox{4968}{is} \markbox{4969}{a} contradiction. Therefore, \markbox{4970}{we} \markbox{4971}{conclude} \markbox{4972}{that} \markbox{m4973}{$\sum_{i=0}^{d} d_{2i}\lambda^i \neq 0$} \markbox{4974}{for} \markbox{4975}{all} \markbox{m4976}{$\lambda\in \mathbb Q$}. 
		%Thus, the sum  $\sum _{i=1}^d d_{2i}\lambda^i$ is nonzero for every integer value of $\lambda.$ 
		\iffalse	Now, \markbox{4977}{suppose} \markbox{4978}{for} \markbox{4979}{contradiction} \markbox{4980}{that} \markbox{4981}{there} \markbox{4982}{exists} \markbox{4983}{a} \markbox{4984}{rational} \markbox{4985}{number} \markbox{m4986}{$\lambda = a/b$}  \markbox{4987}{written} \markbox{4988}{in} \markbox{4989}{lowest} \markbox{4990}{terms} (i.e., \markbox{m4991}{$\gcd(a,b)=1$}) \markbox{4992}{such} \markbox{4993}{that} \markbox{4994}{the} \markbox{4995}{sum} vanishes. \markbox{4996}{Note} \markbox{4997}{that} \markbox{4998}{in} \markbox{4999}{the} \markbox{5000}{polynomial} \markbox{m5001}{$\sum_{i=0}^{d} d_{2i}\lambda^i$}, \markbox{5002}{both} \markbox{5003}{the} \markbox{5004}{leading} \markbox{5005}{coefficient} \markbox{m5006}{$d_d$} \markbox{5007}{and} \markbox{5008}{the} \markbox{5009}{constant} \markbox{5010}{term} \markbox{m5011}{$d_0$} \markbox{5012}{are} \markbox{5013}{equal} \markbox{5014}{to} \markbox{m5015}{$1$}. \markbox{5016}{By} \markbox{5017}{the} \markbox{5018}{Rational} \markbox{5019}{Root} Theorem, \markbox{5020}{the} \markbox{5021}{rational} \markbox{5022}{root} \markbox{m5023}{$\lambda = a/b$} \markbox{5024}{must} \markbox{5025}{satisfy} \markbox{m5026}{$a \mid 1$} \markbox{5027}{and} \markbox{m5028}{$b \mid 1$}, \markbox{5029}{hence} \markbox{m5030}{$\lambda = \pm 1\in \mathbb Z$}. \markbox{5031}{Since} \markbox{5032}{this} \markbox{5033}{is} \markbox{5034}{not} possible, \fi
	\end{proof}
	
	
	
	Denote \markbox{5035}{the} \markbox{m5036}{$i$}-th \markbox{5037}{homology} \markbox{5038}{groups} 
	\markbox{m5039}{$H_i(\mathbb{C}G_{n,k}; \mathbb{Q}), \, H_i(\mathbb{S}^m; \mathbb{Q})$} \markbox{5040}{and} 
	\markbox{m5041}{$H_i(\mathbb{S}^m \times \mathbb{C}G_{n,k}; \mathbb{Q})$}, \markbox{5042}{by} \markbox{m5043}{$H_i^{\mathbb{C}G}, H_i^{\mathbb{S}}$} \markbox{5044}{and} \markbox{m5045}{$H_i^{\times}$}, respectively. \markbox{5046}{Let} \markbox{m5047}{$d$} \markbox{5048}{denote} \markbox{5049}{the} \markbox{5050}{complex} \markbox{5051}{dimension} \markbox{5052}{of} \markbox{m5053}{$\mathbb CG_{n,k}$}, \markbox{5054}{given} \markbox{5055}{by} \markbox{m5056}{$d = k(n - k)$}. \markbox{5057}{Then} \markbox{5058}{we} \markbox{5059}{have} \markbox{5060}{the} \markbox{5061}{following} proposition.
	\begin{proposition}\label{CP of CGnk}
		Consider \markbox{5062}{a} \markbox{5063}{complex} \markbox{5064}{Grassmannian} \markbox{m5065}{$\mathbb{C}G_{n,k}$} \markbox{5066}{such} \markbox{5067}{that} \markbox{5068}{the} \markbox{5069}{hypothesis} \eqref{Homer} \markbox{5070}{is} \markbox{5071}{satisfied} \markbox{5072}{and} \markbox{m5073}{$k(n-k)$} \markbox{5074}{is} even. \markbox{5075}{Let} \markbox{m5076}{$g $} \markbox{5077}{be} \markbox{5078}{a} \markbox{5079}{continuous} \markbox{5080}{map} \markbox{5081}{on} \markbox{m5082}{$\mathbb{C}G_{n,k}$} \markbox{5083}{with} \markbox{5084}{nonzero} \markbox{5085}{Brouwer} degree. \markbox{5086}{Then} \markbox{5087}{the} \markbox{5088}{pair} \markbox{m5089}{$(\mathbb{C}G_{n,k}, g)$} \markbox{5090}{has} \markbox{5091}{the} \markbox{5092}{coincidence} property.
	\end{proposition}
	
	
	\begin{proof}
	Self-maps \markbox{5093}{with} \markbox{5094}{nonzero} \markbox{5095}{Brouwer} \markbox{5096}{degree} \markbox{5097}{induces} \markbox{5098}{automorphisms} \markbox{5099}{in} \markbox{5100}{the} \markbox{5101}{rational} \markbox{5102}{cohomology} algebra. \markbox{5103}{Using} \markbox{5104}{Theorem} \ref{hom and hof} \markbox{5105}{part} \textit{(i)}, \markbox{5106}{there} \markbox{5107}{exist} \markbox{5108}{a} \markbox{5109}{nonzero} \markbox{5110}{rational} \markbox{m5111}{$\lambda$} \markbox{5112}{such} \markbox{5113}{that} \markbox{m5114}{$g^*(c_i)=\lambda^ic_i, \forall i\in I.$} \markbox{5115}{Let} \markbox{m5116}{$f$} \markbox{5117}{be} \markbox{5118}{a} \markbox{5119}{continuous} \markbox{5120}{map} \markbox{5121}{on} \markbox{m5122}{$\mathbb CG_{n,k}$} \markbox{5123}{and} \markbox{5124}{using} \thmref{hom and hof} \markbox{5125}{part} \textit{(i)}, \markbox{5126}{there} \markbox{5127}{exists} \markbox{m5128}{$\mu \in \mathbb Q$} \markbox{5129}{such} that
		\[
		f^*(c_i)=\mu^ic_i, \forall i\in I.
		\]
		Then \markbox{5130}{by} \markbox{5131}{the} \markbox{5132}{Universal} \markbox{5133}{Coefficient} Theorem, \markbox{m5134}{$ \hom_{\mathbb{Q}} (H_i^{\mathbb CG};\mathbb Q) \cong H^i_{\mathbb CG}$} non-canonically \markbox{5135}{which} \markbox{5136}{implies} \markbox{5137}{that} 
		\begin{align*}
			\varphi \circ f_* &= f^*(\varphi) , \, \forall
			\varphi \in \hom _{\mathbb Q}\big(H_{2i}^{\mathbb CG}, \mathbb Q\big)\cong H^{2i}_{\mathbb CG}.\\
			\varphi(f_*(x))&=(f^*(\varphi))(x)= \mu^i\varphi(x)=\varphi(\mu^ix),\; \forall x\in H_{2i}^{\mathbb CG}.
		\end{align*}
		The \markbox{5138}{last} \markbox{5139}{equation} \markbox{5140}{implies} \markbox{5141}{that} \markbox{m5142}{$f_*(x)=\mu^ix,\, \forall x\in  H_{2i}^{\mathbb CG}.$}
		%This implies for all $\varphi\in \hom_{\mathbb Q}(H_{2i}^{\mathbb CG},\mathbb Q)$ and $x\in H_{2i}^{\mathbb CG}$, we have
		Now \markbox{5143}{observe} \markbox{5144}{that} \markbox{m5145}{$D\circ g^*\circ D^{-1}\circ f_*: H_{2i}^{\mathbb CG}\to H_{2i}^{\mathbb CG}$} \markbox{5146}{is} \markbox{5147}{given} \markbox{5148}{by} 
		\[
		D\circ g^*\circ D^{-1}\circ f_*(x)=D\circ g^*\circ D^{-1}(\mu^ix)=\mu^iD\circ g^*(D^{-1}x)=\mu^iD(\lambda^{d-i}D^{-1}x)=\mu^i\lambda^{d-i}x.
		\]
		Thus \markbox{5149}{for} \markbox{m5150}{$x\in H_{2i}^{\mathbb{C}G}$}, \markbox{5151}{the} \markbox{5152}{Lefschetz} \markbox{5153}{coincidence} \markbox{5154}{number} \markbox{5155}{is} \markbox{5156}{given} by
		\[
		\begin{array}{ll}
			L(f,g) &= \sum_{i=0}^d (-1)^{2i}
			\mathrm{tr}(D\circ g^*\circ D^{-1}\circ f_*(x)) \\[6pt]
			&=  \sum_{i=0}^d d_{2i}\mu^i\lambda^{d-i}  \\[6pt]
			&= \lambda^d \sum_{i=0}^d d_{2i}(\mu/\lambda)^i \neq 0 \quad (\because \lambda\neq 0) 
		\end{array}
		\]
		where \markbox{m5157}{$d_{2i}$} \markbox{5158}{denotes} \markbox{m5159}{$\dim_{\mathbb Q}H^{2i}_{\mathbb CG}$} \markbox{5160}{and} \markbox{5161}{the} \markbox{5162}{last} \markbox{5163}{equation} \markbox{5164}{holds} \markbox{5165}{by} \markbox{5166}{using} \lemref{sum neq 0}. Therefore, \markbox{5167}{using} \thmref{LCT} \markbox{5168}{the} \markbox{5169}{pair} \markbox{m5170}{$(\mathbb{C}G_{n,k},g)$} \markbox{5171}{has} \markbox{5172}{the} \markbox{5173}{coincidence} property.
	\end{proof}
	
	\iffalse{}  \textcolor{teal}{
		In the case of a quaternionic Grassmannian $\mathbb H G_{n,k}$, an Adams operation of degree $-1$ cannot be induced by any self-map, since its action on $H^*(\mathbb H G_{n,k};\mathbb Z/3)$ does not commute with the reduced third power operation (see Theorem~2(2) of \cite{glover-homer}). Consequently, the value $\lambda=-1$ cannot occur in the sum $\sum_{i=0}^{d} d_{2i}\lambda^i$ from Lemma~\ref{sum neq 0}, so the sum is nonzero in all cases, regardless of the parity of $d=k(n-k)$. Proceeding as in Proposition~\ref{CP of CGnk}, one obtains the following extension of Theorem~2(2) in \cite{glover-homer}:
		\begin{proposition}\label{CP of HGnk}
			For any homotopy equivalence $g$ of a quaternionic Grassmannian $\mathbb H G_{n,k}$, the pair $(\mathbb H G_{n,k},g)$ has the coincidence property.
		\end{proposition}
	}
	\fi
	
	
	
	
	\subsection{} \markbox{5174}{Denote} \markbox{5175}{by} \markbox{m5176}{$H_*^\times = \bigoplus_{i\geq 0} H_i^{\times},\, H_*^{\mathbb{C}G} = \bigoplus_{i\geq 0} H_i^{\mathbb{C}G},\, H_*^{\mathbb{S}} = \bigoplus_{i\geq 0} H_i^{\mathbb{S}}$} \markbox{5177}{and} \markbox{m5178}{$\vartheta$} \markbox{5179}{the} \markbox{5180}{fundamental} \markbox{5181}{class} \markbox{m5182}{$[\mathbb S^m]\in H_m^{\mathbb S}$}. \markbox{5183}{Let} \markbox{m5184}{$\{v_q\}$} \markbox{5185}{be} \markbox{5186}{a} \markbox{5187}{homogeneous} \markbox{5188}{basis} \markbox{5189}{of} \markbox{m5190}{$H_*^{\mathbb{C}G}$}, \markbox{5191}{and} \markbox{5192}{let} \markbox{m5193}{$\{\delta_{v_q}\}$} \markbox{5194}{denote} \markbox{5195}{the} \markbox{5196}{corresponding} \markbox{5197}{dual} \markbox{5198}{basis} \markbox{5199}{of}  
	\markbox{m5200}{$\operatorname{Hom}(H_*^{\mathbb{C}G}, \mathbb Q) \cong H^*_{\mathbb{C}G}$}, \markbox{5201}{such} \markbox{5202}{that}  
	\markbox{m5203}{$ \delta_{v_q} (v_p) = \delta_{qp}$} \markbox{5204}{where} \markbox{m5205}{$\delta_{qp}$} \markbox{5206}{is} \markbox{5207}{the} \markbox{5208}{Kronecker} \markbox{5209}{delta} function. \markbox{5210}{Without} \markbox{5211}{loss} \markbox{5212}{of} generality, \markbox{5213}{assume} \markbox{5214}{that} \markbox{m5215}{$1=v_0 \in \{v_i\}$} \markbox{5216}{represents} \markbox{5217}{the} \markbox{5218}{generator} \markbox{5219}{of} \markbox{m5220}{$H_0^{\mathbb{C}G} \cong \mathbb Q$}.
	
	Over \markbox{m5221}{$\mathbb Q$}, \markbox{5222}{the} K\"unneth \markbox{5223}{Theorem} \markbox{5224}{yields} \markbox{5225}{the} \markbox{5226}{following} \markbox{5227}{decompositions} 
	\iffalse	Set \markbox{m5228}{$d = 2k(n - k)$}, \markbox{m5229}{$m=2s$} \markbox{5230}{and} \markbox{m5231}{$t={n \choose k}$}. 
	
	By Remark~\ref{lift}, \markbox{5232}{every} \markbox{5233}{continuous} \markbox{5234}{map} \markbox{m5235}{$f$} \markbox{5236}{on} \markbox{m5237}{$P(m,n,k)$} \markbox{5238}{admits} \markbox{5239}{a} \markbox{5240}{lift} 
	\markbox{m5241}{$\tilde f$} \markbox{5242}{on} \markbox{m5243}{$\mathbb{S}^m \times \mathbb{C}G_{n,k}$} \markbox{5244}{satisfying} 
	\markbox{m5245}{$f \circ \pi = \pi \circ \tilde f,$}
	where \markbox{m5246}{$\pi : \mathbb{S}^m \times \mathbb{C}G_{n,k} \to P(m,n,k)$} \markbox{5247}{is} \markbox{5248}{the} \markbox{5249}{double} \markbox{5250}{covering} map. 
	We \markbox{5251}{continue} \markbox{5252}{to} \markbox{5253}{denote} \markbox{5254}{such} \markbox{5255}{lifts} \markbox{5256}{by} \markbox{5257}{the} \markbox{5258}{corresponding} \markbox{5259}{maps} \markbox{5260}{on} \markbox{m5261}{$P(m,n,k)$} \markbox{5262}{with} \markbox{5263}{a} tilde.
	By Theorem~\ref{main thm}, \markbox{5264}{there} \markbox{5265}{exists}  \markbox{m5266}{$\lambda \in \mathbb{Q}$} \markbox{5267}{or} \markbox{m5268}{$P_i\in H^{2i-m}_{\mathbb CG} \;\forall i\in I$} \markbox{5269}{with} \markbox{5270}{some} \markbox{m5271}{$P_i\neq 0$}, \markbox{5272}{such} \markbox{5273}{that} \
	
	\begin{equation}
		\text{either (i) } 
		\tilde f^*(c_i) = \lambda^i c_i,\forall i\in I, \text{or (ii) } \tilde f^*(c_i)=uP_i, \  \forall i\in I , 
	\end{equation}
	\begin{equation}
		\text{and either (iii) }\tilde f^*(u) = \mu u \text{ for some }\mu\in \mathbb Q, \text{or (iv) }\tilde f^*(u) \in H^m_{\mathbb{C}G}.
	\end{equation}
	
	
	%We have the induced maps in homology and cohomology $\tilde f_*$ and $\tilde f^*$, respectively, related by the Kronecker pairing 
	%$\langle \;,\;\rangle:H^i_\times\times H_i^\times\to \mathbb Q$ defined by
	%\[
	%\langle \tilde f^* \varphi, x\rangle = \langle \varphi, \tilde f_* x\rangle 
	%\quad \text{for all } \varphi \in H^i_\times,\, x \in H_i^\times.
	%\]
	%and $u \in H^*_\times$ corresponds to the fundamental class 
	%of $[\mathbb S^m] \in H^{\mathbb S}_m \subseteq H^\times_m.$
	
	%Fix a basis $\{1, v_1, v_2, \ldots, v_t\}$ for $H_*^{\mathbb{C}G}$. 
	%The corresponding dual basis of $\hom(H_*^{\mathbb{C}G
		%}, \mathbb{Q}) \cong H^*_{\mathbb{C}G}$ is 
	%$\{\delta_1, \delta_{v_1}, \delta_{v_2}, \ldots, \delta_{v_t}\}$, 
	%where \ $\delta_{v_i}(x) = \langle \delta_{v_i}, x \rangle$ defined to be $1$ if and only if $x=v_i$, and $0$ otherwise, for all $v_i$.
	\fi  
	\begin{equation}\label{kunneth}
		H_i^\times \cong H_i^{\mathbb{C}G} \oplus (\vartheta \otimes H_{i-m}^{\mathbb{C}G}), 
		\qquad 
		H^i_\times \cong H^i_{\mathbb{C}G} \oplus uH^{i-m}_{\mathbb{C}G},
	\end{equation}
	where  \markbox{m5274}{$u \in H^m_{\times} \cong \operatorname{Hom}(H_m^{\times}, \mathbb Q)$} \markbox{5275}{corresponds} \markbox{5276}{to} \markbox{5277}{the} \markbox{5278}{element} \markbox{m5279}{$\delta_{\vartheta\otimes 1}$}.
	
	
	Using \eqref{kunneth}, \markbox{5280}{we} \markbox{5281}{can} \markbox{5282}{extend} \markbox{5283}{the} \markbox{5284}{chosen} \markbox{5285}{basis} \markbox{m5286}{$\{v_q\}$} \markbox{5287}{of} \markbox{m5288}{$H_*^{\mathbb{C}G}$} \markbox{5289}{to} \markbox{m5290}{$	\{v_q\} \cup \{\vartheta \otimes v_q\}$} \markbox{5291}{of} \markbox{m5292}{$H_*^\times$} \markbox{5293}{such} \markbox{5294}{that} \markbox{5295}{the} \markbox{5296}{corresponding} \markbox{5297}{dual} \markbox{5298}{basis} \markbox{5299}{can} \markbox{5300}{also} \markbox{5301}{be} \markbox{5302}{extended} \markbox{5303}{from} \markbox{m5304}{$\{\delta_{v_q}\}$} \markbox{5305}{of} \markbox{m5306}{$ \hom (H_*^{\mathbb{C}G};\mathbb{Q})$} \markbox{5307}{to} \markbox{m5308}{$\{\delta_{v_q}\} \cup \{\delta_{\vartheta \otimes v_q}\} $} \markbox{5309}{of} \markbox{m5310}{$\hom (H_*^{\times};\mathbb{Q})$} satisfying:
	%\[\{v_i\} \cup \{\vartheta \otimes v_i\} \subseteq H_*^\times, \qquad 	\{\delta_{v_i}\} \cup \{\delta_{\vartheta \otimes v_i}\} \subseteq H^*_\times.\]
	%Since $\delta_{U \otimes 1} = \delta_U = u \in H^m_{\mathbb S}$, we write $\delta_{U \otimes v_i}$ simply as $u\,\delta_{v_i}$.With respect to these bases, the Kronecker pairing  $\langle\,\cdot\,,\cdot\,\rangle : H^*_\times \times H_*^\times \longrightarrow \mathbb Q$satisfies 
	\begin{equation}\label{Kronecker relations}
		\delta_{v_q}( v_p) = \delta_{qp}, \quad
		\delta_{v_q}( \vartheta \otimes v_p)= 0, \quad
		\delta_{\vartheta\otimes v_q}(v_p) = 0, \quad
		\delta_{\vartheta\otimes v_q}(\vartheta \otimes v_p)= \delta_{qp}.
	\end{equation}
	%since $\langle u, \vartheta\rangle=1$ and $u$ (resp. $\vartheta$) kills the other basis elements.Thus the matrix of the pairing in these bases is the identity, hence the  map$\kappa_i: H^i_\times \to \operatorname{Hom}(H_i^\times,\mathbb Q)$, $\kappa_i(\varphi)(x)=\langle \varphi, x\rangle$, is an isomorphism for all $i$. 
	Let \markbox{m5311}{$f$} \markbox{5312}{be} \markbox{5313}{a} \markbox{5314}{continuous} \markbox{5315}{function} \markbox{5316}{on} \markbox{m5317}{$P(m,n,k)$}. \markbox{5318}{Using} \remref{lift} \markbox{5319}{and} \markbox{5320}{the} \markbox{5321}{Universal} \markbox{5322}{Coefficient} Theorem, \markbox{5323}{there} \markbox{5324}{exist} \markbox{5325}{a} \markbox{5326}{lift} \markbox{m5327}{$\tilde{f}$} \markbox{5328}{on} \markbox{m5329}{$\mathbb{S}^m\times \mathbb{C}G_{n,k}$} \markbox{5330}{satisfying} \begin{equation}\label{comm with phi}
		\varphi \circ \tilde{f}_* = \tilde{f}^*(\varphi) , \, \forall
		\varphi \in \hom _{\mathbb Q}\big(H_{2i}^{\mathbb CG}, \mathbb Q\big)\cong H^{2i}_{\mathbb CG}.
	\end{equation}
	
		Poincar\'e \markbox{5331}{duality} \markbox{5332}{on} \markbox{m5333}{$\mathbb S^{m}\times \mathbb C G_{n,k}$} \markbox{5334}{can} \markbox{5335}{be} \markbox{5336}{described} \markbox{5337}{in} \markbox{5338}{terms} \markbox{5339}{of} \markbox{5340}{the} \markbox{5341}{duality} \markbox{5342}{on} \markbox{5343}{the} \markbox{5344}{Grassmannian} factor. \markbox{5345}{Let} 
		\markbox{m5346}{$D_{\mathbb C G}\colon H^{i}_{\mathbb C G}\to H_{2d-i}^{\mathbb C G}$} 
		be \markbox{5347}{the} Poincar\'e \markbox{5348}{duality} \markbox{5349}{isomorphism} \markbox{5350}{defined} \markbox{5351}{in} \eqref{poinc dua} \markbox{5352}{for} \markbox{m5353}{$\mathbb C G_{n,k}$}, \markbox{5354}{where} \markbox{m5355}{$d=k(n-k)$}.  
		The Poincar\'e \markbox{5356}{duality} \markbox{5357}{isomorphism} \markbox{5358}{on} \markbox{5359}{the} product
		is \markbox{5360}{then} \markbox{5361}{determined} \markbox{5362}{on} \markbox{5363}{the} \markbox{5364}{basis} \markbox{5365}{elements} \markbox{5366}{by}  
		\begin{equation}
			D\colon H^{j}_{\times}\to H_{m+2d-j}^{\times}, \quad \delta_{v_i}\mapsto \vartheta\otimes D_{\mathbb C G}(\delta_{v_i})\quad\text{and}\quad \delta_{\vartheta\otimes v_i} \mapsto D_{\mathbb C G}(\delta_{v_i}).
		\end{equation}
		We \markbox{5367}{are} \markbox{5368}{now} \markbox{5369}{ready} \markbox{5370}{to} \markbox{5371}{establish} \markbox{5372}{the} \markbox{5373}{following} lemmas, \markbox{5374}{which} \markbox{5375}{will} \markbox{5376}{be} \markbox{5377}{useful} \markbox{5378}{in} \markbox{5379}{the} sequel.
	\begin{lemma}\label{image of homf}
		Let \markbox{m5380}{$f$} \markbox{5381}{be} \markbox{5382}{a} \markbox{5383}{continuous} \markbox{5384}{function} \markbox{5385}{on} \markbox{m5386}{$P(m,n,k)$} \markbox{5387}{and} \markbox{m5388}{$\tilde{f}$} \markbox{5389}{be} \markbox{5390}{the} \markbox{5391}{lift} \markbox{5392}{defined} \markbox{5393}{in} \remref{lift} \markbox{5394}{such} \markbox{5395}{that} \markbox{m5396}{$\tilde{f}^*(c_1)\neq au,\, a\in \mathbb{Q}$} \markbox{5397}{and} \markbox{m5398}{$k<n-k$}. \markbox{5399}{Then} \markbox{5400}{there} \markbox{5401}{exist} \markbox{m5402}{$\lambda \in \mathbb{Q}\backslash \{0\}$} \markbox{5403}{and} \markbox{m5404}{$\mu \in \mathbb{Q}$} \markbox{5405}{such} \markbox{5406}{that} \markbox{5407}{the} \markbox{5408}{induced} \markbox{5409}{map} \markbox{m5410}{$\tilde{f}_*$} \markbox{5411}{on} \markbox{m5412}{$H_*^{\times}$} \markbox{5413}{is} \markbox{5414}{of} \markbox{5415}{the} \markbox{5416}{following} form.
		\begin{enumerate}
			\item Either \markbox{m5417}{$\tilde{f}_*(\vartheta\otimes x) = \mu \lambda^i (\vartheta \otimes x),\, \forall x\in H_{2i}^{\mathbb{C}G}$} \markbox{5418}{or} \markbox{m5419}{$\tilde{f}_*(\vartheta\otimes x) \in H_*^{\mathbb{CG}},\, \forall x \in H_*^{\mathbb{C}G}$}.
			\item \markbox{m5420}{$\tilde{f}_*(x) = \lambda^i x + \vartheta\otimes y,$} \markbox{5421}{for} \markbox{5422}{some} \markbox{m5423}{$y \in H_{2i-m}^{\mathbb{C}G}, \, \forall x \in H_{2i}^{\mathbb{C}G}.$}
		\end{enumerate}
		Moreover, \markbox{m5424}{$y=0$} \markbox{5425}{in} \textit{(2)} \markbox{5426}{if} \markbox{m5427}{$\tilde{f}_*(\vartheta\otimes x) = \mu \lambda^i (\vartheta \otimes x),\, \forall x\in H_{2i}^{\mathbb{C}G}$}.
	\end{lemma}
	\begin{proof}
		Using \thmref{main thm}, \markbox{5428}{there} \markbox{5429}{exist} \markbox{m5430}{$\lambda \in \mathbb{Q}\backslash \{0\}$} \markbox{5431}{such} \markbox{5432}{that} \markbox{m5433}{$\tilde{f}^*(c_i) = \lambda^i c_i, \forall i \in I$} \markbox{5434}{and} \markbox{5435}{either} \markbox{m5436}{$\tilde{f}^*(u) = \mu u,\, \mu \in \mathbb{Q}$} \markbox{5437}{or} \markbox{m5438}{$\tilde{f}^*(u)\in H^*_{\mathbb{C}G}$}. \markbox{5439}{It} \markbox{5440}{is} \markbox{5441}{sufficient} \markbox{5442}{to} \markbox{5443}{prove} \markbox{5444}{the} \markbox{5445}{result} \markbox{5446}{for} \markbox{5447}{the} \markbox{5448}{chosen} \markbox{5449}{basis} \markbox{m5450}{$\{v_q\}\cup\{\vartheta\otimes v_q\}$} \markbox{5451}{of} \markbox{m5452}{$H_{*}^{\times}$}. 
		
		Let \markbox{5453}{us} \markbox{5454}{consider} \markbox{5455}{the} \markbox{5456}{first} \markbox{5457}{case} \markbox{5458}{where} \markbox{m5459}{$\tilde{f}^*(u) = \mu u$}. \markbox{5460}{Using} \markbox{m5461}{$H^*_{\mathbb{C}G}\cong \hom(H_*^{\mathbb{C}G},\mathbb{Q})$}, \markbox{5462}{we} \markbox{5463}{have} \begin{equation}\label{fstarco}
			\tilde{f}^*(\delta_{v_p}) = \lambda^i \delta_{v_{p}}, \, \forall v_{p}\in H_{2i}^{\mathbb{C}G}, \quad \tilde{f}^*(\delta_{\vartheta\otimes v_{p}})  = \mu \lambda^i (\delta_{\vartheta \otimes v_{p}}), \, \forall v_{p}\in H_{2i}^{\mathbb{C}G}.
		\end{equation}
		If \markbox{m5464}{$m$} \markbox{5465}{is} odd, \markbox{5466}{then} \markbox{5467}{the} \markbox{5468}{coefficient} \markbox{5469}{of} \markbox{5470}{any} \markbox{5471}{basis} \markbox{5472}{element} \markbox{m5473}{$v_p \in H_*^{\mathbb{C}G}$} \markbox{5474}{in} \markbox{m5475}{$\tilde{f}_*(\vartheta \otimes v_q)$} \markbox{5476}{and} \markbox{m5477}{$\vartheta \otimes v_p$} \markbox{5478}{in} \markbox{m5479}{$\tilde{f}_*(v_q)$} \markbox{5480}{is} \markbox{5481}{zero} \markbox{5482}{because} \markbox{m5483}{$\tilde{f}_*$} \markbox{5484}{is} \markbox{5485}{a} \markbox{5486}{graded} map. \markbox{5487}{Let} \markbox{5488}{us} \markbox{5489}{consider} \markbox{5490}{the} \markbox{5491}{case} \markbox{5492}{where} \markbox{m5493}{$m=2s$}.
		By \eqref{comm with phi} \markbox{5494}{and} \eqref{fstarco}, \markbox{5495}{the} \markbox{5496}{coefficient} \markbox{5497}{of} \markbox{5498}{a} \markbox{5499}{basis} \markbox{5500}{element} \markbox{m5501}{$v_p\in H_{2i+m}^{\mathbb{C}G}$} \markbox{5502}{in} \markbox{m5503}{$\tilde{f}_*(\vartheta \otimes v_q)$} \markbox{5504}{written} \markbox{5505}{as} \markbox{5506}{a} \markbox{m5507}{$\mathbb{Q}$}-linear \markbox{5508}{combination} \markbox{5509}{of} \markbox{5510}{the} \markbox{5511}{basis} \markbox{5512}{elements} \markbox{5513}{from} \markbox{m5514}{$\{v_q\}\cup\{\vartheta\otimes v_q\}$} \markbox{5515}{is} \markbox{5516}{the} following:
		$$	 \delta_{v_p}\circ \tilde f_*(\vartheta\otimes v_q)
		=  \tilde f^* (\delta_{v_p}) (\vartheta\otimes v_q)
		=  \lambda^{i+s} \delta_{v_p}(\vartheta\otimes v_q)
		= 0,\, \forall v_q \in H_{2i}^{\mathbb{C}G}$$
		and \markbox{5517}{the} \markbox{5518}{coefficient} \markbox{5519}{of} \markbox{5520}{a} \markbox{5521}{basis} \markbox{5522}{element} \markbox{m5523}{$\vartheta \otimes v_p\in \vartheta \otimes H_{2i}^{\mathbb{C}G}$} \markbox{5524}{in} \markbox{m5525}{$\tilde{f}_*(\vartheta \otimes v_q)$} is
		$$ \delta_{\vartheta\otimes v_p}\circ \tilde f_*(\vartheta\otimes v_q)
		=  \tilde f^*(\delta_{\vartheta\otimes v_p})(\vartheta\otimes v_q)
		=  \mu \lambda^{i}\;\delta_{\vartheta \otimes v_p}(\vartheta\otimes v_q)= \mu\lambda^{i}\delta_{pq},\, \forall v_q \in H^{\mathbb{C}G}_{2i}.$$
		This \markbox{5526}{implies} \markbox{5527}{that} $$\tilde{f}_*(\vartheta \otimes v_q) = \mu \lambda^i (\vartheta \otimes v_q), \, \forall v_q \in H_{2i}^{\mathbb{C}G}.$$ 
		Using \markbox{5528}{similar} \markbox{5529}{calculations} \markbox{5530}{given} above, \markbox{5531}{it} \markbox{5532}{is} \markbox{5533}{easy} \markbox{5534}{to} \markbox{5535}{show} \markbox{5536}{that} $$\delta_{v_p}\circ \tilde{f}_*(v_q) = \lambda^i \delta_{pq},\, \forall v_q \in H_{2i}^{\mathbb{C}G},\quad \delta_{\vartheta \otimes v_p}\circ \tilde{f}_*(v_q)=0,\, \forall v_q\in H_{2i}^{\mathbb{C}G}.$$ Therefore, \markbox{m5537}{$\tilde{f}_*(v_q) = \lambda^i v_q,\, \forall v_q \in H_{2i}^{\mathbb{C}G}$}.\\
		
		If \markbox{m5538}{$\tilde{f}^*(u)\in H^*_{\mathbb{C}G}$}. \markbox{5539}{Again} \markbox{5540}{using} \markbox{m5541}{$H^*_{\mathbb{C}G}\cong \hom(H_*^{\mathbb{C}G},\mathbb{Q})$}, \markbox{5542}{we} \markbox{5543}{have} \begin{equation}\label{second u}
			\tilde{f}^*(\delta_{v_p}) = \lambda^i \delta_{v_{p}}, \, \forall v_{p}\in H_{2i}^{\mathbb{C}G}, \quad \tilde{f}^*(\delta_{\vartheta\otimes v_{p}})  \in H^*_{\mathbb{C}G},\, \forall  v_{p}\in H_{2i}^{\mathbb{C}G}.
		\end{equation}
		By \eqref{comm with phi} \markbox{5544}{and} \eqref{second u}, \markbox{5545}{we} \markbox{5546}{get} \markbox{m5547}{$\delta_{v_p}\circ \tilde{f}_*(v_q) = \lambda^i \delta_{pq},\, \forall v_q \in H_{2i}^{\mathbb{C}G},$} \markbox{5548}{which} \markbox{5549}{implies} \markbox{5550}{that} \markbox{m5551}{$\tilde{f}_*(x) = \lambda^i x + \vartheta\otimes y,$} \markbox{5552}{for} \markbox{5553}{some} \markbox{m5554}{$y \in H_{2i-m}^{\mathbb{C}G}, \, \forall x \in H_{2i}^{\mathbb{C}G}.$} \markbox{5555}{Note} \markbox{5556}{that} \markbox{m5557}{$\tilde f^*(\delta_{\vartheta\otimes v_p})\in H_{\mathbb CG}^*$} \markbox{5558}{and} \markbox{5559}{equal} \markbox{5560}{to} \markbox{5561}{some} \markbox{m5562}{$\sum a_j\delta_{v_j}$}. \markbox{5563}{Then} 
		%coefficient of $\vartheta \otimes v_p $ in $\tilde f_*(\vartheta \otimes v_q)$ is 
		\begin{equation}\label{computation}
			\delta_{\vartheta\otimes v_p}\circ \tilde f_*(\vartheta\otimes v_q)= \tilde f^*(\delta_{\vartheta\otimes v_p})(\vartheta\otimes v_q)=\sum a_j\delta_{v_j}(\vartheta\otimes v_q)=0.
		\end{equation}
		Hence, \markbox{m5564}{$\tilde f_*(\vartheta \otimes v_q)\in H_*^{\mathbb CG}$} \markbox{5565}{for} \markbox{5566}{all} \markbox{m5567}{$\vartheta \otimes v_q\in\vartheta\otimes H_*^{\mathbb CG} $}.
		%	$ \text{ or } \tilde{f}^*(\vartheta\otimes v_{p}) \in H_*^{\mathbb{C}G},\, \forall v_{p}\in H_*^{\mathbb{C}G}$
	\end{proof}
\iffalse	\begin{remark}\label{lemrem}
		For \markbox{5568}{the} \markbox{5569}{case} \markbox{m5570}{$k=n-k$} \markbox{5571}{in} \lemref{image of homf}, \markbox{5572}{if} \markbox{m5573}{$\tilde{f}^*(c_i) = \lambda^i c_i$} \markbox{5574}{then} \markbox{5575}{we} \markbox{5576}{will} \markbox{5577}{get} \markbox{5578}{the} \markbox{5579}{same} \markbox{5580}{result} \markbox{5581}{otherwise} \markbox{5582}{when} \markbox{m5583}{$\tilde{f}^*(c_i) = (-\lambda)^i c_i$} \markbox{5584}{then} \markbox{m5585}{$\lambda$} \markbox{5586}{will} \markbox{5587}{be} \markbox{5588}{replaced} \markbox{5589}{by} \markbox{m5590}{$-\lambda$} \markbox{5591}{in} \lemref{image of homf}. 
	\end{remark}\fi
	\begin{lemma}\label{image of homf under hom}
		Assume \markbox{5592}{that} \markbox{5593}{the} \markbox{5594}{hypothesis} \eqref{Homer} \markbox{5595}{is} satisfied. 	Let \markbox{m5596}{$f$} \markbox{5597}{be} \markbox{5598}{a} \markbox{5599}{continuous} \markbox{5600}{function} \markbox{5601}{on} \markbox{m5602}{$P(m,n,k)$} \markbox{5603}{and} \markbox{m5604}{$\tilde{f}$} \markbox{5605}{be} \markbox{5606}{the} \markbox{5607}{lift} \markbox{5608}{defined} \markbox{5609}{in} \remref{lift} \markbox{5610}{such} \markbox{5611}{that} \markbox{m5612}{$\tilde{f}^*(c_1)= au,\, a\in \mathbb{Q}$}. \markbox{5613}{Then} \markbox{5614}{the} \markbox{5615}{induced} \markbox{5616}{map} \markbox{m5617}{$\tilde{f}_*$} \markbox{5618}{on} \markbox{m5619}{$H_{*}^{\times}$} \markbox{5620}{is} \markbox{5621}{of} \markbox{5622}{the} \markbox{5623}{following} form.
		\begin{enumerate}
			\item \markbox{m5624}{$\tilde f_* (x)\in \vartheta\otimes H_{2i-m}^{\mathbb CG},\, \forall x\in H_{2i}^{\mathbb CG},\, \forall i>0$}.
			\item \markbox{m5625}{$\tilde{f}_*(\vartheta\otimes 1) = \mu (\vartheta\otimes 1 )+y, \, y\in H_{m}^{\mathbb{C}G}, \quad\\ \tilde f_*(\vartheta \otimes x)\in H_{2i+m}^{\mathbb{C}G},\, \forall x \in H_{2i}^{\mathbb{C}G}, i>0\;$} \markbox{5626}{if} \markbox{m5627}{$\tilde{f}^*(u) = \mu u,\,  \mu \in \mathbb{Q}$} 
		\end{enumerate}
	\end{lemma}
	\begin{proof}
		Using \propref{main thm 2}, \markbox{5628}{we} \markbox{5629}{have} \markbox{m5630}{$\tilde{f}^*(c_i) = u P_i,$} \markbox{5631}{for} \markbox{5632}{some} \markbox{m5633}{$P_i\in H^*_{\mathbb{C}G}$} \markbox{5634}{and} \markbox{5635}{either} \markbox{m5636}{$\tilde{f}^*(u) = \mu u,\, \mu \in \mathbb{Q}$} \markbox{5637}{or} \markbox{m5638}{$\tilde{f}^*(u)\in H^*_{\mathbb{C}G}$}. \\
		Let \markbox{5639}{us} \markbox{5640}{consider} \markbox{5641}{the} \markbox{5642}{first} \markbox{5643}{case} \markbox{5644}{where} \markbox{m5645}{$\tilde{f}^*(u) = \mu u$}. \markbox{5646}{Using} \markbox{m5647}{$H^*_{\mathbb{C}G}\cong \hom(H_*^{\mathbb{C}G},\mathbb{Q})$}, \markbox{5648}{we} \markbox{5649}{have} \markbox{5650}{for} \markbox{m5651}{$i>0$} 
		\begin{equation}\label{fstarcom}
			\tilde{f}^*(\delta_{v_p}) = \sum a_{jp}\delta_{\vartheta \otimes v_j} , \, \forall v_{p}\in H_{2i}^{\mathbb{C}G}, \quad \tilde{f}^*(\delta_{\vartheta\otimes 1})  =\mu \delta_{\vartheta \otimes 1}, \quad \tilde{f}^*(\delta_{\vartheta\otimes v_{p}})  = 0, \, \forall v_{p}\in H_{2i}^{\mathbb{C}G}.
		\end{equation}
		Using \eqref{comm with phi}, \eqref{fstarcom} \markbox{5652}{and} \markbox{5653}{similar} \markbox{5654}{calculations} \markbox{5655}{given} \markbox{5656}{in} \markbox{5657}{the} \markbox{5658}{proof} \markbox{5659}{of} \lemref{image of homf}, \markbox{5660}{we} \markbox{5661}{have} \markbox{5662}{for} \markbox{m5663}{$ v_p\neq 1$}
		$$\delta_{v_p}\circ \tilde{f}_*(v_q) = 0,\, \forall v_q \in H_{2i}^{\mathbb{C}G},\quad \delta_{\vartheta\otimes v_p}\circ \tilde{f}_*(\vartheta\otimes v_q) =0,\, \forall v_q \in H_{2i}^{*}$$ \markbox{5664}{that} \markbox{5665}{concludes} \markbox{5666}{the} result.
		
		When \markbox{m5667}{$\tilde{f}^*(u)\in H^*_{\mathbb{C}G}$} \markbox{5668}{then} \markbox{5669}{also} \markbox{5670}{we} \markbox{5671}{have} \markbox{m5672}{$\tilde{f}^*(\delta_{v_p}) = \sum a_{jp}\delta_{\vartheta \otimes v_j} , \, \forall v_{p}\in H_{2i}^{\mathbb{C}G},\, \forall i>0$} \markbox{5673}{which} \markbox{5674}{implies} \markbox{5675}{that} \markbox{m5676}{$\delta_{v_p}\circ \tilde{f}_*(v_q) = 0,\, \forall v_q \in H_{2i}^{\mathbb{C}G},\,\forall i>0.$}   		
	\end{proof}
	\iffalse	Consequently, \markbox{5677}{for} \markbox{5678}{the} \markbox{5679}{lift} \markbox{m5680}{$\tilde f:\mathbb S^m\times \mathbb{C}G_{n,k}\to \mathbb S^m\times \mathbb{C}G_{n,k}$}, \markbox{5681}{the} \markbox{5682}{induced} \markbox{5683}{maps} 
	\markbox{m5684}{$\tilde f^*:H^i_\times\to H^i_\times$} \markbox{5685}{and} \markbox{m5686}{$\tilde f_*:H_i^\times\to H_i^\times$} \markbox{5687}{are} \markbox{5688}{adjoint} \markbox{5689}{with} \markbox{5690}{respect} \markbox{5691}{to} \markbox{5692}{this} \markbox{5693}{perfect} pairing, i.e.,   
	\[
	\langle \tilde f^* \varphi, x\rangle=\langle \varphi, \tilde f_* x\rangle \text{ for all }\varphi\in H^i_{\times},x\in H_i^\times.
	\]\fi
	
	%To determine $\tilde f_*(U)$ when (iii) holds, write 
	%$\tilde f_*(U) = aU + \sum b_i v_i \in H_m^{\mathbb{S}} \oplus H_m^{\mathbb{C}G}\cong H_m^\times$.  
	%Then 
	%$\langle u, \tilde f_* U \rangle = \langle u, aU \rangle + \langle u, \sum b_i v_i \rangle = a + 0 = a$.  
	%On the other hand, 
	%$\langle u, \tilde f_* U \rangle = \langle \tilde f^* u, U \rangle = \langle \mu u, U \rangle = \mu$, 
	%so $a = \mu$.  
	%Proceeding similarly replacing  $u=\delta_U$  by $\delta_{v_i}$ yields $b_i = 0$ for all $i$.  
	%Hence, $\tilde f_*(U) = \mu U$.
	
	\iffalse	\underline{When (i) and (iii) hold}, \markbox{5694}{for} \markbox{5695}{any} \markbox{5696}{basis} \markbox{5697}{element} \markbox{m5698}{$v_q \in H_{2i}^{\mathbb{C}G}$}, \markbox{5699}{the} \markbox{5700}{coefficient} \markbox{5701}{of} \markbox{m5702}{$v_p$} \markbox{5703}{in} \markbox{m5704}{$\tilde f_*(v_q)$}, , \markbox{5705}{is} 
	\begin{equation}\label{i,ii,1}
		\langle \delta_{v_p}, \tilde f_*v_q\rangle
		= \langle \tilde f^* \delta_{v_p}, v_q\rangle
		= \langle \lambda^i \delta_{v_p}, v_q\rangle
		= \lambda^i \delta_{pq},
	\end{equation}
	and \markbox{5706}{the} \markbox{5707}{coefficient} \markbox{5708}{of} \markbox{m5709}{$\vartheta\otimes v_p \in \vartheta\otimes H_{2i}^{\mathbb{C}G}$} \markbox{5710}{in} \markbox{m5711}{$\tilde f_*(v_q)$} \markbox{5712}{is} 
	\begin{equation}\label{1,iii,2}
		\langle \delta_{\vartheta\otimes v_p}, \tilde f_*v_q\rangle
		= \langle \tilde f^*(\delta_{\vartheta\otimes v_p}), v_q\rangle
		= \langle \mu\lambda^{i}\delta_{\vartheta\otimes v_p}, v_q\rangle
		= 0.
	\end{equation}
	Thus, \markbox{m5713}{$\tilde f_*(x) = \lambda^i x$}, \markbox{5714}{for} \markbox{5715}{all} \markbox{m5716}{$x \in H_{2i}^{\mathbb{C}G}$}. 
	Also, \markbox{5717}{the} \markbox{5718}{coefficient} \markbox{5719}{of} \markbox{m5720}{$v_p \in H_m^{\mathbb{C}G}$} \markbox{5721}{in} \markbox{m5722}{$\tilde f_*(\vartheta\otimes v_q)$} \markbox{5723}{is} 
	t \markbox{5724}{of} \markbox{m5725}{$\vartheta\otimes v_p \in \vartheta \otimes H_{2i}^{{\mathbb CG}}$} \markbox{5726}{in} \markbox{m5727}{$\tilde f_*(\vartheta\otimes v_q)$} \markbox{5728}{is} 
	This \markbox{5729}{shows} \markbox{5730}{that} \markbox{m5731}{$\tilde f_*(\vartheta \otimes x) = \mu \lambda^{i}\vartheta \otimes x$} \markbox{5732}{for} \markbox{5733}{all} \markbox{m5734}{$x\in \vartheta\otimes H_{2i}^{\mathbb CG}$}.\fi
	
	\iffalse	\underline{When (ii) and (iii) hold},  \markbox{5735}{the} \markbox{5736}{coefficient} \markbox{5737}{of} \markbox{m5738}{$v_p$} (\markbox{m5739}{$p\neq 0$}) \markbox{5740}{in} \markbox{m5741}{$\tilde f_*(v_q)$}, \markbox{5742}{if} \markbox{m5743}{$\tilde f^*\delta_{v_p}=\sum a_i\delta_{\vartheta \otimes v_i} $},  is
	\[
	\langle \delta_{v_p}, \tilde f_*v_q\rangle=\langle\tilde f^*\delta_{v_p},v_q\rangle=\langle \sum a_i\delta_{\vartheta \otimes v_i},v_q\rangle=\sum 
	a_i\langle  \delta_{\vartheta \otimes v_i},v_q\rangle=0,
	\]
	Thus, \markbox{m5744}{$\tilde f_* (x)\in \vartheta\otimes H_*^{\mathbb CG}$} \markbox{5745}{for} \markbox{5746}{all} \markbox{m5747}{$x\in H_*^{\mathbb CG}$}. Also, \markbox{5748}{the} \markbox{5749}{coefficient} \markbox{5750}{of} \markbox{m5751}{$\vartheta\otimes v_p$} \markbox{5752}{in} \markbox{m5753}{$\tilde f_*(\vartheta\otimes v_q)$} is
	\[
	\langle \delta_{\vartheta \otimes v_p},\tilde f_*(\vartheta\otimes v_q)\rangle=\langle\tilde f^*(\delta_{\vartheta \otimes v_p}), \vartheta \otimes v_q\rangle =
	\begin{cases}
		\langle \mu \delta_{\vartheta \otimes v_0}, \vartheta \otimes v_q\rangle=\mu\delta_{0q}& \text{if } p=0,\\
		\langle 0, \vartheta \otimes v_q\rangle =0& \text{if }p\neq 0.
	\end{cases}
	\]
	Therefore, \markbox{5754}{for} \markbox{5755}{all} \markbox{m5756}{$x\in H_{2i}^{\mathbb CG}$}, \markbox{m5757}{$\tilde f_*(\vartheta\otimes x)=\begin{cases}
		\mu \vartheta\otimes x +y \;\text{ for some }y\in H_{2i+m}^{\mathbb CG}  &\text{if } i=0 ,\\
		\quad \quad0 &\text{if }i>0.
	\end{cases}$} \fi
	
	\iffalse	\underline{When (i) and (iv) hold}, \markbox{5758}{the} \markbox{5759}{coefficient} \markbox{5760}{of} \markbox{m5761}{$v_p\in H_{2i}^{\mathbb CG}$} \markbox{5762}{in} \markbox{m5763}{$\tilde f_*(v_q)$} \markbox{5764}{is} 
	\[
	\langle \delta_{v_p},\tilde f_*(v_q)\rangle=\langle\tilde f^*\delta_{v_p},v_q\rangle=\langle \lambda^i\delta_{v_p},v_q\rangle=\lambda^i\delta_{pq}.
	\]
	This \markbox{5765}{implies} \markbox{5766}{that} \markbox{m5767}{$\tilde f_*(x)=\lambda^ix+\vartheta \otimes y$} \markbox{5768}{for} \markbox{5769}{some} \markbox{m5770}{$\vartheta \otimes y\in \vartheta \otimes H_{2i}^{\mathbb CG}$}, \markbox{5771}{for} \markbox{5772}{all} \markbox{m5773}{$x\in H_{2i}^{\mathbb CG}$}. \fi
	
	
	\iffalse	\underline{When (ii) and (iv) hold}, \markbox{5774}{we} \markbox{5775}{can} \markbox{5776}{write} \markbox{m5777}{$\tilde f^*(\delta_{v_p})=\sum a_j\delta_{\vartheta\otimes v_j}$} %and $\tilde f^*(\delta_{\vartheta\otimes v_p})=\sum b_j\delta_{\vartheta\otimes v_j}$
	for \markbox{5778}{some} \markbox{m5779}{$a_j\in \mathbb Q$} \markbox{5780}{and} \markbox{m5781}{$p\neq 0$}.
	Now \markbox{5782}{the} \markbox{5783}{coefficient} \markbox{5784}{of} \markbox{m5785}{$v_p$} \markbox{5786}{in} \markbox{m5787}{$\tilde f_*(v_q)$} is
	\[
	\langle \delta_{v_p},\tilde f_*v_q\rangle=\langle\tilde f^*\delta_{v_p}, v_q\rangle=\langle\sum a_j\delta_{\vartheta\otimes v_j}, v_q\rangle=0,
	\]
	implying \markbox{m5788}{$\tilde f_*(x)\in \vartheta\otimes H_*^{\mathbb CG}$} \markbox{5789}{for} \markbox{5790}{all} \markbox{m5791}{$x\in H_*^{\mathbb CG}$}. \fi
	\iffalse
	Also, \markbox{5792}{the} \markbox{5793}{coefficient} \markbox{5794}{of} \markbox{m5795}{$ v_p$} \markbox{5796}{in} \markbox{m5797}{$\tilde f_*(\vartheta\otimes v_q)$} is
	\[
	\langle \delta_{v_p},\tilde f_*(\vartheta\otimes v_q)\rangle=\langle\tilde f^*\delta_{v_p}, \vartheta\otimes v_q\rangle=\langle\sum b_j\delta_{v_j},\vartheta\otimes v_q\rangle=0.
	\]
	This \markbox{5798}{ensures} \markbox{5799}{that} \markbox{m5800}{$\tilde f_*(\vartheta\otimes x)\in H_*^{\mathbb CG}$} \markbox{5801}{for} \markbox{5802}{all} \markbox{m5803}{$\vartheta\otimes x\in \vartheta\otimes H_*^{\mathbb CG}$}.
	
	
	We \markbox{5804}{refer} \markbox{5805}{to} \markbox{5806}{the} \markbox{5807}{above} \markbox{5808}{observations} \markbox{5809}{from} \markbox{5810}{any} \markbox{5811}{combination} \markbox{5812}{of} \markbox{5813}{one} \markbox{5814}{condition} \markbox{5815}{from} 
	\markbox{m5816}{$(\mathrm{i}),(\mathrm{ii})$} \markbox{5817}{and} \markbox{5818}{one} \markbox{5819}{from} \markbox{m5820}{$(\mathrm{iii}),(\mathrm{iv})$}, \markbox{5821}{by} \markbox{5822}{the} \markbox{5823}{symbol} \markbox{m5824}{$\mathscr{X}$}.
	
	
	
	%Summarizing above, we have for all $x\in H_{2i}^{\mathbb CG}$,
	%\[
	%\tilde f_*(x)=
	%\begin{cases}
	%   \lambda^ix  &\text{ if (i) and (iii) hold},\\
	%    \lambda^ix +\vartheta\otimes y, \text{ for some }y\in H_*^{\mathbb CG}& \text{ if (i) and (iv) hold}.
	%^\end{cases}
%\]


\vspace{.2in}\fi

\iffalse
******

Thus, \markbox{5825}{for} \markbox{5826}{all} \markbox{m5827}{$x \in H^{\mathbb CG}_{2i}$}, \markbox{m5828}{$\varphi \in H^{2i}_{\mathbb CG}$}, \markbox{5829}{and} \markbox{m5830}{$y \in H^m_\times$}, \markbox{5831}{we} have
\begin{equation}\label{}
	\langle \varphi, \tilde f_* x\rangle =
	\begin{cases}
		\langle \varphi, \lambda_1^i x\rangle & \text{if (i) holds,} \\
		\langle u P_{\varphi}, x\rangle & \text{if (ii) holds},\\
	\end{cases}
	\quad \text{and} \quad
	\langle u, \tilde f_* y\rangle =
	\begin{cases}
		\langle u, \mu_1 y\rangle & \text{if (iii) holds,} \\
		& \text{if (iv) holds}.\\
	\end{cases}
\end{equation}

Therefore, \markbox{5832}{for} \markbox{5833}{all} \markbox{m5834}{$x\in H_{2i}^{\mathbb CG}$}, \markbox{m5835}{$\tilde f_*(x)=\lambda_1^ix$} \markbox{5836}{if} (i) holds, \markbox{m5837}{$\tilde f_*(x)\in [\mathbb S^m]\otimes H_{2i-m}^{\mathbb CG}$} \markbox{5838}{if} (ii) holds, \markbox{5839}{and} \markbox{m5840}{$\tilde f_*([\mathbb S^m])=\mu_1 [\mathbb S^m]$} \markbox{5841}{if} (iii) holds, \markbox{5842}{where} \markbox{m5843}{$[\mathbb S^m]\in H_m^{\mathbb S}$} \markbox{5844}{denotes} \markbox{5845}{the} \markbox{5846}{fundamental} \markbox{5847}{class} \markbox{5848}{of} \markbox{m5849}{$\mathbb S^m$} \markbox{5850}{satisfying} \markbox{m5851}{$\langle u,[\mathbb S^m]\rangle=1.$}
\fi


\subsection{} \markbox{5852}{The} \markbox{5853}{following} \markbox{5854}{theorems} \markbox{5855}{provide} \markbox{5856}{a} \markbox{5857}{criteria} \markbox{5858}{for} \markbox{5859}{the} \markbox{5860}{existence} \markbox{5861}{of} \markbox{5862}{coincidence} \markbox{5863}{points} \markbox{5864}{between} \markbox{5865}{a} \markbox{5866}{pair} \markbox{5867}{of} \markbox{5868}{continuous} \markbox{5869}{functions} \markbox{5870}{on} \markbox{m5871}{$P(m,n,k)$}.

\begin{theorem}\label{coincidence thm}
	Let \markbox{m5872}{$P(m,n,k)$} \markbox{5873}{be} \markbox{5874}{a} \markbox{5875}{generalized} \markbox{5876}{Dold} \markbox{5877}{manifold} \markbox{5878}{with} \markbox{m5879}{$k<n-k$} \markbox{5880}{and} \markbox{m5881}{$k(n-k)$} even. \markbox{5882}{Let} \markbox{m5883}{$f$} \markbox{5884}{and} \markbox{m5885}{$g$} \markbox{5886}{be} \markbox{5887}{two} \markbox{5888}{continuous} \markbox{5889}{maps} \markbox{5890}{on} \markbox{m5891}{$P(m,n,k)$} \markbox{5892}{and} \markbox{m5893}{$\tilde f, \tilde g$} \markbox{5894}{be} \markbox{5895}{their} \markbox{5896}{lifts} \markbox{5897}{as} \markbox{5898}{defined} \markbox{5899}{in}  \markbox{5900}{Remark} \ref{lift} \markbox{5901}{such} that
	%and suppose that the induced endomorphisms in cohomology satisfies:	
	\begin{enumerate}	
		\item \markbox{m5902}{$g^*$} \markbox{5903}{is} \markbox{5904}{an} \markbox{5905}{automorphism} \markbox{5906}{of} \markbox{m5907}{$H^*(P(m,n,k);\mathbb Q)$}. 
		\item \markbox{m5908}{$\tilde{f}^*(c_1) \neq au,\, a \in \mathbb{Q}$}.
		\item \markbox{m5909}{$\deg(p\circ g \circ s)\neq -\deg (p\circ f\circ s)$} \markbox{5910}{if} \markbox{m5911}{$m$} \markbox{5912}{is} odd.
	\end{enumerate}
	where \markbox{m5913}{$s$} \markbox{5914}{denotes} \markbox{5915}{a} \markbox{5916}{section} \markbox{5917}{of} \markbox{5918}{the} \markbox{m5919}{$X$}-bundle \markbox{5920}{projection} \markbox{m5921}{$p$} \markbox{5922}{defined} \markbox{5923}{in} \eqref{sectio} \markbox{5924}{and} \eqref{proj}.	Then, \markbox{5925}{there} \markbox{5926}{is} \markbox{5927}{a} \markbox{5928}{point} \markbox{5929}{of} \markbox{5930}{coincidence} \markbox{5931}{of} \markbox{m5932}{$f$} \markbox{5933}{and}  \markbox{m5934}{$g$}.
\end{theorem}
\begin{proof}
	Using \corref{automor}, \markbox{5935}{we} \markbox{5936}{have} \markbox{m5937}{$\tilde g^*$} \markbox{5938}{is} \markbox{5939}{an} \markbox{5940}{automorphism} \markbox{5941}{on} \markbox{m5942}{$H^*_{\times}$} \markbox{5943}{given} by
	\markbox{m5944}{$\tilde g^*(c_i) = \lambda_1^i c_i$}, \markbox{5945}{and} \markbox{m5946}{$\tilde g^*(u) = \mu_1 u$} \markbox{5947}{for} \markbox{5948}{some} \markbox{m5949}{$\lambda_1, \mu_1 \in \mathbb{Q}\backslash \{0\}$} \markbox{5950}{if} \markbox{m5951}{$k<n-k$}.
	
	Using \lemref{image of homf}, \markbox{5952}{there} \markbox{5953}{exist} \markbox{m5954}{$\lambda \in \mathbb{Q}\backslash \{0\}$} \markbox{5955}{and} \markbox{m5956}{$\mu \in \mathbb{Q}$} \markbox{5957}{such} \markbox{5958}{that} \markbox{m5959}{$\tilde{f}_*$} \markbox{5960}{is} \markbox{5961}{of} \markbox{5962}{the} \markbox{5963}{following} form, 
	\begin{equation}\label{flower star}
		\begin{split}
			\tilde f_*(x)=\lambda^ix+\vartheta\otimes y, \text{ for some }y\in H_{2i-m}^{\mathbb CG}, \, \forall x \in H_{2i}^{\mathbb{C}G}\\
			\tilde f_*(\vartheta\otimes x)=\mu\lambda^i (\vartheta\otimes x) , \text{ or } \tilde{f}_*(\vartheta \otimes x) = z, \text{ for some }z\in H_{2i+m}^{\mathbb CG},\, \forall x \in H_{2i}^{\mathbb{C}G}
		\end{split}
	\end{equation}
	To \markbox{5964}{prove} \markbox{5965}{that} \markbox{m5966}{$f$} \markbox{5967}{has} \markbox{5968}{a} \markbox{5969}{point} \markbox{5970}{of} \markbox{5971}{coincidence} \markbox{5972}{with} \markbox{m5973}{$g$}, \markbox{5974}{it} \markbox{5975}{is} \markbox{5976}{sufficient} \markbox{5977}{to} \markbox{5978}{prove} \markbox{5979}{that} \markbox{5980}{either} \markbox{m5981}{$\tilde{f}$} \markbox{5982}{or} \markbox{5983}{the} \markbox{5984}{composition} \markbox{m5985}{$\theta \circ \tilde{f} $} \markbox{5986}{has} \markbox{5987}{a} \markbox{5988}{point} \markbox{5989}{of} \markbox{5990}{coincidence} \markbox{5991}{with} \markbox{m5992}{$g$} \markbox{5993}{where} \markbox{m5994}{$\theta = \alpha \times \sigma$} \markbox{5995}{defined} \markbox{5996}{in} \secref{gds}. \markbox{5997}{By} \thmref{LCT}, \markbox{5998}{we} \markbox{5999}{need} \markbox{6000}{to} \markbox{6001}{compute} \markbox{m6002}{$L(\tilde f, \tilde g)$} \markbox{6003}{and} \markbox{m6004}{$L(\theta \circ \tilde f, \tilde g)$}.
	
	For \markbox{m6005}{$x\in H_{2i}^{\mathbb{C}G}$}, \markbox{6006}{we} \markbox{6007}{have} 
	\begin{equation*}\label{D cal}
		\begin{split}
			D\tilde g^* D^{-1} \tilde f_*(x)=\mu_1 \lambda^i\lambda_1^{d-i} x + \vartheta\otimes y'\text{ for some }y' \in H^{\mathbb CG}_{2i-m}\\
			D\tilde g^* D^{-1} \tilde f_*(\vartheta\otimes x)= \mu \lambda^i\lambda_1^{d-i}(\vartheta\otimes x)+z' \text{ for some }z'\in H_{2i+m}^{\mathbb CG}.
		\end{split}
	\end{equation*}
	where \markbox{m6008}{$z^{'} =0$} \markbox{6009}{or} \markbox{m6010}{$\mu =0$} \markbox{6011}{depending} \markbox{6012}{on} \markbox{6013}{the} \markbox{6014}{image} \markbox{6015}{of} \markbox{m6016}{$\tilde{f}_*(\vartheta \otimes x)$}.
	%Now we compute the Lefschetz number of the map $L(\tilde{f},\tilde g)$  and $L(\theta\circ \tilde f, \tilde g)$ under the consideration that $\lambda=0$ if (ii) holds and $\mu=0$ if (iv) holds. 
	Recall \markbox{6017}{that}  \markbox{m6018}{$d_{2i}$} \markbox{6019}{denote} \markbox{6020}{the} \markbox{6021}{dimension} \markbox{m6022}{$\dim H^{2i}_{\mathbb CG}$}. \markbox{6023}{The} \markbox{6024}{Lefschetz} \markbox{6025}{number}  \markbox{m6026}{$L(\tilde{f},\tilde g)$} is
	\begin{equation*}\label{L(f,g)}
		L(\tilde f,\tilde g) =(\mu_1 + \mu) \sum_{i=0}^{k(n-k)} d_{2i} \lambda^i\lambda_1^{d-i}.
	\end{equation*}
	Using \markbox{6027}{the} \lemref{sum neq 0} \markbox{6028}{and} \markbox{6029}{the} \markbox{6030}{fact} \markbox{6031}{that} \markbox{m6032}{$\lambda_1\neq 0$}, \markbox{6033}{the} sum
	\[
	\sum_{i=0}^{k(n-k)} d_{2i} \lambda^i\lambda_1^{d-i}=\lambda_1^d\sum_{i=0}^{k(n-k)} d_{2i} (\lambda/\lambda_1)^i\neq 0,
	\]
	Since \markbox{m6034}{$\tilde f\circ\theta=\theta\circ \tilde f$}, \markbox{6035}{it} \markbox{6036}{follows} \markbox{6037}{that} $$(\theta\circ \tilde  f )^*(c_i)= (-1)^i \tilde{f}^*(c_i), \forall i \in I, \quad (\theta \circ\tilde  f)^*(u)=\begin{cases}
		-\tilde  f^*(u), \text{ if } m \text{ is even,}\\
		\tilde  f^*(u), \text{ if } m \text{ is odd}.
	\end{cases}$$ 
%	If $m$ is odd, using $\deg(p\circ g \circ s)\neq -\deg (p\circ f\circ s)$ i.e. $\mu_1 \neq -\mu$, we have $L(\theta\circ\tilde f, \tilde g) = L(\tilde f, \tilde g)\neq 0$ and we have the result. \\
	If \markbox{m6038}{$m$} \markbox{6039}{is} even, \markbox{6040}{then}  
	\begin{equation*}\label{D with theta}
		\begin{split}
			D\tilde g^* D^{-1} (\theta\circ \tilde f)_*(x)=\mu_1(- \lambda)^i\lambda_1^{d-i} x + \vartheta\otimes y''\text{ for some }y'' \in H^{\mathbb CG}_{2i-m} \\
			D\tilde g^* D^{-1} (\theta\circ \tilde f)_*(\vartheta\otimes x)= -\mu (-\lambda)^i\lambda_1^{d-i}\vartheta\otimes x+z'' \text{ for some }z''\in H_{2i+m}^{\mathbb CG}.
		\end{split}
	\end{equation*}
	Thus,  \markbox{6041}{the} \markbox{6042}{Lefschetz} \markbox{6043}{number} is
	\begin{equation*}\label{L(theta f,g)}
		L(\theta \circ\tilde f,\tilde g) =(\mu_1 - \mu)\sum_{i=0}^{k(n-k)} d_{2i} (-\lambda)^i\lambda_1^{d-i}.
	\end{equation*}
	Also, \markbox{6044}{using}  \markbox{m6045}{$\mu_1\neq 0$} \markbox{6046}{and} \lemref{sum neq 0}  \markbox{6047}{it} \markbox{6048}{follows} \markbox{6049}{that} \markbox{6050}{that} \markbox{6051}{either} \markbox{m6052}{$L(\tilde f, \tilde g)$} \markbox{6053}{or} \markbox{m6054}{$L(\theta\circ \tilde f,\tilde g)$} \markbox{6055}{is} nonzero. 
	
	If \markbox{m6056}{$m$} \markbox{6057}{is} odd, \markbox{m6058}{$	L(\theta \circ\tilde f,\tilde g) =(\mu_1 + \mu)\sum_{i=0}^{k(n-k)} d_{2i} (-\lambda)^i\lambda_1^{d-i}.$} \markbox{6059}{Using}  \lemref{sum neq 0} \markbox{6060}{and} \markbox{m6061}{$\deg(p\circ g \circ s)\neq -\deg (p\circ f\circ s)$} \markbox{6062}{that} \markbox{6063}{is} \markbox{m6064}{$\mu_1 \neq -\mu$}, \markbox{6065}{we} \markbox{6066}{have} \markbox{6067}{both} \markbox{m6068}{$L(\tilde{f},\tilde{g})$} \markbox{6069}{and} \markbox{m6070}{$L(\theta \circ\tilde f,\tilde g)$} \markbox{6071}{are} nonzero. 
	This \markbox{6072}{ensures} \markbox{6073}{that} \markbox{6074}{there} \markbox{6075}{exist} \markbox{6076}{a} \markbox{6077}{point} \markbox{6078}{of} \markbox{6079}{conincidence} \markbox{6080}{between} \markbox{m6081}{$f $} \markbox{6082}{and} \markbox{m6083}{$g$}.
\end{proof}
%	Using  \remref{lemrem} in the case $k=n-k$, we need to replace $\lambda$ by $-\lambda$ in \eqref{flower star} if $\tilde{f}^*(c_i) = -\lambda^i c_i$. In the rest of the calculations, $\lambda$ will be replaced by $-\lambda$ and we get the same result.
\begin{theorem}\label{coincidence thm under hom}
	Let \markbox{m6084}{$P(m,n,k)$} \markbox{6085}{be} \markbox{6086}{a} \markbox{6087}{generalized} \markbox{6088}{Dold} \markbox{6089}{manifold} \markbox{6090}{with} \markbox{m6091}{$k(n-k)$} \markbox{6092}{even} \markbox{6093}{and} \markbox{6094}{assume} \markbox{6095}{that} \markbox{6096}{the} \markbox{6097}{hypothesis} \eqref{Homer} \markbox{6098}{is} satisfied. \markbox{6099}{Let} \markbox{m6100}{$g$} \markbox{6101}{and} \markbox{m6102}{$f$} \markbox{6103}{are} \markbox{6104}{two} \markbox{6105}{continuous} \markbox{6106}{maps} \markbox{6107}{on} \markbox{m6108}{$P(m,n,k)$} \markbox{6109}{and} \markbox{m6110}{$\tilde g, \tilde f$} \markbox{6111}{be} \markbox{6112}{their} \markbox{6113}{lifts} \markbox{6114}{as} \markbox{6115}{defined} \markbox{6116}{in}  \markbox{6117}{Remark} \ref{lift} \markbox{6118}{such} that
	%and suppose that the induced endomorphisms in cohomology satisfies:	
	\begin{enumerate}	
		\item \markbox{m6119}{$g^*$} \markbox{6120}{is} \markbox{6121}{an} \markbox{6122}{automorphism} \markbox{6123}{of} \markbox{m6124}{$H^*(P(m,n,k);\mathbb Q)$}. 
		\item \markbox{m6125}{$\tilde{f}^*(u) = \mu u,\, \mu \in \mathbb{Q}$} \markbox{6126}{if} \markbox{m6127}{$\tilde{f}^*(H^*_{\mathbb CG}) \nsubseteq H^*_{\mathbb CG}$} \markbox{6128}{and} \markbox{m6129}{$m$} \markbox{6130}{is} even.
		\item \markbox{m6131}{$\deg(p\circ g \circ s)\neq -\deg (p\circ f\circ s)$} \markbox{6132}{if} \markbox{m6133}{$m$} \markbox{6134}{is} odd.
	\end{enumerate}
	\markbox{m6135}{$s$} \markbox{6136}{denotes} \markbox{6137}{a} \markbox{6138}{section} \markbox{6139}{of} \markbox{6140}{the} \markbox{m6141}{$X$}-bundle \markbox{6142}{projection} \markbox{m6143}{$p$} \markbox{6144}{defined} \markbox{6145}{in} \eqref{sectio} \markbox{6146}{and} \eqref{proj}. Then, \markbox{6147}{there} \markbox{6148}{is} \markbox{6149}{a} \markbox{6150}{point} \markbox{6151}{of} \markbox{6152}{coincidence} \markbox{6153}{of} \markbox{m6154}{$f$} \markbox{6155}{and}  \markbox{m6156}{$g$}.
\end{theorem}
\begin{proof}
	If \markbox{m6157}{$\tilde{f}^*(c_1) \neq au, \, a\in \mathbb{Q}$} \markbox{6158}{then} \markbox{6159}{we} \markbox{6160}{have} \markbox{6161}{the} \markbox{6162}{result} \markbox{6163}{by} \thmref{coincidence thm}.\\ \markbox{6164}{Let} \markbox{6165}{us} \markbox{6166}{consider} \markbox{6167}{the} \markbox{6168}{other} \markbox{6169}{case} \markbox{6170}{when} \markbox{m6171}{$\tilde{f}^*(c_1) = au, \, a\in \mathbb{Q}$}, \markbox{6172}{using} \thmref{main thm 2} \markbox{6173}{we} \markbox{6174}{have} \markbox{m6175}{$\tilde{f}^*(c_i) = uP_i, \text{ for some } P_i\in H^{2i-m}_{\mathbb{C}G}.$} 
	
	If \markbox{m6176}{$P_i \neq 0$} \markbox{6177}{for} \markbox{6178}{some} \markbox{m6179}{$i$} \markbox{6180}{in} \markbox{m6181}{$I$} \markbox{6182}{then} \markbox{m6183}{$\tilde{f}^*(H^*_{\mathbb CG}) \nsubseteq H^*_{\mathbb CG}$}. \markbox{6184}{Since} \markbox{m6185}{$\tilde{f}^*$} \markbox{6186}{is} \markbox{6187}{graded} \markbox{6188}{and} \markbox{6189}{by} \textit{(2)} \markbox{6190}{we} \markbox{6191}{have} \markbox{m6192}{$\tilde{f}^*(u) = \mu u,\, \mu \in \mathbb{Q}$}. \markbox{6193}{Using} \lemref{image of homf under hom}, \markbox{m6194}{$\tilde{f}_*$} \markbox{6195}{is} \markbox{6196}{of} \markbox{6197}{the} \markbox{6198}{following} form, \begin{equation}\label{uin CG}
		\begin{split}
			\tilde f_*(x)= \vartheta\otimes y, \text{ for some }y\in H_{2i-m}^{\mathbb CG}, \, \forall x \in H_{2i}^{\mathbb{C}G}, \, i>0\\
			\tilde f_*(\vartheta\otimes x)=\mu(\vartheta\otimes x) +z, \text{ for some }z\in H_{2i+m}^{\mathbb CG},\, \forall x \in H_{2i}^{\mathbb{C}G}
		\end{split}
	\end{equation}
	where \markbox{m6199}{$\mu =0$} \markbox{6200}{if} \markbox{m6201}{$i>0$}. \markbox{6202}{By} \corref{automor}, \markbox{6203}{we} \markbox{6204}{have} \markbox{m6205}{$\tilde g^*$} \markbox{6206}{is} \markbox{6207}{an} \markbox{6208}{automorphism} \markbox{6209}{on} \markbox{m6210}{$H^*_{\times}$} \markbox{6211}{given} by
	\markbox{m6212}{$\tilde g^*(c_i) = \lambda_1^i c_i$}, \markbox{6213}{and} \markbox{m6214}{$\tilde g^*(u) = \mu_1 u$} \markbox{6215}{for} \markbox{6216}{some} \markbox{m6217}{$\lambda_1, \mu_1 \in \mathbb{Q}\backslash \{0\}$}. \markbox{6218}{Using} \thmref{LCT} \markbox{6219}{and} \markbox{6220}{the} \markbox{6221}{similar} \markbox{6222}{calculations} \markbox{6223}{as} \markbox{6224}{done} \markbox{6225}{in} \markbox{6226}{the} \markbox{6227}{proof} \markbox{6228}{of} \thmref{coincidence thm}, \markbox{6229}{we} \markbox{6230}{get} $$L(\tilde f,\tilde g) =(\mu_1 + \mu) d_0 \lambda_1^d, \quad L(\theta \circ\tilde f,\tilde g) =\begin{cases}
		(\mu_1 - \mu)d_{0} \lambda_1^{d},  \text{ if } m \text{ is even,}\\
		(\mu_1 +\mu)d_{0} \lambda_1^{d},  \text{ if } m \text{ is odd.}
	\end{cases}$$ \markbox{6231}{Using} \markbox{m6232}{$\lambda_1 \neq 0$} \markbox{6233}{and} \markbox{m6234}{$\mu_1 \neq 0$}, \markbox{6235}{either} \markbox{m6236}{$L(\tilde f,\tilde g) $} \markbox{6237}{or} \markbox{m6238}{$ L(\theta \circ\tilde f,\tilde g)$} \markbox{6239}{is} \markbox{6240}{non} \markbox{6241}{zero} \markbox{6242}{if} \markbox{m6243}{$m$} \markbox{6244}{is} even. \markbox{6245}{Using} \markbox{m6246}{$\deg(p\circ g \circ s)\neq -\deg (p\circ f\circ s)$} i.e. \markbox{m6247}{$\mu_1 \neq -\mu$} \markbox{6248}{we} \markbox{6249}{have} \markbox{m6250}{$L(\tilde f, \tilde g) = L(\theta \circ \tilde f, \tilde g) \neq 0$}.  Hence, \markbox{6251}{we} \markbox{6252}{get} \markbox{6253}{the} result.
	
	Let \markbox{6254}{us} \markbox{6255}{consider} \markbox{6256}{the} \markbox{6257}{case} \markbox{6258}{when} \markbox{m6259}{$P_i =0,\, \forall i \in I$}, \markbox{6260}{if} \markbox{m6261}{$\tilde{f}^*(u) = \mu u, \mu \in \mathbb{Q}$} \markbox{6262}{then} \markbox{6263}{the} \markbox{6264}{proof} \markbox{6265}{remains} \markbox{6266}{exactly} \markbox{6267}{the} \markbox{6268}{same} \markbox{6269}{as} \markbox{6270}{given} above. \markbox{6271}{We} \markbox{6272}{need} \markbox{6273}{to} \markbox{6274}{focus} \markbox{6275}{on} \markbox{6276}{the} \markbox{6277}{case} \markbox{6278}{when} \markbox{m6279}{$\tilde{f}^*(u)\in H^*_{\mathbb{C}G}$}. \markbox{6280}{Using} \lemref{image of homf under hom} \markbox{6281}{and} \eqref{computation}, \markbox{6282}{we} \markbox{6283}{have} $$\tilde{f}_*(x) = \vartheta \otimes y, \text{ for some }y\in H_{2i-m}^{\mathbb CG}, \, \forall x \in H_{2i}^{\mathbb{C}G}, \, i>0, \quad \tilde{f}_*(\vartheta \otimes x) \in H_*^{\mathbb{C}G}, \forall x\in  H_*^{\mathbb{C}G}.$$ \markbox{6284}{This} \markbox{6285}{is} \markbox{6286}{exactly} \markbox{6287}{the} \markbox{6288}{same} \markbox{6289}{if} \markbox{6290}{we} \markbox{6291}{take} \markbox{m6292}{$\mu =0$} \markbox{6293}{in} \eqref{uin CG}. \markbox{6294}{The} \markbox{6295}{rest} \markbox{6296}{of} \markbox{6297}{the} \markbox{6298}{calculations} \markbox{6299}{also} \markbox{6300}{remains} \markbox{6301}{the} \markbox{6302}{same} \markbox{6303}{and} \markbox{6304}{we} \markbox{6305}{get} \markbox{6306}{the} result.
\end{proof}



\begin{remark}
	There \markbox{6307}{are} \markbox{6308}{many} \markbox{6309}{situations} \markbox{6310}{when} \markbox{6311}{the} \markbox{6312}{map} \markbox{m6313}{$f$} \markbox{6314}{satisfies} \markbox{6315}{the} \markbox{6316}{required} \markbox{6317}{hypothesis} \textit{(2)} \markbox{6318}{considered} \markbox{6319}{in} \thmref{coincidence thm} \markbox{6320}{or} \thmref{coincidence thm under hom}. \markbox{6321}{Some} \markbox{6322}{of} \markbox{6323}{them} \markbox{6324}{are} \markbox{6325}{as} follows: 
	\begin{enumerate}
		\item The \markbox{6326}{lift} \markbox{m6327}{$\tilde f$} \markbox{6328}{stabilizes} \markbox{6329}{a} \markbox{6330}{copy} \markbox{6331}{of} Grassmannian, i.e., \markbox{m6332}{$\tilde f(\{x_0\}\times\mathbb CG_{n,k})\subseteq \{x_0\}\times\mathbb CG_{n,k}$} \markbox{6333}{for} \markbox{6334}{some} \markbox{m6335}{$x_0\in \mathbb S^m$}. 
		\item  The \markbox{6336}{map} \markbox{m6337}{$p_1\circ \tilde f^*\circ i_1: H^*_{\mathbb CG}\to H^*_{\mathbb C G}$} \markbox{6338}{is} \markbox{6339}{an} automorphism, equivalently, \markbox{m6340}{$f^*(c_1^2)=\lambda^2c_1^2,\, \lambda\in \mathbb Q\backslash \{0\},$} \markbox{6341}{where} \markbox{m6342}{$p_1$} \markbox{6343}{and} \markbox{m6344}{$i_1$} \markbox{6345}{are} \markbox{6346}{defined} \markbox{6347}{in} \eqref{comm diagram}.
		\item The \markbox{6348}{map} \markbox{m6349}{$p_2\circ \tilde f\circ i_1:\mathbb S^m\to \mathbb CG_{n,k}$} \markbox{6350}{is} \markbox{6351}{rationally} \markbox{6352}{null} homotopic, \markbox{6353}{where} \markbox{m6354}{$p_2$} \markbox{6355}{is} \markbox{6356}{the} \markbox{6357}{projection} \markbox{6358}{onto} \markbox{6359}{the} \markbox{6360}{second} \markbox{6361}{summand} \markbox{6362}{and} \markbox{m6363}{$i_1$} \markbox{6364}{is} \markbox{6365}{the} \markbox{6366}{inclusion} \markbox{6367}{into} \markbox{6368}{the} \markbox{6369}{first} summand. %$i_1:\mathbb S^m\hookrightarrow\mathbb S^m\times \mathbb CG_{n,k}$ be the inclusion into the first component and $p_2:\mathbb S^m\times \mathbb CG_{n,k}\to \mathbb CG_{n,k}$ be the projection onto the second component.
	\end{enumerate} 
\end{remark}



	Under \markbox{6370}{the} \markbox{6371}{assumption} \markbox{m6372}{$m>2k$}, \markbox{6373}{any} \markbox{6374}{continuous} \markbox{6375}{map} \markbox{m6376}{$f$} \markbox{6377}{on} \markbox{6378}{the} \markbox{6379}{generalized} \markbox{6380}{Dold} \markbox{6381}{space} \markbox{m6382}{$P(m,n,k)$}, \markbox{6383}{the} \markbox{6384}{lift} \markbox{m6385}{$\tilde f$} (from Remark~\ref{lift}) \markbox{6386}{satisfies} \markbox{m6387}{$\tilde f^{*}(c_i)=\lambda^ic_i$} \markbox{6388}{for} \markbox{6389}{all} \markbox{m6390}{$i\in I$}. \markbox{6391}{Hence} condition~\textit{(2)} \markbox{6392}{of} Theorem~\ref{coincidence thm under hom} \markbox{6393}{may} \markbox{6394}{be} omitted, \markbox{6395}{and} \markbox{6396}{one} \markbox{6397}{obtains} \markbox{6398}{the} \markbox{6399}{following} consequence.
	\begin{corollary}
		Let \markbox{m6400}{$P(m,n,k)$} \markbox{6401}{be} \markbox{6402}{a} \markbox{6403}{generalized} \markbox{6404}{Dold} \markbox{6405}{space} \markbox{6406}{with} \markbox{m6407}{$m$} \markbox{6408}{and} \markbox{m6409}{$k(n-k)$} \markbox{6410}{both} even. \markbox{6411}{Assume} \markbox{m6412}{$m>2k$}, \markbox{6413}{and} \markbox{6414}{the} \markbox{6415}{hypothesis} \eqref{Homer} \markbox{6416}{is} satisfied.
		Then, \markbox{6417}{for} \markbox{6418}{any} \markbox{6419}{continuous} \markbox{6420}{function} \markbox{m6421}{$g$} \markbox{6422}{on} \markbox{m6423}{$P(m,n,k)$} \markbox{6424}{that} \markbox{6425}{induces} \markbox{6426}{an} \markbox{6427}{automorphism} \markbox{6428}{on} \markbox{m6429}{$H^*(P(m,n,k);\mathbb{Q})$}, \markbox{6430}{the} \markbox{6431}{pair} \markbox{m6432}{$(P(m,n,k),g)$} \markbox{6433}{has} \markbox{6434}{the} \markbox{6435}{coincidence} property. \\ \markbox{6436}{In} particular, \markbox{6437}{for} \markbox{m6438}{$g=\mathrm{id}$}, \markbox{6439}{the} \markbox{6440}{space} \markbox{m6441}{$P(m,n,k)$} \markbox{6442}{has} \markbox{6443}{the} fixed-point property.
	\end{corollary}










In \thmref{coincidence thm under hom}, \markbox{6444}{the} \markbox{6445}{first} \markbox{6446}{assumption} \markbox{6447}{that} \markbox{m6448}{$g^*$} \markbox{6449}{is} \markbox{6450}{an} \markbox{6451}{automorphism} \markbox{6452}{of} \markbox{m6453}{$H^*(P(m,n,k); \mathbb{Q})$} \markbox{6454}{can} \markbox{6455}{be} \markbox{6456}{relaxed} \markbox{6457}{by} \markbox{6458}{assuming} \markbox{m6459}{$\mu$} \markbox{6460}{is} nonzero, \markbox{6461}{which} \markbox{6462}{leads} \markbox{6463}{to} \markbox{6464}{the} \markbox{6465}{following} proposition. 
\begin{proposition}
	Let \markbox{m6466}{$P(m,n,k)$} \markbox{6467}{be} \markbox{6468}{a} \markbox{6469}{generalized} \markbox{6470}{Dold} \markbox{6471}{manifold} \markbox{6472}{with} \markbox{m6473}{$k(n-k)$} \markbox{6474}{even} \markbox{6475}{and} \markbox{6476}{assume} \markbox{6477}{that} \markbox{6478}{the} \markbox{6479}{hypothesis} \eqref{Homer} \markbox{6480}{is} satisfied. \markbox{6481}{Let} \markbox{m6482}{$g$} \markbox{6483}{and} \markbox{m6484}{$f$} \markbox{6485}{are} \markbox{6486}{two} \markbox{6487}{continuous} \markbox{6488}{maps} \markbox{6489}{on} \markbox{m6490}{$P(m,n,k)$} \markbox{6491}{and} \markbox{m6492}{$\tilde g, \tilde f$} \markbox{6493}{be} \markbox{6494}{their} \markbox{6495}{lifts} \markbox{6496}{as} \markbox{6497}{defined} \markbox{6498}{in}  \markbox{6499}{Remark} \ref{lift} \markbox{6500}{such} that
	\begin{enumerate}
		\item \markbox{m6501}{$\tilde g^*(H^*_{\mathbb CG})= H^*_{\mathbb CG}$}.
		\item \markbox{m6502}{$\tilde f^*(u)=\mu u,\, \mu\in \mathbb Q\backslash \{0\}$}
	\end{enumerate}
Then, \markbox{6503}{there} \markbox{6504}{is} \markbox{6505}{a} \markbox{6506}{point} \markbox{6507}{of} \markbox{6508}{coincidence} \markbox{6509}{of} \markbox{m6510}{$f$} \markbox{6511}{and}  \markbox{m6512}{$g$}.
\end{proposition}
The \markbox{6513}{proof} \markbox{6514}{of} \markbox{6515}{the} \markbox{6516}{above} \markbox{6517}{proposition} \markbox{6518}{is} \markbox{6519}{similar} \markbox{6520}{to} \markbox{6521}{the} \markbox{6522}{proof} \markbox{6523}{of} \thmref{coincidence thm under hom}. Therefore, \markbox{6524}{we} \markbox{6525}{omit} \markbox{6526}{the} details.

\iffalse \begin{proposition}
	Let \markbox{m6527}{$P(m,n,k)$} \markbox{6528}{be} \markbox{6529}{a} \markbox{6530}{generalized} \markbox{6531}{Dold} manifold. \markbox{6532}{Consider} \markbox{6533}{two} \markbox{6534}{continuous} \markbox{6535}{maps} \markbox{m6536}{$f,g$} \markbox{6537}{on} \markbox{m6538}{$P(m,n,k)$} \markbox{6539}{and} \markbox{m6540}{$\tilde f, \tilde g$} \markbox{6541}{be} \markbox{6542}{their} \markbox{6543}{lifts} \markbox{6544}{as} \markbox{6545}{defined} \markbox{6546}{in}  \markbox{6547}{Remark} \ref{lift}.
	Suppose \markbox{6548}{that} \markbox{6549}{the} \markbox{6550}{induced} \markbox{6551}{endomorphisms} \markbox{6552}{in} \markbox{6553}{cohomology} satisfies:
	\begin{enumerate}
		\item \markbox{m6554}{$\tilde g^*(c_i)\in uH^*_{\mathbb CG},\forall i\in I$} \markbox{6555}{and} \markbox{m6556}{$\tilde g^*(u)=\mu_1u$} \markbox{6557}{for} \markbox{6558}{some} \markbox{m6559}{$\mu_1\in \mathbb Q$},
		\item \markbox{m6560}{$\tilde f$} \markbox{6561}{has} \markbox{6562}{nonzero} Brouwer-degree.
	\end{enumerate}
	Then, \markbox{6563}{there} \markbox{6564}{is} \markbox{6565}{a} \markbox{6566}{point} \markbox{6567}{of} \markbox{6568}{coincidence} \markbox{6569}{of} \markbox{m6570}{$f$} \markbox{6571}{and}  \markbox{m6572}{$g$}.
\end{proposition}
\fi




\iffalse If \markbox{6573}{we} \markbox{6574}{assume} \markbox{m6575}{$m> 2k$}, \markbox{6576}{then} \markbox{6577}{we} \markbox{6578}{have} \markbox{6579}{more} \markbox{6580}{information} \markbox{6581}{about} \markbox{m6582}{$\tilde f^*$} \markbox{6583}{and} \markbox{6584}{in} \markbox{6585}{that} \markbox{6586}{case} \markbox{6587}{we} don't \markbox{6588}{have} \markbox{6589}{to} \markbox{6590}{assume} \markbox{6591}{hypothesis} \textit{(2)}.

We \markbox{6592}{need} \markbox{6593}{to} \markbox{6594}{talk} \markbox{6595}{about} \markbox{6596}{the} \markbox{6597}{similar} \markbox{6598}{situations} \markbox{6599}{when} \markbox{6600}{we} \markbox{6601}{assume} \markbox{6602}{Hoffman} \markbox{6603}{instead} \markbox{6604}{of} Glover-Homer.

We \markbox{6605}{should} \markbox{6606}{be} \markbox{6607}{able} \markbox{6608}{to} \markbox{6609}{find} \markbox{6610}{some} \markbox{6611}{GDS} \markbox{m6612}{$P(m,n,k)$} \markbox{6613}{which} \markbox{6614}{has} FPP.

What \markbox{6615}{happens} \markbox{6616}{if} \markbox{6617}{we} \markbox{6618}{allow} \markbox{m6619}{$m$} \markbox{6620}{to} \markbox{6621}{be} \markbox{6622}{odd} \markbox{6623}{in} \markbox{m6624}{$P(m,n,k)$}?
\fi

\section*{Acknowledgements}
Part \markbox{6625}{of} \markbox{6626}{this} \markbox{6627}{work} \markbox{6628}{was} \markbox{6629}{carried} \markbox{6630}{out} \markbox{6631}{while} \markbox{6632}{the} \markbox{6633}{first} \markbox{6634}{author} \markbox{6635}{was} \markbox{6636}{a} \markbox{6637}{postdoctoral} \markbox{6638}{fellow} \markbox{6639}{at} \markbox{6640}{IISER} Berhampur, \markbox{6641}{which} \markbox{6642}{the} \markbox{6643}{author} \markbox{6644}{gratefully} acknowledges.







%\section*{References}

\begin{thebibliography}{99}
	
	%\bibitem[A]{adams} Adams, J. F. Vector fields on spheres.
	%Ann. of Math. {\bf 75} (1962), 603--632.
	%\bibitem[ABS]{ABS} Atiyah, M. F.; Bott, R., Shapiro, A. Clifford modules. Topology {\bf 3} (1964), no. suppl, 3--38.
	%\bibitem[AH]{atiyah-hirzebruch} Atiyah, M. F.; Hirzebruch, F.
	%Vector bundles and homogeneous spaces.Proc. Sympos. Pure Math., Vol. III, 7--38
	%American Mathematical Society, Providence, RI, 1961.
	
	%\bibitem[B]{borel} Borel, A. Sur la cohomologie des espaces fibr{\'e}s principaux et des espaces homogènes de groupes de Lie compacts. Ann. of Math, Vol. {\bf 57} (1953) 115--207.
	
	\bibitem[B]{brewster} S. Brewster, \textit{Automorphisms of the cohomology ring of finite Grassmann manifolds}, \markbox{6645}{Thesis} (Ph.D.)---The \markbox{6646}{Ohio} \markbox{6647}{State} University, \markbox{6648}{ProQuest} LLC, \markbox{6649}{Ann} Arbor, \markbox{6650}{MI} (1978), 102 pp. \url{http://gateway.proquest.com/openurl?url_ver=Z39.88-2004&rft_val_fmt=info:ofi/fmt:kev:mtx:dissertation&res_dat=xri:pqdiss&rft_dat=xri:pqdiss:7908118}
	
	\bibitem[BH]{brewster-homer} S. \markbox{6651}{Brewster} \markbox{6652}{and} W. Homer, \textit{Rational automorphisms of Grassmann manifolds}, Proc. Amer. Math. Soc. \textbf{88} (1983), no. 1, 181--183, \url{https://doi.org/10.2307/2045137}.
	
	%\bibitem[B2]{borel} Borel, A. Cohomologie mod $2$
	% de certains espaces homogènes.
	%Comment. Math. Helv. {\bf 27} (1953)165--197 (1953).
	%\bibitem[B3]{borel-lag} Borel, A. Linear algebraic groups. Springer-Verlag, New York.
	%\bibitem[CF]{conner-floyd} Conner, P. E.; Floyd, E. E. {\it Differentiable periodic maps.} Ergebnisse der Mathematik...
	%\bibitem[D]{davis} Davis, D. Projective product spaces. J. Topol. {\bf 3} (2010), 265--279.
	%\bibitem[DJ]{dj} Davis, M.; Januszkiewicz, T. Convex polytopes...
	
	\bibitem[Do]{dold} A. Dold, \textit{Erzeugende der Thomschen Algebra }\markbox{m6653}{$\mathfrak{N}$}, Math. Z. \textbf{65} (1956), 25--35, \url{https://doi.org/10.1007/BF01473868}.

    \bibitem[Du]{duan} H. Duan, \textit{Self-maps of the Grassmannian of complex structures}, \markbox{6654}{Compositio} Math. \textbf{132} (2002), no. 2, 159--175, \url{https://doi.org/10.1023/A:1015885227445}.
	
	\bibitem[DF]{duan-fang} H. \markbox{6655}{Duan} \markbox{6656}{and} L. Fang, \textit{Homology rigidity of Grassmannians}, \markbox{6657}{Acta} Math. Sci. Ser. \markbox{6658}{B} \textbf{29} (2009), no. 3, 697--704, \url{https://doi.org/10.1016/S0252-9602(09)60065-5}.

    \bibitem[DZ]{duan-zhao} H. \markbox{6659}{Duan} \markbox{6660}{and} X. Zhao, \textit{The classification of cohomology endomorphisms of certain flag manifolds}, \markbox{6661}{Pacific} J. Math. \textbf{192} (2000), no. 1, 93--102, \url{https://doi.org/10.2140/pjm.2000.192.93}.
	%\bibitem[F1]{fujii-66} Fujii, M. $K_U$-groups of Dold manifolds...
	%\bibitem[F2]{fujii-69} Fujii, M. Ring structures...
	
	%\bibitem[F]{fulton} Fulton, W. *Introduction to Toric Varieties*...
	%\bibitem[K]{karoubi} Karoubi, M. *K-theory*...
	%\bibitem[HM]{hm} Hattori, A.; Masuda, M...
	%\bibitem[H]{husemoller} Husemoller, D. *Fibre bundles*...
	
	\bibitem[GH1]{glover-homer} H. \markbox{6662}{Glover} \markbox{6663}{and} B. Homer, \textit{Endomorphisms of the cohomology ring of finite Grassmann manifolds}, \markbox{6664}{Geometric} \markbox{6665}{applications} \markbox{6666}{of} \markbox{6667}{homotopy} \markbox{6668}{theory} (Proc. Conf., Evanston, Ill., 1977), I, pp. 170--193.
	

    \bibitem[GH2]{glover-homer coin} H. \markbox{6669}{Glover} \markbox{6670}{and} W. Homer, \textit{Fixed points on flag manifolds}, \markbox{6671}{Pacific} J. Math. \textbf{101} (1982), no. 2, 303--306, \url{http://projecteuclid.org/euclid.pjm/1102724776}.
	
	\bibitem[GS]{goswami-sarkar} A. \markbox{6672}{Goswami} \markbox{6673}{and} S. Sarkar, \textit{Endomorphisms of the Cohomology Algebra of Certain Homogeneous Spaces}, \url{https://arxiv.org/abs/2509.09363}

	\bibitem[Ho1]{hoffman} M. Hoffman, \textit{Endomorphisms of the cohomology of complex Grassmannians}, Trans. Amer. Math. Soc. \textbf{281} (1984), no. 2, 745--760, \url{https://doi.org/10.2307/2000083}.
    
    \bibitem[Ho2]{hoffman-noncoin} M. Hoffman, \markbox{6674}{Noncoincidence} \markbox{6675}{index} \markbox{6676}{of} manifolds, \markbox{6677}{Pacific} J. Math. \textbf{115}(2) (1984), 373--383, \url{http://projecteuclid.org/euclid.pjm/1102708254}.

    \bibitem[HH]{hoffman-homer} M. \markbox{6678}{Hoffman} \markbox{6679}{and} W. Homer, \textit{On cohomology automorphisms of complex flag manifolds}, Proc. Amer. Math. Soc. \textbf{91} (1984), no. 4, 643--648, \url{https://doi.org/10.2307/2044817}.
	
	

 %   \bibitem[Kh]{khare} Khare, S.S. On Dold manifolds, Topology and its Applications, Volume 33, Issue 3, 1989, Pages 297-307.

  %  \bibitem[Ko]{korbas} Korbaš, J. On parallelizability and span of the Dold manifolds,  Proc. Amer. Math. Soc. \textbf{141} (2013), no.~8, 2933--2939.

    
	\bibitem[L]{lin} X. Z. Lin, \textit{Geometric realization of Adams maps}, \markbox{6680}{Acta} Math. Sin. \textbf{27} (2011), no. 5, 863--870, \url{https://doi.org/10.1007/s10114-011-0164-y}.

    \bibitem[M]{mandal} M. Mandal, \markbox{6681}{Cohomology} \markbox{6682}{of} \markbox{6683}{generalized} \markbox{6684}{Dold} manifolds, \markbox{6685}{Thesis} (Ph.D.), \markbox{6686}{Homi} \markbox{6687}{Bhabha} \markbox{6688}{National} Institute, \markbox{6689}{The} \markbox{6690}{Institute} \markbox{6691}{of} \markbox{6692}{Mathematical} \markbox{6693}{Sciences} (2024).

%    \bibitem[Mu]{mukerjee} Mukerjee, H.~K. Classification of homotopy Dold manifolds,  New York J. Math. \textbf{9} (2003), 271--293.
	
	\bibitem[MS1]{mandal-sankaran} M. \markbox{6694}{Mandal} \markbox{6695}{and} P. Sankaran, \textit{Cohomology of generalized Dold spaces}, \markbox{6696}{Topology} Appl. \textbf{310} (2022), \markbox{6697}{Paper} No. 108040, 16 pp, \url{https://doi.org/10.1016/j.topol.2022.108040}.
    
	\bibitem[MS2]{mandal-sankaran2} M. \markbox{6698}{Mandal} \markbox{6699}{and} P. Sankaran, \textit{Cohomology and K-theory of generalized Dold manifolds fibred by complex flag manifolds}, \url{https://arxiv.org/abs/2407.03932}.


	
	%\bibitem[MiSt]{milnor-stasheff} Milnor, J. W.; Stasheff, J. D. {\it Characteristic classes.} Annals of Mathematics Studies, {\bf 76}, Princeton University Press, Princeton, NJ., 1974.
	
	\bibitem[NS]{nath-sankaran} A. \markbox{6700}{Nath} \markbox{6701}{and} P. Sankaran, \textit{On generalized Dold manifolds}, \markbox{6702}{Osaka} J. Math. \textbf{56} (2019), no. 1, 75--90, Errata, \textbf{57} (2020), no. 2, 505--506, \url{https://projecteuclid.org/euclid.ojm/1547607627}.
	
	\bibitem[O]{O} L.S. O'Neill, \textit{On the fixed point property for Grassmann manifolds}, \markbox{6703}{Thesis} (Ph.D.)---The \markbox{6704}{Ohio} \markbox{6705}{State} University
ProQuest LLC, \markbox{6706}{Ann} Arbor, \markbox{6707}{MI} (1974). 52 pp, \url{http://gateway.proquest.com/openurl?url_ver=Z39.88-2004&rft_val_fmt=info:ofi/fmt:kev:mtx:dissertation&res_dat=xri:pqdiss&rft_dat=xri:pqdiss:7511411}.	

	\bibitem[P]{Papadima} S. Papadima, \textit{Rigidity properties of compact Lie groups modulo maximal tori}, Math. Ann. \textbf{275} (1986), no. 4, 637--652, \url{https://doi.org/10.1007/BF01459142}.

	
	%\bibitem[Sp]{spanier} Spanier, E. H. {\em Algebraic Topology}. McGraw Hill, 1966. Reprinted by Springer-Verlag, New York, 1975.
	
	\bibitem[ST1]{shiga-tezuka} H. \markbox{6708}{Shiga} \markbox{6709}{and} M. Tezuka, \textit{Rational fibrations, homogeneous spaces with positive Euler characteristics and Jacobians}, Ann. Inst. \markbox{6710}{Fourier} (Grenoble) \textbf{37} (1987), no. 1, 81--106, \url{https://doi.org/10.5802/aif.1078}.
	
	\bibitem[ST2]{shiga-tezuka2} H. \markbox{6711}{Shiga} \markbox{6712}{and} M. Tezuka, \textit{Cohomology automorphisms of some homogeneous spaces}, \markbox{6713}{Singapore} \markbox{6714}{topology} \markbox{6715}{conference} (Singapore, 1985), \markbox{6716}{Topology} Appl. \textbf{25} (1987), no. 2, 143--150, Errata, \textbf{34} (1990), no. 2, 207, \url{https://doi.org/10.1016/0166-8641(87)90007-1}.
	
%	\bibitem[U]{ucci} Ucci, J. J. Immersions and embeddings of Dold manifolds. {\it Topology} {\bf 4} (1965), 283--293.

    \bibitem[W]{wong} P. Wong, \textit{Fixed point theory for homogeneous spaces. II}, Fund. Math. \textbf{186} (2005), no. 2, 161--175, \url{https://doi.org/10.4064/fm186-2-4}.


	
\end{thebibliography}
\end{document}