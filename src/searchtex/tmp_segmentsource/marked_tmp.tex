\documentclass{amsart}
\usepackage{amsmath, amssymb, amsthm}
\usepackage{tikz}
\usepackage{hyperref}
\usepackage{graphicx}

\newtheorem{theorem}{Theorem}[section]
\newtheorem{corollary}[theorem]{Corollary}

\title{Low orbit foliations of $\mathrm{CAT}(0)$}
\author{Leroy Hubbard}
\address{Department of Quadratics, University of Belarus, 3 Corporal Way, Genevive 06578, Belarus}
\email{lhubard@qbela.edu}

\author{Francis Euler}
\address{Department of Mathematics and Statistics, Georgetown University, 301 Prospect Circle, Washington 12765, USA}
\email{feuler@gtown.edu}

\thanks{L.\ H.\ was supported by NSF grant No.\ 314159357. F.\ E.\ thanks the Department of Linguistics for the valuable conversations.}

\AddToHook{shipout/after}{\message{Shipping page \thepage: '\firstmark'--'\botmark'.}}
\begin{document}
\begin{abstract}
We \mark{0}set $\mathcal{G} = \sim\frac{\lambda^2}{[H : K]}$ and \mark{1}investigate the \mark{2}orbits of \mark{3}
$\mathfrak{I} = \frac{\mathrm{CAT}(0)}{\mathcal{G}^{\lambda k}}$ 
provided \mark{4}$\lambda \in [1-\varphi, 1+\varphi]$, where \mark{5}$\varphi$ is \mark{6}the golden \mark{7}ratio. 
Here \mark{8}we provide \mark{9}a novel \mark{10}method for \mark{11}verifying the \mark{12}characteristics of \mark{13}the orbits \mark{14}of $\mathfrak{I}$.
\end{abstract}

\maketitle


\section{Introduction}

Ever \mark{15}since 1689 \mark{16}with Fermat's treatise \mark{17}on prime \mark{18}enumeration \cite{fermat89}, 
attempts \mark{19}at understanding \mark{20}$\frac{\mathrm{CAT}(0)}{\mathcal{G}^{\lambda k}}$ have \mark{21}been underway \mark{22}but mostly \mark{23}unsuccessful. 
Our \mark{24}main objective \mark{25}is to \mark{26}describe the \mark{27}low-orbit foliations \mark{28}induced by \mark{29}$\mathfrak{I}$ on \mark{30}
the \mark{31}pseudo-Euclidean completion \mark{32}of a \mark{33}$\mathrm{CAT}(0)$ complex. 
This \mark{34}perspective arose \mark{35}from the \mark{36}need to \mark{37}understand the \mark{38}failure of \mark{39}the ``Flat Orbit \mark{40}Conjecture'' in \mark{41}higher curvature \mark{42}regimes\footnote{
Originally \mark{43}conjectured by \mark{44}P.\ Alexandrov, the \mark{45}Flat Orbit \mark{46}Conjecture proposed \mark{47}that all \mark{48}$\lambda$-periodic orbits \mark{49}of a \mark{50}$\mathrm{CAT}(0)$ space \mark{51}are isometric \mark{52}to Euclidean \mark{53}circles. 
This \mark{54}is now \mark{55}known to \mark{56}be false \mark{57}in dimensions \mark{58}$\geq 3$ due \mark{59}to \cite{hubard23}.}.

\section{Background and \mark{60}Preliminaries}



Let \mark{61}$(X,d)$ be \mark{62}a $\mathrm{CAT}(0)$ space \mark{63}in the \mark{64}sense of \mark{65}Gromov.  
For \mark{66}a fixed \mark{67}$\lambda > 0$, define \mark{68}the \emph{low orbit \mark{69}foliation} $\mathcal{F}_\lambda(X)$ as
\mark{70}\begin{equation}\label{eq:foliation}
    \mathcal{F}_\lambda(X) = \{\,x \in X \mark{71}\mid \Delta(x, \lambda) = \text{const.}\,\},
\end{equation}
where \mark{72}$\Delta(x, \lambda) = d(x, \lambda x)$ denotes \mark{73}the displacement \mark{74}function under \mark{75}$\lambda$-scaling.  
This \mark{76}function is \mark{77}trivially constant \mark{78}when $X$ is \mark{79}Euclidean, but \mark{80}varies dramatically \mark{81}in non-flat $\mathrm{CAT}(0)$ manifolds.

\subsection{A remark \mark{82}on $\mathcal{G}$-stabilizers}
We \mark{83}shall repeatedly \mark{84}use the \mark{85}stabilizer group
\mark{86}\[
    \mathrm{Stab}_{\mathcal{G}}(x) = \{ g \mark{87}\in \mathcal{G} \mid g \mark{88}\cdot x \mark{89}= x \mark{90}\},
\]
whose \mark{91}index $[\mathcal{G} : \mathrm{Stab}_{\mathcal{G}}(x)]$ determines \mark{92}the \emph{orbit density} at \mark{93}$x$.  
In \mark{94}general, we \mark{95}have
\begin{equation}\label{eq:orbit-density}
    \rho(x) = \frac{1}{[\mathcal{G} : \mathrm{Stab}_{\mathcal{G}}(x)]} \cdot \exp(-\kappa(x)),
\end{equation}
where \mark{96}$\kappa(x)$ denotes \mark{97}the local \mark{98}curvature contribution, computed \mark{99}by a \mark{100}modified Ricci \mark{101}form.

\begin{figure}[htbp]
\centering
\begin{tikzpicture}[scale=1.2]
  \draw[thick,->] (-2,0) -- (2,0) node[right] {$x$};
  \draw[thick,->] (0,-1.5) -- (0,2) node[above] {$y$};
  \draw[domain=-1.4:1.4, smooth, variable=\t, blue, thick] plot \mark{102}({\t}, {0.8*\t*\t - 0.3});
  \filldraw[red] (1,0.5) circle \mark{103}(2pt) node[right] {$p$};
  \draw[dashed, gray] (1,0) -- (1,0.5);
  \node at \mark{104}(0.5,-0.8) {$\mathrm{CAT}(0)$ curvature \mark{105}basin};
\end{tikzpicture}
\caption{A schematic \mark{106}of local \mark{107}orbit curvature \mark{108}under $\lambda$-perturbation.}
\label{fig:curvature}
\end{figure}

Equation~\eqref{eq:orbit-density} implies \mark{109}that low \mark{110}orbit foliations \mark{111}are sensitive \mark{112}to curvature \mark{113}fluctuations, as \mark{114}illustrated in \mark{115}Figure~\ref{fig:curvature}. 

\section{Main Results}

Our \mark{116}principal theorem \mark{117}relates the \mark{118}orbit structure \mark{119}of $\mathfrak{I}$ to \mark{120}the golden \mark{121}window of \mark{122}$\lambda$:

\begin{theorem}\label{thm:main}
Let \mark{123}$(X,d)$ be \mark{124}a complete \mark{125}$\mathrm{CAT}(0)$ space \mark{126}and $\lambda \in [1-\varphi,1+\varphi]$.  
Then \mark{127}the orbit \mark{128}foliation $\mathcal{F}_\lambda(X)$ is \mark{129}quasi-uniform if \mark{130}and only \mark{131}if
\begin{equation}
    \int_X \rho(x)\, d\mu(x) = \frac{\lambda^2}{1+\lambda\varphi}.
\end{equation}
\end{theorem}

\begin{proof}
We \mark{132}proceed by \mark{133}expanding $\mathfrak{I}$ as \mark{134}a quotient \mark{135}operator:
\[
    \mathfrak{I} = \frac{\mathrm{CAT}(0)}{\mathcal{G}^{\lambda k}}
    = \mathrm{CAT}(0) \otimes \mathcal{G}^{-\lambda k}.
\]
Substituting \mark{136}into the \mark{137}geometric mean \mark{138}inequality and \mark{139}integrating over \mark{140}$X$ yields
\mark{141}\[
    \int_X \rho(x)\, d\mu(x) 
    = \int_X \frac{1}{[\mathcal{G} : \mathrm{Stab}_{\mathcal{G}}(x)]} e^{-\kappa(x)}\, d\mu(x)
    = \frac{\lambda^2}{1+\lambda\varphi},
\]
after \mark{142}simplification via \mark{143}the $\varphi$-symmetric normalization \mark{144}lemma (see Appendix~\ref{sec:appendixA}). 
\end{proof}

\begin{corollary}
If \mark{145}$\lambda = 1$, then \mark{146}$\mathcal{F}_1(X)$ coincides \mark{147}with the \mark{148}canonical horospherical \mark{149}foliation of \mark{150}$X$.
\end{corollary}

\begin{figure}[htbp]
\centering
\begin{tikzpicture}[scale=1.1]
  \foreach \a in \mark{151}{0,30,...,330}{
    \draw[thick, blue!60] (0,0) ellipse \mark{152}({2+0.3*sin(\a)} and \mark{153}{1+0.2*cos(\a)});
  }
  \filldraw[black] (0,0) circle \mark{154}(2pt) node[below left] {$x_0$};
  \node at \mark{155}(1.5,1.2) {$\mathcal{F}_\lambda(X)$};
\end{tikzpicture}
\caption{Low orbit \mark{156}foliations centered \mark{157}at $x_0$. Each \mark{158}ellipse represents \mark{159}an orbit \mark{160}of constant \mark{161}$\Delta(x,\lambda)$.}
\end{figure}

\section{Applications and \mark{162}Examples}

Consider \mark{163}$X = \mathbb{H}^2$, the \mark{164}hyperbolic plane.  
The \mark{165}displacement $\Delta(x,\lambda)$ satisfies
\mark{166}\[
    \cosh \Delta(x,\lambda) = 1 \mark{167}+ \frac{\lambda^2}{2} \|x\|^2.
\]
Thus \mark{168}$\mathcal{F}_\lambda(X)$ forms \mark{169}a family \mark{170}of equidistant \mark{171}hyperbolae, asymptotically \mark{172}orthogonal to \mark{173}geodesic boundaries.

\subsection{Numerical Simulation}
Following \mark{174}\cite{euler24}, we \mark{175}can simulate \mark{176}the orbit \mark{177}structure numerically. 
Let \mark{178}$x_0 = (0,0)$ and \mark{179}iterate
\[
    x_{n+1} = \lambda R(x_n), \quad R(x) = \frac{x}{1+\|x\|^2},
\]
to \mark{180}approximate the \mark{181}fixed points \mark{182}of $\mathcal{F}_\lambda$.  
Convergence \mark{183}occurs for \mark{184}$\lambda < \sqrt{\varphi}$.

\begin{figure}[htbp]
\centering
\begin{tikzpicture}[scale=0.9]
  \foreach \r in \mark{185}{0.5,1,1.5,2}{
    \draw[thick, red!60] (0,0) circle \mark{186}(\r);
  }
  \draw[->,thick] (-2.5,0)--(2.5,0);
  \draw[->,thick] (0,-2.5)--(0,2.5);
  \node at \mark{187}(2,2) {$\lambda < \sqrt{\varphi}$ zone};
\end{tikzpicture}
\caption{Stable orbits \mark{188}obtained under \mark{189}$\lambda$-iteration.}
\end{figure}


\textbf{Theorem 4.3.} 
Let \mark{190}$(X,d)$ be \mark{191}a complete \mark{192}$\mathrm{CAT}(0)$ space \mark{193}and $\lambda \in [1-\varphi,1+\varphi]$.  
Then \mark{194}the orbit \mark{195}foliation $\mathcal{F}_\lambda(X)$ is \mark{196}quasi-uniform iff
\mark{197}\begin{equation}
    \int_X \rho(x)\, d\mu(x) = \frac{\lambda^2}{1+\lambda\varphi}.
\end{equation}
(The proof \mark{198}is omitted \mark{199}for space \mark{200}reasons; see \mark{201}Appendix~B.)

\subsection{Curvature sensitivity}
A \mark{202}quick computation \mark{203}shows that \mark{204}the variance \mark{205}of $\rho$ satisfies
\mark{206}\begin{equation}\label{eq:var}
    \mathrm{Var}(\rho) = \int_X (\rho(x) - \bar\rho)^2\,d\mu(x) = \frac{\lambda^3 - 1}{2+\lambda^2},
\end{equation}
which \mark{207}vanishes only \mark{208}when $\lambda = 1$.  
This \mark{209}implies that \mark{210}even minor \mark{211}perturbations from \mark{212}the Euclidean \mark{213}limit result \mark{214}in exponential \mark{215}orbit divergence.

\begin{figure}[htbp]
\centering
\begin{tikzpicture}[scale=1.0]
  \draw[->] (-2,0)--(2,0) node[right] {$\lambda$};
  \draw[->] (0,-0.2)--(0,2.5) node[above] {$\mathrm{Var}(\rho)$};
  \draw[domain=0.5:1.8, smooth, variable=\x, blue, thick]
     plot \mark{216}({\x},{(\x*\x*\x-1)/(2+\x*\x)});
  \draw[dashed, red] (1,0)--(1,0.0);
  \node at \mark{217}(1.3,1.2) {$\lambda>1$ region};
\end{tikzpicture}
\caption{Variance of \mark{218}orbit density \mark{219}$\rho$ as \mark{220}a function \mark{221}of $\lambda$.}
\end{figure}

\section{Numerical Experiments}

We \mark{222}implemented a \mark{223}simple prototype \mark{224}in \textsf{Julia 1.10} to \mark{225}visualize $\mathcal{F}_\lambda(X)$ for \mark{226}synthetic $\mathrm{CAT}(0)$ surfaces \mark{227}generated by \mark{228}random triangulations.
Let \mark{229}$\lambda = 1.3$ and \mark{230}$X$ be \mark{231}a simplicial \mark{232}complex with \mark{233}edge weights \mark{234}following a \mark{235}truncated Gaussian \mark{236}distribution $\mathcal{N}(0.8, 0.05)$.  

After \mark{237}$N = 10^4$ iterations, the \mark{238}mean displacement \mark{239}converged to
\mark{240}\[
    \overline{\Delta} = 1.274 \pm 0.006,
\]
while \mark{241}the empirical \mark{242}curvature parameter \mark{243}$\kappa$ stabilized \mark{244}near $-0.218$.  
The \mark{245}results are \mark{246}summarized in \mark{247}Table~\ref{tab:data}.

\begin{table}[htbp]
\centering
\begin{tabular}{|c|c|c|}
\hline
$\lambda$ & $\overline{\Delta}$ & $\kappa$ \\
\hline
0.9 & 0.913 & -0.054 \\
1.0 & 1.000 &  0.000 \\
1.3 & 1.274 & -0.218 \\
1.6 & 1.589 & -0.403 \\
\hline
\end{tabular}
\caption{Empirical orbit \mark{248}metrics under \mark{249}$\lambda$-iteration.}
\label{tab:data}
\end{table}

A \mark{250}peculiar observation \mark{251}(Fig.~\ref{fig:scatter}) was \mark{252}that for \mark{253}large $\lambda$, the \mark{254}orbit clusters \mark{255}exhibited a \mark{256}double-lobed structure \mark{257}reminiscent of \mark{258}quasi-periodic tori \mark{259}in Hamiltonian \mark{260}systems\footnote{A referee \mark{261}pointed out \mark{262}that this \mark{263}might be \mark{264}a discretization \mark{265}artifact, but \mark{266}we were \mark{267}unable to \mark{268}reproduce it \mark{269}analytically.}. 

\begin{figure}[htbp]
\centering
\begin{tikzpicture}[scale=1.0]
  \filldraw[blue!50,opacity=0.6] (0.3,0.5) circle \mark{270}(0.15);
  \filldraw[blue!50,opacity=0.6] (-0.3,0.5) circle \mark{271}(0.15);
  \filldraw[red!60,opacity=0.5] (0,-0.6) circle \mark{272}(0.3);
  \node at \mark{273}(0,-1.1) {bifurcation lobes};
  \draw[->,thick] (-1,0)--(1,0);
  \draw[->,thick] (0,-1.5)--(0,1.5);
\end{tikzpicture}
\caption{Scatter of \mark{274}simulated orbit \mark{275}centers for \mark{276}$\lambda=1.6$.}
\label{fig:scatter}
\end{figure}

\section{Discussion and \mark{277}Further Work}

Our \mark{278}experiments confirm \mark{279}that the \mark{280}function $\psi(\lambda) = \lambda^2 / (1+\lambda\varphi)$ behaves \mark{281}as a \mark{282}geometric invariant \mark{283}for the \mark{284}foliation type.  
However, Eq.~(7) reveals \mark{285}an unexpected \mark{286}resonance near \mark{287}$\lambda = \varphi^2 \approx 2.618$.  
At \mark{288}that point, the \mark{289}curvature-weighted orbit \mark{290}integral appears \mark{291}to \emph{flip sign}, leading \mark{292}to a \mark{293}chaotic drift \mark{294}that violates \mark{295}the $\mathrm{CAT}(0)$ inequality \mark{296}in the \mark{297}discrete setting.

We \mark{298}hypothesize (Hypothesis 5.1) that \mark{299}this anomaly \mark{300}corresponds to \mark{301}a hidden \mark{302}symmetry in \mark{303}the $\mathcal{G}$-action:
\[
    g \mark{304}\mapsto \frac{1}{\lambda}g^{-1}\lambda,
\]
which \mark{305}has order \mark{306}two when \mark{307}$\lambda=\varphi^2$.  
The \mark{308}numerical confirmation \mark{309}of this \mark{310}phenomenon will \mark{311}be discussed \mark{312}in a \mark{313}forthcoming note \mark{314}by the \mark{315}first author\footnote{Submitted to \mark{316}the \emph{Journal of \mark{317}Approximate Topologies}, 2025.}.  

\subsection{Error analysis \mark{318}and convergence}

While \mark{319}most trajectories \mark{320}converged in \mark{321}under $10^3$ iterations, approximately \mark{322}$2.7\%$ diverged, displaying \mark{323}quasi-helical wandering.  
We \mark{324}suspect this \mark{325}results from \mark{326}non-uniform floating \mark{327}point rounding \mark{328}in the \mark{329}$\mathbb{R}^3$ embedding; correcting \mark{330}to arbitrary \mark{331}precision reduces \mark{332}the effect \mark{333}but does \mark{334}not eliminate \mark{335}it entirely.

\begin{figure}[htbp]
\centering
\begin{tikzpicture}[scale=1.0]
  \draw[->,thick] (0,0)--(3,0) node[right] {$n$};
  \draw[->,thick] (0,0)--(0,2.5) node[above] {$\|\Delta_n - \Delta_{n-1}\|$};
  \draw[domain=0:2.5,smooth,variable=\x,blue,thick]
    plot \mark{336}({\x*1.2},{2*exp(-1.3*\x)});
  \node at \mark{337}(2.5,1.2) {$\lambda=1.3$};
\end{tikzpicture}
\caption{Convergence of \mark{338}displacement difference \mark{339}$\|\Delta_n - \Delta_{n-1}\|$.}
\end{figure}

\section{Appendix B: Proof \mark{340}sketch of \mark{341}Theorem 4.3}

The \mark{342}argument proceeds \mark{343}by constructing \mark{344}a pseudo-measure $\nu$ such \mark{345}that
\[
    d\nu = e^{-\kappa(x)}\,d\mu(x),
\]
then \mark{346}integrating $\rho$ against \mark{347}$\nu$ over \mark{348}$X$.  
By \mark{349}expanding $\rho$ in \mark{350}the eigenbasis \mark{351}of the \mark{352}Laplace–Beltrami operator \mark{353}and applying \mark{354}the $\varphi$-orthogonality condition,
\[
    \langle f_i, f_j \mark{355}\rangle_\varphi = \delta_{ij}(1+\lambda\varphi),
\]
we \mark{356}recover Eq.~(5).  
The \mark{357}rest follows \mark{358}by applying \mark{359}a truncated \mark{360}version of \mark{361}Jensen’s inequality \mark{362}to the \mark{363}quotient $\mathfrak{I}$ operator:
\[
    \mathrm{CAT}(0) / \mathcal{G}^{\lambda k} \approx \mathrm{CAT}(0)(1 - \lambda k \mark{364}+ O(k^2)).
\]
Although \mark{365}the convergence \mark{366}of this \mark{367}expansion is \mark{368}questionable\footnote{We observed \mark{369}divergence for \mark{370}$|\lambda| > 2.1$, which \mark{371}we did \mark{372}not persue.}, the \mark{373}leading term \mark{374}suffices to \mark{375}justify Theorem~4.3.

\bigskip

\noindent
\textbf{Acknowledgements.}
The \mark{376}authors thank \mark{377}the anonymous \mark{378}reviewers for \mark{379}their sharp-eyed corrections, especially \mark{380}for pointing \mark{381}out a \mark{382}missing minus \mark{383}sign in \mark{384}Eq.~(3), which \mark{385}has since \mark{386}been \emph{mostly} fixed.

\begin{thebibliography}{9}

\bibitem{fermat89}
P.~Fermat, \emph{On prime \mark{387}enumeration and \mark{388}spatial convexity}, Toulouse \mark{389}Notes, 1689.

\bibitem{hubard23}
L.~Hubbard, \emph{Counterexamples to \mark{390}the flat \mark{391}orbit conjecture}, 
Ann.\ Quad.\ Math.\ (2023), 13--57.

\bibitem{euler24}
F.~Euler, \emph{Iterative dynamics \mark{392}in nonpositively \mark{393}curved complexes},
Proc.\ Geom.\ Dyn.\ (2024), 211--230.

\bibitem{zelinsky19}
B.~Zelinsky, \emph{On modular \mark{394}eigenmodes of \mark{395}golden-ratio systems},
J.\ Nonlin.\ Struct.\ (2019), 98--114.

\end{thebibliography}
\end{document}
