\documentclass{amsart}
\usepackage{amsmath}
\usepackage{amssymb}
\usepackage[mathscr]{eucal}

% --- Other utilities ---
\usepackage[shortlabels]{enumitem}
\usepackage{graphicx}
\usepackage[all]{xy}
\usepackage[dvipsnames]{xcolor}
\usepackage{tikz-cd}

% --- Hyperref should be loaded last ---
\usepackage[colorlinks=true,citecolor=red,urlcolor=blue,linkcolor=blue]{hyperref}

% --- Theorem environments ---
\newtheorem{theorem}{Theorem}[section]
\newtheorem{proposition}[theorem]{Proposition}
\newtheorem{lemma}[theorem]{Lemma}
\newtheorem{remark}[theorem]{Remark}
\newtheorem{definition}[theorem]{Definition}
\newtheorem{example}[theorem]{Example}
\newtheorem{corollary}[theorem]{Corollary}

% --- Commands ---
\newcommand{\thmref}[1]{Theorem~\ref{#1}}
\newcommand{\thmmref}[1]{Theorem$^\prime$~\ref{#1}}
\newcommand{\secref}[1]{Section~\ref{#1}}
\newcommand{\lemref}[1]{Lemma~\ref{#1}}
\newcommand{\propref}[1]{Proposition~\ref{#1}}
\newcommand{\propnref}[1]{Proposition$'$~\ref{#1}}
\newcommand{\corref}[1]{Corollary~\ref{#1}}
\newcommand{\remref}[1]{Remark~\ref{#1}}
\newcommand{\defref}[1]{Definition~\ref{#1}}
\newcommand{\subsecref}[1]{Subsection~\ref{#1}}
\newcommand{\homeo}{\mathrm{Homeo}}
\newcommand{\gal}{\mathrm{Gal}}
\newcommand{\bz}{\mathbb{Q}}
\newcommand{\bn}{\mathbb{N}}
\newcommand{\bq}{\mathbb{Q}}
\newcommand{\br}{\mathbb{R}}
\newcommand{\bc}{\mathbb{C}}
\newcommand{\bp}{\mathbb{P}}
\newcommand{\bs}{\mathbb{S}}
\newcommand{\co}{\mathcal{O}}
\newcommand{\rank}{\mathrm{Rank}}
\newcommand{\cl}{\mathcal{L}}
\newcommand{\lr}{\longrightarrow}
\renewcommand{\hom}{\mathrm{Hom}}
\newcommand{\wt}{\widetilde}
\newcommand{\im}{\mathrm{Im}}
\newcommand{\re}{\mathrm{Re}}
\newcommand{\Span}{\mathrm{span}}
\newcommand{\tor}{\mathrm{Tor}}
\newcommand{\pic}{\mathrm{Pic}}
\newcommand{\flag}{\mathrm{Flag}}
\newcommand{\ul}{\underline}
\newcommand{\fix}{\mathrm{Fix}}

\title[Graded Endomorphisms of $H^*(\mathbb{S}^m \times \mathbb{C}G_{n,k};\mathbb Q)$]{Rational Cohomology Endomorphisms of product of Sphere with Grassmannian and coincidence theory}

\author[M. Mandal]{Manas Mandal}
\address{Indian Institute of Technology Kanpur, Kanpur 208016, India}
	\email{manasm.imsc@gmail.com}
	
	\author[D. Setia]{Divya Setia}
	\address{Institute of Mathematics, Polish Academy of Sciences, Krak{\'o}w 31-027, Poland}
	\email{divyasetia01@gmail.com}
	
	\subjclass[2020]{Primary: 55S37, 08A35,  55M20; Secondary: 14M15}
	\keywords{cohomology endomorphisms, complex Grassmann manifolds, generalized Dold spaces,  fixed point theory, coincidence theory}


\newwrite\markfile
\immediate\openout\markfile=boxpositions_endomorphisms.txt

\newcommand{\markbox}[2]{%
  \setbox0=\hbox{#2}%
  \immediate\write\markfile{#1:pwhd:\the\value{page}:\the\wd0:\the\ht0:\the\dp0}%
  \pdfsavepos
  \write\markfile{#1:spxy:\the\value{page}:\the\pdflastxpos:\the\pdflastypos:}%
  #2% 
  \pdfsavepos
  \write\markfile{#1:epxy:\the\value{page}:\the\pdflastxpos:\the\pdflastypos:}%
}

\begin{document}
\begin{abstract}
We classified graded endomorphisms of the rational cohomology algebra of the product of a sphere and a complex Grassmannian, whose images are nonzero  in the second cohomology of the Grassmannian.

We also derive necessary conditions for the generalized Dold spaces to satisfy the coincidence property, in particular the fixed-point property. As an application of our results, we obtain several sufficient conditions for the existence of a point of coincidence between a pair of continuous functions on certain generalized Dold spaces. 
\end{abstract}
	
\maketitle
	
	\section{Introduction} \label{intro}
	\markbox{0}{The} \markbox{1}{classification} \markbox{2}{of} \markbox{3}{endomorphisms} \markbox{4}{of} \markbox{5}{the} \markbox{6}{rational} \markbox{7}{cohomology} \markbox{8}{algebra} \markbox{9}{of} \markbox{10}{formal} \markbox{11}{spaces} \markbox{12}{was} \markbox{13}{greatly} \markbox{14}{motivated} \markbox{15}{by} Sullivan's \markbox{16}{theory} \markbox{17}{where} \markbox{18}{it} \markbox{19}{was}  \markbox{20}{proved} \markbox{21}{that} \markbox{22}{rational} \markbox{23}{homotopy} \markbox{24}{class} \markbox{25}{of} self-maps \markbox{26}{are} \markbox{27}{completely} \markbox{28}{determined} \markbox{29}{by} \markbox{30}{the} \markbox{31}{induced} \markbox{32}{graded} \markbox{33}{endomorphisms} \markbox{34}{of} \markbox{35}{their} \markbox{36}{rational} \markbox{37}{cohomology} algebras. 
	
	\markbox{38}{In} \cite{brewster}, \markbox{39}{the} \markbox{40}{authors} \markbox{41}{developed} \markbox{42}{the} \markbox{43}{foundational} \markbox{44}{work} \markbox{45}{by} \markbox{46}{classifying} \markbox{47}{automorphisms} \markbox{48}{of} \markbox{49}{the} \markbox{50}{rational} \markbox{51}{cohomology} \markbox{52}{algebra} \markbox{53}{of} \markbox{54}{complex} Grassmannian. \markbox{55}{Their} \markbox{56}{results} \markbox{57}{were} \markbox{58}{generalized} \markbox{59}{in} \cite{hoffman}, \markbox{60}{where} \markbox{61}{the} \markbox{62}{author} \markbox{63}{classified} \markbox{64}{graded} \markbox{65}{endomorphisms} \markbox{66}{of} \markbox{67}{the} \markbox{68}{rational} \markbox{69}{cohomology} \markbox{70}{algebra} \markbox{71}{of} \markbox{72}{complex} \markbox{73}{Grassmannian} \markbox{74}{which} \markbox{75}{are} \markbox{76}{nonzero} \markbox{77}{on} \markbox{78}{dimension} \markbox{m79}{$2$}. Further, \markbox{80}{he} \markbox{81}{conjectured} \markbox{82}{that} \markbox{83}{every} \markbox{84}{graded} \markbox{85}{endomorphism}  \markbox{86}{vanishing} \markbox{87}{on} \markbox{88}{dimension} \markbox{89}{two} \markbox{90}{is} \markbox{91}{necessarily} trivial. \markbox{92}{This} \markbox{93}{conjecture} \markbox{94}{was} \markbox{95}{proved} \markbox{96}{in} \cite{glover-homer} \markbox{97}{for} \markbox{98}{several} cases.
	
	\markbox{99}{The} \markbox{100}{cohomology} \markbox{101}{endomorphisms} \markbox{102}{are} \markbox{103}{also} \markbox{104}{studied} \markbox{105}{for} \markbox{106}{a} \markbox{107}{variety} \markbox{108}{of} \markbox{109}{homogeneous} \markbox{110}{spaces} \markbox{m111}{$G/H$}, \markbox{112}{where} \markbox{m113}{$G$} \markbox{114}{is} \markbox{115}{a} \markbox{116}{compact} \markbox{117}{connected} \markbox{118}{Lie} \markbox{119}{group} \markbox{120}{and} \markbox{m121}{$H$} \markbox{122}{is} \markbox{123}{a} \markbox{124}{closed}
	\markbox{125}{subgroup} \markbox{126}{of} \markbox{127}{maximal} rank. \markbox{128}{This} \markbox{129}{is} \markbox{130}{a} \markbox{131}{topic} \markbox{132}{of} \markbox{133}{interest} \markbox{134}{since} \markbox{135}{past} \markbox{136}{fifty} \markbox{137}{years} \markbox{138}{and} \markbox{139}{are} \markbox{140}{studied} \markbox{141}{in} \markbox{142}{several} \markbox{143}{papers} \cite{shiga-tezuka2, brewster-homer, hoffman-homer, Papadima, duan, duan-fang, duan-zhao, lin, goswami-sarkar}. 
	
	%However, we are interested in product spaces because a little is known about how cohomology endomorphisms behave under products.
    However, \markbox{144}{the} \markbox{145}{behavior} \markbox{146}{of} \markbox{147}{cohomology} \markbox{148}{endomorphisms} \markbox{149}{for} \markbox{150}{the} \markbox{151}{product} \markbox{152}{spaces} \markbox{153}{is} \markbox{154}{comparatively} \markbox{155}{less} explored, \markbox{156}{and} \markbox{157}{this} \markbox{158}{provides} \markbox{159}{a} \markbox{160}{direction} \markbox{161}{for} study.
    \markbox{162}{We} \markbox{163}{are} \markbox{164}{mainly} \markbox{165}{interested} \markbox{166}{in} \markbox{167}{the} \markbox{168}{product} \markbox{169}{of} \markbox{170}{a} \markbox{171}{sphere}
	\markbox{172}{with} \markbox{173}{complex} \markbox{174}{Grassmannian} \markbox{m175}{$\mathbb S^m\times \mathbb CG_{n,k}$} \markbox{176}{because} \markbox{177}{the} \markbox{178}{sphere} \markbox{m179}{$\mathbb{S}^m$} \markbox{180}{has} \markbox{181}{a} simple, \markbox{182}{singly} \markbox{183}{generated} \markbox{184}{cohomology} algebra, \markbox{185}{while} \markbox{186}{the} \markbox{187}{Grassmannian} \markbox{m188}{$\mathbb{C}G_{n,k}$} (of \markbox{m189}{$k$}-planes \markbox{190}{in} \markbox{m191}{$\mathbb C^n$})
	\markbox{192}{carries} \markbox{193}{the} \markbox{194}{rich} \markbox{195}{structure} \markbox{196}{arising} \markbox{197}{from} \markbox{198}{Schubert} calculus. \markbox{199}{We} \markbox{200}{have} \markbox{201}{classified} \markbox{202}{graded} \markbox{203}{endomorphisms} \markbox{204}{of} \markbox{205}{the} \markbox{206}{rational} \markbox{207}{cohomology} \markbox{208}{algebra} \markbox{m209}{$H^*(\mathbb{S}^m \times \mathbb{C}G_{n,k};\mathbb{Q})$}, \markbox{210}{whose} \markbox{211}{image} \markbox{212}{has} \markbox{213}{a} \markbox{214}{nonzero} \markbox{215}{component} \markbox{216}{in} \markbox{m217}{$H^2(\mathbb CG_{n,k},\mathbb Q)$}.  \markbox{218}{As} \markbox{219}{an} \markbox{220}{application} \markbox{221}{we} \markbox{222}{obtain} \markbox{223}{useful} \markbox{224}{results} \markbox{225}{in} \markbox{226}{coincidence} \markbox{227}{theory} \markbox{228}{and} \markbox{229}{in} particular, fixed-point theory.  
	
	\markbox{230}{The} \markbox{231}{rational} \markbox{232}{cohomology} \markbox{233}{algebra} \markbox{234}{of} \markbox{235}{the} \markbox{236}{product}
	\[
	H^*(\mathbb S^m \times \mathbb CG_{n,k};\mathbb Q)\cong H^*(\mathbb S^m,\mathbb Q)\otimes H^*(\mathbb CG_{n,k};\mathbb Q)
	\]
	\markbox{237}{is} \markbox{238}{generated} \markbox{239}{by} \markbox{m240}{$u,c_1,c_2,\dots,c_k$}, \markbox{241}{where} \markbox{m242}{$H^*(\mathbb{C}G_{n,k};\mathbb{Q})$} (resp. \markbox{m243}{$H^*(\mathbb S^m;\mathbb Q)$}) \markbox{244}{is} \markbox{245}{generated} \markbox{246}{by} \markbox{247}{certain} \markbox{248}{Chern} \markbox{249}{classes} \markbox{m250}{$c_1, \dots, c_k$} (resp. \markbox{m251}{$u$}). Now, \markbox{252}{we} \markbox{253}{are} \markbox{254}{ready} \markbox{255}{to} \markbox{256}{state} \markbox{257}{one} \markbox{258}{of} \markbox{259}{the} \markbox{260}{main} \markbox{261}{results} \markbox{262}{of} \markbox{263}{our} \markbox{264}{paper} \markbox{265}{which} \markbox{266}{proves} \markbox{267}{that} \markbox{268}{the} \markbox{269}{rigidity} \markbox{270}{of} \markbox{m271}{$H^*(\mathbb CG_{n,k};\mathbb Q)$} \markbox{272}{persists} \markbox{273}{in} \markbox{m274}{$H^*(\mathbb S^m \times \mathbb CG_{n,k};\mathbb Q)$} \markbox{275}{even} \markbox{276}{in} \markbox{277}{the} \markbox{278}{presence} \markbox{279}{of} \markbox{280}{spherical} \markbox{281}{cohomology} classes.
	\begin{theorem}
		\markbox{282}{Let} \markbox{m283}{$\phi$} \markbox{284}{be} \markbox{285}{a} \markbox{286}{graded} \markbox{287}{endomorphism} \markbox{288}{of} 
		\markbox{m289}{$H^*(\mathbb S^{m}\times\mathbb C G_{n,k};\mathbb Q)$} 
		\markbox{290}{satisfying} \markbox{m291}{$\phi(c_1)\neq a u, \, a\in\mathbb Q$}.  
		\markbox{292}{Then} \markbox{293}{there} \markbox{294}{exists} \markbox{295}{a} \markbox{296}{nonzero} \markbox{297}{rational} \markbox{m298}{$\lambda$} \markbox{299}{such} \markbox{300}{that} \markbox{301}{the} \markbox{302}{following} holds.
		\begin{enumerate}\label{result main}
			\item If \markbox{m303}{$k < n - k,$} 
			$$ \phi(c_i) = \lambda^i c_i, \forall i \in \{1,2,\dots,k\}$$
			\markbox{304}{If} \markbox{m305}{$k = n - k$}, \markbox{306}{there} \markbox{307}{is} \markbox{308}{an} \markbox{309}{additional} \markbox{310}{possibility} \markbox{311}{of} \markbox{m312}{$\phi$} \markbox{313}{that} \markbox{314}{it} \markbox{315}{is} \markbox{316}{induced} \markbox{317}{by} \markbox{318}{the} \markbox{319}{homeomorphism} 
			\[
			\mathbb{C}G_{2k,k} \longrightarrow \mathbb{C}G_{2k,k}, 
			\quad L \longmapsto L^{\perp},
			\]
			\markbox{320}{where} \markbox{m321}{$L^{\perp}$} \markbox{322}{denotes} \markbox{323}{the} \markbox{324}{orthogonal} \markbox{325}{complement} \markbox{326}{of} \markbox{327}{the} \markbox{m328}{$k$}-plane \markbox{m329}{$L$} \markbox{330}{in} \markbox{m331}{$\mathbb{C}^{2k}$}.
			
			\item 
			The \markbox{332}{image} \markbox{333}{of} \markbox{m334}{$H^*(\mathbb{S}^m;\mathbb{Q})$} \markbox{335}{under} \markbox{m336}{$\phi$} \markbox{337}{lies} \markbox{338}{in} 
			\markbox{m339}{$H^*(\mathbb{S}^m;\mathbb{Q})$} \markbox{340}{or} \markbox{341}{in} \markbox{m342}{$H^*(\mathbb{C}G_{n,k};\mathbb{Q})$} i.e. $$\phi(u) = \mu u,\, \mu \in \mathbb{Q}, \text{ or } \phi(u) \in H^*(\mathbb{C}G_{n,k};\mathbb{Q}), \text{ with } (\phi(u))^2 =0.$$
			
		\end{enumerate}
	\end{theorem}
	\markbox{343}{Unlike} \markbox{344}{the} \markbox{345}{case} \markbox{346}{of} \markbox{347}{the} \markbox{348}{complex} Grassmannian, \markbox{349}{we} \markbox{350}{cannot} \markbox{351}{expect} \markbox{352}{a} \markbox{353}{graded}
	\markbox{354}{endomorphism} \markbox{355}{of} \markbox{m356}{$H^*(\mathbb S^m \times \mathbb CG_{n,k};\mathbb Q)$} \markbox{357}{to} \markbox{358}{be}
	\markbox{359}{trivial} \markbox{360}{merely} \markbox{361}{because} \markbox{362}{it} \markbox{363}{vanishes} \markbox{364}{in} \markbox{m365}{$H^2(\mathbb{C}G_{n,k}; \mathbb{Q})$}. 
	\markbox{366}{In} fact, \markbox{367}{we} \markbox{368}{proved} \markbox{369}{that} \markbox{370}{for} \markbox{371}{any} \markbox{372}{choice} \markbox{373}{of} \markbox{m374}{$P_i \in H^{2i-m}(\mathbb{C}G_{n,k};\mathbb{Q})$} \markbox{375}{and} \markbox{m376}{$Q\in \mathbb Qu\cup  H^*(\mathbb CG_{n,k};\mathbb Q)$} \markbox{377}{with} \markbox{m378}{$Q^2=0$}, \markbox{379}{there} \markbox{380}{exist} \markbox{381}{a} \markbox{382}{graded} \markbox{383}{endomorphism} \markbox{m384}{$\phi$} \markbox{385}{on} \markbox{m386}{$H^*(\mathbb S^m \times \mathbb CG_{n,k};\mathbb Q)$} \markbox{387}{such} \markbox{388}{that} \markbox{m389}{$\phi(c_i) = uP_i, \, \forall i$} \markbox{390}{and} \markbox{m391}{$\phi(u)=Q$}.
	%In fact, we proved that for any choice of $P_i \in H^{2i-m}(\mathbb{C}G;\mathbb{Q})$ and $Q =au, a\in \mathbb{Q}$, there exist a graded endomorphism $\phi$ on $H^*(\mathbb S^m \times \mathbb CG_{n,k};\mathbb Q)$ such that $\phi(c_i) = uP_i, \, \forall i$ and $\phi(u)=Q$.
	We \markbox{392}{also} \markbox{393}{proved} \markbox{394}{that} \markbox{395}{if} \markbox{396}{a} \markbox{397}{continuous} \markbox{398}{function} \markbox{399}{on} \markbox{m400}{$\mathbb S^m \times \mathbb CG_{n,k}$} \markbox{401}{stabilizes} \markbox{402}{a} \markbox{403}{copy} \markbox{404}{of} \markbox{405}{Grassmannian} \markbox{406}{then} \markbox{407}{the} \markbox{408}{induced} \markbox{409}{cohomology} \markbox{410}{endomorphism} \markbox{411}{stabilizes} \markbox{412}{the} \markbox{413}{subalgebra} \markbox{m414}{$H^*(\mathbb S^m ;\mathbb Q)$}.
	
	\markbox{415}{Our} \markbox{416}{study} \markbox{417}{is} \markbox{418}{also} \markbox{419}{motivated} \markbox{420}{by} \markbox{421}{the} \markbox{422}{theory} \markbox{423}{of} \markbox{424}{generalized} \markbox{425}{Dold} \markbox{426}{spaces} \markbox{427}{because} \markbox{428}{the} \markbox{429}{product} \markbox{430}{space} \markbox{m431}{$\mathbb S^m \times \mathbb CG_{n,k}$} \markbox{432}{is} \markbox{433}{a} \markbox{434}{double} \markbox{435}{cover} \markbox{436}{of} \markbox{437}{certain}  \markbox{438}{generalized} \markbox{439}{Dold} \markbox{440}{spaces} (GDS). %denoted by $P(m,n,k)$.
	The \markbox{441}{classical} \markbox{442}{Dold} \markbox{443}{manifolds} \markbox{444}{were} \markbox{445}{introduced} \markbox{446}{in} \cite{dold} \markbox{447}{to} \markbox{448}{construct} odd-dimensional \markbox{449}{generators} \markbox{450}{for} Thom's \markbox{451}{unoriented} \markbox{452}{cobordism} ring. \markbox{453}{In} \markbox{454}{this} paper, \markbox{455}{we} \markbox{456}{are} \markbox{457}{interested} \markbox{458}{in} \markbox{459}{the} \markbox{460}{GDS} \markbox{461}{introduced} \markbox{462}{in} \cite{nath-sankaran} \markbox{463}{and} \markbox{464}{defined} \markbox{465}{as}  %These spaces have been extensively studied from various aspects (see \cite{ucci, khare, korbas, mukerjee}). 
	\[
	P(m,n,k):=\mathbb S^m\times \mathbb CG_{n,k}/\!\!\sim, \text { where } (s,L)\sim (-s,\bar L).
	\]
	\markbox{466}{As} \markbox{467}{an} \markbox{468}{application} \markbox{469}{of} \thmref{result main}, \markbox{470}{we} \markbox{471}{describe} \markbox{472}{endomorphisms} \markbox{473}{of} \markbox{m474}{$H^*(P(m,n,k);\mathbb{Q})$} \markbox{475}{induced} \markbox{476}{by} \markbox{477}{continuous} \markbox{478}{functions} \markbox{479}{on} \markbox{m480}{$P(m,n,k)$}. \markbox{481}{Using} \markbox{482}{this} description, \markbox{483}{we} \markbox{484}{prove} \markbox{485}{that} \markbox{486}{every} \markbox{487}{automorphism} \markbox{488}{of} \markbox{m489}{$H^*(P(m,n,k);\mathbb{Q})$} \markbox{490}{induced} \markbox{491}{by} \markbox{492}{a} \markbox{493}{continuous} \markbox{494}{function} \markbox{495}{on} \markbox{m496}{$P(m,n,k)$} \markbox{497}{lifts} \markbox{498}{to} \markbox{499}{an} \markbox{500}{automorphism} \markbox{501}{of} \markbox{m502}{$H^*(\mathbb S^m\times \mathbb CG_{n,k}; \mathbb{Q})$} \markbox{503}{if} \markbox{m504}{$n>2$}. 
	
	\markbox{505}{Our} \markbox{506}{broader} \markbox{507}{aim} \markbox{508}{is} \markbox{509}{to} \markbox{510}{apply} \thmref{result main} \markbox{511}{to} \markbox{512}{obtain} \markbox{513}{results} \markbox{514}{in} \markbox{515}{coincidence} theory. \markbox{516}{Coincidence} \markbox{517}{theory} \markbox{518}{has} \markbox{519}{been} \markbox{520}{extensively} \markbox{521}{studied} \markbox{522}{in} \cite{hoffman-noncoin, glover-homer coin, wong}. %and fixed point theory.
	A \markbox{523}{pair} \markbox{m524}{$(X,g)$} \markbox{525}{where} \markbox{m526}{$g$} \markbox{527}{is} \markbox{528}{a} \markbox{529}{continuous} \markbox{530}{function} \markbox{531}{on} \markbox{m532}{$X$} \markbox{533}{is} \markbox{534}{said} \markbox{535}{to} \markbox{536}{have} \markbox{537}{the} \markbox{538}{coincidence} \markbox{539}{property} \markbox{540}{if} \markbox{m541}{$g$} \markbox{542}{has} \markbox{543}{a} \markbox{544}{point} \markbox{545}{of} \markbox{546}{coincidence} \markbox{547}{with} \markbox{548}{every} \markbox{549}{continuous} \markbox{550}{function} \markbox{551}{on} \markbox{m552}{$X$}. \markbox{553}{In} particular, \markbox{554}{if} \markbox{m555}{$g$} \markbox{556}{is} \markbox{557}{the} \markbox{558}{identity} map, \markbox{559}{then} \markbox{560}{the} \markbox{561}{coincidence} \markbox{562}{property} \markbox{563}{is} \markbox{564}{same} \markbox{565}{as} \markbox{566}{the} fixed-point \markbox{567}{property} \markbox{568}{of} \markbox{m569}{$X$}.
	%We are providing necessary conditions for certain generalized Dold spaces to satisfy coincidence property.
	
	We \markbox{570}{have} \markbox{571}{generalized} \markbox{572}{Theorem} \markbox{m573}{$2$} \markbox{574}{of} \cite{glover-homer} \markbox{575}{to} \markbox{576}{the} \markbox{577}{setting} \markbox{578}{of} \markbox{579}{coincidence} \markbox{580}{theory} \markbox{581}{and} \markbox{582}{proved} \markbox{583}{that} \markbox{584}{the} \markbox{585}{pair} \markbox{m586}{$(\mathbb{C}G_{n,k},g)$} \markbox{587}{satisfies} \markbox{588}{the} \markbox{589}{coincidence} \markbox{590}{property} \markbox{591}{if} \markbox{m592}{$k(n-k)$} \markbox{593}{is} \markbox{594}{even} \markbox{595}{and} \markbox{m596}{$g$} \markbox{597}{has} \markbox{598}{nonzero} \markbox{599}{Brouwer} degree.
	\markbox{600}{To} conclude, \markbox{601}{using} \markbox{602}{the} \markbox{603}{Lefschetz} \markbox{604}{Coincidence} \markbox{605}{Theorem} \markbox{606}{and} \thmref{result main}, \markbox{607}{we} \markbox{608}{obtained} \markbox{609}{multiple} \markbox{610}{situations} \markbox{611}{when} \markbox{612}{two} \markbox{613}{continuous} \markbox{614}{functions} \markbox{615}{on} \markbox{616}{the} \markbox{617}{generalized} \markbox{618}{Dold} \markbox{619}{space} \markbox{m620}{$P(m,n,k)$} \markbox{621}{are} \markbox{622}{guaranteed} \markbox{623}{to} \markbox{624}{have} \markbox{625}{a} \markbox{626}{point} \markbox{627}{of} coincidence, \markbox{628}{and} \markbox{629}{found} \markbox{630}{certain} \markbox{631}{pairs} \markbox{m632}{$(P(m,n,k),g)$} \markbox{633}{satisfying} \markbox{634}{the} \markbox{635}{coincidence} property.
	%Using Lefschetz Coincidence Theorem and \thmref{result main}, we succeed in obtaining sufficient conditions for the existence of a point of coincidence between any two continuous functions on $P(m,n,k)$ when $k(n-k)$ is even.
	%At last, we explored when two continuous functions on the generalized Dold space $P(m,n,k)$ has a point of coincidence. Using Lefschetz Coincidence Theorem and \thmref{result main}, we succeed in obtaining sufficient conditions for the existence of a point of coincidence between any two continuous functions on $P(m,n,k)$ when $k(n-k)$ is even.
	
	The \markbox{636}{paper} \markbox{637}{is} \markbox{638}{organized} \markbox{639}{as} follows: \\
	\markbox{640}{In} \secref{section 2} \markbox{641}{we} \markbox{642}{develop} \markbox{643}{the} \markbox{644}{necessary} \markbox{645}{background} \markbox{646}{and} \markbox{647}{recall} \markbox{648}{some} \markbox{649}{relevant} results. \secref{section 3} \markbox{650}{is} \markbox{651}{devoted} \markbox{652}{to} \markbox{653}{the} \markbox{654}{study} \markbox{655}{of} \markbox{656}{graded} \markbox{657}{endomorphisms} \markbox{658}{of} \markbox{659}{the} \markbox{660}{rational} \markbox{661}{cohomology} \markbox{662}{algebra} \markbox{663}{of} \markbox{m664}{$\mathbb S^{m}\times \mathbb C G_{n,k}$}, \markbox{665}{from} \markbox{666}{which} \markbox{667}{we} \markbox{668}{extract} \markbox{669}{several} consequences. \markbox{670}{These} \markbox{671}{are} \markbox{672}{applied} \markbox{673}{in} \secref{section 4} \markbox{674}{to} \markbox{675}{obtain} \markbox{676}{the} coincidence-theoretic results.
	
	\iffalse For \markbox{677}{formal} spaces, Sullivan's \markbox{678}{theory} \markbox{679}{shows} \markbox{680}{that} \markbox{681}{the} \markbox{682}{rational} \markbox{683}{homotopy} \markbox{684}{class} \markbox{685}{of} \markbox{686}{a} self-maps \markbox{687}{are} \markbox{688}{completely} \markbox{689}{determined} \markbox{690}{by} \markbox{691}{the} \markbox{692}{induced} \markbox{693}{graded} \markbox{694}{endomorphisms} \markbox{695}{of} \markbox{696}{their} \markbox{697}{rational} \markbox{698}{cohomology} algebras. \markbox{699}{This} \markbox{700}{viewpoint} \markbox{701}{has} \markbox{702}{motivated} \markbox{703}{extensive} \markbox{704}{work} \markbox{705}{on} \markbox{706}{cohomology} \markbox{707}{endomorphisms} \markbox{708}{of} \markbox{709}{various} \markbox{710}{classical}  spaces.
	%, particularly homogeneous spaces of compact Lie groups.
	%Initial results for complex Grassmannians were obtained by Brewster \cite{}, Glover-Homer \cite{}, and Hoffman \cite{}, followed by contributions of  Shiga-Tezuka \cite{}, Papadima \cite{}, and Duan et al. (\cite{},\cite{},\cite{}) for broader classes of homogeneous spaces.
	% $G/H$, where $G$ is complact connected Lie group and $H$ is a closed subgroup of maximal rank.
	
	
	Early \markbox{711}{foundational} \markbox{712}{work} \markbox{713}{on} \markbox{714}{cohomology} \markbox{715}{endomorphisms} \markbox{716}{of} \markbox{717}{complex} \markbox{718}{Grassmannians}
	\markbox{719}{was} \markbox{720}{carried} \markbox{721}{out} \markbox{722}{in} 1978 \markbox{723}{by} Brewster, \markbox{724}{in} \markbox{725}{his} Ph.D.\ thesis \cite{brewster} \markbox{726}{on} \markbox{727}{cohomology}
	automorphisms, \markbox{728}{and} \markbox{729}{by} \markbox{730}{Glover} \markbox{731}{and} \markbox{732}{Homer} \cite{glover-homer}.
	\markbox{733}{This} \markbox{734}{was} \markbox{735}{followed} \markbox{736}{by} Hoffman's 1984 \markbox{737}{work}
	\cite{hoffman}, \markbox{738}{which} \markbox{739}{classified} \markbox{740}{all} \markbox{741}{rational} \markbox{742}{cohomology} \markbox{743}{endomorphisms} \markbox{744}{that} \markbox{745}{are}
	non-trivial \markbox{746}{in} \markbox{747}{degree} two. \markbox{748}{He} \markbox{749}{showed} \markbox{750}{that} \markbox{751}{for} \markbox{752}{the} \markbox{753}{Grassmannians} \markbox{m754}{$\mathbb CG_{n,k}$} (of \markbox{m755}{$k$}-planes \markbox{756}{in} \markbox{m757}{$\mathbb{C}^n$}), \markbox{758}{such} \markbox{759}{endomorphisms} \markbox{760}{are} \markbox{761}{precisely} \emph{Adams maps}: \markbox{762}{each} \markbox{763}{cohomology} \markbox{764}{class} \markbox{m765}{$x$} \markbox{766}{of} \markbox{767}{degree} \markbox{m768}{$2i$} \markbox{769}{maps} \markbox{770}{to} \markbox{771}{a} \markbox{772}{scalar} \markbox{773}{multiple} \markbox{m774}{$\lambda^i x$} \markbox{775}{for} \markbox{776}{some} \markbox{777}{fixed} \markbox{778}{nonzero} \markbox{m779}{$\lambda \in \mathbb{Q}$}, \markbox{780}{except} \markbox{781}{in} \markbox{782}{special} \markbox{783}{symmetric} \markbox{784}{cases} \markbox{m785}{$n=2k$}, \markbox{786}{where} \markbox{787}{additional} \markbox{788}{involutive} \markbox{789}{automorphisms} appear. \markbox{790}{Hoffman} \markbox{791}{further} \markbox{792}{conjectured} \markbox{793}{that} \markbox{794}{every} \markbox{795}{endomorphism} \markbox{796}{vanishing} \markbox{797}{in} \markbox{798}{degree} \markbox{799}{two} \markbox{800}{is} \markbox{801}{necessarily} trivial. \markbox{802}{The} \markbox{803}{results} \markbox{804}{of} \markbox{805}{Glover} \markbox{806}{and} \markbox{807}{Homer} (see \cite{glover-homer}) \markbox{808}{support} \markbox{809}{this} \markbox{810}{conjecture} \markbox{811}{in} \markbox{812}{several} cases, \markbox{813}{under} \markbox{814}{a} \markbox{815}{certain} hypothesis, \markbox{816}{say} (GH).
	
	
	Later, \markbox{817}{cohomology} \markbox{818}{endomorphisms} \markbox{819}{were} \markbox{820}{studied} \markbox{821}{for} \markbox{822}{a} \markbox{823}{variety} \markbox{824}{of} \markbox{825}{homogeneous}
	\markbox{826}{spaces} \markbox{m827}{$G/H$}, \markbox{828}{where} \markbox{m829}{$G$} \markbox{830}{is} \markbox{831}{a} \markbox{832}{compact} \markbox{833}{connected} \markbox{834}{Lie} \markbox{835}{group} \markbox{836}{and} \markbox{m837}{$H$} \markbox{838}{is} \markbox{839}{a} \markbox{840}{closed}
	\markbox{841}{subgroup} \markbox{842}{of} \markbox{843}{maximal} rank. \markbox{844}{For} examples,
	\markbox{845}{Shiga} \markbox{846}{and} \markbox{847}{Tezuka} \markbox{848}{examined} \markbox{849}{cohomology}
	\markbox{850}{automorphisms} \markbox{851}{of} \markbox{852}{homogeneous} \markbox{853}{spaces} \markbox{854}{of} \markbox{855}{simple} \markbox{856}{Lie} \markbox{857}{groups} \cite{shiga-tezuka2}, \markbox{858}{and} \markbox{859}{Papadima} \markbox{860}{described} \markbox{861}{all} \markbox{862}{cohomology} \markbox{863}{automorphisms} \markbox{864}{when} \markbox{865}{the} \markbox{866}{subgroup} \markbox{867}{is} \markbox{868}{a} \markbox{869}{maximal} torus,
	\markbox{870}{working} \markbox{871}{over} \markbox{872}{both} \markbox{m873}{$\mathbb Q$} \markbox{874}{and} \markbox{m875}{$\mathbb R$} \cite{Papadima}. \markbox{876}{Hoffman} \markbox{877}{and} \markbox{878}{Homer} \markbox{879}{also}
	\markbox{880}{proposed} \markbox{881}{a} \markbox{882}{general} \markbox{883}{classification} \markbox{884}{for} \markbox{885}{homogeneous} \markbox{886}{spaces} \markbox{887}{arising} \markbox{888}{from} \markbox{889}{unitary}
	groups, \markbox{890}{together} \markbox{891}{with} \markbox{892}{partial} \markbox{893}{results} \markbox{894}{supporting} \markbox{895}{their} conjecture. \markbox{896}{Further}
	\markbox{897}{progress} \markbox{898}{includes} \markbox{899}{the} \markbox{900}{study} \markbox{901}{of} \markbox{902}{integral} \markbox{903}{cohomology} \markbox{904}{endomorphisms} \markbox{905}{of} \markbox{906}{the} \markbox{907}{Grassmannian} \markbox{908}{of} \markbox{909}{complex} \markbox{910}{structures}
	\markbox{m911}{$SO(2n)/U(n)$} \cite{duan}, \markbox{912}{and} \markbox{913}{the} \markbox{914}{work} \markbox{915}{of} \markbox{916}{Duan} \markbox{917}{and} \markbox{918}{Fang} \markbox{919}{on} \markbox{920}{the} \markbox{921}{eight} \markbox{922}{exceptional}
	\markbox{923}{Grassmannians} \cite{duan-fang}. \markbox{924}{Related} \markbox{925}{developments} \markbox{926}{appear} \markbox{927}{in} \cite{duan-zhao}, \cite{lin}, \markbox{928}{and} \cite{goswami-sarkar}.
	
	
	\markbox{929}{Most} \markbox{930}{of} \markbox{931}{these} results, however, \markbox{932}{concern} \markbox{933}{specific} \markbox{934}{classes} \markbox{935}{of} \markbox{936}{homogeneous} spaces.
	\markbox{937}{Comparatively} \markbox{938}{little} \markbox{939}{is} \markbox{940}{known} \markbox{941}{about} \markbox{942}{how} \markbox{943}{cohomology} \markbox{944}{endomorphisms} \markbox{945}{behave} \markbox{946}{for}
	products. \markbox{947}{The} \markbox{948}{product} \markbox{949}{of} \markbox{950}{a} \markbox{951}{sphere}
	\markbox{952}{with} \markbox{953}{a} \markbox{954}{complex} \markbox{955}{Grassmannian} \markbox{m956}{$\mathbb S^m\times \mathbb CG_{n,k}$} \markbox{957}{offers} \markbox{958}{a} \markbox{959}{particularly} \markbox{960}{useful} \markbox{961}{setting} \markbox{962}{for} \markbox{963}{such} questions: \markbox{964}{the}
	\markbox{965}{sphere} \markbox{966}{has} \markbox{967}{a} simple, \markbox{968}{singly} \markbox{969}{generated} \markbox{970}{cohomology} algebra, \markbox{971}{while} \markbox{972}{the} \markbox{973}{Grassmannian}
	\markbox{974}{carries} \markbox{975}{the} \markbox{976}{rich} \markbox{977}{structure} \markbox{978}{arising} \markbox{979}{from} \markbox{980}{Schubert} calculus.  
	\markbox{981}{Although} \markbox{982}{the} \markbox{983}{garded} \markbox{984}{endomorphisms} \markbox{985}{of} \markbox{986}{the} \markbox{987}{rational} \markbox{988}{cohomology} \markbox{989}{algebra} \markbox{990}{of} \markbox{991}{the} product, \markbox{m992}{$\mathrm{End}\big(H^*(\mathbb S^m\times \mathbb CG_{n,k};\mathbb Q)\big)$}, \markbox{993}{are} \markbox{994}{interesting} \markbox{995}{in} \markbox{996}{their} \markbox{997}{own} right, \markbox{998}{our} \markbox{999}{broader} \markbox{1000}{aim} \markbox{1001}{is} \markbox{1002}{to} \markbox{1003}{apply} \markbox{1004}{them} \markbox{1005}{to} \markbox{1006}{the} \textit{coincidence theory} \markbox{1007}{of} \markbox{1008}{certain} \textit{generalized Dold space}s.
	\markbox{1009}{Coincidence} \markbox{1010}{theory} \markbox{1011}{studies} \markbox{1012}{the} \markbox{1013}{set} \markbox{1014}{of} \markbox{1015}{points} \markbox{m1016}{$x$} \markbox{1017}{for} \markbox{1018}{which} \markbox{1019}{two} \markbox{1020}{maps} 
	\markbox{m1021}{$f,g : X \to X$} \markbox{1022}{satisfy} \markbox{m1023}{$f(x)=g(x)$}, \markbox{1024}{and} \markbox{1025}{seeks} \markbox{1026}{criteria} \markbox{1027}{for} \markbox{1028}{the} 
	\markbox{1029}{existence} \markbox{1030}{or} \markbox{1031}{elimination} \markbox{1032}{of} \markbox{1033}{such} points. 
	
	
	%For a continuous map $g$ on a space $X$, the pair $(X,g)$ is said to have the \emph{coincidence property} (abbreviated as CP) if every continuous map $f:X\to X$ has a coincidence point with $g$; that is, $f(x)=g(x)$ for some $x\in X$.
	The \markbox{1034}{classical} \markbox{1035}{Dold} \markbox{1036}{manifolds} \markbox{m1037}{$P(m,n):=\mathbb S^{m}\times\mathbb CP^{n}/\!\sim$}, 
	\markbox{1038}{where} \markbox{m1039}{$(s,L)\sim(-s,\bar L)$}, \markbox{1040}{were} \markbox{1041}{introduced} \markbox{1042}{by} \markbox{1043}{Dold} \cite{dold} \markbox{1044}{to} \markbox{1045}{construct} odd-dimensional \markbox{1046}{generators} \markbox{1047}{for} Thom's \markbox{1048}{unoriented} \markbox{1049}{cobordism} ring. \markbox{1050}{These} \markbox{1051}{spaces} \markbox{1052}{have} \markbox{1053}{been} \markbox{1054}{extensively} \markbox{1055}{studied} \markbox{1056}{and} generalized. \markbox{1057}{Nath} \markbox{1058}{and} \markbox{1059}{Sankaran} \cite{nath-sankaran} \markbox{1060}{extended} \markbox{1061}{the} \markbox{1062}{construction} \markbox{1063}{by} \markbox{1064}{replacing} \markbox{m1065}{$\mathbb CP^{n}$} \markbox{1066}{with} \markbox{1067}{a} \markbox{1068}{Hausdorff} \markbox{1069}{space} \markbox{m1070}{$X$} \markbox{1071}{equipped} \markbox{1072}{with} \markbox{1073}{an} \markbox{1074}{involution} \markbox{m1075}{$\sigma$} \markbox{1076}{with} \markbox{1077}{nonempty} \markbox{1078}{fixed} points. \markbox{1079}{A} \markbox{1080}{further} \markbox{1081}{generalization} \markbox{1082}{in} \cite{mandal-sankaran} \markbox{1083}{replaces} \markbox{1084}{the} \markbox{1085}{sphere} \markbox{m1086}{$\mathbb S^{m}$} \markbox{1087}{by} \markbox{1088}{a} \markbox{1089}{space} \markbox{m1090}{$S$} \markbox{1091}{equipped} \markbox{1092}{with} \markbox{1093}{a} \markbox{1094}{free} \markbox{1095}{involution} \markbox{m1096}{$\alpha$}, \markbox{1097}{leading} \markbox{1098}{to}
	\[
	P(S,\alpha,X,\sigma):=S\times X/\!\sim,\text{ where } (s,x)\sim(\alpha(s),\sigma(x)),
	\]
	\markbox{1099}{which} \markbox{1100}{is} \markbox{1101}{called} \markbox{1102}{a} \emph{generalized Dold space} (GDS).
	\markbox{1103}{Here} \markbox{1104}{the} \markbox{1105}{quotient} \markbox{1106}{map} \markbox{m1107}{$\pi: S\times X\to P(S,\alpha,X,\sigma)$} \markbox{1108}{is} \markbox{1109}{a} \markbox{1110}{double} covering.
	%The space $P(S,\alpha,X,\sigma)$ has a $X$-bundle structure over $Y:=S/\!\sim,$ where $s\sim \alpha(s)$.
	
	
	The \markbox{1111}{class} \markbox{1112}{of} \markbox{1113}{GDS} \markbox{1114}{which} \markbox{1115}{we} \markbox{1116}{are} \markbox{1117}{interested} \markbox{1118}{in} \markbox{1119}{is} \markbox{m1120}{$P(m,n,k):=P(\mathbb S^m,\alpha,\mathbb CG_{n,k},\sigma)$}, \markbox{1121}{where} \markbox{m1122}{$\alpha $} \markbox{1123}{is} \markbox{1124}{the} \markbox{1125}{antipodal} \markbox{1126}{map} \markbox{1127}{on} \markbox{1128}{the} \markbox{1129}{sphere} \markbox{1130}{and} \markbox{1131}{the} \markbox{1132}{involution} \markbox{m1133}{$\sigma$} \markbox{1134}{on} \markbox{m1135}{$\mathbb CG_{n,k}$} \markbox{1136}{is} \markbox{1137}{induced} \markbox{1138}{from} \markbox{1139}{the} \markbox{1140}{standard} \markbox{1141}{complex} \markbox{1142}{conjugation} \markbox{1143}{on} \markbox{m1144}{$\mathbb C^n$}. \markbox{1145}{One} \markbox{1146}{observes} \markbox{1147}{that} \markbox{1148}{for} \markbox{1149}{any} \markbox{1150}{map} \markbox{m1151}{$f$} \markbox{1152}{on} \markbox{m1153}{$P(m,n,k)$}, \markbox{1154}{there} \markbox{1155}{exist} \markbox{1156}{a} \markbox{m1157}{$\alpha\times \sigma$}-equivariant \markbox{1158}{map} \markbox{m1159}{$\tilde f$} \markbox{1160}{on} \markbox{m1161}{$\mathbb S^m\times\mathbb CG_{n,k}$} \markbox{1162}{such} \markbox{1163}{that} \markbox{m1164}{$\pi\circ \tilde f=f\circ \pi.$}
	
	
	\markbox{1165}{Let} \markbox{1166}{us} \markbox{1167}{regard} \markbox{m1168}{$H^*(\mathbb S^m;\mathbb Q)$} \markbox{1169}{and} \markbox{m1170}{$H^*(\mathbb CG_{n,k};\mathbb Q)$} \markbox{1171}{as}
	\markbox{1172}{subalgebras} \markbox{1173}{of} \markbox{m1174}{$H^*(\mathbb S^m \times \mathbb CG_{n,k};\mathbb Q)$}.
	\markbox{1175}{Since} \markbox{m1176}{$H^*(\mathbb CG_{n,k};\mathbb Q)$} \markbox{1177}{is} \markbox{1178}{generated} \markbox{1179}{by} \markbox{1180}{certain} \markbox{1181}{Chern} \markbox{1182}{classes}
	\markbox{m1183}{$c_1,c_2,\dots,c_k$}, \markbox{1184}{it} \markbox{1185}{follows} \markbox{1186}{that} 
	\[
	H^*(\mathbb S^m \times \mathbb CG_{n,k};\mathbb Q)\cong H^*(\mathbb S^m,\mathbb Q)\otimes H^*(\mathbb CG_{n,k};\mathbb Q)
	\]
	\markbox{1187}{is} \markbox{1188}{generated} \markbox{1189}{by} \markbox{m1190}{$u,c_1,c_2,\dots,c_k$}, \markbox{1191}{where} \markbox{m1192}{$u$} \markbox{1193}{generates} \markbox{m1194}{$H^m(\mathbb S^m;\mathbb Q)$}.
	\markbox{1195}{We} \markbox{1196}{are} \markbox{1197}{now} \markbox{1198}{ready} \markbox{1199}{to} \markbox{1200}{present} \markbox{1201}{our} \markbox{1202}{main} results; \markbox{1203}{the} \markbox{1204}{following} \markbox{1205}{is} \markbox{1206}{one} \markbox{1207}{of} them, \markbox{1208}{which} \markbox{1209}{shows} \markbox{1210}{that} \markbox{1211}{the} \markbox{1212}{rigidity} \markbox{1213}{of}
	\markbox{m1214}{$H^*(\mathbb CG_{n,k};\mathbb Q)$} \markbox{1215}{persists} \markbox{1216}{in} \markbox{1217}{the} \markbox{1218}{cohomology} \markbox{1219}{of} \markbox{1220}{the}  product: \markbox{1221}{even} \markbox{1222}{in} \markbox{1223}{the} \markbox{1224}{presence}
	\markbox{1225}{of} \markbox{1226}{spherical} \markbox{1227}{cohomology} classes, \markbox{1228}{any} \markbox{1229}{graded} \markbox{1230}{endomorphism} \markbox{1231}{whose} \markbox{1232}{image} \markbox{1233}{has} \markbox{1234}{a} \markbox{1235}{nonzero} \markbox{1236}{component} \markbox{1237}{in} \markbox{m1238}{$H^{2}(\mathbb C G_{n,k};\mathbb Q)$} \markbox{1239}{forces}  \markbox{m1240}{$H^*(\mathbb CG_{n,k};\mathbb Q)$} \markbox{1241}{to} \markbox{1242}{behave} \markbox{1243}{exactly} \markbox{1244}{as} \markbox{1245}{it} \markbox{1246}{does} \markbox{1247}{on} \markbox{1248}{its} own.
	\markbox{1249}{More} precisely, \markbox{1250}{we} obtain: %that any graded endomorphism which is nontrivial on $H^2(\mathbb CG_{n,k};\mathbb Q)$ necessarily preserves  $H^*(\mathbb CG_{n,k};\mathbb Q)$.  Furthermore, the image of $H^*(\mathbb S^m;\mathbb Q)$ under such an endomorphism lies either entirely in $H^*(\mathbb S^m;\mathbb Q)$ or entirely in $H^*(\mathbb CG_{n,k};\mathbb Q)$.
	\begin{theorem}
		\markbox{1251}{Let} \markbox{m1252}{$\phi$} \markbox{1253}{be} \markbox{1254}{a} \markbox{1255}{graded} \markbox{1256}{endomorphism} \markbox{1257}{of} 
		\markbox{m1258}{$H^*(\mathbb S^{m}\times\mathbb C G_{n,k};\mathbb Q)$} 
		\markbox{1259}{satisfying} \markbox{m1260}{$\phi(c_1)\neq a u$} \markbox{1261}{for} \markbox{1262}{any} \markbox{m1263}{$a\in\mathbb Q$}.  
		\markbox{1264}{Then} \markbox{1265}{the} \markbox{1266}{following} \markbox{1267}{statements} hold:
		\begin{enumerate}
			\item If \markbox{m1268}{$k < n - k$}, \markbox{1269}{there} \markbox{1270}{exists} \markbox{1271}{a} \markbox{1272}{nonzero} \markbox{1273}{rational} \markbox{1274}{number} \markbox{m1275}{$\lambda$} \markbox{1276}{such} \markbox{1277}{that}
			\[
			\phi(x) = \lambda^i x 
			\quad \text{for all } x \in H^{2i}(\mathbb{C}G_{n,k}; \mathbb{Q}).
			\]
			\markbox{1278}{If} \markbox{m1279}{$k = n - k$}, \markbox{1280}{there} \markbox{1281}{is} \markbox{1282}{an} \markbox{1283}{additional} \markbox{1284}{possibility} \markbox{1285}{of} \markbox{m1286}{$\phi$} \markbox{1287}{induced} \markbox{1288}{by} \markbox{1289}{the} \markbox{1290}{homeomorphism} 
			\[
			\mathbb{C}G_{2k,k} \longrightarrow \mathbb{C}G_{2k,k}, 
			\quad L \longmapsto L^{\perp},
			\]
			\markbox{1291}{where} \markbox{m1292}{$L^{\perp}$} \markbox{1293}{denotes} \markbox{1294}{the} \markbox{1295}{orthogonal} \markbox{1296}{complement} \markbox{1297}{of} \markbox{1298}{the} \markbox{m1299}{$k$}-plane \markbox{m1300}{$L$} \markbox{1301}{in} \markbox{m1302}{$\mathbb{C}^{2k}$}.
			
			\item 
			The \markbox{1303}{image} \markbox{1304}{of} \markbox{m1305}{$H^*(\mathbb{S}^m;\mathbb{Q})$} \markbox{1306}{under} \markbox{m1307}{$\phi$} \markbox{1308}{lies} \markbox{1309}{either} \markbox{1310}{in} 
			\markbox{m1311}{$H^*(\mathbb{S}^m;\mathbb{Q})$} \markbox{1312}{or} \markbox{1313}{in} \markbox{m1314}{$H^*(\mathbb{C}G_{n,k};\mathbb{Q})$}.
			
		\end{enumerate}
	\end{theorem}
	
	\markbox{1315}{Unlike} \markbox{1316}{the} \markbox{1317}{case} \markbox{1318}{of} \markbox{1319}{the} \markbox{1320}{complex} Grassmannian, \markbox{1321}{we} \markbox{1322}{cannot} \markbox{1323}{expect} \markbox{1324}{a} \markbox{1325}{graded}
	\markbox{1326}{endomorphism} \markbox{1327}{of} \markbox{m1328}{$H^*(\mathbb S^m \times \mathbb CG_{n,k};\mathbb Q)$} \markbox{1329}{to} \markbox{1330}{be}
	\markbox{1331}{trivial} \markbox{1332}{merely} \markbox{1333}{because} \markbox{1334}{it} \markbox{1335}{vanishes} \markbox{1336}{in} \markbox{1337}{degree} two. Indeed, \markbox{1338}{for} \markbox{1339}{any} \markbox{1340}{choice} \markbox{1341}{of}
	\markbox{1342}{elements} \markbox{m1343}{$P_i \in H^{2i-m}(\mathbb S^m \times \mathbb CG_{n,k};\mathbb Q)$},
	\markbox{m1344}{$i=1,\dots,k$}, \markbox{1345}{one} \markbox{1346}{obtains} \markbox{1347}{a} well-defined \markbox{1348}{graded} \markbox{1349}{endomorphism} \markbox{m1350}{$\phi$} \markbox{1351}{of}
	\markbox{m1352}{$H^*(\mathbb S^m \times \mathbb CG_{n,k};\mathbb Q)$} \markbox{1353}{by} \markbox{1354}{setting}
	\markbox{m1355}{$\phi(c_i) = uP_i$} \markbox{1356}{for} \markbox{m1357}{$i=1,\dots,k$} \markbox{1358}{and} \markbox{m1359}{$\phi(u)=u$}, \markbox{1360}{when} \markbox{1361}{the} \markbox{1362}{hypothesis} (GH) holds. Hence, \markbox{1363}{even} \markbox{1364}{in} \markbox{1365}{the} \markbox{1366}{case} \markbox{m1367}{$P_1 = 0$}, \markbox{1368}{a} \markbox{1369}{large} \markbox{1370}{family} \markbox{1371}{of} \markbox{1372}{nontrivial} \markbox{1373}{graded} \markbox{1374}{endomorphisms} \markbox{1375}{still} exists. \markbox{1376}{See} \markbox{1377}{Proposition} \ref{main thm 2}.
	\markbox{1378}{We} \markbox{1379}{show} \markbox{1380}{that} \markbox{1381}{if} \markbox{1382}{a} \markbox{1383}{map} \markbox{1384}{on} \markbox{m1385}{$\mathbb S^{m}\times \mathbb C G_{n,k}$} \markbox{1386}{stabilizes} \markbox{1387}{a} \markbox{1388}{Grassmannian} factor, \markbox{1389}{then} \markbox{1390}{the} \markbox{1391}{induced} \markbox{1392}{cohomology} \markbox{1393}{endomorphism} \markbox{1394}{stabilizes} \markbox{1395}{the} \markbox{1396}{subalgebra} \markbox{m1397}{$H^*(\mathbb S^{m};\mathbb Q)$}. \markbox{1398}{See} \thmref{ind from top}.
	
	%(see Theorem \ref{ind from top}). 
	%\textcolor{gray}{If the map on $\mathbb S^m \times \mathbb CG_{n,k}$ has nonzero  Brouwer degree, the induced map in cohomology splits as a tensor product of two maps: one acting on the cohomology of the sphere and the other on the cohomology of the Grassmannian. (see  \ref{}).}
	
	\iffalse
	Using \markbox{1399}{the} \markbox{1400}{study} \markbox{1401}{of} \markbox{m1402}{$\mathrm{End}(H^*(\mathbb S^{m}\times \mathbb C G_{n,k};\mathbb Q))$}, \markbox{1403}{we} \markbox{1404}{obtain} \markbox{1405}{several} coincidence-theoretic consequences. \markbox{1406}{The} \markbox{1407}{next} \markbox{1408}{result} \markbox{1409}{provides} \markbox{1410}{a} \markbox{1411}{necessary} \markbox{1412}{criterion} \markbox{1413}{for} \markbox{1414}{determining} \markbox{1415}{when} \markbox{1416}{the} \markbox{1417}{pair} \markbox{m1418}{$\big(P(S,\alpha,X,\sigma),g\big)$} \markbox{1419}{have} \markbox{1420}{the} \markbox{1421}{coincidence} property, \markbox{1422}{for} \markbox{1423}{certain} \markbox{1424}{maps} \markbox{m1425}{$g$} \markbox{1426}{on} \markbox{1427}{the} GDS, \markbox{1428}{expressed} \markbox{1429}{in} \markbox{1430}{terms} \markbox{1431}{of} \markbox{1432}{its} \markbox{1433}{fibre} \markbox{m1434}{$X$} \markbox{1435}{and} \markbox{1436}{its} \markbox{1437}{base} \markbox{m1438}{$Y:=S/\!\sim_{\alpha}$}.
	
	
	
	
	\begin{proposition}
		\markbox{1439}{For}  \markbox{1440}{a} \markbox{1441}{continuous} \markbox{1442}{map} \markbox{m1443}{$g$} \markbox{1444}{on} \markbox{1445}{the} \markbox{1446}{generalized} \markbox{1447}{Dold} \markbox{1448}{space} \markbox{m1449}{$P(S,\alpha, X,\sigma)$}, \markbox{1450}{the} \markbox{1451}{pair} \markbox{m1452}{$\big(P(S,\alpha,X,\sigma),g\big)$} \markbox{1453}{does} \markbox{1454}{not} \markbox{1455}{have} \markbox{1456}{the} \markbox{1457}{CP} \markbox{1458}{if} \markbox{1459}{one} \markbox{1460}{of} \markbox{1461}{the} \markbox{1462}{following} hold:
		\begin{enumerate}
			\item The \markbox{1463}{pair} \markbox{m1464}{$\big(Y,p \circ g \circ s\big)$} \markbox{1465}{does} \markbox{1466}{not} \markbox{1467}{have} \markbox{1468}{the} CP, 
			\markbox{1469}{where} \markbox{m1470}{$s$} \markbox{1471}{denotes} \markbox{1472}{a} \markbox{1473}{global} \markbox{1474}{section} \markbox{1475}{of} \markbox{1476}{the} \markbox{m1477}{$X$}-bundle  
			\markbox{m1478}{$p: P(S, X) \to Y$}, \markbox{1479}{and} \markbox{m1480}{$g$} \markbox{1481}{is} \markbox{1482}{a} \markbox{1483}{fiber} \markbox{1484}{bundle} map.
			\item  There \markbox{1485}{exists} \markbox{1486}{a} \markbox{m1487}{$\sigma$}-equivariant \markbox{1488}{map} \markbox{m1489}{$f$}  \markbox{1490}{on} \markbox{m1491}{$X$} \markbox{1492}{and} \markbox{1493}{a}  \markbox{m1494}{$\alpha \times \sigma$}-equivariant \markbox{1495}{map} \markbox{m1496}{$\tilde g$} \markbox{1497}{on} \markbox{m1498}{$S\times X$} \markbox{1499}{inducing} \markbox{m1500}{$g$} \markbox{1501}{such} \markbox{1502}{that} \markbox{m1503}{$\mathrm{id}_S \times f$} \markbox{1504}{coincides} \markbox{1505}{with} \markbox{1506}{neither} \markbox{m1507}{$\tilde{g}$} \markbox{1508}{nor} \markbox{m1509}{$(\alpha \times \sigma) \circ \tilde{g}$}.
		\end{enumerate}
	\end{proposition}
	\fi
	
	Using \markbox{1510}{the} \markbox{1511}{study} \markbox{1512}{of} \markbox{m1513}{$\mathrm{End}\bigl(H^*(\mathbb S^{m}\times \mathbb C G_{n,k};\mathbb Q)\bigr),$}
	\markbox{1514}{we} \markbox{1515}{obtain} \markbox{1516}{several} coincidence-theoretic consequences. First, \markbox{1517}{we} \markbox{1518}{extend} Theorem~2 \markbox{1519}{of} \cite{glover-homer} \markbox{1520}{to} \markbox{1521}{the} \markbox{1522}{setting} \markbox{1523}{of} \markbox{1524}{coincidence} \markbox{1525}{theory} \markbox{1526}{and} \markbox{1527}{show} \markbox{1528}{that} \markbox{1529}{for} \markbox{1530}{a} \markbox{1531}{continuous} self-map \markbox{m1532}{$g$} \markbox{1533}{with} \markbox{1534}{nonzero} \markbox{1535}{Brouwer} \markbox{1536}{degree} \markbox{1537}{on} \markbox{m1538}{$\mathbb C G_{n,k}$} \markbox{1539}{with} \markbox{m1540}{$k(n-k)$} even, \markbox{1541}{any} \markbox{1542}{continuous} \markbox{1543}{map} \markbox{m1544}{$f$} \markbox{1545}{on} \markbox{m1546}{$\mathbb CG_{n,k}$} \markbox{1547}{has} \markbox{1548}{a} \markbox{1549}{coincidence} \markbox{1550}{with} \markbox{m1551}{$g$}. \markbox{1552}{See} Proposition~\ref{CP of CGnk}.
	
	\markbox{1553}{We} \markbox{1554}{then} \markbox{1555}{investigate} \markbox{1556}{the} \markbox{1557}{coincidence} \markbox{1558}{behaviour} \markbox{1559}{of} \markbox{1560}{the} \markbox{1561}{generalized} \markbox{1562}{Dold} \markbox{1563}{spaces} \markbox{m1564}{$P(m,n,k)$}. \markbox{1565}{For} \markbox{1566}{any} \markbox{1567}{continuous} self-map \markbox{m1568}{$g$} \markbox{1569}{with} \markbox{1570}{nonzero} \markbox{1571}{Brouwer} \markbox{1572}{degree} \markbox{1573}{on} \markbox{m1574}{$P(m,n,k)$}  \markbox{1575}{with} \markbox{m1576}{$k(n-k)$} \markbox{1577}{even} \markbox{1578}{and} \markbox{m1579}{$m$} even, \markbox{1580}{every} \markbox{1581}{map} \markbox{m1582}{$f$} \markbox{1583}{satisfying} \markbox{m1584}{$\tilde f^{*}(c_1)\neq a u$} \markbox{1585}{for} \markbox{1586}{any} \markbox{m1587}{$a\in\mathbb Q$} \markbox{1588}{has} \markbox{1589}{a} \markbox{1590}{coincidence} \markbox{1591}{with} \markbox{m1592}{$g$}; \markbox{1593}{when} \markbox{m1594}{$m$} \markbox{1595}{is} odd, \markbox{1596}{one} \markbox{1597}{additionally} \markbox{1598}{requires} \markbox{m1599}{$\tilde f^{*}(u)\neq -\tilde g^{*}(u)$}. \markbox{1600}{See} Theorem~\ref{coincidence thm}.
	
	\markbox{1601}{Under} \markbox{1602}{hypothesis} (GH), \markbox{1603}{if} \markbox{m1604}{$g$} \markbox{1605}{is} \markbox{1606}{a} \markbox{1607}{homotopy} \markbox{1608}{equivalence} \markbox{1609}{on} \markbox{m1610}{$P(m,n,k)$} \markbox{1611}{with} \markbox{m1612}{$k(n-k)$} even, \markbox{1613}{then} \markbox{m1614}{$g$} \markbox{1615}{has} \markbox{1616}{a} \markbox{1617}{coincidence} \markbox{1618}{with} \markbox{1619}{any} \markbox{1620}{map} \markbox{m1621}{$f$} \markbox{1622}{on} \markbox{m1623}{$P(m,n,k)$} \markbox{1624}{such} that: \\ 
	-- \markbox{1625}{when} \markbox{m1626}{$m$} \markbox{1627}{is} even,  \markbox{m1628}{$\tilde f^{*}(c_1)=a u, a\in \mathbb Q$} \markbox{1629}{implies} \markbox{m1630}{$\tilde f^{*}(u)=b u$} \markbox{1631}{for} \markbox{1632}{some} \markbox{m1633}{$b\in\mathbb Q$};  \\
	-- \markbox{1634}{when} \markbox{m1635}{$m$} \markbox{1636}{is} odd,  \markbox{m1637}{$\tilde f^{*}(u)\neq -\tilde g^{*}(u)$} holds. \markbox{1638}{See} Theorem~\ref{coincidence thm under hom}.  \\
	Moreover, \markbox{1639}{if} \markbox{m1640}{$m>2k$}, \markbox{1641}{no} \markbox{1642}{additional} \markbox{1643}{assumptions} \markbox{1644}{on} \markbox{m1645}{$f$} \markbox{1646}{are} needed.
	
	
	
	
	\markbox{1647}{The} \markbox{1648}{paper} \markbox{1649}{is} \markbox{1650}{organized} \markbox{1651}{as} follows: \markbox{1652}{In} Section~2 \markbox{1653}{we} \markbox{1654}{develop} \markbox{1655}{the} \markbox{1656}{necessary} \markbox{1657}{background} \markbox{1658}{and} \markbox{1659}{recall} \markbox{1660}{some} \markbox{1661}{relevant} results. Section~3 \markbox{1662}{is} \markbox{1663}{devoted} \markbox{1664}{to} \markbox{1665}{the} \markbox{1666}{study} \markbox{1667}{of} \markbox{1668}{graded} \markbox{1669}{endomorphisms} \markbox{1670}{of} \markbox{1671}{the} \markbox{1672}{rational} \markbox{1673}{cohomology} \markbox{1674}{algebra} \markbox{1675}{of} \markbox{m1676}{$\mathbb S^{m}\times \mathbb C G_{n,k}$}, \markbox{1677}{from} \markbox{1678}{which} \markbox{1679}{we} \markbox{1680}{extract} \markbox{1681}{several} consequences. \markbox{1682}{These} \markbox{1683}{are} \markbox{1684}{applied} \markbox{1685}{in} Section~4 \markbox{1686}{to} \markbox{1687}{obtain} \markbox{1688}{the} coincidence-theoretic results.\fi
	
	
	
	
	
	
	%%%%%%%%%%%%%%%%%%%%%%%%%%%%%%%
	\section{Preliminaries}  \label{section 2}
	
	\markbox{1689}{In} \markbox{1690}{this} section, \markbox{1691}{we} \markbox{1692}{discuss} \markbox{1693}{some} \markbox{1694}{preliminaries} \markbox{1695}{and} \markbox{1696}{recall} \markbox{1697}{some} \markbox{1698}{results} \markbox{1699}{that} \markbox{1700}{will} \markbox{1701}{be} \markbox{1702}{required} \markbox{1703}{to} \markbox{1704}{proceed} \markbox{1705}{with} \markbox{1706}{our} study.
	
	
	
	
	
	
	
	\subsection{Cohomology of complex Grassmannians}
	
	
	\markbox{1707}{Let} \markbox{m1708}{$\mathbb{C}G_{n,k}$} \markbox{1709}{denote} \markbox{1710}{the} \markbox{1711}{complex} \markbox{1712}{Grassmannian} \markbox{1713}{consisting} \markbox{1714}{of} \markbox{1715}{complex} \markbox{m1716}{$k$}-planes \markbox{1717}{in} \markbox{m1718}{$\mathbb{C}^n$}. 
	%As a homogeneous space, it is diffeomorphic to $U(n)/U(k) \times U(n-k)$, where $U(r)$ denotes the group of unitary matrices of order $r$. 
	Let \markbox{m1719}{$\gamma_{n,k}$} \markbox{1720}{and} \markbox{m1721}{$\beta_{n,k}$} \markbox{1722}{denote} \markbox{1723}{the} \markbox{1724}{canonical} \markbox{1725}{complex} \markbox{m1726}{$k$}-plane \markbox{1727}{and} \markbox{m1728}{$(n-k)$}-plane bundles, respectively, \markbox{1729}{over} \markbox{m1730}{$\mathbb{C}G_{n,k}$}.
	\markbox{1731}{Let} \markbox{1732}{the} \markbox{1733}{total} \markbox{1734}{Chern} \markbox{1735}{classes} \markbox{1736}{of} \markbox{1737}{the} \markbox{1738}{vector} \markbox{1739}{bundles} \markbox{m1740}{$\gamma_{n,k}$} \markbox{1741}{and} \markbox{m1742}{$\beta_{n,k}$} \markbox{1743}{be} \markbox{1744}{denoted} \markbox{1745}{by} \markbox{m1746}{$c(\gamma_{n,k}) = c$} \markbox{1747}{and} \markbox{m1748}{$c(\beta_{n,k}) = \bar{c}$}, respectively. Thus,
	$$c = 1 + c_1 + c_2 + \cdots + c_k, \quad \bar{c} = 1 + \bar{c}_1 + \bar{c}_2 + \cdots + \bar{c}_{n-k},$$
	\markbox{1749}{where} \markbox{m1750}{$c_i$} \markbox{1751}{and} \markbox{m1752}{$\bar{c}_i$} \markbox{1753}{denote} \markbox{1754}{the} \markbox{m1755}{$i$}-th \markbox{1756}{Chern} \markbox{1757}{classes} \markbox{1758}{of} \markbox{m1759}{$\gamma_{n,k}$} \markbox{1760}{and} \markbox{m1761}{$\beta_{n,k}$}, respectively.
	\markbox{1762}{Since} \markbox{m1763}{$\gamma_{n,k} \oplus \beta_{n,k} \cong \varepsilon_{\mathbb{C}}^n$}, \markbox{1764}{it} \markbox{1765}{follows} \markbox{1766}{that}  \markbox{m1767}{$c \cdot \bar{c} = 1$}.
	\markbox{1768}{The}  \markbox{1769}{cohomology} \markbox{1770}{ring} \markbox{1771}{of} \markbox{1772}{the} \markbox{1773}{complex} \markbox{1774}{Grassmannian} \markbox{1775}{is} \markbox{1776}{well} \markbox{1777}{known} \markbox{1778}{and} \markbox{1779}{given} \markbox{1780}{by}  
	$$
	H^*_{\mathbb{C}G}:=H^*(\mathbb{C}G_{n,k};\mathbb{Q}) \cong \mathbb{Q}[c_1, c_2, \dots, c_k, \bar{c}_1, \bar{c}_2, \dots, \bar{c}_{n-k}]/\langle h_r: 1\leq r\leq n\rangle,$$  
	\markbox{1781}{where}  \markbox{1782}{the} \markbox{1783}{relations} \markbox{m1784}{$h_r$} \markbox{1785}{for} \markbox{m1786}{$r = 1, 2, \dots, n$} \markbox{1787}{are} \markbox{1788}{induced} \markbox{1789}{from} \markbox{1790}{the} \markbox{1791}{homogeneous} \markbox{1792}{parts} \markbox{1793}{of} \markbox{1794}{the} \markbox{1795}{equation} \markbox{m1796}{$c\cdot \bar c=1$} \markbox{1797}{and} \markbox{1798}{given} \markbox{1799}{by}  
	\[
	h_r := \sum_{i+j=r} c_i \bar{c}_j.
	\]  
	\markbox{1800}{Using} \markbox{1801}{the} \markbox{1802}{relations} \markbox{m1803}{$h_r, r=1,2,...,n-k$}, \markbox{1804}{the} \markbox{1805}{generators} \markbox{m1806}{$\bar{c}_i$} \markbox{1807}{for} \markbox{m1808}{$i = 1, 2, \dots, n-k$} \markbox{1809}{can} \markbox{1810}{be} \markbox{1811}{expressed} \markbox{1812}{inductively} \markbox{1813}{in} \markbox{1814}{terms} \markbox{1815}{of} \markbox{m1816}{$c_i$} \markbox{1817}{for} \markbox{m1818}{$i = 1, 2, \dots, k$}.  Consequently, \markbox{1819}{the} \markbox{1820}{relations} \markbox{m1821}{$h_r$} \markbox{1822}{for} \markbox{m1823}{$r = n-k+1, \dots, n$} \markbox{1824}{become} \markbox{1825}{homogeneous} \markbox{1826}{polynomials} \markbox{1827}{in} \markbox{m1828}{$c_i$} \markbox{1829}{of} \markbox{1830}{degree} \markbox{m1831}{$2r$}, \markbox{1832}{where} \markbox{1833}{the} \markbox{1834}{degree} \markbox{1835}{of} \markbox{1836}{each} \markbox{m1837}{$c_i$} \markbox{1838}{is} \markbox{m1839}{$2i$}. \markbox{1840}{Then} \markbox{1841}{the} \markbox{1842}{cohomology} \markbox{1843}{ring} \markbox{m1844}{$H^*_{\mathbb{C}G}$} \markbox{1845}{can} \markbox{1846}{be} \markbox{1847}{rewritten} \markbox{1848}{as}  
	\begin{equation}\label{cohomo of grass}
		\mathbb{Q}[c_1, c_2, \dots, c_k]/\langle h_{n-k+1}, h_{n-k+2}, \dots, h_n \rangle.
	\end{equation}
	\markbox{1849}{Since} \markbox{1850}{there} \markbox{1851}{are} \markbox{1852}{no} \markbox{1853}{relations} \markbox{1854}{among} \markbox{1855}{the} \markbox{1856}{generators} \markbox{m1857}{$c_i$} \markbox{1858}{for} \markbox{m1859}{$i = 1, 2, \dots, k$} \markbox{1860}{up} \markbox{1861}{to} \markbox{1862}{degree} \markbox{m1863}{$2(n-k)$}, \markbox{1864}{the} \markbox{1865}{set} \markbox{1866}{of} \markbox{1867}{all} \markbox{1868}{monomials} \markbox{1869}{of} \markbox{1870}{degree} \markbox{m1871}{$2r$} \markbox{1872}{in} \markbox{1873}{terms} \markbox{1874}{of} \markbox{m1875}{$c_1, c_2, \ldots, c_k$} \markbox{1876}{forms} \markbox{1877}{a} \markbox{m1878}{$\mathbb{Q}$}-basis \markbox{1879}{of} \markbox{m1880}{$H^{2r}(\mathbb{C}G_{n,k};\mathbb{Q})$} \markbox{1881}{for} \markbox{m1882}{$r \leq n-k$}.
	
	\markbox{1883}{From} \markbox{1884}{now} on, \markbox{1885}{we} \markbox{1886}{denote} \markbox{1887}{the} \markbox{1888}{indexing} \markbox{1889}{set} \markbox{m1890}{$\{1,2,\dots, k\}$} \markbox{1891}{by} \markbox{m1892}{$I$}.
	\begin{remark}
		\markbox{1893}{We} \markbox{1894}{can} \markbox{1895}{assume}  \markbox{m1896}{$k\leq n-k$} \markbox{1897}{for}  \markbox{m1898}{$\mathbb{C}G_{n,k}$} \markbox{1899}{as} \markbox{m1900}{$\mathbb{C}G_{n,k}$} \markbox{1901}{is} \markbox{1902}{homeomorphic} \markbox{1903}{to} \markbox{m1904}{$\mathbb{C}G_{n,n-k}$} \markbox{1905}{by} \markbox{1906}{using} \markbox{1907}{orthogonal} complementation.
	\end{remark}
	\markbox{1908}{The} \markbox{1909}{complex} \markbox{1910}{Grassmannian} \markbox{m1911}{$\mathbb{C}G_{n,k}$} \markbox{1912}{is} \markbox{1913}{a} \markbox{1914}{homogeneous} \markbox{1915}{space} \markbox{1916}{and} \markbox{1917}{can} \markbox{1918}{be} \markbox{1919}{represented} \markbox{1920}{as} \markbox{1921}{the} \markbox{1922}{quotient} \markbox{1923}{of} \markbox{1924}{the} \markbox{1925}{unitary} \markbox{1926}{group} \markbox{m1927}{$U(n)$} \markbox{1928}{by} \markbox{1929}{the} \markbox{1930}{stabilizer} \markbox{1931}{subgroup} \markbox{m1932}{$U(k)\times U(n-k)$} \markbox{1933}{that} \markbox{1934}{is} 
	\begin{equation}\label{cgn as hom}
		\mathbb{C}G_{n,k} = U(n)/ (U(k)\times U(n-k)). 
	\end{equation} \markbox{1935}{Now} \markbox{1936}{we} \markbox{1937}{recall} \markbox{1938}{a} \markbox{1939}{result} \markbox{1940}{given} \markbox{1941}{in} \cite{shiga-tezuka}.    
	\begin{theorem}[\cite{shiga-tezuka}, \markbox{1942}{Theorem} \markbox{m1943}{\(A^{'}\)}]\label{Tezuka}
		\markbox{1944}{Let} \markbox{m1945}{$D_i(H^*(G/H;\mathbb{Q}))$} \markbox{1946}{be} \markbox{1947}{the} \markbox{m1948}{$\mathbb{Q}$}-vector \markbox{1949}{space} \markbox{1950}{of} \markbox{m1951}{$\mathbb{Q}$}-derivations \markbox{1952}{of} \markbox{m1953}{$H^*(G/H;\mathbb{Q})$} \markbox{1954}{which} \markbox{1955}{decreases} \markbox{1956}{the} \markbox{1957}{degree} \markbox{1958}{by} \markbox{m1959}{$i>0$} \markbox{1960}{where} \markbox{m1961}{$G$} \markbox{1962}{is} \markbox{1963}{a} connected, \markbox{1964}{compact} \markbox{1965}{Lie} \markbox{1966}{group} \markbox{1967}{and} \markbox{m1968}{$H$} \markbox{1969}{is} \markbox{1970}{a} \markbox{1971}{closed} \markbox{1972}{subgroup} \markbox{1973}{of} \markbox{1974}{maximal} rank.
		%such that $\rank (H) = \rank (G)$.
		Then, \markbox{1975}{for} \markbox{1976}{all} \markbox{m1977}{$i$}, $$D_i(H^*(G/H;\mathbb{Q})) =0.$$  
	\end{theorem}
	\subsection{Graded endomorphisms on $\mathbf{H^*_{\mathbb{C}G}}$}
	\markbox{1978}{It} \markbox{1979}{was} \markbox{1980}{conjectured} \markbox{1981}{in} \cite{O} \markbox{1982}{that} \markbox{1983}{any} \markbox{1984}{graded} \markbox{1985}{endomorphism} \markbox{m1986}{$\phi$} \markbox{1987}{of} \markbox{1988}{the} \markbox{1989}{cohomology} \markbox{1990}{algebra} \markbox{m1991}{$H^*_{\mathbb{C}G}$} \markbox{1992}{is} an\textit{ Adams }map \markbox{1993}{when} \markbox{m1994}{$k < n - k$}; \markbox{1995}{that} is, \markbox{1996}{there} \markbox{1997}{exists} \markbox{1998}{a} \markbox{1999}{rational} \markbox{m2000}{$\lambda$} \markbox{2001}{such} \markbox{2002}{that}
	\markbox{m2003}{$\phi(c_i) = \lambda^i c_i$}, \markbox{2004}{for} \markbox{2005}{all} \markbox{m2006}{$ i \in I.$} \markbox{2007}{Glover} \markbox{2008}{and} \markbox{2009}{Homer} (see \cite{glover-homer}) \markbox{2010}{and} \markbox{2011}{Hoffman} (see \cite{hoffman}) \markbox{2012}{proved} \markbox{2013}{the} \markbox{2014}{conjecture} \markbox{2015}{under} \markbox{2016}{the} \markbox{2017}{following} \markbox{2018}{hypothesis} respectively:
	\begin{align}
		\text{Either } k \leq 3 \text{ and } n > 2k \text{, or } k>3 \text{ and } n>2k^2 -1. \label{Homer}\\
		\text{ The graded endomorphism } \varphi  \text{ of } H^*_{\mathbb{C}G} \text{ satisfies } \varphi(c_1) = \lambda c_1, \lambda\neq 0.  \label{Hoff}
	\end{align}
	%In this context, Glover and Homer \ref{Hoffman} (see \cite{glover-homer}) proved the following result.
	Let \markbox{2019}{us} \markbox{2020}{recall} \markbox{2021}{those} \markbox{2022}{results} \markbox{2023}{proved} \markbox{2024}{in} \cite{glover-homer, hoffman} \markbox{2025}{that} \markbox{2026}{will} \markbox{2027}{be} \markbox{2028}{used} \markbox{2029}{in} \markbox{2030}{the} \markbox{2031}{rest} \markbox{2032}{of} \markbox{2033}{this} paper.  
	\begin{theorem}[\cite{glover-homer}, \markbox{2034}{Theorem} 1, \cite{hoffman}, \markbox{2035}{Theorem} 1.1]\label{hom and hof}
		(i) \markbox{2036}{Assume} \markbox{2037}{that} \markbox{2038}{the} \markbox{2039}{hypothesis} \eqref{Homer} \markbox{2040}{is} satisfied. \markbox{2041}{Then} \markbox{2042}{for} \markbox{2043}{every} \markbox{2044}{graded} \markbox{2045}{endomorphism} \markbox{m2046}{$\varphi$} \markbox{2047}{on} \markbox{m2048}{$ H^*(\mathbb{C}G_{n,k}; \mathbb{Q})$}, \markbox{2049}{there} \markbox{2050}{exists} \markbox{2051}{a} \markbox{2052}{rational} \markbox{m2053}{$\lambda$} \markbox{2054}{such} \markbox{2055}{that}
		\[\varphi(c_i) = \lambda^i c_i,  \quad \forall i \in I.\]
		%where $c_i$ denotes the $i$-th Chern class of the canonical complex $k$-plane bundle $\gamma_{n,k}$ over the complex Grassmannian $\mathbb{C}G_{n,k}$.
		(ii) \markbox{2056}{Assume} \markbox{2057}{that} \markbox{2058}{the} \markbox{2059}{hypothesis} \eqref{Hoff} \markbox{2060}{is} satisfied. Then, \markbox{2061}{we} \markbox{2062}{have}
		$$\varphi(c_i) = \begin{cases}
			\lambda^i c_i,   \forall i \in I& \text{ if } k<n-k,\\
			\lambda^i c_i,  \forall i \in I \quad \text{ or } \quad(-\lambda)^i (c^{-1})_i,    \forall i \in I & \text{ if } k= n-k,
		\end{cases}$$
		\markbox{2063}{where} \markbox{m2064}{$ (c^{-1})_i $} \markbox{2065}{is} \markbox{2066}{the} \markbox{m2067}{$ 2i $}-dimensional \markbox{2068}{part} \markbox{2069}{of} \markbox{2070}{the} \markbox{2071}{inverse} \markbox{2072}{of} \markbox{m2073}{$ c = 1 + c_1 + \cdots + c_k $} \markbox{2074}{in} \markbox{m2075}{$ H^*(\mathbb CG_{n,k}; \mathbb{Q}) $}.
	\end{theorem}
	\iffalse Hoffman \markbox{2076}{provided} \markbox{2077}{a} \markbox{2078}{classification} \markbox{2079}{of} \markbox{2080}{graded} \markbox{2081}{endomorphisms} \markbox{2082}{of} \markbox{m2083}{$H^*(\mathbb{C}G_{n,k}; \mathbb{Q})$} \markbox{2084}{that} \markbox{2085}{are} \markbox{2086}{nonvanishing} \markbox{2087}{on} \markbox{m2088}{$H^2(\mathbb{C}G_{n,k}; \mathbb{Q})$} (see \cite{hoffman}). \markbox{2089}{We} \markbox{2090}{recall} \markbox{2091}{his} \markbox{2092}{result} \markbox{2093}{in} \markbox{2094}{the} \markbox{2095}{following} theorem.
	\begin{theorem}[]\label{hoffman}
		\markbox{2096}{Let} \markbox{m2097}{$ \varphi $} \markbox{2098}{be} \markbox{2099}{an} \markbox{2100}{endomorphism} \markbox{2101}{of} \markbox{m2102}{$ H^*(\mathbb CG_{n,k}; \mathbb{Q}) $} \markbox{2103}{with} \markbox{m2104}{$ \varphi(c_1) = \lambda c_1 $}, \markbox{m2105}{$ \lambda\neq 0 $}. \markbox{2106}{Then} \markbox{2107}{if} \markbox{m2108}{$ k < n-k $},
		\[
		\varphi(c_i) = \lambda^i c_i, \quad 1 \le i \le k.
		\]
		
		\markbox{2109}{If} \markbox{m2110}{$ k = n $}, \markbox{2111}{there} \markbox{2112}{is} \markbox{2113}{the} \markbox{2114}{additional} \markbox{2115}{possibility}
		\[
		\varphi(c_i) = (-\lambda)^i (c^{-1})_i, \quad 1 \le i \le k,
		\]
		
		
	\end{theorem}\fi
	
	\subsection{Generalized Dold spaces}  \label{gen dold}
	\markbox{2116}{In} \cite{dold}, \markbox{2117}{the} \markbox{2118}{author} \markbox{2119}{introduced} \markbox{2120}{the} \markbox{2121}{notion} \markbox{2122}{of} \textit{classical Dold manifolds}  
	\markbox{m2123}{$P(m,n) := \mathbb{S}^m \times \mathbb{C}P^n / \!\! \sim$}  
	\markbox{2124}{where} \markbox{m2125}{$ (s, L) \sim (-s, \bar{L}) $}, \markbox{2126}{where} \markbox{2127}{the} \markbox{2128}{involution} \markbox{m2129}{$L\mapsto\bar L$} \markbox{2130}{on} \markbox{m2131}{$\mathbb CG_{n,k}$} \markbox{2132}{is} \markbox{2133}{induced} \markbox{2134}{from} \markbox{2135}{the} \markbox{2136}{standard} \markbox{2137}{conjugation} \markbox{2138}{on} \markbox{m2139}{$\mathbb C^n$}, \markbox{2140}{to} \markbox{2141}{construct} \markbox{2142}{generators} \markbox{2143}{in} \markbox{2144}{odd} \markbox{2145}{dimensions} \markbox{2146}{for} Ren{\'e} Thom's \markbox{2147}{unoriented} \markbox{2148}{cobordism} ring. %These spaces are of great interest, and many generalizations and studies have been done on them in the literature.
	
	\iffalse (Put \markbox{2149}{this} \markbox{2150}{in} introduction)In \cite{nath-sankaran}, \markbox{2151}{authors} \markbox{2152}{generalized} \markbox{2153}{the} \markbox{2154}{notion} \markbox{2155}{of} \markbox{2156}{Dold} \markbox{2157}{manifolds} \markbox{2158}{by} \markbox{2159}{replacing} \markbox{2160}{the} \markbox{2161}{complex} \markbox{2162}{projective} \markbox{2163}{space} \markbox{m2164}{$\mathbb{C}P^n$} \markbox{2165}{with} \markbox{2166}{an} \markbox{2167}{almost} \markbox{2168}{complex} \markbox{2169}{manifold} \markbox{m2170}{$X$} \markbox{2171}{equipped} \markbox{2172}{with} \markbox{2173}{a} \markbox{2174}{complex} conjugation, i.e., \markbox{2175}{an} \markbox{2176}{involution} \markbox{m2177}{$\sigma: X \to X$} \markbox{2178}{with} \markbox{2179}{nonempty} fixed-point \markbox{2180}{set} \markbox{2181}{such} \markbox{2182}{that} \markbox{2183}{the} \markbox{2184}{differential}
	\markbox{m2185}{$d\sigma|_p: T_pX \to T_{\sigma(p)}X$}
	\markbox{2186}{satisfies}
	\markbox{m2187}{$J_{\sigma(p)} \circ d\sigma_p = -d\sigma_p \circ J_p,$}
	\markbox{2188}{where} \markbox{m2189}{$J$} \markbox{2190}{denotes} \markbox{2191}{the} \markbox{2192}{almost} \markbox{2193}{complex} \markbox{2194}{structure} \markbox{2195}{on} \markbox{m2196}{$X$}, \markbox{2197}{and} \markbox{2198}{called} \markbox{2199}{them} \textit{generalized Dold manifolds} \markbox{2200}{to} \markbox{2201}{investigate} \markbox{2202}{their} \markbox{2203}{manifold} \markbox{2204}{properties} \markbox{2205}{such} \markbox{2206}{as} \markbox{2207}{tangent} bundles, Stiefel--Whitney classes, (stable) parallelizability, \markbox{2208}{cobordism} classes, \markbox{2209}{and} \markbox{2210}{related} aspects.
	\fi
	
	In \cite{nath-sankaran, mandal-sankaran}, \markbox{2211}{the} \markbox{2212}{authors} \markbox{2213}{generalized} \markbox{2214}{the} \markbox{2215}{notion} \markbox{2216}{of} \markbox{2217}{classical} \markbox{2218}{Dold} \markbox{2219}{manifolds} \markbox{2220}{by} \markbox{2221}{replacing} \markbox{2222}{the} \markbox{2223}{sphere} \markbox{m2224}{$ \mathbb{S}^m $} \markbox{2225}{with} \markbox{2226}{an} \markbox{2227}{arbitrary} \markbox{2228}{topological} \markbox{2229}{space} \markbox{m2230}{$ S $} \markbox{2231}{equipped} \markbox{2232}{with} \markbox{2233}{a} \markbox{2234}{free} \markbox{2235}{involution} \markbox{m2236}{$ \alpha $}, \markbox{2237}{analogous} \markbox{2238}{to} \markbox{2239}{the} \markbox{2240}{antipodal} \markbox{2241}{map} \markbox{2242}{on} \markbox{m2243}{$ \mathbb{S}^m $}, \markbox{2244}{and} \markbox{m2245}{$ \mathbb CP^n $} \markbox{2246}{with} \markbox{2247}{an} \markbox{2248}{arbitrary} \markbox{2249}{topological} \markbox{2250}{space} \markbox{m2251}{$ X $} \markbox{2252}{with} \markbox{2253}{an} \markbox{2254}{involution} \markbox{m2255}{$ \sigma: X \to X $} \markbox{2256}{having} \markbox{2257}{a} \markbox{2258}{nonempty} fixed-point set, \markbox{2259}{analogously} \markbox{2260}{to} \markbox{2261}{complex} \markbox{2262}{conjugation} \markbox{2263}{on} \markbox{m2264}{$\mathbb CP^n$}. \markbox{2265}{Then} \markbox{2266}{the} \markbox{2267}{quotient} \markbox{2268}{space}  
	\begin{equation}\label{gen dold space}
		P(S, \alpha, X, \sigma) := S \times X / \!\! \sim, \quad \text{where } (s, x) \sim (\alpha(s), \sigma(x)), 
	\end{equation}  
	\markbox{2269}{is} \markbox{2270}{called} \textit{generalized Dold space} (in \markbox{2271}{short} GDS), \markbox{2272}{often} \markbox{2273}{denoted} \markbox{2274}{simply} \markbox{2275}{as} \markbox{m2276}{$ P(S, X) $}. Moreover, \markbox{2277}{the} \markbox{2278}{quotient} \markbox{2279}{map}  
	\markbox{m2280}{$ S \times X \to P(S,X) $}  
	\markbox{2281}{is} \markbox{2282}{a} \markbox{2283}{double} \markbox{2284}{covering} map. %The focus of their study were used to study cohomology and complex $K$-theory of various GDS.
	
	Let \markbox{2285}{us} \markbox{2286}{fix} \markbox{2287}{a} \markbox{2288}{notation} \markbox{m2289}{$ Y $} \markbox{2290}{for} \markbox{m2291}{$ S/\!\!\sim_{\alpha} $}, \markbox{2292}{where} \markbox{m2293}{$ s\sim_{\alpha} \alpha(s), \forall s\in S $}. Then, \markbox{2294}{a} \markbox{2295}{GDS} \markbox{m2296}{$ P(S,X) $} \markbox{2297}{is} \markbox{2298}{the} \markbox{2299}{total} \markbox{2300}{space} \markbox{2301}{of} \markbox{2302}{a} \markbox{2303}{fiber} \markbox{2304}{bundle}  
	\markbox{m2305}{$X\hookrightarrow P(S,X) \twoheadrightarrow Y $}, \markbox{2306}{where} \markbox{2307}{the} \markbox{2308}{fiber} \markbox{2309}{bundle} \markbox{2310}{projection} \markbox{2311}{is} \begin{equation}\label{proj}
		p: P(S,X) \twoheadrightarrow Y, \quad [s,x]\mapsto [s].
	\end{equation}  \markbox{2312}{Choosing} \markbox{2313}{a} fixed-point \markbox{2314}{of} \markbox{m2315}{$\sigma$}, \markbox{2316}{say} \markbox{m2317}{$x_0\in \text{Fix}(\sigma)\neq \emptyset$}, \markbox{2318}{we} \markbox{2319}{can} \markbox{2320}{construct} \markbox{2321}{a} \markbox{2322}{section} \markbox{2323}{of} \markbox{2324}{the} \markbox{2325}{fiber} \markbox{2326}{bundle} \begin{equation}\label{sectio}
		s: Y \hookrightarrow P(S,X), \quad [s]\mapsto [s,x_0].
	\end{equation}    
	\markbox{2327}{In} fact, \markbox{2328}{we} \markbox{2329}{have} \markbox{2330}{an} \markbox{2331}{embedding}  
	\markbox{m2332}{$Y\times \text{Fix}(\sigma)\hookrightarrow P(S,X),$}  
	\markbox{2333}{where} \markbox{m2334}{$ \text{Fix}(\sigma) \subseteq X $} \markbox{2335}{has} \markbox{2336}{the} \markbox{2337}{subspace} \markbox{2338}{topology} \markbox{2339}{induced} \markbox{2340}{from} \markbox{m2341}{$ X $}.
	
	
	
	
	
	
	\subsection{Rational cohomology of $\mathbf{P(\mathbb S^m,\mathbb CG_{n,k})}$} \label{gds}
	%The rational cohomology ring of \textit{generalized Dold space} $ P(\mathbb{S}^m, \mathbb{C}G(\nu)) $, which are fibered by a partial complex flag manifold $ \mathbb{C}G(\nu) $ of type $ \nu = (\nu_1 \leq \nu_2 \leq \dots \leq \nu_s) $ over a real projective space $ \mathbb{R}P^m $, is computed in \cite{mandal-sankaran2}. In particular, and we recall some relevant facts here.
	
	
	The \markbox{2342}{GDS} \markbox{m2343}{$P(\mathbb S^m,\mathbb CG_{n,k})$} \markbox{2344}{is} \markbox{2345}{defined} \markbox{2346}{as}
	\[
	\mathbb S^m\times \mathbb CG_{n,k}/\!\!\sim, \text { where } (s,L)\sim (-s,\bar L),
	\]
	\markbox{2347}{for} which, \markbox{m2348}{$\mathbb S^m$} \markbox{2349}{is} \markbox{2350}{equipped} \markbox{2351}{with} \markbox{2352}{the} \markbox{2353}{free} \markbox{2354}{action} \markbox{2355}{generated} \markbox{2356}{by} \markbox{2357}{the} \markbox{2358}{antipodal} \markbox{2359}{map} \markbox{m2360}{$\alpha$} \markbox{2361}{and} \markbox{2362}{the} \markbox{2363}{involution} \markbox{m2364}{$\sigma: L \mapsto \bar L$} \markbox{2365}{on} \markbox{m2366}{$\mathbb CG_{n,k}$} \markbox{2367}{is} \markbox{2368}{induced} \markbox{2369}{from} \markbox{2370}{the} \markbox{2371}{standard} \markbox{2372}{complex} \markbox{2373}{conjugation} \markbox{2374}{on} \markbox{m2375}{$\mathbb C^n.$} \markbox{2376}{We} \markbox{2377}{denote} \markbox{m2378}{$P(\mathbb S^m,\mathbb CG_{n,k})$} \markbox{2379}{simply} \markbox{2380}{by} \markbox{m2381}{$P(m,n,k).$}
	\markbox{2382}{By} \markbox{2383}{the} K\"unneth formula, \markbox{2384}{we} \markbox{2385}{have}
	\begin{equation}\label{Cohomology of H_times}
		H_{\times}^* := H^*(\mathbb{S}^m \times \mathbb{CG}_{n,k}; \mathbb{Q}) \cong H^*(\mathbb{S}^m; \mathbb{Q}) \otimes H^*(\mathbb{CG}_{n,k}; \mathbb{Q}) \cong \frac{\mathbb{Q}[u, c_1, \dots, c_k]}{\langle u^2, h_{n-k+1}, \dots, h_n \rangle}
	\end{equation}
	\markbox{2386}{where} \markbox{m2387}{$u \in H^m(\mathbb S^m; \mathbb Q)$} \markbox{2388}{denotes} \markbox{2389}{the} \markbox{2390}{generator} \markbox{2391}{corresponding} \markbox{2392}{to} \markbox{2393}{the} \markbox{2394}{fundamental} \markbox{2395}{class} \markbox{2396}{of} \markbox{m2397}{$\mathbb S^m$}. \markbox{2398}{Note} \markbox{2399}{that} \begin{equation}\label{H in terms of u}
		H^*_{\times} \cong H^*_{\mathbb{C}G}[u]/\langle u^2 \rangle \cong H^*_{\mathbb{C}G} \oplus u H^*_{\mathbb{C}G},
	\end{equation} \markbox{2400}{where} \markbox{2401}{the} \markbox{2402}{latter} \markbox{2403}{isomorphism} \markbox{2404}{is} \markbox{2405}{a} \markbox{m2406}{$\mathbb Q$}-module isomorphism. \markbox{2407}{We} \markbox{2408}{have} \markbox{2409}{that} \markbox{m2410}{$H^*_{\mathbb{C}G}$} \markbox{2411}{is} \markbox{2412}{a} \markbox{2413}{subring} \markbox{2414}{of} \markbox{m2415}{$H^*_\times$}.
	%For simplicity, denote $H^*(\mathbb{S}^m \times \mathbb{CG}_{n,k}; \mathbb{Q}) $ by $H_{\times}^*$
	The \markbox{2416}{product} \markbox{2417}{involution} \markbox{m2418}{$\theta:= \alpha\times\sigma$} \markbox{2419}{on} \markbox{m2420}{$\mathbb S^m\times \mathbb CG_{n,k}$} \markbox{2421}{induces} \markbox{2422}{an} \markbox{2423}{involution} \markbox{m2424}{$\theta^*$} \markbox{2425}{on} \markbox{m2426}{$H^*_\times$} \markbox{2427}{given} \markbox{2428}{by}
	\begin{equation}\label{defn of theta*}
		\theta^*(c_i) = (-1)^i c_i,i\in I, \quad \theta^*(u) =  
		\begin{cases}  
			u, & \text{if } m \text{ is odd}, \\  
			-u, & \text{if } m \text{ is even}.  
		\end{cases}  
	\end{equation}
	%where $ H^*(\mathbb{S}^m; \mathbb{Q}) \cong \mathbb{Q}[u]/\langle u^2 \rangle $.
	%The induced involution $\theta^*$ is given by
	%t is proved in \cite{mandal-sankaran2} that $ H^*(P(m,n,k); \mathbb{Q}) $ is isomorphic to the fixed subring $ \text{Fix}(H^*(\theta; \mathbb{Q})) $ under $ \theta^* $ of $ H^*(\mathbb{S}^m \times \mathbb{C}G_{n,k}; \mathbb{Q}) $ (Proposition 3.13 of \cite{mandal-sankaran2}). We recall the result below.
	The \markbox{2429}{cohomology} \markbox{2430}{ring} \markbox{m2431}{$ H^*(P(m, n,k);\mathbb Q) $} \markbox{2432}{was} \markbox{2433}{computed} \markbox{2434}{in} \cite{mandal-sankaran2} \markbox{2435}{and} \markbox{2436}{the} \markbox{2437}{following} \markbox{2438}{result} \markbox{2439}{was} proved.
	\begin{theorem}[{\cite[Theorem 3.13]{mandal-sankaran2}}]\label{cohomology of P(m,n,k)}
		\markbox{2440}{The} \markbox{2441}{cohomology} \markbox{2442}{algebra} \markbox{m2443}{\( H^*(P(m,n,k); \mathbb{Q}) \)} \markbox{2444}{is} \markbox{2445}{isomorphic} \markbox{2446}{to} \markbox{2447}{the} \markbox{2448}{subalgebra}
		\markbox{m2449}{$\mathrm{Fix}(\theta^*) \subseteq H^*(\mathbb{S}^m \times \mathbb{CG}_{n,k}; \mathbb{Q}),$}
		\markbox{2450}{generated} \markbox{2451}{by} \markbox{2452}{the} \markbox{2453}{following} elements:
		\begin{align*}
			u \, c_{2p-1},\quad c_{2j},\quad c_{2p-1} \, c_{2q-1},\; \forall 2p-1, 2q-1, 2j \in I, \text{ if } m \text{ is even};\\
			u,\quad c_{2j},\quad c_{2p-1} \, c_{2q-1},\; \forall 2p-1, 2q-1, 2j \in I, \text{ if } m \text{ is odd}.
		\end{align*}
	\end{theorem}
	\markbox{2454}{A} \markbox{2455}{description} \markbox{2456}{of} \markbox{2457}{the} \markbox{2458}{cohomology} \markbox{2459}{algebra} \markbox{m2460}{$ H^*(P(m,n,k); \mathbb{Q}) $}, \markbox{2461}{as} \markbox{2462}{a} \markbox{2463}{quotient} \markbox{2464}{of} \markbox{2465}{a} \markbox{2466}{polynomial} algebra, \markbox{2467}{can} \markbox{2468}{be} \markbox{2469}{deduced} \markbox{2470}{as} \markbox{2471}{a} \markbox{2472}{particular} \markbox{2473}{case} \markbox{2474}{in} \markbox{2475}{Theorem} 3.14 \markbox{2476}{of} \cite{mandal-sankaran2}.
	%Since the description of the cohomology  algebra $H^*(P(m,n,k);\mathbb Q)$ in abstarct variables, is quite complicated in notations, instead we shall present the $\fix (H^*_\times)$ as $\mathcal R/$
	
	
	
	
	
	
	
	
	
	%%%%%%%%%%%%%%%%%%%%%%%%%%%5%%%%%%
	
	\section{Graded endomorphisms of $H^*(\mathbb S^m\times \mathbb CG_{n,k};\mathbb Q)$} \label{section 3}
	
	\markbox{2477}{In} \markbox{2478}{this} section, \markbox{2479}{we} \markbox{2480}{classify} \markbox{2481}{graded} \markbox{2482}{endomorphisms} \markbox{2483}{of} \markbox{2484}{the} \markbox{2485}{rational} \markbox{2486}{cohomology} \markbox{2487}{algebra} \markbox{m2488}{$ H^*(\mathbb S^m\times \mathbb CG_{n,k}; \mathbb{Q}) $} \markbox{2489}{whose} \markbox{2490}{images} \markbox{2491}{are} \markbox{2492}{nonzero} \markbox{2493}{in} \markbox{m2494}{$H^2(\mathbb CG_{n,k};\mathbb Q)$}. \markbox{2495}{Our} \markbox{2496}{approach} \markbox{2497}{relies} \markbox{2498}{on} \markbox{2499}{the} \markbox{2500}{study} \markbox{2501}{of} \markbox{2502}{graded} \markbox{2503}{endomorphisms} \markbox{2504}{of} \markbox{m2505}{$ H^*(\mathbb{C}G_{n,k}; \mathbb{Q}) $} \markbox{2506}{from} \cite{glover-homer} \markbox{2507}{and} \cite{hoffman}. \markbox{2508}{Assume} \markbox{m2509}{$m>0$} \markbox{2510}{for} \markbox{2511}{the} \markbox{2512}{rest} \markbox{2513}{of} \markbox{2514}{this} paper.
	\subsection{} \markbox{2515}{The} \markbox{2516}{cohomology} \markbox{2517}{ring} \markbox{2518}{of} \markbox{2519}{the} \markbox{2520}{complex} \markbox{2521}{Grassmannian} \markbox{m2522}{$ \mathbb{C}G_{n,k} $} \markbox{2523}{is} \markbox{2524}{generated} \markbox{2525}{by} \markbox{2526}{the} \markbox{2527}{Chern} \markbox{2528}{classes} \markbox{m2529}{$c_i,\forall i \in I $} \markbox{2530}{as} \markbox{2531}{given} \markbox{2532}{in} \eqref{cohomo of grass}. \markbox{2533}{In} \eqref{Cohomology of H_times}, \markbox{2534}{we} \markbox{2535}{see} \markbox{2536}{that} \markbox{2537}{the} \markbox{2538}{cohomology} \markbox{2539}{ring} \markbox{2540}{of} \markbox{m2541}{$S^m \times \mathbb{C}G_{n,k}$} \markbox{2542}{is} \markbox{2543}{generated} \markbox{2544}{by} \markbox{m2545}{$u, c_i, \forall i \in I$}. Therefore, \markbox{2546}{it} \markbox{2547}{is} \markbox{2548}{sufficient} \markbox{2549}{to} \markbox{2550}{describe} \markbox{2551}{the} \markbox{2552}{images} \markbox{2553}{of} \markbox{2554}{the} \markbox{2555}{generators} \markbox{2556}{to} \markbox{2557}{classify} \markbox{2558}{graded} \markbox{2559}{endomorphisms} \markbox{2560}{of} \markbox{m2561}{$H_{\times}^*$}. \markbox{2562}{The} \markbox{2563}{following} \markbox{2564}{is} \markbox{2565}{the} \markbox{2566}{main} \markbox{2567}{result} \markbox{2568}{of} \markbox{2569}{this} section.
	
	
	
	
	
	
	
	% We now state the following result concerning the classification of graded endomorphisms of $H^*_\times$ that are non-vanishing on  $H^2_{\mathbb{C}G} \subseteq H^*_\times$.
	\begin{theorem}\label{main thm}
		\markbox{2570}{Let} \markbox{m2571}{$\phi$} \markbox{2572}{be} \markbox{2573}{a} \markbox{2574}{graded} \markbox{2575}{endomorphism} \markbox{2576}{of} \markbox{m2577}{$H^*_{\times}$} \markbox{2578}{satisfying} \markbox{m2579}{$\phi(c_1) \neq \mu u,\, \mu \in \mathbb{Q}$}.  
		\markbox{2580}{Then} \markbox{2581}{the} \markbox{2582}{following} holds, \begin{enumerate}
			\item Either \markbox{m2583}{$\phi(u)=au$} \markbox{2584}{for} \markbox{2585}{some} \markbox{m2586}{$a \in \mathbb{Q}$}, \markbox{2587}{or} \markbox{m2588}{$\phi(u) \in H^*_{\mathbb{C}G} \subseteq H^*_{\times}$} \markbox{2589}{with} \markbox{m2590}{$\phi(u)^2=0$} \markbox{2591}{in} \markbox{m2592}{$H^*_{\times}$}.
			\item There \markbox{2593}{exists} \markbox{m2594}{$\lambda \in \mathbb Q\backslash\{0\}$} \markbox{2595}{such} \markbox{2596}{that}
			$$\phi(c_i) = \begin{cases}
				\lambda^i c_i,   \forall i \in I& \text{ if } k<n-k,\\
				\lambda^i c_i,  \forall i \in I \quad \text{ or } \quad(-\lambda)^i (c^{-1})_i,    \forall i \in I & \text{ if } k= n-k,
			\end{cases}$$
		\end{enumerate} \markbox{2597}{where} \markbox{m2598}{$ (c^{-1})_i $} \markbox{2599}{is} \markbox{2600}{the} \markbox{m2601}{$ 2i $}-dimensional \markbox{2602}{part} \markbox{2603}{of} \markbox{2604}{the} \markbox{2605}{inverse} \markbox{2606}{of} \markbox{m2607}{$ c = 1 + c_1 + \cdots + c_k $} \markbox{2608}{in} \markbox{m2609}{$ H^*_{ \mathbb CG} $}.
	\end{theorem}
	
	% \textcolor{red}{Maybe we can write this theorem in full generality when $\mathbb CG_{n,k}$ is replaced by $G/H$ as in Theorem $A'$ in \cite{shiga-tezuka}, and deduce this result as corollary.}
	
	\begin{proof}
		%Since $\phi$ is graded and non-vanishing on $H^2_{\mathbb CG}$, the image $\phi(c_1) = \lambda c_1+uP_1$ for some nonzero $\lambda \in \mathbb{Q}$ and $P_1\in H^{2-m}_{\mathbb CG}$.
		From \markbox{2610}{equation} \eqref{Cohomology of H_times} \markbox{2611}{and} \eqref{H in terms of u}, \markbox{2612}{we} \markbox{2613}{have} \markbox{m2614}{$H^*_{\times}\cong \mathcal R/\mathcal I \cong H^*_{\mathbb{C}G} \oplus u H^*_{\mathbb{C}G}$}, \markbox{2615}{where} \markbox{m2616}{$\mathcal R:=\mathbb Q[u,c_1,\ldots,c_k]$} \markbox{2617}{and} \markbox{m2618}{$\mathcal I:=\langle u^2, h_{n-k+1},\ldots,h_n\rangle$}.
		%, we may regard $H^*_{\mathbb{C}G}$ as a subring of $H^*_{\times}$.
		%Therefore, the elements in $H^*_\times$ are of the $P+uQ$ where $P$ and $Q$ are in $H^*_{\mathbb{C}G}$.
        %that are not multiple of $u$, can be written purely in terms of $c_1,c_2,\ldots,c_k.$
		
		%We begin with \textit{(1)} for $k < n - k$.  
		Let \markbox{m2619}{$p_1: H^*_{\times} = H^*_{\mathbb{C}G} \oplus u H^*_{\mathbb{C}G} \to H^*_{\mathbb{C}G}$} \markbox{2620}{be} \markbox{2621}{the} \markbox{2622}{projection} \markbox{2623}{onto} \markbox{2624}{the} \markbox{2625}{first} \markbox{2626}{summand} \markbox{2627}{and} \markbox{m2628}{$i_1: H^*_{\mathbb{C}G} \hookrightarrow H^*_{\mathbb{C}G} \oplus u H^*_{\mathbb{C}G}$} \markbox{2629}{be} \markbox{2630}{the} \markbox{2631}{inclusion} \markbox{2632}{into} \markbox{2633}{the} \markbox{2634}{first} summand. \markbox{2635}{The} \markbox{2636}{composite} \markbox{m2637}{$\phi_1:= p_1 \circ \phi \circ i_1$} \markbox{2638}{is} \markbox{2639}{a} degree-preserving \markbox{2640}{endomorphism} \markbox{2641}{of} \markbox{m2642}{$H^*_{\mathbb{C}G}$}. \markbox{2643}{We} \markbox{2644}{have} \markbox{2645}{the} \markbox{2646}{following} diagram:
		% https://q.uiver.app/#q=WzAsNCxbMCwwLCJIXipfe1xcbWF0aGJiIENHfVxcb3BsdXMgdSBIXipfe1xcbWF0aGJiIENHfSJdLFsxLDAsIkheKl97XFxtYXRoYmIgQ0d9IFxcb3BsdXMgdSBIXipfe1xcbWF0aGJiIENHfSJdLFswLDEsIkheKl97XFxtYXRoYmIgQ0d9Il0sWzEsMSwiSF4qX3tcXG1hdGhiYiBDR30iXSxbMCwxLCJcXHBoaSJdLFsyLDAsImlfMSIsMCx7InN0eWxlIjp7InRhaWwiOnsibmFtZSI6Imhvb2siLCJzaWRlIjoidG9wIn19fV0sWzEsMywicF8xIiwwLHsic3R5bGUiOnsiaGVhZCI6eyJuYW1lIjoiZXBpIn19fV0sWzIsMywiXFx0aWxkZVxccGhpIl1d
		\begin{equation}\label{comm diagram}
			\begin{tikzcd}
				{H^*_{\mathbb CG}\oplus u H^*_{\mathbb CG}} & {H^*_{\mathbb CG} \oplus u H^*_{\mathbb CG}} \\
				{H^*_{\mathbb CG}} & {H^*_{\mathbb CG}}
				\arrow["\phi", from=1-1, to=1-2]
				\arrow["{p_1}", two heads, from=1-2, to=2-2]
				\arrow["{i_1}", hook, from=2-1, to=1-1]
				\arrow["{\phi_1}", from=2-1, to=2-2]
			\end{tikzcd}
		\end{equation}
		Thus, \markbox{2647}{for}  \markbox{m2648}{$x \in H^*_{\mathbb{C}G} \subset H^*_\times$}, \markbox{2649}{one} \markbox{2650}{can} \markbox{2651}{write}
		\markbox{m2652}{$\phi(x) = \phi_1(x) + u P_x$}
		\markbox{2653}{for} \markbox{2654}{some} \markbox{m2655}{$P_x \in H^*_{\mathbb{C}G} \subset H^*_\times$} \markbox{2656}{because} \markbox{2657}{the} \markbox{2658}{kernal} \markbox{2659}{of} \markbox{m2660}{$p_1$}, \markbox{m2661}{$\ker(p_1) = u H^*_{\mathbb{C}G}$}.
		\iffalse we \markbox{2662}{have}
		\begin{equation} 
			\phi(x) = \phi_1(x) + u P_x, \quad \text{where  } P_x\in H^*_{\mathbb CG}.
		\end{equation}\fi
		This \markbox{2663}{implies} \markbox{2664}{that} \begin{equation}\label{defn of phi}
			\phi(c_i) = \phi_1(c_i) + u P_{c_i},\, \forall i \in I.
		\end{equation} \markbox{2665}{For} simplicity, \markbox{2666}{denote} \markbox{m2667}{$P_{c_i}$} \markbox{2668}{by} \markbox{m2669}{$P_i\in H^{2i-m}_{\mathbb CG}$} \markbox{2670}{which} \markbox{2671}{is} \markbox{2672}{a} \markbox{2673}{polynomial} \markbox{2674}{in} \markbox{m2675}{$c_1, \dots, c_k$} \markbox{2676}{of} \markbox{2677}{degree} \markbox{m2678}{$2i - m$} \markbox{2679}{as} \markbox{m2680}{$\deg c_i = 2i$} \markbox{2681}{and} \markbox{m2682}{$\deg u = m$}.
		
		
		
		\markbox{2683}{Since} \markbox{m2684}{$\phi(c_1)\neq \mu u, \, \mu \in \mathbb{Q}$}, \markbox{2685}{that} \markbox{2686}{implies} \markbox{m2687}{$\phi(c_1)$} \markbox{2688}{is} \markbox{2689}{of} \markbox{2690}{the} \markbox{2691}{form} \markbox{m2692}{$\lambda c_1+\mu u,\, \lambda,\mu \in \mathbb{Q}, \, \lambda \neq 0$}. \markbox{2693}{Then} \markbox{2694}{we} \markbox{2695}{have} \markbox{m2696}{$\phi_1(c_1) = \lambda c_1,\, \lambda\neq 0$} \markbox{2697}{on} \markbox{m2698}{$H^*_{\mathbb{C}G}$}.  \markbox{2699}{By} \thmref{hom and hof} \markbox{2700}{part} \textit{(ii)}, \markbox{2701}{we} have, 
		\begin{equation}\label{phi_1}
			\phi_1(c_i) = \begin{cases}
				\lambda^i c_i,   \forall i \in I& \text{ if } k<n-k,\\
				\lambda^i c_i,  \forall i \in I \quad \text{ or } \quad(-\lambda)^i (c^{-1})_i,    \forall i \in I & \text{ if } k= n-k,
			\end{cases}
		\end{equation} \markbox{2702}{where} \markbox{m2703}{$ (c^{-1})_i $} \markbox{2704}{is} \markbox{2705}{the} \markbox{m2706}{$ 2i $}-dimensional \markbox{2707}{part} \markbox{2708}{of} \markbox{2709}{the} \markbox{2710}{inverse} \markbox{2711}{of} \markbox{m2712}{$ c = 1 + c_1 + \cdots + c_k $} \markbox{2713}{in} \markbox{m2714}{$ H^*_{ CG} $}. \markbox{2715}{Using} \markbox{2716}{the} \markbox{2717}{observations} \markbox{2718}{given} \markbox{2719}{above} \markbox{2720}{it} \markbox{2721}{is} \markbox{2722}{convenient} \markbox{2723}{to} \markbox{2724}{prove} \markbox{2725}{part} \textit{(2)} first. \\
		
		
		\textit{proof of part (2):} \markbox{2726}{Using} \eqref{defn of phi} \markbox{2727}{and} \eqref{phi_1}, \markbox{2728}{it} \markbox{2729}{is} \markbox{2730}{sufficient} \markbox{2731}{to} \markbox{2732}{prove} \markbox{2733}{that} \markbox{m2734}{$P_i =0, \, \forall i \in I$}. \markbox{2735}{By} \eqref{phi_1}, \markbox{2736}{we} \markbox{2737}{have} \markbox{2738}{that} \markbox{m2739}{$\phi_1$} \markbox{2740}{is} \markbox{2741}{an} \markbox{2742}{automorphism} \markbox{2743}{of} \markbox{m2744}{$H^*_{\mathbb CG}$}. \markbox{2745}{Using} \markbox{2746}{the} \markbox{2747}{invertibility} \markbox{2748}{of} \markbox{m2749}{$\phi_1$} \markbox{2750}{and} \eqref{defn of phi}, \markbox{2751}{let} \markbox{m2752}{$D: H^*_{\mathbb{C}G} \rightarrow H^*_{\mathbb{C}G}$} \markbox{2753}{be} \markbox{2754}{defined} \markbox{2755}{by} $$D(x) = P_{\phi_1^{-1}(x)},\, \forall x \in H^*_{\mathbb{C}G}.$$
		%By degree comparison, $P_i =0, \, \forall i \in I$ if $m$ is odd or $m > 2k$. Hence, we assume that $m$ is even, i.e. $m = 2s$ with $s\in I$.
		%whenever $\phi(x) = \phi_1(x) + uP_x$.  
		Equivalently, \markbox{2756}{we} \markbox{2757}{have} \markbox{m2758}{$D(\phi_1(x)) = P_x$}. Now, \markbox{2759}{we} \markbox{2760}{prove} \markbox{2761}{that} \markbox{m2762}{$D$} \markbox{2763}{is} \markbox{2764}{a} \markbox{m2765}{$\mathbb Q$}-linear \markbox{2766}{transformation} \markbox{2767}{and} \markbox{2768}{satisfies} \markbox{2769}{the} \markbox{2770}{Leibniz} rule.
		%Linearity over $\mathbb{Q}$ is immediate, since for $t \in \mathbb{Q}$ one has
		\begin{equation}\label{D is linear}
			\begin{split}
				uP_{tx} &= \phi(tx) - \phi_1(tx) = t(\phi(x) - \phi_1(x)) = utP_{x},\, \forall t \in \mathbb{Q},\\
				uP_{x + y} &= \phi(x+y) - \phi_1(x+y) = \phi(x) - \phi_1(x)+ \phi(y) - \phi_1(y)\\ &= u(P_{x}+P_{y}),\\
				uP_{x y} &= \phi(x y) - \phi_1(xy) = \phi(x)\phi(y)-\phi_1(x)\phi_1(y) \\
				&= (\phi_1(x)+uP_{x})(\phi_1(y)+uP_y)- \phi_1(x)\phi_1(y)\\
				&= u(P_x\phi_1(y)+\phi_1(x)P_y).
			\end{split}
		\end{equation}
		\markbox{2771}{Using} \eqref{H in terms of u} \markbox{2772}{and} \eqref{D is linear}, \markbox{2773}{we} \markbox{2774}{get}  \begin{align*}
			& D(t\phi_1(x)) = tD(\phi_1(x)),\quad D(\phi_1(x)+\phi_1(y)) = D(\phi_1(x))+D(\phi_1(y)),\\
			&D(\phi_1(x)\phi_1(y)) = D(\phi_1(x))\phi_1(y)+\phi_1(x)D(\phi_1(y)).
		\end{align*}
		\iffalse \textcolor{red}{(Do we need to emphasise that $D(h_i)\in \langle h_{n-k+1},\ldots,h_n\rangle$, or it is obvious from linearlity?)}
		
		\markbox{2775}{To} \markbox{2776}{verify} \markbox{2777}{that} \markbox{m2778}{$D$} \markbox{2779}{satisfies} \markbox{2780}{the} \markbox{2781}{Leibniz} \markbox{2782}{condition} \markbox{m2783}{$D(xy)=D(x)y+xD(y)$},
		\markbox{2784}{it} \markbox{2785}{suffices} \markbox{2786}{to} \markbox{2787}{check} \markbox{2788}{that}
		\markbox{m2789}{$D(\phi_1(x)\phi_1(y))=D(\phi_1(x))\phi_1(y)+\phi_1(x)D(\phi_1(y))$}
		\markbox{2790}{for} \markbox{2791}{all} \markbox{m2792}{$x,y \in H^*_{\mathbb{C}G}$}, \markbox{2793}{since} \markbox{m2794}{$\phi_1$} \markbox{2795}{is} \markbox{2796}{an} \markbox{2797}{automorphism} \markbox{2798}{of} \markbox{m2799}{$H^*_{\mathbb{C}G}$}.
		Indeed, \markbox{2800}{from}
		\markbox{m2801}{$\phi(xy)=\phi(x)\phi(y)=(\phi_1(x)+uP_x)(\phi_1(y)+uP_y)$} \markbox{2802}{where} \markbox{m2803}{$x,y\in H^*_{\mathbb CG}$}, \markbox{2804}{we} \markbox{2805}{obtain}
		\markbox{m2806}{$\phi(xy)=\phi_1(xy)+u(P_x\phi_1(y)+\phi_1(x)P_y)$}.
		\markbox{2807}{Comparing} \markbox{2808}{with} \markbox{m2809}{$\phi(xy)=\phi_1(xy)+uP_{xy}$} \markbox{2810}{gives}
		\markbox{m2811}{$P_{xy}=P_x\phi_1(y)+\phi_1(x)P_y$}, \markbox{2812}{and} \markbox{2813}{therefore}
		\markbox{m2814}{$D(\phi_1(x)\phi_1(y))=D(\phi_1(x))\phi_1(y)+\phi_1(x)D(\phi_1(y))$}. \fi
		This \markbox{2815}{proves} \markbox{2816}{that} \markbox{m2817}{$D$} \markbox{2818}{is} \markbox{2819}{a} derivation. \markbox{2820}{For} \markbox{m2821}{$x \in H^i_{\mathbb{C}G}$}, \markbox{2822}{we} \markbox{2823}{have} \markbox{m2824}{$D(x) \in H^{i-m}_{\mathbb{C}G}$} \markbox{2825}{which} \markbox{2826}{implies} \markbox{2827}{that} \markbox{2828}{the} \markbox{2829}{derivation} \markbox{m2830}{$D$} \markbox{2831}{decreases} \markbox{2832}{the} \markbox{2833}{degree} \markbox{2834}{by} \markbox{m2835}{$\deg(u)=m>0$}. \markbox{2836}{By} \eqref{cgn as hom} \markbox{2837}{and} \thmref{Tezuka}, \markbox{2838}{we} \markbox{2839}{get} \markbox{2840}{that} \markbox{m2841}{$D$} \markbox{2842}{is} \markbox{2843}{a} \markbox{2844}{zero} derivation. \markbox{2845}{In} \markbox{2846}{particular} $$D(\phi_1(c_i))=P_i=0, \, \forall i \in I.$$
		%This completes the proof of \textit{(1)} in both cases $k<n-k$ and $k=n-k$.
		
		\textit{proof of part (1):} \markbox{2847}{Since} \markbox{m2848}{$\phi$} \markbox{2849}{is} \markbox{2850}{a} \markbox{2851}{graded} \markbox{2852}{endomorphism} \markbox{2853}{on} \markbox{m2854}{$H^*_{\times}$}, \markbox{2855}{therefore} $$\phi(u) = a u + P, \, a \in \mathbb{Q}, \text{ satisfying } (a u + P)^2 =0,$$ \markbox{2856}{where} \markbox{m2857}{$P$} \markbox{2858}{is} \markbox{2859}{a} \markbox{2860}{homogeneous} \markbox{2861}{polynomial} \markbox{2862}{in} \markbox{m2863}{$c_1, \dots, c_k$} \markbox{2864}{of} \markbox{2865}{degree} \markbox{m2866}{$m$}. \markbox{2867}{We} \markbox{2868}{have} \markbox{m2869}{$P^2 + 2 a u P =0$} \markbox{2870}{in} \markbox{m2871}{$H^*_{\times}$}. \markbox{2872}{Using} \eqref{H in terms of u}, \markbox{2873}{we} \markbox{2874}{get} \markbox{2875}{that} \markbox{m2876}{$2aP =0$} \markbox{2877}{in} \markbox{m2878}{$H^*_{\times}=\mathcal{R}/\mathcal{I}$}. Hence, \markbox{2879}{either} \markbox{m2880}{$a=0$} \markbox{2881}{or} \markbox{m2882}{$P\in \mathcal{I}$}.
		%for some $f, f_i\in \mathcal R$ and generators $h_i$ of $\mathcal{I}\subset \mathcal R$. Write $f_i = g_i + u l_i$ with $g_i, l_i$ in $\mathbb{Q}[c_1, \dots, c_k]$, then comparing terms not in $u^2$ gives:\[P^2 + 2 a u P = \sum g_i h_i + u \sum l_i h_i.\]Thus, $2 a u P = u \sum l_i h_i \in \mathcal{I}$, so if $a \neq 0$, then $P \in \mathcal{I}$ and hence $\phi(u) = a u$ in $H^*_{\times}$. If $a = 0$, then $\phi(u) = P(c_1, \dots, c_k)$ with $P^2 \in \mathcal{I}$.This completes the proof.
	\end{proof}
	
	
		\begin{remark}
			\thmref{main thm} \markbox{2883}{classifies} \markbox{2884}{all} \markbox{2885}{graded} \markbox{2886}{endomorphisms} \markbox{m2887}{$\phi$} \markbox{2888}{of} \markbox{m2889}{$H^*_\times$} \markbox{2890}{whose} \markbox{2891}{image} \markbox{2892}{is} \markbox{2893}{nonzero} \markbox{2894}{in} \markbox{m2895}{$H^2_{\mathbb CG}$} \markbox{2896}{if} \markbox{m2897}{$n>2$}. \markbox{2898}{In} fact, \markbox{m2899}{$n>2$} \markbox{2900}{implies} \markbox{m2901}{$c_1^2\neq 0$} \markbox{2902}{and} \markbox{m2903}{$\phi(u) \neq ac_1,\, a \in \mathbb{Q}\setminus\{0\}$} \markbox{2904}{as} \markbox{m2905}{$\phi(u)^2=0$}. Therefore, \markbox{2906}{the} \markbox{2907}{only} \markbox{2908}{remaining} \markbox{2909}{possibility} \markbox{2910}{is} \markbox{m2911}{$\phi(c_1)\neq \mu u,\, \mu \in \mathbb Q.$} 
			
			\markbox{2912}{On} \markbox{2913}{the} \markbox{2914}{other} hand, \markbox{2915}{when} \markbox{m2916}{$n=2,$} \markbox{m2917}{$\mathbb CG_{n,k}$} \markbox{2918}{is} \markbox{2919}{either} \markbox{2920}{a} \markbox{2921}{point} \markbox{2922}{or} \markbox{m2923}{$\mathbb S^2$} \markbox{2924}{and} \markbox{2925}{the} \markbox{2926}{classification} \markbox{2927}{of} \markbox{2928}{graded} \markbox{2929}{endomorphisms} \markbox{2930}{of} \markbox{m2931}{$H^*_\times$} \markbox{2932}{is} easy.
		\end{remark}
	
	
	
	\subsection{} \markbox{2933}{In} \thmref{main thm}, \markbox{2934}{we} \markbox{2935}{assume} \markbox{2936}{that} \markbox{m2937}{$\phi(c_1) \neq \mu u$}. \markbox{2938}{Let} \markbox{2939}{us} \markbox{2940}{try} \markbox{2941}{to} \markbox{2942}{look} \markbox{2943}{at} \markbox{2944}{the} \markbox{2945}{other} \markbox{2946}{case} \markbox{2947}{where} \markbox{m2948}{$\phi(c_1) = \mu u$}. \markbox{2949}{To} \markbox{2950}{address} this, \markbox{2951}{we} \markbox{2952}{use} \markbox{2953}{part} (i) \markbox{2954}{of} \thmref{hom and hof} \markbox{2955}{which} \markbox{2956}{leads} \markbox{2957}{to} \markbox{2958}{the} \markbox{2959}{following} proposition.
	
	
	\begin{proposition}\label{main thm 2}
		\markbox{2960}{Assume} \markbox{2961}{that} \markbox{2962}{hypothesis} \eqref{Homer} \markbox{2963}{is} satisfied.
		\markbox{2964}{Let} \markbox{m2965}{$\phi$} \markbox{2966}{be} \markbox{2967}{a} \markbox{2968}{graded} \markbox{2969}{endomorphism} \markbox{2970}{such} \markbox{2971}{that} \markbox{m2972}{$\phi(c_1)=\mu u,\, \mu \in \mathbb{Q}$} \markbox{2973}{in} \markbox{m2974}{$H^*_{\times}$}. \markbox{2975}{Then}
		\begin{enumerate}
			\item Either \markbox{m2976}{$\phi(u)=a u$} \markbox{2977}{for} \markbox{2978}{some} \markbox{m2979}{$a \in \mathbb{Q}$}, \markbox{2980}{or} \markbox{m2981}{$\phi(u) \in H^*_{\mathbb{C}G} \subseteq H^*_{\times}$} \markbox{2982}{with} \markbox{m2983}{$\phi(u)^2=0$} \markbox{2984}{in} \markbox{m2985}{$H^*_{\times}$}.
			\item  \markbox{m2986}{$\phi(c_i) = uP_i, \, \forall i >1,$} \markbox{2987}{where} \markbox{m2988}{$P_i \in H^{2i-m}_{\mathbb CG}\subseteq H^*_\times$}. 
		\end{enumerate}
	\end{proposition}
	\begin{proof} \textit{(1):} \markbox{2989}{The} \markbox{2990}{proof} \markbox{2991}{of} \markbox{2992}{part} \textit{(1)} \markbox{2993}{is} \markbox{2994}{exactly} \markbox{2995}{the} \markbox{2996}{same} \markbox{2997}{as} \markbox{2998}{the} \markbox{2999}{proof} \markbox{3000}{of} \markbox{3001}{part} \textit{(1)} \markbox{3002}{of} \thmref{main thm}. Therefore, \markbox{3003}{we} \markbox{3004}{omit} \markbox{3005}{the} details.
		
		\textit{(2):} \markbox{3006}{Using} \eqref{comm diagram}, \markbox{3007}{we} \markbox{3008}{have} \markbox{3009}{that} \markbox{3010}{the} \markbox{3011}{map} \markbox{m3012}{$\phi_1$} \markbox{3013}{is} \markbox{3014}{a} \markbox{3015}{graded} \markbox{3016}{endomorphism} \markbox{3017}{on} \markbox{m3018}{$H^*_{\mathbb{C}G}$} \markbox{3019}{such} \markbox{3020}{that} \markbox{m3021}{$\phi_1(c_1) =0$}. \markbox{3022}{By} \thmref{hom and hof}, \markbox{m3023}{$\phi_1(c_i) =0, \, \forall i\in I$}, \markbox{3024}{then} \markbox{3025}{by} \eqref{defn of phi}, \markbox{3026}{we} \markbox{3027}{get} \markbox{m3028}{$\phi(c_i) = uP_i$} \markbox{3029}{for} \markbox{3030}{some} \markbox{m3031}{$P_i \in H^*_{\mathbb{C}G}$}, \markbox{3032}{with} \markbox{m3033}{$\deg(P_i) = 2i - m$}. \end{proof}
	%  When $\lambda \neq 0$, the proof follows directly from Theorem~\ref{main thm}.
	\begin{remark}
		\markbox{3034}{In} \thmref{main thm} \markbox{3035}{and} \propref{main thm 2}, \markbox{3036}{if} \markbox{3037}{we} \markbox{3038}{assume} \markbox{m3039}{$2m \leq n-k$}  \markbox{3040}{then} \markbox{m3041}{$\phi(u)=0$} \markbox{3042}{whenever} \markbox{m3043}{$\phi(u) \in H^*_{\mathbb{C}G}$}. \markbox{3044}{This} \markbox{3045}{is} \markbox{3046}{because} \markbox{m3047}{$H^*_{\mathbb{C}G}$} \markbox{3048}{has} \markbox{3049}{no} \markbox{3050}{nontrivial} \markbox{3051}{relations} \markbox{3052}{up} \markbox{3053}{to} \markbox{3054}{degree} \markbox{m3055}{$2(n-k)$} \markbox{3056}{and} \markbox{m3057}{$u^2=0$} \markbox{3058}{implies} \markbox{3059}{that} \markbox{m3060}{$\phi(u)^2=0$} \markbox{3061}{forcing} \markbox{m3062}{$\phi(u)=0$}.
		
	\end{remark}
	%Now suppose $\lambda = 0$, i.e., $\phi(c_1) = 0$ in $H^*_\times$.
	%Unlike the case of Grassmannian, the following proposition guarantees the existence of non-trivial graded endomorphisms $\phi$ on $H^*_\times$ even if $\phi(c_1) =0$.
   A \markbox{3063}{graded} \markbox{3064}{endomorphism} \markbox{3065}{of} \markbox{m3066}{$H^*_{\mathbb CG}$} \markbox{3067}{that} \markbox{3068}{vanishes} \markbox{3069}{on}
   \markbox{m3070}{$H^2_{\mathbb CG}$} \markbox{3071}{is} \markbox{3072}{expected} \markbox{3073}{to} \markbox{3074}{be} trivial, \markbox{3075}{in} \markbox{3076}{view} \markbox{3077}{of} Hoffman's \markbox{3078}{conjecture} \cite{hoffman}. However, \markbox{3079}{unlike} \markbox{3080}{the} \markbox{3081}{case} \markbox{3082}{of} \markbox{3083}{the} \markbox{3084}{complex} Grassmannian, \markbox{3085}{there} \markbox{3086}{exist} \markbox{3087}{many} non-trivial \markbox{3088}{graded} \markbox{3089}{endomorphisms} \markbox{3090}{of} \markbox{m3091}{$H^*_\times$} \markbox{3092}{that} \markbox{3093}{vanish} \markbox{3094}{on} \markbox{m3095}{$H^2_{\mathbb CG}$}. \markbox{3096}{The} \markbox{3097}{following} \markbox{3098}{proposition} \markbox{3099}{provides}   \markbox{3100}{such} \markbox{3101}{examples} \markbox{3102}{when} \markbox{m3103}{$m$} \markbox{3104}{is} \markbox{3105}{even} \markbox{3106}{and}   \markbox{m3107}{$1\le m \le 2k$}.



	
	
	\begin{proposition}
		\markbox{3108}{For} \markbox{3109}{each} \markbox{m3110}{$i\in I$}, \markbox{3111}{choose} \markbox{m3112}{$P_i \in H^{2i-m}_{\mathbb C G} \subseteq H^*_\times$} \markbox{3113}{and} \markbox{3114}{either} \markbox{m3115}{$Q = au,\, a\in \mathbb Q$}, \markbox{3116}{or}  \markbox{m3117}{$Q\in H^*_{\mathbb CG}\subseteq H^*_{\times}$} \markbox{3118}{with} \markbox{m3119}{$Q^2=0$}  \markbox{3120}{in} \markbox{m3121}{$ H^*_{\times}$}. \markbox{3122}{Then} \markbox{3123}{there} \markbox{3124}{exist} \markbox{3125}{a} \markbox{3126}{graded} \markbox{3127}{endomorphism} \markbox{m3128}{$\phi$} \markbox{3129}{on} \markbox{m3130}{$H^*_\times$} \markbox{3131}{such} \markbox{3132}{that} 
		\[
		\phi(c_i)=uP_i, \; \forall i\in I, \text{ and  } \quad \phi(u)=Q.
		\]
	\end{proposition}
	\begin{proof}
		\markbox{3133}{Define} \markbox{m3134}{$\phi$} \markbox{3135}{on} \markbox{m3136}{$H^*_{\times} = \mathcal{R}/\mathcal{I}$} \markbox{3137}{by} \markbox{m3138}{$\phi(c_i)=uP_i, \; \forall i\in I, \text{ and } \phi(u)=Q.$} \markbox{3139}{It} \markbox{3140}{is} \markbox{3141}{sufficient} \markbox{3142}{to} \markbox{3143}{prove} \markbox{3144}{that} \markbox{m3145}{$\phi$} \markbox{3146}{is} \markbox{3147}{well} defined, \markbox{3148}{that} is, \markbox{m3149}{$\mathcal{I}\subseteq \ker (\phi)$}. \markbox{3150}{Observe} \markbox{3151}{that} \markbox{m3152}{$u^2 =0$} \markbox{3153}{in} \markbox{m3154}{$H^*_{\times}$} \markbox{3155}{which} \markbox{3156}{implies} \markbox{3157}{that} \begin{equation}\label{ideal cicj}
			\phi(c_i c_j) = \phi(c_i) \phi(c_j) = uP_i \cdot uP_j = u^2 P_i P_j = 0.
		\end{equation} \markbox{3158}{Using} \eqref{ideal cicj} \markbox{3159}{and} \markbox{m3160}{$\phi(u^2)=Q^2 =0$}, \markbox{3161}{we} \markbox{3162}{have} \markbox{m3163}{$\mathcal{I}\subseteq \langle u^2, c_i c_j \,|\, i,j \in I \rangle \subseteq \ker(\phi).$} 
	\end{proof}
	
	
	\subsection{} \markbox{3164}{In} \markbox{3165}{this} subsection, \markbox{3166}{we} \markbox{3167}{derive} \markbox{3168}{some} \markbox{3169}{immediate} \markbox{3170}{applications} \markbox{3171}{of} \thmref{main thm}.
	\begin{corollary}
		%It is natural to consider a more general situation with several spheres instead of a single one.  
		Let \markbox{3172}{us} \markbox{3173}{consider} \markbox{m3174}{$X = \mathbb{S}^{2m_1} \times \cdots \times \mathbb{S}^{2m_r} \times \mathbb{C}G_{n,k}$} \markbox{3175}{and} \markbox{3176}{denote} \markbox{3177}{by} \markbox{m3178}{$u_j$} \markbox{3179}{the} \markbox{3180}{generator} \markbox{3181}{of} \markbox{m3182}{$H^{2m_j}(\mathbb{S}^{2m_j}; \mathbb{Q})$} \markbox{3183}{corresponding} \markbox{3184}{to} \markbox{3185}{the} \markbox{3186}{fundamental} \markbox{3187}{class} \markbox{3188}{of} \markbox{m3189}{$\mathbb S^{2m_j}$} \markbox{3190}{for} \markbox{3191}{all} \markbox{m3192}{$1\leq j \leq r.$} \markbox{3193}{Define}  
		\[
		H^*_{\mathbf{m}, \mathbb{C}G} := H^*(\mathbb{S}^{2m_1} \times \cdots \times \mathbb{S}^{2m_r} \times \mathbb{C}G_{n,k}; \mathbb{Q})
		\;\cong\; H^*_{\mathbb{C}G}[u_1,\ldots,u_r] \big/ \langle u_1^2,\ldots,u_r^2 \rangle,
		\] \markbox{3194}{where} \markbox{m3195}{$\mathbf{m} = (m_1, \ldots, m_r)$}.
		%where each $u_j$ denotes the degree-$m_j$ generator of $H^*(\mathbb{S}^{m_j}; \mathbb{Q})$.  
		Suppose \markbox{m3196}{$\phi: H^*_{\mathbf{m}, \mathbb{C}G} \to H^*_{\mathbf{m}, \mathbb{C}G}$} \markbox{3197}{is} \markbox{3198}{a} \markbox{3199}{graded} \markbox{3200}{endomorphism} \markbox{3201}{satisfying} \markbox{m3202}{$\phi(c_1)=\lambda c_1,\, \lambda \neq 0$}. \markbox{3203}{Then} $$\phi(c_i) = \begin{cases}
			\lambda^i c_i,   \forall i \in I& \text{ if } k<n-k,\\
			\lambda^i c_i,  \forall i \in I \quad \text{ or } \quad(-\lambda)^i (c^{-1})_i,    \forall i \in I & \text{ if } k= n-k,
		\end{cases}$$
		\markbox{3204}{where} \markbox{m3205}{$ (c^{-1})_i $} \markbox{3206}{is} \markbox{3207}{the} \markbox{m3208}{$ 2i $}-dimensional \markbox{3209}{part} \markbox{3210}{of} \markbox{3211}{the} \markbox{3212}{inverse} \markbox{3213}{of} \markbox{m3214}{$ c = 1 + c_1 + \cdots + c_k $} \markbox{3215}{in} \markbox{m3216}{$ H^*_{ \mathbb CG} $}. 
	\end{corollary}
	\begin{proof}
		\markbox{3217}{The} \markbox{3218}{proof} \markbox{3219}{of} \markbox{3220}{this} \markbox{3221}{corollary} \markbox{3222}{is} \markbox{3223}{similar} \markbox{3224}{to} \markbox{3225}{the} \markbox{3226}{proof} \markbox{3227}{of} \markbox{3228}{part} \textit{2} \markbox{3229}{of} \thmref{main thm}. \markbox{3230}{Apply} \markbox{3231}{induction} \markbox{3232}{on} \markbox{m3233}{$r$} \markbox{3234}{and} \markbox{3235}{replace} \markbox{m3236}{$\mathbb{C}G_{n,k}$} \markbox{3237}{with} \markbox{m3238}{$\hat{X} := \mathbb{S}^{2m_1}\times\cdots\times \mathbb{S}^{2m_{i-1}}\times \mathbb{S}^{2m_{i+1}}\times\cdots\times \mathbb{S}^{2m_r}\times \mathbb{C}G_{n,k},$} \markbox{3239}{and} \markbox{3240}{the} \markbox{3241}{sphere} \markbox{m3242}{$\mathbb{S}^m$} \markbox{3243}{with} \markbox{m3244}{$\mathbb{S}^{2m_i}$} \markbox{3245}{in} \thmref{main thm}. \markbox{3246}{Since} \begin{equation}\label{Sm as hom}
			\mathbb S^{2m_j}=SO(2m_j+1)/SO(2m_j)
		\end{equation}
		\markbox{3247}{where} \markbox{3248}{the} \markbox{3249}{orthogonal} \markbox{3250}{groups} \markbox{m3251}{$SO(2m_j+1)$} \markbox{3252}{and} \markbox{m3253}{$SO(2m_j)$} \markbox{3254}{have} \markbox{3255}{the} \markbox{3256}{same} \markbox{3257}{rank} \markbox{m3258}{$m_j$}. \markbox{3259}{Using} \eqref{Sm as hom} \markbox{3260}{and} \eqref{cgn as hom}, \markbox{m3261}{$\hat{X}$} \markbox{3262}{satisfies} \markbox{3263}{the} \markbox{3264}{hypothesis} \markbox{3265}{of} \thmref{Tezuka}. Therefore, \markbox{3266}{every} \markbox{m3267}{$\mathbb{Q}$}-linear \markbox{3268}{derivation} \markbox{3269}{of} \markbox{m3270}{$H^*(\hat{X};\mathbb{Q})$} \markbox{3271}{that} \markbox{3272}{decreases} \markbox{3273}{the} \markbox{3274}{degree} \markbox{3275}{by} \markbox{m3276}{$2m_i$} \markbox{3277}{is} trivial.
		%Since the $m_i$ are even, the    and $\mathbb CG_{n,k}=U(n)/U(k)\times U(n-k)$, the space $X$ can be realized as a homogeneous space $G/H$, where $G$ is a connected compact Lie group and $H$ a closed subgroup of maximal rank.  This satisfies the hypotheses of Shiga--Tezuka's Theorem~$A'$ \cite{sigha-tezuka}, which then implies that  The argument then proceeds as in Theorem~\ref{main thm}, which completes the proof.  
	\end{proof}
	\markbox{3278}{Let} \markbox{3279}{us} \markbox{3280}{turn} \markbox{3281}{our} \markbox{3282}{attention} \markbox{3283}{to} \markbox{3284}{the} \markbox{3285}{generalized} \markbox{3286}{Dold} \markbox{3287}{spaces} \markbox{m3288}{$P(m,n,k)$} \markbox{3289}{defined} \markbox{3290}{in} \subsecref{gds}. \markbox{3291}{The} \markbox{3292}{following} \markbox{3293}{remark} \markbox{3294}{helps} \markbox{3295}{us} \markbox{3296}{to} \markbox{3297}{describe} \markbox{3298}{endomorphisms} \markbox{3299}{of} \markbox{m3300}{$H^*(P(m,n,k);\mathbb{Q})$} \markbox{3301}{induced} \markbox{3302}{by} \markbox{3303}{continuous} \markbox{3304}{functions} \markbox{3305}{on} \markbox{m3306}{$P(m,n,k)$}. \markbox{3307}{These} \markbox{3308}{observations} \markbox{3309}{will} \markbox{3310}{be} \markbox{3311}{used} \markbox{3312}{in} \secref{section 4}.
	%to understand how continuous self-maps of $P(m,n,k)$ induce homomorphisms on the cohomology ring $H^*(P(m,n,k);\mathbb{Q})$. 
	%Let us record a few useful observations.
	\begin{remark}\label{lift}
		\markbox{3313}{For} \markbox{3314}{a} \markbox{3315}{continuous} \markbox{3316}{map} \markbox{m3317}{$f$} \markbox{3318}{on} \markbox{m3319}{$P(m,n,k)$}, \markbox{3320}{we} \markbox{3321}{have}  
		\begin{equation} \label{lift of f}
			f_*\circ \pi_*\big(\pi_1(\mathbb{S}^m\times \mathbb{C}G_{n,k})\big)
			\subseteq \pi_*\big(\pi_1(\mathbb{S}^m\times \mathbb{C}G_{n,k})\big),
		\end{equation}
		\markbox{3322}{where} \markbox{m3323}{$\pi_1(X)$} \markbox{3324}{denotes} \markbox{3325}{the} \markbox{3326}{fundamental} \markbox{3327}{group} \markbox{3328}{of} \markbox{3329}{a} \markbox{3330}{topological} \markbox{3331}{space} \markbox{m3332}{$X$}. Hence, \markbox{3333}{the} \markbox{3334}{composite} \markbox{m3335}{$f\circ \pi$} \markbox{3336}{admits} \markbox{3337}{a} \markbox{3338}{lift} \markbox{m3339}{$\tilde f$} \markbox{3340}{on}
		\markbox{m3341}{$\mathbb{S}^m\times \mathbb{C}G_{n,k}$} \markbox{3342}{for} \markbox{3343}{the} \markbox{3344}{double} \markbox{3345}{covering}  
		\markbox{m3346}{$\pi:\mathbb{S}^m\times \mathbb{C}G_{n,k}\to P(m,n,k)$}.
	\end{remark}
	\markbox{3347}{Using} \remref{lift}, \markbox{3348}{we} \markbox{3349}{get} \markbox{3350}{the} \markbox{3351}{following} \markbox{3352}{commutative} diagram,
	
	\begin{equation}\label{comm diag on H}
		\begin{tikzcd}
			{H^*(P(m,n,k);\mathbb Q)}  & {H^*(\mathbb S^m\times \mathbb CG_{n,k};\mathbb Q)} \\
			{H^*(P(m,n,k);\mathbb Q)} & {H^*(\mathbb S^m\times \mathbb CG_{n,k};\mathbb Q).}
			\arrow["{\pi^*}", from=1-1, to=1-2]
			\arrow["{f^*}"', from=1-1, to=2-1]
			%\arrow[hook, from=1-2, to=1-3]
			\arrow["{\bar f^*}", from=1-2, to=2-2]
			%\arrow["{\bar f^*}", from=1-3, to=2-3]
			\arrow["{\pi^*}", from=2-1, to=2-2]
			%\arrow[hook, from=2-2, to=2-3]
		\end{tikzcd}
	\end{equation}
	\markbox{3353}{where} \markbox{m3354}{$\pi^*$} \markbox{3355}{is} \markbox{3356}{an} \markbox{3357}{injective} map. \markbox{3358}{Using} \thmref{cohomology of P(m,n,k)} \markbox{3359}{and} \eqref{comm diag on H} \markbox{3360}{we} \markbox{3361}{obtain} \markbox{3362}{the} \markbox{3363}{following} \markbox{3364}{two} corollaries. 
	%as an immediate application of respectively. 	
	\begin{corollary}\label{cor3}
		\markbox{3365}{Let} \markbox{m3366}{$f^*$} \markbox{3367}{be} \markbox{3368}{an} \markbox{3369}{endomorphism} \markbox{3370}{of} \markbox{m3371}{$H^*(P(m,n,k); \mathbb{Q})$} \markbox{3372}{induced} \markbox{3373}{by} \markbox{3374}{a} \markbox{3375}{continuous} \markbox{3376}{function} \markbox{m3377}{$f$} \markbox{3378}{on} \markbox{m3379}{$P(m,n,k)$} \markbox{3380}{satisfying} \markbox{m3381}{$f^*(c_1^2) \ne 0$}. \markbox{3382}{Then}
		\markbox{m3383}{$f^*$} \markbox{3384}{is} \markbox{3385}{the} \markbox{3386}{restriction} \markbox{3387}{of} \markbox{3388}{a} \markbox{3389}{graded} \markbox{3390}{endomorphism} \markbox{m3391}{$\tilde{f}^*$} \markbox{3392}{on}  \markbox{m3393}{$H^*_\times$} \markbox{3394}{satisfying} \markbox{m3395}{$\tilde{f}^*(c_1) = \lambda c_1, \lambda \neq 0$},  \markbox{3396}{to} \markbox{3397}{the} \markbox{3398}{fixed} \markbox{3399}{subring} \markbox{m3400}{$\mathrm{Fix}(\theta^*)$} \markbox{3401}{of} \markbox{m3402}{$H^*_\times$} \markbox{3403}{where} \markbox{m3404}{$\theta = \alpha\times \sigma$}.
	\end{corollary}
	\begin{corollary}\label{cor4}
		\markbox{3405}{Let} \markbox{m3406}{$f^*$} \markbox{3407}{be} \markbox{3408}{an} \markbox{3409}{endomorphism} \markbox{3410}{of} \markbox{m3411}{$H^*(P(m,n,k); \mathbb{Q})$} \markbox{3412}{induced} \markbox{3413}{by} \markbox{3414}{a} \markbox{3415}{continuous} \markbox{3416}{function} \markbox{m3417}{$f$} \markbox{3418}{on} \markbox{m3419}{$P(m,n,k)$} \markbox{3420}{satisfying} \markbox{m3421}{$f^*(c_1^2) =0$} \markbox{3422}{and} \markbox{m3423}{$n>2$}. \markbox{3424}{Then}
		\markbox{m3425}{$f^*$} \markbox{3426}{is} \markbox{3427}{the} \markbox{3428}{restriction} \markbox{3429}{of} \markbox{3430}{a} \markbox{3431}{graded} \markbox{3432}{endomorphism} \markbox{m3433}{$\tilde{f}^*$} \markbox{3434}{on}  \markbox{m3435}{$H^*_\times$} \markbox{3436}{satisfying} \markbox{m3437}{$\tilde{f}^*(c_1) = au, a \in \mathbb{Q}$},  \markbox{3438}{to} \markbox{3439}{the} \markbox{3440}{fixed} \markbox{3441}{subring} \markbox{m3442}{$\mathrm{Fix}(\theta^*)$} \markbox{3443}{of} \markbox{m3444}{$H^*_\times$} \markbox{3445}{where} \markbox{m3446}{$\theta = \alpha\times \sigma$}.
	\end{corollary}
	\markbox{3447}{Using} \thmref{main thm} \markbox{3448}{in} \corref{cor3}, \markbox{3449}{and} \propref{main thm 2} \markbox{3450}{in} \corref{cor4} \markbox{3451}{along} \markbox{3452}{with} \markbox{3453}{hypothesis} \eqref{Homer}, \markbox{3454}{we} \markbox{3455}{can} \markbox{3456}{determine} \markbox{m3457}{$f^*$}.
    %if we add the assumption that the hypothesis \eqref{Homer} is satisfied in \corref{cor4}.
	
	Moreover, \markbox{3458}{there} \markbox{3459}{exist} \markbox{3460}{graded} \markbox{3461}{endomorphisms} \markbox{3462}{of} \markbox{m3463}{$H^*(P(m,n,k))$} \markbox{3464}{that} \markbox{3465}{are} \markbox{3466}{not} \markbox{3467}{induced} \markbox{3468}{by} \markbox{3469}{any} \markbox{3470}{continuous} self-map \markbox{3471}{of} \markbox{m3472}{$P(m,n,k)$}, \markbox{3473}{and} \markbox{3474}{cannot} \markbox{3475}{be} \markbox{3476}{realized} \markbox{3477}{as} \markbox{3478}{restrictions} \markbox{3479}{of} \markbox{3480}{graded} \markbox{3481}{endomorphisms} \markbox{3482}{of} \markbox{m3483}{$H^*_{\times}$}. \markbox{3484}{Let} \markbox{3485}{us} \markbox{3486}{see} \markbox{3487}{an} \markbox{3488}{example} \markbox{3489}{of} \markbox{3490}{such} \markbox{3491}{graded} endomorphism.
	
	
	%Also, there exist graded endomorphisms of $H^*(P(m,n,k); \mathbb{Q})$ which are not restriction of any graded endomorphism of $H^*_\times$, if the graded endomorphism is not induced from a continuous function on $P(m,n,k)$. Let us see an example of such graded endomorphism of $H^*(P(m,n,k);\mathbb{Q})$.
	\begin{example}
		\markbox{3492}{If} \markbox{m3493}{$m$} odd, \markbox{m3494}{$n>2$} \markbox{3495}{and} \markbox{m3496}{$k = 1$}, \markbox{3497}{then} \markbox{m3498}{$P(m,n,1)$} \markbox{3499}{is} \markbox{3500}{fibered} \markbox{3501}{by} \markbox{3502}{the} \markbox{3503}{complex} \markbox{3504}{projective} \markbox{3505}{space}  \markbox{m3506}{$\mathbb{C} P^{n-1}$} \markbox{3507}{over} \markbox{3508}{the} \markbox{3509}{real} \markbox{3510}{projective} \markbox{3511}{space} \markbox{m3512}{$\mathbb{R} P^m$}. \markbox{3513}{In} \markbox{3514}{this} case,  \markbox{m3515}{$H^*_\times\cong \mathbb Q[u,c_1]/\langle u^2, c_1^n\rangle$} \markbox{3516}{and} \markbox{3517}{using} \eqref{defn of theta*}\ and \thmref{cohomology of P(m,n,k)}, \markbox{3518}{the} \markbox{3519}{rational} \markbox{3520}{cohomology} \markbox{3521}{ring} $$H^*(P(m,n,1); \mathbb{Q}) \cong \mathbb{Q}[u, b] / \langle u^2, b^{\lfloor (n+1)/2 \rfloor} \rangle,$$ \markbox{3522}{where} \markbox{m3523}{$u$} \markbox{3524}{is} \markbox{3525}{a} \markbox{3526}{generator} \markbox{3527}{of} \markbox{m3528}{$H^m(\mathbb{R}P^m;\mathbb Q)$} \markbox{3529}{and} \markbox{m3530}{$b$} \markbox{3531}{restricts} \markbox{3532}{to} \markbox{m3533}{$c_1^2\in H^2(\mathbb CP^{n-1};\mathbb Q)$} \markbox{3534}{under} \markbox{3535}{the} \markbox{3536}{fiber} inclusion. 
		%Here, $\lfloor x\rfloor$ denotes the greatest integer less than or equal to $x$.
		
		Consider \markbox{3537}{the} \markbox{3538}{endomorphism}
		\[
		\phi \colon H^*(P(m,n,1); \mathbb{Q}) \to H^*(P(m,n,1); \mathbb{Q}), \quad \text{defined by }\quad u \mapsto u,  b \mapsto -b.
		\]
		\markbox{3539}{Then} \markbox{m3540}{$\phi$} \markbox{3541}{is} \markbox{3542}{a} well-defined \markbox{3543}{graded} \markbox{3544}{endomorphism} \markbox{3545}{but} \markbox{3546}{it} \markbox{3547}{cannot} \markbox{3548}{be} \markbox{3549}{a} \markbox{3550}{restriction} \markbox{3551}{of} \markbox{3552}{a} \markbox{3553}{graded} \markbox{3554}{endomorphism} \markbox{3555}{of} \markbox{m3556}{$H^*_\times$} \markbox{3557}{because} \markbox{3558}{any} \markbox{3559}{such} \markbox{3560}{map} \markbox{3561}{induces} \markbox{m3562}{$c_1^2 \mapsto \lambda^2 c_1^2$} \markbox{3563}{for} \markbox{3564}{some} \markbox{m3565}{$\lambda \in \mathbb{Q}$}, \markbox{3566}{and} \markbox{m3567}{$\lambda^2 \neq -1$}.
	\end{example}
	
	
	\iffalse
	\begin{corollary}
		\markbox{3568}{Assume} \markbox{3569}{that} \markbox{m3570}{$P(m,n,k)$} \markbox{3571}{be} \markbox{3572}{an} \markbox{3573}{orientable} \markbox{3574}{manifold} \markbox{3575}{and} \markbox{m3576}{$n>2$}. \markbox{3577}{Let} \markbox{m3578}{$f$} \markbox{3579}{be} \markbox{3580}{a} \markbox{3581}{continuous} \markbox{3582}{map} \markbox{3583}{on} \markbox{m3584}{$P(m,n,k)$} \markbox{3585}{with} \markbox{3586}{nonzero} \markbox{3587}{Brouwer} degree. \markbox{3588}{Then}  
		\markbox{3589}{the} \markbox{3590}{induced} \markbox{3591}{map} \markbox{m3592}{$f^*$} \markbox{3593}{is} \markbox{3594}{an} \markbox{3595}{automorphism} \markbox{3596}{on} \markbox{m3597}{$H^*(P(m,n,k);\mathbb Q)$}.
	\end{corollary}
	\begin{proof}
		\markbox{3598}{Using} \remref{lift}, \markbox{3599}{there} \markbox{3600}{exists} \markbox{3601}{a} \markbox{3602}{lift} \markbox{m3603}{$\tilde f: \mathbb{S}^m\times \mathbb{C}G_{n,k}\to \mathbb{S}^m\times \mathbb{C}G_{n,k}$} 
		\markbox{3604}{satisfying} \markbox{m3605}{$\pi\circ \tilde f = f\circ \pi$}. \markbox{3606}{Using} \corref{cor3} \markbox{3607}{and} \corref{cor4}, \markbox{3608}{we} \markbox{3609}{have} \markbox{m3610}{$f^*$} \markbox{3611}{is} \markbox{3612}{the} \markbox{3613}{restriction} \markbox{3614}{of} \markbox{3615}{a} \markbox{3616}{graded} \markbox{3617}{endomorphism} \markbox{m3618}{$\tilde{f}^*$} \markbox{3619}{on}  \markbox{m3620}{$H^*_\times$} \markbox{3621}{satisfying} \markbox{3622}{either} $$\tilde{f}^*(c_1) = \lambda c_1, \lambda \neq 0, \text{ if } f^*(c_1^2) \neq 0, \text{ or } \tilde f^*(c_1) =au, a\in \mathbb{Q}, \text{ if } f^*(c_1^2) =0,$$  \markbox{3623}{to} \markbox{3624}{the} \markbox{3625}{fixed} \markbox{3626}{subring} \markbox{m3627}{$\mathrm{Fix}(\theta^*)$} \markbox{3628}{of} \markbox{m3629}{$H^*_\times$}.
		
		\markbox{3630}{Let} \markbox{m3631}{$d=k(n-k)$}.  \markbox{3632}{The} \markbox{3633}{top} \markbox{3634}{cohomology} \markbox{m3635}{$H^{2d+m}_\times\cong\mathbb{Q}$} \markbox{3636}{is} \markbox{3637}{generated} \markbox{3638}{by} 
		\markbox{m3639}{$uc_1^d$}.  \markbox{3640}{The} \markbox{3641}{nonzero} \markbox{3642}{Brouwer} \markbox{3643}{degree} \markbox{3644}{of} \markbox{m3645}{$f$} \markbox{3646}{implies} \markbox{3647}{nonzero} \markbox{3648}{Brouwer} \markbox{3649}{degree} \markbox{3650}{of} \markbox{m3651}{$\tilde f$} i.e. \begin{equation}\label{brouwer}
			\tilde f^*(uc_1^d)=\nu\,uc_1^d \in uH^{*}_{\mathbb{C}G}, \, \nu \neq 0.
		\end{equation}
		
		\markbox{3652}{Let} \markbox{3653}{us} \markbox{3654}{consider} \markbox{3655}{the} \markbox{3656}{first} \markbox{3657}{case} \markbox{3658}{where} \markbox{m3659}{$f^*(c_1^2) \neq 0$} \markbox{3660}{then} \markbox{m3661}{$\tilde f^*(c_1) = \lambda c_1, \lambda \neq 0$}. \markbox{3662}{Using} \thmref{main thm}, \markbox{3663}{we} \markbox{3664}{have} \markbox{m3665}{$\tilde f^*(c_i) = \lambda^i c_i$}. Also, $$\tilde f^*(u) = \mu u, \mu \in \mathbb{Q}, \text{ or } \tilde f^*(u)\in H^*_{\mathbb{C}G}.$$ \markbox{3666}{If} \markbox{m3667}{$\tilde f^*(u)\in H^*_{\mathbb{C}G}$}, \markbox{3668}{then} \markbox{m3669}{$\tilde f^*(uc_1^d) \in H^*_{\mathbb{C}G}$}
		\markbox{3670}{which} \markbox{3671}{is} \markbox{3672}{a} \markbox{3673}{contradiction} \markbox{3674}{to} \eqref{brouwer}. Therefore, \markbox{m3675}{$\tilde f^*(u) = \mu u $} \markbox{3676}{where} \markbox{m3677}{$\mu \neq 0$} \markbox{3678}{because} \markbox{m3679}{$\nu \neq 0$}. So, \markbox{m3680}{$\tilde f^*$} \markbox{3681}{is} \markbox{3682}{an} automorphism.
		
		\markbox{3683}{Let} \markbox{3684}{us} \markbox{3685}{consider} \markbox{3686}{the} \markbox{3687}{other} \markbox{3688}{case} \markbox{3689}{when} \markbox{m3690}{$f^*(c_1^2) = 0$} \markbox{3691}{then} \markbox{m3692}{$\tilde f^*(c_1) = au, a \in \mathbb{Q}$}. Since, \markbox{m3693}{$u^2 =0$} \markbox{3694}{and} \markbox{m3695}{$d\geq 2$} \markbox{3696}{we} \markbox{3697}{have} $$\tilde f^*(uc_1^d) = \tilde f^*(u) (\tilde f^*(c_1))^d = \tilde f^*(u) a^d u^d =0$$ \markbox{3698}{which} \markbox{3699}{is} \markbox{3700}{a} \markbox{3701}{contradiction} \markbox{3702}{to} \eqref{brouwer}. Hence, \markbox{3703}{this} \markbox{3704}{case} \markbox{3705}{would} \markbox{3706}{not} arise. 
	\end{proof}
	\fi
	
	The \markbox{3707}{following} \markbox{3708}{corollary} \markbox{3709}{helps} \markbox{3710}{us} \markbox{3711}{to} \markbox{3712}{understand} \markbox{3713}{the} \markbox{3714}{relationship} \markbox{3715}{between} \markbox{3716}{the} \markbox{3717}{automorphisms} \markbox{3718}{of} \markbox{m3719}{$H^*(P(m,n,k))$} \markbox{3720}{with} \markbox{3721}{the} \markbox{3722}{automorphisms} \markbox{3723}{of} \markbox{m3724}{$H^*_{\times}$}.
	\begin{corollary}\label{automor}
		\markbox{3725}{Let} \markbox{m3726}{$f^*$} \markbox{3727}{be} \markbox{3728}{an} \markbox{3729}{automorphism} \markbox{3730}{of} \markbox{m3731}{$H^*(P(m,n,k);\mathbb{Q})$} \markbox{3732}{induced} \markbox{3733}{by} \markbox{3734}{a} \markbox{3735}{continuous} \markbox{3736}{function} \markbox{m3737}{$f$} \markbox{3738}{on} \markbox{m3739}{$P(m,n,k)$} \markbox{3740}{and} \markbox{3741}{assume} \markbox{3742}{that} \markbox{m3743}{$n> 2$}. \markbox{3744}{Then} \markbox{m3745}{$\tilde{f}^*$} \markbox{3746}{is} \markbox{3747}{an} \markbox{3748}{automorphism} \markbox{3749}{of} \markbox{m3750}{$H^*_{\times}$}, \markbox{3751}{where} \markbox{m3752}{$\tilde{f}$} \markbox{3753}{is} \markbox{3754}{as} \markbox{3755}{in} \remref{lift}. \\ \markbox{3756}{Moreover} \markbox{3757}{there} \markbox{3758}{exist} \markbox{m3759}{$\lambda, \mu \in \mathbb{Q}\backslash \{0\}$} \markbox{3760}{such} \markbox{3761}{that} \markbox{m3762}{$\tilde{f}^*(u) = \mu u$} \markbox{3763}{and} \markbox{m3764}{$\tilde{f}^*(c_i)$} \markbox{3765}{is} \markbox{3766}{of} \markbox{3767}{the} \markbox{3768}{form} \markbox{3769}{given} \markbox{3770}{in} \textit{(2)} \markbox{3771}{of} \thmref{main thm}.
	\end{corollary}
	\begin{proof}
		\markbox{3772}{Using} \remref{lift}, \markbox{3773}{we} \markbox{3774}{have} \markbox{m3775}{$\tilde{f}^*$} \markbox{3776}{is} \markbox{3777}{a} \markbox{3778}{graded} \markbox{3779}{endomorphism} \markbox{3780}{of} \markbox{m3781}{$H^*_{\times}$}. \markbox{3782}{When} \markbox{m3783}{$n>2$}, \markbox{3784}{we} \markbox{3785}{have} \markbox{m3786}{$c_1^2\neq 0$} \markbox{3787}{in} \markbox{m3788}{$\fix (\theta^*)\subseteq H^*_\times$}, \markbox{3789}{where} Fix\markbox{m3790}{$(\theta^*)$} \markbox{3791}{is} \markbox{3792}{the} \markbox{3793}{fixed} \markbox{3794}{subring} \markbox{3795}{under} \markbox{m3796}{$\theta^*$} \markbox{3797}{defined} \markbox{3798}{in} \eqref{defn of theta*}. \markbox{3799}{Since} \markbox{m3800}{$f^*$} \markbox{3801}{is} \markbox{3802}{an} automorphism, \markbox{3803}{we} \markbox{3804}{have} \markbox{m3805}{$f^*(c_1^2) \neq 0$}. \markbox{3806}{Using} \corref{cor3}, \markbox{3807}{there} \markbox{3808}{exist} \markbox{m3809}{$\lambda \in \mathbb{Q}$} \markbox{3810}{such} \markbox{3811}{that} \markbox{m3812}{$\tilde{f}^*(c_1) = \lambda c_1, \lambda \neq 0$}.\\
		\markbox{3813}{By} \thmref{main thm}, \markbox{m3814}{$\tilde{f}^*(c_i)$} \markbox{3815}{is} \markbox{3816}{of} \markbox{3817}{the} \markbox{3818}{form} \markbox{3819}{given} \markbox{3820}{in} \textit{(2)} \markbox{3821}{of} \thmref{main thm}. Also, $$\tilde{f}^*(u) = \mu u, \, \mu \in \mathbb{Q} \quad \text{ or } \quad \tilde{f}^*(u) = Q$$ \markbox{3822}{where} \markbox{m3823}{$Q$} \markbox{3824}{is} \markbox{3825}{a} \markbox{3826}{polynomial} \markbox{3827}{of} \markbox{3828}{degree} \markbox{m3829}{$m$} \markbox{3830}{in} \markbox{m3831}{$H^*_{\mathbb{C}G}$} \markbox{3832}{with} \markbox{m3833}{$Q^2 =0$}. \markbox{3834}{To} \markbox{3835}{conclude} \markbox{3836}{the} result, \markbox{3837}{we} \markbox{3838}{need} \markbox{3839}{to} \markbox{3840}{prove} \markbox{3841}{that} \markbox{m3842}{$\tilde{f}^*(u) = \mu u$} \markbox{3843}{where} \markbox{m3844}{$\mu \neq 0$}. \\
		\markbox{3845}{Suppose} \markbox{3846}{that} \markbox{m3847}{$\tilde{f}^*(u) = Q$}, \markbox{3848}{then} \markbox{3849}{the} \markbox{3850}{image} \markbox{3851}{set} \markbox{m3852}{$\im \tilde{f}^*\subseteq H^*_{\mathbb{C}G}$}. \markbox{3853}{Using} \corref{cor3}, \markbox{3854}{we} \markbox{3855}{get} $$\im f^*  \cong \im \tilde{f^*}|_{Fix (\theta^*)}\subseteq H^*_{\mathbb{C}G}.$$ \markbox{3856}{This} \markbox{3857}{is} \markbox{3858}{a} \markbox{3859}{contradiction} \markbox{3860}{to} \markbox{3861}{the} \markbox{3862}{assumption} \markbox{3863}{that} \markbox{m3864}{$f^*$} \markbox{3865}{is} \markbox{3866}{an} \markbox{3867}{automorphism} \markbox{3868}{because} \markbox{3869}{using} \thmref{cohomology of P(m,n,k)}, \markbox{3870}{either} \markbox{m3871}{$u$} \markbox{3872}{or} \markbox{m3873}{$uc_{1}$} (depending \markbox{3874}{on} \markbox{3875}{the} \markbox{3876}{parity} \markbox{3877}{of} \markbox{m3878}{$m$}) \markbox{3879}{is} \markbox{3880}{in} \markbox{m3881}{$\im f^*=\fix (\theta^*) \cong H^*(P(m,n,k);\mathbb{Q})$}. Therefore, \markbox{m3882}{$\tilde{f}^*(u) = \mu u, \mu \in \mathbb{Q}$} \markbox{3883}{and} \markbox{m3884}{$\mu \neq 0$} \markbox{3885}{because} \markbox{m3886}{$f^*$} \markbox{3887}{is} \markbox{3888}{an} automorphism.
	\end{proof}
	\subsection{} \markbox{3889}{The} \markbox{3890}{following} \markbox{3891}{theorem} \markbox{3892}{provides} \markbox{3893}{a} \markbox{3894}{criterion} \markbox{3895}{for} \markbox{3896}{the} \markbox{3897}{image} \markbox{3898}{of} \markbox{3899}{the} \markbox{3900}{spherical} \markbox{3901}{cohomology} \markbox{3902}{class} \markbox{3903}{mapped} \markbox{3904}{to} \markbox{3905}{a} \markbox{3906}{scaler} \markbox{3907}{multiple} \markbox{3908}{of} \markbox{3909}{itself} \markbox{3910}{under} \markbox{3911}{the} \markbox{3912}{graded} \markbox{3913}{endomorphism} \markbox{3914}{on}
	\markbox{m3915}{$H^*(\mathbb{S}^m \times \mathbb{C}G_{n,k};\mathbb Z)$} \markbox{3916}{induced} \markbox{3917}{from} \markbox{3918}{a} \markbox{3919}{continuous} map.
	
	
	
	\begin{theorem}\label{ind from top}
		\markbox{3920}{Let} \markbox{m3921}{$f$} \markbox{3922}{be} \markbox{3923}{a} \markbox{3924}{continuous} \markbox{3925}{map} \markbox{3926}{on} \markbox{m3927}{$\mathbb S^m\times \mathbb CG_{n,k}$} \markbox{3928}{such} \markbox{3929}{that} \markbox{3930}{it} \markbox{3931}{stabilizes} \markbox{3932}{a} \markbox{3933}{copy} \markbox{3934}{of} \markbox{3935}{Grassmannian} \markbox{m3936}{$\{{x_0}\}\times \mathbb CG_{n,k}$} \markbox{3937}{for} \markbox{3938}{some} \markbox{m3939}{$x_0\in \mathbb S^m.$} \markbox{3940}{Then} \markbox{3941}{the} \markbox{3942}{induced} \markbox{3943}{endomorphism} \markbox{3944}{in} \markbox{3945}{cohomology} \markbox{3946}{satisfies} \markbox{m3947}{$f^*(u) = \mu u$} \markbox{3948}{for} \markbox{3949}{some} \markbox{m3950}{$\mu \in \mathbb Z$}.
	\end{theorem}
	
	
	%\begin{theorem}\label{ind from top}
	%Let $x_0 \in \mathbb{S}^m$ and consider a graded endomorphism $f^*$ on $H^*_{\times}$  induced from a topological map $f$ on $\mathbb S^m\times \mathbb CG_{n,k}$ such that $f (\{x_0 \}\times\mathbb CG_{n,k})\subseteq \{x_0\}\times \mathbb{C}G_{n,k}$.  Then $f^*(u)=\mu u$ for some $\mu\in \mathbb Q.$
	%\end{theorem}
	
	
	\begin{proof}
		\markbox{3951}{Let} \markbox{m3952}{$\mathbb T^m$} \markbox{3953}{be} \markbox{3954}{the} \markbox{3955}{torus} \markbox{m3956}{$(\mathbb S^1)^m$} \markbox{3957}{and}
		\markbox{m3958}{$q:\mathbb T^m\to \mathbb S^m$} \markbox{3959}{be} \markbox{3960}{the} \markbox{3961}{quotient} \markbox{3962}{map} \markbox{3963}{that} \markbox{3964}{collapses} \markbox{3965}{the} \markbox{3966}{complement} \markbox{m3967}{$C$} \markbox{3968}{of} \markbox{3969}{an} \markbox{3970}{open} \markbox{3971}{disk} \markbox{m3972}{$D\subset \mathbb T^m$} \markbox{3973}{to} \markbox{3974}{the} \markbox{3975}{point} \markbox{m3976}{$x_0$} \markbox{3977}{in} \markbox{m3978}{$\mathbb{S}^m$}.  \markbox{3979}{Denote} \markbox{m3980}{$p_i$} \markbox{3981}{the} \markbox{m3982}{$i$}-th \markbox{3983}{projection} \markbox{3984}{map} \markbox{3985}{on} \markbox{m3986}{$\mathbb S^m\times \mathbb CG_{n,k}$} \markbox{3987}{for} \markbox{m3988}{$i=1,2$} \markbox{3989}{and} \markbox{m3990}{$s:\mathbb S^m\setminus\{x_0\}\to D$} \markbox{3991}{is} \markbox{3992}{the} \markbox{3993}{inverse} \markbox{3994}{of} \markbox{3995}{the} \markbox{3996}{restriction} \markbox{m3997}{$q|_D$}.
		\markbox{3998}{Since} \markbox{m3999}{$f$} \markbox{4000}{stabilizes} \markbox{m4001}{$\{x_0\}\times \mathbb{C}G_{n,k}$}, \markbox{4002}{define} \markbox{4003}{continuous} \markbox{4004}{maps}
		\markbox{m4005}{$g:\mathbb{C}G_{n,k}\to \mathbb{C}G_{n,k}$} \markbox{4006}{by} \markbox{m4007}{$(x_0,g(y)) = f(x_0,y)$} \markbox{4008}{and} \markbox{m4009}{$\tilde f:\mathbb T^m\times \mathbb{C}G_{n,k}\to \mathbb T^m\times \mathbb{C}G_{n,k}$} \markbox{4010}{by}
		\[
		\tilde f(x,y)=
		\begin{cases}
			\big(s\circ p_1\circ f(q(x),y),\,\,p_2\circ f(q(x),y)\big), & x\in D,\\[2pt]
			\big(x,g(y)\big), & x\in C.
		\end{cases}
		\]
		%where and $p_i$ denotes the $i$-th projection.  
		Then \markbox{4011}{it} \markbox{4012}{is} \markbox{4013}{easy} \markbox{4014}{to} \markbox{4015}{check} \markbox{4016}{that} \markbox{4017}{the} \markbox{4018}{following} \markbox{4019}{diagram} commutes:
		\[ \begin{tikzcd}
			{\mathbb T^m\times \mathbb{C}G_{n,k}} & {\mathbb T^m\times \mathbb{C}G_{n,k}} \\
			{\mathbb S^m\times \mathbb{C}G_{n,k}} & {\mathbb S^m\times \mathbb{C}G_{n,k}}
			\arrow["{\tilde f}", from=1-1, to=1-2]
			\arrow["{q\times \mathrm{id}}"', two heads, from=1-1, to=2-1]
			\arrow["{q\times \mathrm{id}}", two heads, from=1-2, to=2-2]
			\arrow["f", from=2-1, to=2-2]
		\end{tikzcd}
		\]
		
		Since, \markbox{4020}{the} \markbox{4021}{quotient} \markbox{4022}{map} \markbox{m4023}{$q$} \markbox{4024}{has} \markbox{4025}{Brouwer} \markbox{4026}{degree} 1, \markbox{4027}{the} \markbox{4028}{induced} \markbox{4029}{map} \markbox{4030}{on} \markbox{4031}{rational} \markbox{4032}{cohomology} \markbox{m4033}{$q^*: H^*(\mathbb{S}^m; \mathbb{Z}) \rightarrow H^*(\mathbb{T}^m;\mathbb{Z})$} \markbox{4034}{sends} \markbox{m4035}{$u\mapsto 1\cdot u_1 u_2 \dots u_m$} \markbox{4036}{where} \markbox{m4037}{$u_i$} \markbox{4038}{denote} \markbox{4039}{the} \markbox{4040}{one} \markbox{4041}{dimensional} \markbox{4042}{cohomology} \markbox{4043}{class} \markbox{4044}{corresponding} \markbox{4045}{to} \markbox{4046}{the} \markbox{4047}{fundamental} \markbox{4048}{class} \markbox{4049}{of} \markbox{4050}{the} \markbox{m4051}{$i$}-th \markbox{4052}{circle} \markbox{4053}{factor} \markbox{4054}{of} \markbox{m4055}{$\mathbb T^m$} \markbox{4056}{for} \markbox{m4057}{$i\in \{1,2,\ldots,m\}$} \markbox{4058}{with} \markbox{4059}{appropriate} orientation. 
		%Let $u$ be a generator of $H^m(\mathbb S^m;\mathbb Z)$ so that $q^*(u)=u_1u_2\cdots u_m$.  
		Since \markbox{m4060}{$H^{\mathrm{odd}}(\mathbb{C}G_{n,k};\mathbb Z)=0$}, \markbox{4061}{the} \markbox{4062}{induced} \markbox{4063}{map} \markbox{m4064}{$\tilde f^*$} \markbox{4065}{sends} \markbox{4066}{each} \markbox{m4067}{$u_i$} \markbox{4068}{to} \markbox{4069}{a} \markbox{4070}{polynomial} \markbox{m4071}{$P_i(u_1,\dots,u_m)$}.  \markbox{4072}{We} \markbox{4073}{slightly} \markbox{4074}{abuse} \markbox{4075}{notation} \markbox{4076}{by} \markbox{4077}{using} \markbox{4078}{the} \markbox{4079}{same} \markbox{4080}{symbols} \markbox{4081}{for} \markbox{4082}{the} \markbox{4083}{cohomology} \markbox{4084}{classes} \markbox{4085}{of} \markbox{m4086}{$H^*(\mathbb S^m;\mathbb Z)$} \markbox{4087}{and} \markbox{m4088}{$H^*(\mathbb{C}G_{n,k};\mathbb Z)$} \markbox{4089}{when} \markbox{4090}{viewed} \markbox{4091}{in} \markbox{m4092}{$H^*(\mathbb S^m\times \mathbb CG_{n,k};\mathbb Z)$}.
		\markbox{4093}{The} \markbox{4094}{induced} \markbox{4095}{diagram} \markbox{4096}{in} \markbox{4097}{cohomology} \markbox{4098}{implies} \markbox{4099}{the} \markbox{4100}{following} \markbox{4101}{commutative} diagram.
		\[
		\begin{tikzcd}
			{\prod_{i=1}^m u_i} & {\prod_{i=1}^m P_i(u_1,\ldots,u_m)} \\
			u & {f^*(u)}
			\arrow["{\tilde f^*}", maps to, from=1-1, to=1-2]
			\arrow["{(q\times \mathrm{id})^*}"', maps to, from=2-1, to=1-1]
			\arrow["{f^*}", maps to, from=2-1, to=2-2]
			\arrow["{(q\times \mathrm{id})^*}"', maps to, from=2-2, to=1-2]
		\end{tikzcd}
		\]
		\markbox{4102}{This} \markbox{4103}{implies} \markbox{4104}{that} \markbox{m4105}{$f^*(u)$} \markbox{4106}{does} \markbox{4107}{not} \markbox{4108}{contain} \markbox{4109}{any} \markbox{4110}{nonzero} \markbox{4111}{element} \markbox{4112}{from} \markbox{m4113}{$H^*(\mathbb{C}G_{n,k};\mathbb Z)$}.  
		Thus, \markbox{m4114}{$f^*(u)=\mu u$} \markbox{4115}{for} \markbox{4116}{some} \markbox{m4117}{$\mu\in \mathbb Z$}.
	\end{proof}
	%\begin{remark}
		%To remain consistent with notations, we have written the proof of \thmref{ind from top} over $\mathbb{Q}$, but the same proof also works over $\mathbb{Z}$.
	%\end{remark}
	
	
	
	
	
	
	
	
	
	
	
	
	\section{Coincidence theory of $P(m,n,k)$} \label{section 4}
	\markbox{4118}{In} \markbox{4119}{this} section, \markbox{4120}{we} \markbox{4121}{study} \markbox{4122}{the} \textit{coincidence theory} \markbox{4123}{of} \markbox{4124}{generalized} \markbox{4125}{Dold} \markbox{4126}{spaces} \markbox{m4127}{$P(m,n,k)$} \markbox{4128}{defined} \markbox{4129}{in} \subsecref{gds}. \markbox{4130}{We} \markbox{4131}{establish} \markbox{4132}{the} \markbox{4133}{necessary} \markbox{4134}{conditions} \markbox{4135}{for} \markbox{4136}{a} \markbox{4137}{generalized} \markbox{4138}{Dold} \markbox{4139}{space} \markbox{m4140}{$P(S,X)$} \markbox{4141}{defined} \markbox{4142}{in} \eqref{gen dold space} \markbox{4143}{to} \markbox{4144}{satisfy} \markbox{4145}{the} \markbox{4146}{coincidence} property. 
	%Our study builds upon previous results on the \emph{graded endomorphisms} of the rational cohomology ring $H^*_\times$ in section~\ref{graded endomorphism} and the \emph{rational cohomology ring of generalized Dold spaces} as established in \cite{mandal-sankaran2}.
	
	\subsection{} \markbox{4147}{Let} \markbox{4148}{us} \markbox{4149}{recall} \markbox{4150}{certain} \markbox{4151}{definitions} \markbox{4152}{that} \markbox{4153}{will} \markbox{4154}{be} \markbox{4155}{required} \markbox{4156}{in} \markbox{4157}{the} \markbox{4158}{rest} \markbox{4159}{of} \markbox{4160}{this} section.
	
	\begin{definition}
		\markbox{4161}{Let} \markbox{m4162}{$(X,g)$} \markbox{4163}{be} \markbox{4164}{a} pair, \markbox{4165}{where} \markbox{m4166}{$g$} \markbox{4167}{is} \markbox{4168}{a} \markbox{4169}{continuous} \markbox{4170}{map} \markbox{4171}{on} \markbox{4172}{a} \markbox{4173}{topological} \markbox{4174}{space} \markbox{m4175}{$X$}. \markbox{4176}{The} \markbox{4177}{pair} \markbox{m4178}{$(X,g)$} \markbox{4179}{is} \markbox{4180}{said} \markbox{4181}{to} \markbox{4182}{have} \markbox{4183}{the} \textbf{coincidence property} (in short, CP) if, \markbox{4184}{for} \markbox{4185}{every} \markbox{4186}{continuous} \markbox{4187}{map} \markbox{m4188}{$f : X \to X$}, \markbox{4189}{there} \markbox{4190}{exists} \markbox{4191}{a} \markbox{4192}{point} \markbox{m4193}{$x \in X$} \markbox{4194}{such} \markbox{4195}{that} \markbox{m4196}{$f(x) = g(x)$}.
	\end{definition}
	
	\markbox{4197}{If} \markbox{4198}{we} \markbox{4199}{consider} \markbox{m4200}{$g$} \markbox{4201}{to} \markbox{4202}{be} \markbox{4203}{the} \markbox{4204}{identity} \markbox{4205}{map} \markbox{4206}{on} \markbox{m4207}{$X$}, \markbox{4208}{then} \markbox{4209}{the} \markbox{4210}{notion} \markbox{4211}{of} \markbox{4212}{coincidence} \markbox{4213}{reduces} \markbox{4214}{to} \markbox{4215}{that} \markbox{4216}{of} \markbox{4217}{a} \markbox{4218}{fixed} point, \markbox{4219}{resulting} \markbox{4220}{in} \markbox{4221}{the} \markbox{4222}{following} definition.
	
	
	
	\begin{definition}
		\markbox{4223}{A} \markbox{4224}{topological} \markbox{4225}{space} \markbox{m4226}{$X$} \markbox{4227}{is} \markbox{4228}{said} \markbox{4229}{to} \markbox{4230}{have} \textbf{fixed-point property} (FPP) \markbox{4231}{if} \markbox{4232}{every} \markbox{4233}{continuous} \markbox{4234}{map} \markbox{m4235}{$f : X \to X$} \markbox{4236}{admits} \markbox{4237}{a} fixed-point; \markbox{4238}{that} is, \markbox{4239}{there} \markbox{4240}{exists} \markbox{m4241}{$x \in X$} \markbox{4242}{such} \markbox{4243}{that} \markbox{m4244}{$f(x) = x$}.
	\end{definition}
	
	
	%Our aim is to understand the situations when two continuous maps on the generalized Dold spaces $P(S,X)$ defined in \subsecref{gen dold}, have coincidence.
	 The \markbox{4245}{following} \markbox{4246}{proposition} \markbox{4247}{provides} \markbox{4248}{a} \markbox{4249}{criteria} \markbox{4250}{in} \markbox{4251}{terms} \markbox{4252}{of} \markbox{4253}{the} \markbox{4254}{fiber} \markbox{m4255}{$X$} \markbox{4256}{and} \markbox{4257}{the} \markbox{4258}{base} \markbox{4259}{space} \markbox{m4260}{$Y := S/\!\!\sim_\alpha$}, \markbox{4261}{allowing} \markbox{4262}{one} \markbox{4263}{to} \markbox{4264}{infer} \markbox{4265}{the} \markbox{4266}{coincidence} \markbox{4267}{properties} \markbox{4268}{of} \markbox{4269}{the} \markbox{4270}{total} \markbox{4271}{space} \markbox{m4272}{$P(S,X)$}.
	
	%As a first step, we establish the following proposition, which provides necessary conditions for a generalized Dold space to exhibit certain coincidence properties. 
	
	
	
	
	\iffalse
	\begin{proposition}
		\  \markbox{4273}{A} \markbox{4274}{generalized} \markbox{4275}{Dold} \markbox{4276}{space} \markbox{m4277}{$P(S,\alpha,X,\sigma)$} \markbox{4278}{does} \markbox{4279}{not} \markbox{4280}{have} \markbox{4281}{fixed} \markbox{4282}{point} \markbox{4283}{property} \markbox{4284}{if} \markbox{4285}{any} \markbox{4286}{of} \markbox{4287}{the} \markbox{4288}{following} holds:\\
		(i) \markbox{m4289}{$Y=S/\!\!\sim _{\alpha}$} \markbox{4290}{does} \markbox{4291}{not} \markbox{4292}{have} \markbox{4293}{the} \markbox{4294}{fixed} \markbox{4295}{point} property.\\
		(ii) \markbox{4296}{There} \markbox{4297}{exists} \markbox{4298}{a} \markbox{4299}{map} \markbox{m4300}{$f:X\to X$} \markbox{4301}{having} \markbox{4302}{no} fixed-point \markbox{4303}{and} \markbox{m4304}{$f\circ \sigma=\sigma \circ f$}.
	\end{proposition}
	\begin{proof}
		(i) \markbox{4305}{Let} \markbox{m4306}{$g:Y\to Y$} \markbox{4307}{be} \markbox{4308}{fixed} \markbox{4309}{point} free. \markbox{4310}{Then} \markbox{m4311}{$\phi:=s\circ g\circ p$} \markbox{4312}{on} \markbox{m4313}{$P(S,X)$} \markbox{4314}{also} \markbox{4315}{has} \markbox{4316}{no} \markbox{4317}{fixed} points, \markbox{4318}{where} \markbox{m4319}{$p$} \markbox{4320}{is} \markbox{4321}{the} \markbox{m4322}{$X$}-bundle \markbox{4323}{projection} \markbox{4324}{and} \markbox{m4325}{$s$} \markbox{4326}{is} \markbox{4327}{a} section.  
		(ii).   \markbox{4328}{Suppose} \markbox{4329}{that} \markbox{m4330}{$f:X\to X$} \markbox{4331}{has} \markbox{4332}{no} \markbox{4333}{fixed} \markbox{4334}{points} \markbox{4335}{and} \markbox{4336}{that} \markbox{m4337}{$f\circ \sigma=\sigma \circ f$}.
		\markbox{4338}{Now} \markbox{4339}{define} \markbox{m4340}{$\psi:P(S,X)\to P(S,X)$} \markbox{4341}{as} \markbox{m4342}{$[s,x]\mapsto [s,f(x)]$}. \markbox{4343}{The} well-definedness \markbox{4344}{follows} \markbox{4345}{because} \markbox{m4346}{$\psi([\alpha(s),\sigma(x)])=[\alpha(s),f\circ\sigma(x)]=[\alpha(s),\sigma \circ f(x)]=[s,f(x)]=\psi ([s,x])$}. Clearly, \markbox{m4347}{$\psi$} \markbox{4348}{has} \markbox{4349}{no} \markbox{4350}{fixed} points. \markbox{4351}{This} \markbox{4352}{completes} \markbox{4353}{the} proof.
	\end{proof}
	\fi
	
	
	%Recall that $P(S,X)$ has the fiber bundle structure $X\hookrightarrow P(SX)\twoheadrightarrow Y:=S/\!\! \sim_\alpha$, where $p$ is the $X$-bundle projection and $s$ is a section.  
	\begin{proposition}\label{necessary condition}
		\markbox{4354}{Let} \markbox{m4355}{$(P(S,X),g)$} \markbox{4356}{be} \markbox{4357}{a} pair, \markbox{4358}{where} \markbox{m4359}{$g$} \markbox{4360}{is} \markbox{4361}{a} \markbox{4362}{continuous} \markbox{4363}{map} \markbox{4364}{on} \markbox{4365}{the} \markbox{4366}{generalized} \markbox{4367}{Dold} \markbox{4368}{space} \markbox{m4369}{$P(S,X)$}. \markbox{4370}{Then} \markbox{m4371}{$(P(S,X),g)$} \markbox{4372}{does} \markbox{4373}{not} \markbox{4374}{have} \markbox{4375}{the} \markbox{4376}{CP} \markbox{4377}{if} \markbox{4378}{one} \markbox{4379}{of} \markbox{4380}{the} \markbox{4381}{following} hold:
		\begin{enumerate}
			\item The \markbox{4382}{continuous} \markbox{4383}{map} \markbox{m4384}{$g$} \markbox{4385}{is} \markbox{4386}{a} \markbox{4387}{fiber} \markbox{4388}{bundle} \markbox{4389}{map} \markbox{4390}{and} \markbox{4391}{the} \markbox{4392}{pair} \markbox{m4393}{$(Y,p \circ g \circ s)$} \markbox{4394}{does} \markbox{4395}{not} \markbox{4396}{have} \markbox{4397}{the} CP, 
			\markbox{4398}{where} \markbox{m4399}{$Y = S/\!\!\sim_\alpha$} \markbox{4400}{and} \markbox{m4401}{$s$} \markbox{4402}{denotes} \markbox{4403}{a} \markbox{4404}{section} \markbox{4405}{of} \markbox{4406}{the} \markbox{m4407}{$X$}-bundle \markbox{4408}{projection} \markbox{m4409}{$p$} \markbox{4410}{defined} \markbox{4411}{in} \eqref{sectio} \markbox{4412}{and} \eqref{proj}.
			
			\item  
			There \markbox{4413}{exists} \markbox{4414}{a} \markbox{m4415}{$\sigma$}-equivariant \markbox{4416}{map} \markbox{m4417}{$f$} (i.e. \markbox{m4418}{$f\circ\sigma =\sigma\circ f$}) \markbox{4419}{on} \markbox{m4420}{$X$} \markbox{4421}{and} \markbox{4422}{a} \markbox{m4423}{$\alpha \times \sigma$}-equivariant \markbox{4424}{map} \markbox{m4425}{$\tilde g$} \markbox{4426}{on} \markbox{m4427}{$S\times X$} \markbox{4428}{inducing} \markbox{m4429}{$g$} \markbox{4430}{such} \markbox{4431}{that}
			\markbox{m4432}{$\mathrm{id}_S \times f$} \markbox{4433}{coincides} \markbox{4434}{with} \markbox{4435}{neither} \markbox{m4436}{$\tilde{g}$} \markbox{4437}{nor}
			\markbox{m4438}{$(\alpha \times \sigma) \circ \tilde{g}$}.
		\end{enumerate}
		
		
		
	\end{proposition}
	\begin{proof} \textit{(1)}
		\markbox{4439}{Suppose} \markbox{4440}{that} \markbox{4441}{the} \markbox{4442}{pair} \markbox{m4443}{$(Y,p \circ g \circ s)$} \markbox{4444}{does} \markbox{4445}{not} \markbox{4446}{have} \markbox{4447}{the} CP. 
		\markbox{4448}{Then} \markbox{4449}{there} \markbox{4450}{exists} \markbox{4451}{a} \markbox{4452}{continuous} \markbox{4453}{map} \markbox{m4454}{$f : Y \to Y$} \markbox{4455}{such} \markbox{4456}{that} \begin{equation} \label{pogos}
			f(x) \neq p \circ g \circ s(x), \, \forall x \in Y.
		\end{equation}
		\markbox{4457}{We} \markbox{4458}{are} \markbox{4459}{given} \markbox{4460}{that} \markbox{m4461}{$g$} \markbox{4462}{is} \markbox{4463}{a} \markbox{4464}{fiber} \markbox{4465}{bundle} map, \markbox{4466}{which} \markbox{4467}{implies} \markbox{4468}{that} \markbox{4469}{there} \markbox{4470}{exist}  \markbox{m4471}{$g_1:Y\to Y$}, \markbox{4472}{satisfying}  \markbox{m4473}{$p \circ g = g_1 \circ p.$}
		\markbox{4474}{Consider} \markbox{m4475}{$p\circ g\circ s=g_1\circ p\circ s=g_1 $}.
		Thus, \markbox{m4476}{$p \circ g = g_1 \circ p$} \markbox{4477}{implies}
		\[
		p \circ g(x) = p \circ g \circ s \circ p(x),\, \forall x \in P(S, X).
		\]
		\markbox{4478}{Define} \markbox{4479}{the} \markbox{4480}{map} \markbox{m4481}{$\phi := s \circ f \circ p$} \markbox{4482}{on} \markbox{m4483}{$P(S, X)$}. \markbox{4484}{We} \markbox{4485}{claim} \markbox{4486}{that} \markbox{m4487}{$\phi(y)\neq g(y), \, \forall y \in P(S,X)$}. \\ \markbox{4488}{Suppose} \markbox{4489}{there} \markbox{4490}{exist} \markbox{m4491}{$y \in P(S, X)$} \markbox{4492}{such} \markbox{4493}{that} \markbox{m4494}{$\phi(y) = g(y)$}, \markbox{4495}{then} 
		\[
		p \circ g \circ s(p(y)) = p \circ g(y)
		= p \circ s \circ f \circ p (y) 
		= f(p(y)),
		\]
		\markbox{4496}{which} \markbox{4497}{contradicts} \eqref{pogos}.
		
		
		\textit{(2)} %Let $G$ be a group of order $2$ generated by $\alpha \times \sigma$ acting on the topological space $S\times X$ by composition.
		Let \markbox{m4498}{$G$} \markbox{4499}{denote} \markbox{4500}{the} \markbox{4501}{group} \markbox{4502}{of} \markbox{4503}{deck} \markbox{4504}{transformations} \markbox{4505}{of} \markbox{4506}{the} \markbox{4507}{double} \markbox{4508}{covering} \markbox{m4509}{$\pi : S \times X \to P(S, X)$}, \markbox{4510}{generated} \markbox{4511}{by} \markbox{4512}{the} \markbox{4513}{free} \markbox{4514}{involution} \markbox{m4515}{$\alpha \times \sigma.$}
		\markbox{4516}{The} \markbox{4517}{proof} \markbox{4518}{then} \markbox{4519}{follows} \markbox{4520}{from} \markbox{4521}{a} \markbox{4522}{general} \markbox{4523}{observation} \markbox{4524}{that} \markbox{4525}{if} \markbox{4526}{for} \markbox{4527}{two} \markbox{m4528}{$G$}-equivariant \markbox{4529}{maps} \markbox{m4530}{$\tilde\phi, \tilde\psi$} \markbox{4531}{on} \markbox{m4532}{$S\times X$}, \markbox{4533}{the} \markbox{4534}{maps} \markbox{m4535}{$\tilde \phi$} \markbox{4536}{and} \markbox{m4537}{$t \cdot \tilde\psi$} \markbox{4538}{have} \markbox{4539}{no} \markbox{4540}{point} \markbox{4541}{of} coincidence, \markbox{4542}{for} \markbox{4543}{any} \markbox{m4544}{$t \in G$}; \markbox{4545}{then} \markbox{4546}{the} \markbox{4547}{maps} \markbox{4548}{they} \markbox{4549}{induce} \markbox{4550}{on} \markbox{4551}{the} \markbox{4552}{orbit} \markbox{4553}{space} \markbox{m4554}{$P(S,X)$}, \markbox{4555}{namely} \markbox{m4556}{$\phi, \psi$}, \markbox{4557}{are} \markbox{4558}{also} coincidence-free.
		%$id_{S}\times f$ coincides with neither $\tilde g$ nor $(\alpha \times \sigma)\circ \tilde g$ then the two $G$-equivariant maps $\tilde\phi, \tilde\psi: M \to M$ on a topological space $M$, the maps $\tilde \phi$ and $t \cdot \tilde\psi$ have no point of coincidence, for any $t \in G$; then the maps they induce on the orbit space $M/G$, namely $\phi, \psi: M/G \to M/G$, are coincidence-free.
	\end{proof}
	
	\iffalse
	\begin{proposition}
		
		
		\markbox{4559}{Let} \markbox{m4560}{$g : P(S,X) \to P(S,X)$} \markbox{4561}{be} \markbox{4562}{a} \markbox{4563}{map} \markbox{4564}{induced} \markbox{4565}{by} \markbox{4566}{an} \markbox{m4567}{$\alpha \times \sigma$}-equivariant \markbox{4568}{map} \markbox{m4569}{$\tilde{g} : S \times X \to S \times X$}. 
		\markbox{4570}{Suppose} \markbox{4571}{there} \markbox{4572}{exists} \markbox{4573}{a} \markbox{m4574}{$\sigma$}-equivariant \markbox{4575}{map} \markbox{m4576}{$f : X \to X$} \markbox{4577}{such} \markbox{4578}{that} \markbox{4579}{both} \markbox{4580}{the} \markbox{4581}{pairs} \markbox{m4582}{$id_S\times f, \tilde g$} \markbox{4583}{and} \markbox{m4584}{$id_S\times f,\alpha\times\sigma\circ \tilde g$} \markbox{4585}{dont} \markbox{4586}{have} \markbox{4587}{any} \markbox{4588}{point} \markbox{4589}{of} coincidence.
		\markbox{4590}{Then} \markbox{4591}{the} \markbox{4592}{space} \markbox{m4593}{$P(S,X)$} \markbox{4594}{does} \markbox{4595}{not} \markbox{4596}{have} \markbox{4597}{the} \markbox{m4598}{$g$}-coincidence property.
	\end{proposition}
	\begin{proof}
		\markbox{4599}{Define} \markbox{m4600}{$\phi : P(S,X) \to P(S,X)$} \markbox{4601}{by} \markbox{m4602}{$\phi([s,x]) = [s, f(x)]$}. 
		\markbox{4603}{This} \markbox{4604}{map} \markbox{4605}{is} \markbox{4606}{well} \markbox{4607}{defined} \markbox{4608}{because}
		\markbox{m4609}{$\phi([\alpha(s), \sigma(x)]) = [\alpha(s), f(\sigma(x))] = [\alpha(s), \sigma(f(x))] = [s, f(x)]$}, \markbox{4610}{using} \markbox{4611}{the} \markbox{4612}{fact} \markbox{4613}{that} \markbox{m4614}{$f\circ \sigma=\sigma\circ f$}.
		
		Similarly, \markbox{4615}{define} \markbox{m4616}{$\psi : P(S,X) \to P(S,X)$} \markbox{4617}{by} \markbox{m4618}{$\psi([s,x]) = [g_1(s), g_2(x)]$}, 
		\markbox{4619}{where} \markbox{m4620}{$g_j = p_j \circ \tilde{g} \circ i_j$}, \markbox{4621}{and} \markbox{m4622}{$p_j$} \markbox{4623}{and} \markbox{m4624}{$i_j$} \markbox{4625}{denote} \markbox{4626}{the} \markbox{4627}{projection} \markbox{4628}{and} \markbox{4629}{inclusion} \markbox{4630}{maps} \markbox{4631}{on} \markbox{4632}{the} \markbox{m4633}{$j$}-th \markbox{4634}{factor} (\markbox{m4635}{$j=1,2$}). 
		\markbox{4636}{The} \markbox{4637}{map} \markbox{m4638}{$\psi$} \markbox{4639}{is} \markbox{4640}{well} \markbox{4641}{defined} \markbox{4642}{since} \markbox{m4643}{$\tilde{g}$} \markbox{4644}{is} \markbox{m4645}{$\alpha \times \sigma$}-equivariant.
		
		Since, \markbox{4646}{the} \markbox{4647}{maps} \markbox{m4648}{$f$} \markbox{4649}{and} \markbox{m4650}{$p_2 \circ \tilde{g} \circ i_2 = g_2$} \markbox{4651}{have} \markbox{4652}{no} \markbox{4653}{coincidence} points. 
		Hence, \markbox{4654}{for} \markbox{4655}{any} \markbox{m4656}{$[s,x] \in P(S,X)$}, \markbox{4657}{the} \markbox{4658}{second} \markbox{4659}{coordinates} \markbox{4660}{in} \markbox{m4661}{$\phi([s,x])$} \markbox{4662}{and} \markbox{m4663}{$\psi([s,x])$} \markbox{4664}{are} different. 
		Therefore, \markbox{m4665}{$\phi$} \markbox{4666}{and} \markbox{m4667}{$\psi$} \markbox{4668}{have} \markbox{4669}{no} \markbox{4670}{points} \markbox{4671}{of} coincidence, \markbox{4672}{and} \markbox{m4673}{$P(S,X)$} \markbox{4674}{does} \markbox{4675}{not} \markbox{4676}{possess} \markbox{4677}{the} \markbox{m4678}{$g$}-CP.
	\end{proof}
	\fi
	
		In \markbox{4679}{particular} \markbox{4680}{if} \markbox{4681}{we} \markbox{4682}{take} \markbox{m4683}{$g$} \markbox{4684}{to} \markbox{4685}{be} \markbox{4686}{the} \markbox{4687}{identity} \markbox{4688}{map} \markbox{4689}{in} \propref{necessary condition}, \markbox{4690}{we} \markbox{4691}{recover} Proposition~7.2.1 \markbox{4692}{of} \cite{mandal}, \markbox{4693}{which} \markbox{4694}{proves} \markbox{4695}{that} \markbox{4696}{if} \markbox{4697}{the} \markbox{4698}{base} \markbox{m4699}{$Y$} \markbox{4700}{does} \markbox{4701}{not} \markbox{4702}{have} \markbox{4703}{the} FPP, \markbox{4704}{or} \markbox{4705}{there} \markbox{4706}{exist} \markbox{4707}{a} \markbox{m4708}{$\sigma$}-equivarinat \markbox{4709}{map} \markbox{4710}{on} \markbox{4711}{the} \markbox{4712}{fibre} \markbox{m4713}{$X$} \markbox{4714}{with} \markbox{4715}{no} \markbox{4716}{fixed} \markbox{4717}{point} \markbox{4718}{then} \markbox{m4719}{$P(S,X)$} \markbox{4720}{does} \markbox{4721}{not} \markbox{4722}{have} \markbox{4723}{the} FPP. \markbox{4724}{As} \markbox{4725}{a} consequence, \markbox{m4726}{$P(m,n)$} \markbox{4727}{does} \markbox{4728}{not} \markbox{4729}{have} \markbox{4730}{the} \markbox{4731}{FPP} \markbox{4732}{if} \markbox{4733}{either} \markbox{m4734}{$m$} \markbox{4735}{or} \markbox{m4736}{$n$} \markbox{4737}{is} odd.
	\iffalse
	\begin{remark}\label{necessary criterion for FFP}
		\markbox{4738}{A} \markbox{4739}{generalized} \markbox{4740}{Dold} \markbox{4741}{space} \markbox{m4742}{$P(S,X)$} \markbox{4743}{does} \markbox{4744}{not} \markbox{4745}{have} \markbox{4746}{the} \markbox{4747}{FPP} \markbox{4748}{if} \markbox{4749}{any} \markbox{4750}{of} \markbox{4751}{the} \markbox{4752}{following} holds:
		
		\begin{enumerate}
			\item \markbox{m4753}{$Y=S/\!\!\sim _{\alpha}$} \markbox{4754}{does} \markbox{4755}{not} \markbox{4756}{have} \markbox{4757}{the} FPP.
			\item There \markbox{4758}{exists} \markbox{4759}{a} \markbox{m4760}{$\sigma$}-equivariant \markbox{4761}{map} \markbox{m4762}{$f$} \markbox{4763}{on} \markbox{m4764}{$X$} \markbox{4765}{that} \markbox{4766}{has} \markbox{4767}{no} \markbox{4768}{fixed} point.
		\end{enumerate}
	\end{remark}
	
	
	
	
	
	
	
	
	
	
	\markbox{4769}{We} \markbox{4770}{have} \markbox{4771}{the} \markbox{4772}{following} \markbox{4773}{observation} \markbox{4774}{as} \markbox{4775}{an} \markbox{4776}{immediate} \markbox{4777}{consequence} \markbox{4778}{of} \markbox{4779}{Remark} \ref{necessary criterion for FFP}.
	
	\begin{remark}
		\markbox{4780}{Let} \markbox{m4781}{$P(m,n)$} \markbox{4782}{be} \markbox{4783}{a} \markbox{4784}{classical} \markbox{4785}{Dold} \markbox{4786}{manifold} \markbox{4787}{such} \markbox{4788}{that} \markbox{4789}{either} \markbox{m4790}{$m$} \markbox{4791}{or} \markbox{m4792}{$n$} \markbox{4793}{is} odd. \markbox{4794}{Then}  
		%and the existence of a fixed point free map on $\mathbb CP^n$ which commutes with the conjugation, 
		\markbox{m4795}{$P(m,n)$} \markbox{4796}{does} \markbox{4797}{not} \markbox{4798}{have} \markbox{4799}{the} FPP.
	\end{remark}
	\begin{proof}
		\markbox{4800}{When} \markbox{m4801}{$m$} \markbox{4802}{is} odd, \markbox{4803}{the} \markbox{4804}{proof} \markbox{4805}{follows} \markbox{4806}{from} \markbox{4807}{part} \textit{(1)} \markbox{4808}{of} \remref{necessary criterion for FFP}. \\ \markbox{4809}{When} \markbox{m4810}{$n$} \markbox{4811}{is} odd, \markbox{4812}{we} \markbox{4813}{have} \markbox{4814}{a} \markbox{4815}{continuous} \markbox{m4816}{$\sigma$}-equivariant \markbox{4817}{map}  \markbox{m4818}{$f:\mathbb CP^n\to \mathbb CP^n$} \markbox{4819}{defined} \markbox{4820}{by} 
		\[
		[x_1:x_2:\cdots:x_n:x_{n+1}]\mapsto [ x_2:- x_1:\cdots: x_{n+1}:- x_{n}]
		\] \markbox{4821}{which} \markbox{4822}{does} \markbox{4823}{not} \markbox{4824}{have} \markbox{4825}{any} \markbox{4826}{fixed} point.
		%Clearly, $f\circ \sigma=\sigma\circ f$, where  
		Thus, \markbox{4827}{using} \markbox{4828}{part} \textit{(2)} \markbox{4829}{of} \remref{necessary criterion for FFP}, \markbox{m4830}{$P(m,n)$} \markbox{4831}{does} \markbox{4832}{not} \markbox{4833}{have} fixed-point property.
	\end{proof}
	%\textcolor{red}{Converse of the above remark seems to be true.}
	\fi
	
	
	\iffalse
	We \markbox{4834}{need} \markbox{4835}{the} \markbox{4836}{following} \markbox{4837}{theorems} \markbox{4838}{in} \markbox{4839}{the} sequal.
	\begin{theorem}[Lefschetz Fixed-Point Theorem]\label{Lefschetz theorem}
		\markbox{4840}{Let} \markbox{m4841}{$X$} \markbox{4842}{be} \markbox{4843}{a} \markbox{4844}{finite} \markbox{4845}{simplicial} complex, \markbox{4846}{and} \markbox{4847}{let} \markbox{m4848}{$f : X \to X$} \markbox{4849}{be} \markbox{4850}{a} \markbox{4851}{continuous} map.  
		\markbox{4852}{Define} \markbox{4853}{the} \markbox{4854}{number}
		\[
		\tau(f) = \sum_{n} (-1)^n \operatorname{tr}\!\big(f_* : H_n(X) \to H_n(X)\big),
		\]
		\markbox{4855}{called} \markbox{4856}{the} \emph{Lefschetz number} \markbox{4857}{of} \markbox{m4858}{$f$}.  
		\markbox{4859}{If} \markbox{m4860}{$\tau(f) \ne 0$}, \markbox{4861}{then} \markbox{m4862}{$f$} \markbox{4863}{has} \markbox{4864}{at} \markbox{4865}{least} \markbox{4866}{one} \markbox{4867}{fixed} point.
	\end{theorem}
	\fi
	
	
	%\begin{theorem}[\cite{glover-homer}, Theorem 2]\label{GH2} For $k,n$ as in Theorem \ref{GH1}, the complex Grassmannian $\mathbb CG_{n,k}$ has the fixed-point property if and only if $k(n-k)$ is even. \end{theorem}
	
	
	\iffalse
	In (Theorem 2, \cite{glover-homer}), \markbox{4868}{it} \markbox{4869}{is} \markbox{4870}{proved} \markbox{4871}{that} \markbox{m4872}{$\mathbb CG_{n,k}$} \markbox{4873}{has} fixed-point \markbox{4874}{property} \markbox{4875}{if} \markbox{4876}{and} \markbox{4877}{only} \markbox{4878}{if} \markbox{m4879}{$k(n-k)$} \markbox{4880}{is} even, \markbox{4881}{provided} \markbox{4882}{either} (i) \markbox{m4883}{$k\leq 3$} \markbox{4884}{and} \markbox{m4885}{$n>2k$} \markbox{4886}{or} (ii) \markbox{m4887}{$k>3$} \markbox{4888}{and} \markbox{m4889}{$n> 2k^2-1$}.
	\markbox{4890}{It} \markbox{4891}{is} \markbox{4892}{well} \markbox{4893}{known} \markbox{4894}{that} \markbox{4895}{there} \markbox{4896}{exists} \markbox{4897}{a} fixed-point \markbox{4898}{free} \markbox{4899}{map}   \markbox{m4900}{$ L \mapsto L^\perp   $} \markbox{4901}{on}   \markbox{m4902}{$ \mathbb{C}G_{n,k}   $} \markbox{4903}{when}   \markbox{m4904}{$ k = n - k   $}.
	
	\markbox{4905}{The} \markbox{4906}{following} \markbox{4907}{remark} \markbox{4908}{follows} \markbox{4909}{easily} \markbox{4910}{from} \markbox{4911}{Theorem} \ref{hoffman} \cite{hoffman} \markbox{4912}{and} \markbox{4913}{Theorem} \ref{GH2} \cite{glover-homer}.
	\begin{remark}Let \markbox{m4914}{$f$} \markbox{4915}{be} \markbox{4916}{continuous} self-map \markbox{4917}{on} \markbox{m4918}{$\mathbb CG_{n,k}$} \markbox{4919}{such} \markbox{4920}{that}
		\markbox{m4921}{$ f^*(c_1) = \lambda c_1 $}, \markbox{m4922}{$ \lambda\neq 0 $}. \markbox{4923}{Assume} \markbox{4924}{that} \markbox{m4925}{$k(n-k)$} \markbox{4926}{is} \markbox{4927}{even} \markbox{4928}{and} \markbox{m4929}{$k\neq n-k$}. \markbox{4930}{Then} \markbox{m4931}{$f$} \markbox{4932}{has} \markbox{4933}{a} \markbox{4934}{fixed} point. 
	\end{remark}
	\begin{proof}
		\markbox{4935}{By} Theorem~\ref{hoffman} \cite{hoffman}, \markbox{4936}{the} \markbox{4937}{map} \markbox{m4938}{$f^*$} \markbox{4939}{is} \markbox{4940}{an} \markbox{4941}{Adams} map. \markbox{4942}{Using} \markbox{4943}{the} \markbox{4944}{computation} \markbox{4945}{of} \markbox{4946}{the} \markbox{4947}{Lefschetz} \markbox{4948}{number} \markbox{m4949}{$L(f)$} \markbox{4950}{from} \markbox{4951}{the} \markbox{4952}{proof} \markbox{4953}{of} Theorem~\ref{GH1}, \markbox{4954}{we} \markbox{4955}{observe} \markbox{4956}{that} \markbox{m4957}{$L(f) \ne 0$} \markbox{4958}{because} \markbox{m4959}{$k(n-k)$} \markbox{4960}{is} even. Thus, \markbox{m4961}{$f$} \markbox{4962}{must} \markbox{4963}{have} \markbox{4964}{a} \markbox{4965}{fixed} point. \markbox{4966}{This} \markbox{4967}{completes} \markbox{4968}{the} proof.
	\end{proof}
	\fi
	
	\subsection{} \markbox{4969}{Let} \markbox{4970}{us} \markbox{4971}{recall} \markbox{4972}{a} \markbox{4973}{well} \markbox{4974}{known} \markbox{4975}{result} \markbox{4976}{in} \markbox{4977}{coincidence} theory, \markbox{4978}{the} \markbox{4979}{Lefschetz} \markbox{4980}{Coincidence} Theorem, \markbox{4981}{which} \markbox{4982}{will} \markbox{4983}{be} \markbox{4984}{used} \markbox{4985}{to} \markbox{4986}{prove} \markbox{4987}{results} \markbox{4988}{in} \markbox{4989}{the} \markbox{4990}{rest} \markbox{4991}{of} \markbox{4992}{this} paper.\\
	
	\markbox{4993}{For} \markbox{4994}{a} \markbox{4995}{closed} \markbox{4996}{oriented} \markbox{4997}{manifold} \markbox{m4998}{$M$} \markbox{4999}{of} \markbox{5000}{dimension} \markbox{m5001}{$n$}, \markbox{5002}{let} \markbox{m5003}{$[M]\in H^n(M;\mathbb Q)$} \markbox{5004}{denote} \markbox{5005}{a} \markbox{5006}{chosen} \markbox{5007}{fundamental} class. \markbox{5008}{Then} \markbox{5009}{we} \markbox{5010}{have} \markbox{5011}{the} Poincar\'e  \markbox{5012}{duality} \markbox{5013}{isomorphism} \markbox{m5014}{$D_M:H^k(M;\mathbb Q)\to H_{n-k}(M;\mathbb Q)$}, \markbox{5015}{defined} \markbox{5016}{by} 
	\begin{equation}\label{poinc dua}
		D_M(\alpha)=[M]\frown \alpha, \, \forall \alpha\in H^k(M;\mathbb Z).
	\end{equation}

	
	\begin{theorem}[Lefschetz \markbox{5017}{Coincidence} Theorem]\label{LCT}
		\markbox{5018}{Let} \markbox{m5019}{$f,g$} \markbox{5020}{be} \markbox{5021}{two} \markbox{5022}{continuous} \markbox{5023}{maps} \markbox{5024}{on} \markbox{5025}{a} compact, connected, \markbox{5026}{oriented} \markbox{5027}{manifold} \markbox{m5028}{$M$} \markbox{5029}{of}  \markbox{5030}{dimension} \markbox{m5031}{$n$}. \markbox{5032}{The} \markbox{5033}{Lefschetz} \markbox{5034}{coincidence} \markbox{5035}{number} \markbox{5036}{is} \markbox{5037}{defined} \markbox{5038}{as} 
		$$L(f,g):= \sum_{i=0}^{n} (-1)^i \mathrm{tr}\big(D_M \circ g^* \circ D_M^{-1} \circ f_* \; : \; H_i(M;\mathbb{Q}) \longrightarrow H_i(M;\mathbb{Q})\big).
		$$ \markbox{5039}{If} \markbox{m5040}{$L(f,g) \neq 0$}, \markbox{5041}{then} \markbox{5042}{there} \markbox{5043}{exists} \markbox{m5044}{$x \in M$} \markbox{5045}{such} \markbox{5046}{that} \markbox{m5047}{$f(x) = g(x)$}.
	\end{theorem}
	\markbox{5048}{When}  \markbox{m5049}{$g=\operatorname{id}_M$}, \markbox{5050}{the} \markbox{5051}{theorem} \markbox{5052}{reduces} \markbox{5053}{to} \markbox{5054}{the} \markbox{5055}{Lefschetz} Fixed-Point \markbox{5056}{Theorem} \markbox{5057}{for} \markbox{m5058}{$M$}.\\
	
	\markbox{5059}{To} \markbox{5060}{study} \markbox{5061}{the} \markbox{5062}{coincidence} \markbox{5063}{theory} \markbox{5064}{of} \markbox{5065}{generalized} \markbox{5066}{Dold} \markbox{5067}{spaces} \markbox{5068}{fibred} \markbox{5069}{by} \markbox{5070}{complex} \markbox{5071}{Grassmannians} \markbox{5072}{over} \markbox{5073}{real} \markbox{5074}{projective} spaces, \markbox{5075}{it} \markbox{5076}{is} \markbox{5077}{helpful} \markbox{5078}{to} \markbox{5079}{first} \markbox{5080}{understand} \markbox{5081}{the} \markbox{5082}{coincidence} \markbox{5083}{theory} \markbox{5084}{of} \markbox{5085}{complex} Grassmannians, \markbox{5086}{a} \markbox{5087}{topic} \markbox{5088}{of} \markbox{5089}{independent} interest. \markbox{5090}{We} \markbox{5091}{now} \markbox{5092}{prove} \markbox{5093}{the} \markbox{5094}{following} \markbox{5095}{lemma} (cf. \markbox{5096}{Theorem} 2, \cite{glover-homer}) \markbox{5097}{to} \markbox{5098}{prove} \propref{CP of CGnk}.
	\begin{lemma}\label{sum neq 0}
		\markbox{5099}{Let} \markbox{m5100}{$d_{2i}$} \markbox{5101}{be} \markbox{5102}{the} \markbox{m5103}{$2i$}-th \markbox{5104}{Betti} \markbox{5105}{number} \markbox{5106}{of} \markbox{5107}{a} \markbox{5108}{complex} \markbox{5109}{Grassmannian} \markbox{m5110}{$\mathbb CG_{n,k}$} \markbox{5111}{with} \markbox{m5112}{$d=k(n-k)$} even. \markbox{5113}{Then} \markbox{5114}{the} \markbox{5115}{sum} \markbox{m5116}{\(\sum _{i=0}^d d_{2i}\lambda^i\neq 0,\, \forall \lambda \in \mathbb Q.\)}
	\end{lemma}
	\begin{proof}
		\markbox{5117}{Let} \markbox{5118}{us} \markbox{5119}{consider} \markbox{5120}{the} \markbox{5121}{sum} \markbox{m5122}{$\sum_{i=0}^d d_{2i}\lambda^i$}, \markbox{5123}{when} \markbox{m5124}{$\lambda$} \markbox{5125}{is} \markbox{5126}{an} integer. 
		Clearly, $$\sum_{i=0}^d d_{2i}\lambda^i \equiv 1 \pmod{\lambda}.$$ %(see Theorem 2 in \cite{glover-homer}). 
		Hence, \markbox{m5127}{$\sum_{i=0}^d d_{2i}\lambda^i \neq 0$}, \markbox{5128}{if} \markbox{m5129}{$\lambda \neq \pm 1$}. \markbox{5130}{When} \markbox{m5131}{$\lambda = 1$}, \markbox{5132}{the} \markbox{5133}{sum} \markbox{5134}{is} \markbox{5135}{also} \markbox{5136}{positive} \markbox{5137}{and} \markbox{5138}{therefore} nonzero. \markbox{5139}{It} \markbox{5140}{remains} \markbox{5141}{to} \markbox{5142}{consider} \markbox{5143}{the} \markbox{5144}{case} \markbox{5145}{where} \markbox{m5146}{$\lambda = -1$}.
		\markbox{5147}{Let} \markbox{m5148}{$\chi(\mathbb{R}G_{n,k})$} \markbox{5149}{denote} \markbox{5150}{the} Euler-Poincar{\'e} \markbox{5151}{characteristic} \markbox{5152}{of} \markbox{m5153}{$\mathbb{R}G_{n,k}$} \markbox{5154}{and} \markbox{5155}{be} \markbox{5156}{defined} \markbox{5157}{by} $$\chi(X):=\sum_{i\ge0}\dim H^i(\mathbb{R}G_{n,k};\mathbb{Z}_2)$$ \markbox{5158}{where} \markbox{m5159}{$\mathbb{R}G_{n,k}$} \markbox{5160}{denotes} \markbox{5161}{the} \markbox{5162}{Grassmannian} \markbox{5163}{of} \markbox{5164}{real} \markbox{m5165}{$k$}-planes \markbox{5166}{in} \markbox{m5167}{$\mathbb{R}^n$}.
		\markbox{5168}{Now} \markbox{5169}{we} \markbox{5170}{observe} \markbox{5171}{that} 
		\markbox{m5172}{$\sum_{i=0}^d d_{2i}(-1)^i = \chi(\mathbb{R}G_{n,k})$} \markbox{5173}{where} \markbox{m5174}{$d_{2i} = \operatorname{dim} H^{2i}(\mathbb{C}G_{n,k}; \mathbb{Q}) = \dim H^{i}(\mathbb{R}G_{n,k}; \mathbb{Z}_2)$}. 
		\markbox{5175}{It} \markbox{5176}{is} \markbox{5177}{a} \markbox{5178}{well} \markbox{5179}{known} \markbox{5180}{fact} \markbox{5181}{that} \markbox{m5182}{$\chi(\mathbb{R}G_{n,k}) \neq 0$} \markbox{5183}{if} \markbox{m5184}{$k(n-k)$} \markbox{5185}{is} even. 
		
		\markbox{5186}{Let} \markbox{5187}{us} \markbox{5188}{move} \markbox{5189}{to} \markbox{5190}{the} \markbox{5191}{other} \markbox{5192}{case} \markbox{5193}{where} \markbox{m5194}{$\lambda\in \mathbb{Q}\backslash \mathbb{Z}$}. \markbox{5195}{Suppose} \markbox{m5196}{$\sum_{i=0}^{d} d_{2i}\lambda^i =0$} \markbox{5197}{for} \markbox{5198}{some} \markbox{m5199}{$\lambda = \frac{p}{q}$} \markbox{5200}{where} \markbox{m5201}{$p$} \markbox{5202}{and} \markbox{m5203}{$q$} \markbox{5204}{are} \markbox{5205}{coprime} integers. \markbox{5206}{Since} \markbox{m5207}{$d_0 =d_d =1$}, \markbox{5208}{using} \markbox{5209}{the} \markbox{5210}{rational} \markbox{5211}{root} \markbox{5212}{theorem} \markbox{m5213}{$p|1$} \markbox{5214}{and} \markbox{m5215}{$q|1$}. Hence, \markbox{m5216}{$\lambda = \pm 1$}, \markbox{5217}{which} \markbox{5218}{is} \markbox{5219}{a} contradiction. Therefore, \markbox{5220}{we} \markbox{5221}{conclude} \markbox{5222}{that} \markbox{m5223}{$\sum_{i=0}^{d} d_{2i}\lambda^i \neq 0$} \markbox{5224}{for} \markbox{5225}{all} \markbox{m5226}{$\lambda\in \mathbb Q$}. 
		%Thus, the sum  $\sum _{i=1}^d d_{2i}\lambda^i$ is nonzero for every integer value of $\lambda.$ 
		\iffalse	Now, \markbox{5227}{suppose} \markbox{5228}{for} \markbox{5229}{contradiction} \markbox{5230}{that} \markbox{5231}{there} \markbox{5232}{exists} \markbox{5233}{a} \markbox{5234}{rational} \markbox{5235}{number} \markbox{m5236}{$\lambda = a/b$}  \markbox{5237}{written} \markbox{5238}{in} \markbox{5239}{lowest} \markbox{5240}{terms} (i.e., \markbox{m5241}{$\gcd(a,b)=1$}) \markbox{5242}{such} \markbox{5243}{that} \markbox{5244}{the} \markbox{5245}{sum} vanishes. \markbox{5246}{Note} \markbox{5247}{that} \markbox{5248}{in} \markbox{5249}{the} \markbox{5250}{polynomial} \markbox{m5251}{$\sum_{i=0}^{d} d_{2i}\lambda^i$}, \markbox{5252}{both} \markbox{5253}{the} \markbox{5254}{leading} \markbox{5255}{coefficient} \markbox{m5256}{$d_d$} \markbox{5257}{and} \markbox{5258}{the} \markbox{5259}{constant} \markbox{5260}{term} \markbox{m5261}{$d_0$} \markbox{5262}{are} \markbox{5263}{equal} \markbox{5264}{to} \markbox{m5265}{$1$}. \markbox{5266}{By} \markbox{5267}{the} \markbox{5268}{Rational} \markbox{5269}{Root} Theorem, \markbox{5270}{the} \markbox{5271}{rational} \markbox{5272}{root} \markbox{m5273}{$\lambda = a/b$} \markbox{5274}{must} \markbox{5275}{satisfy} \markbox{m5276}{$a \mid 1$} \markbox{5277}{and} \markbox{m5278}{$b \mid 1$}, \markbox{5279}{hence} \markbox{m5280}{$\lambda = \pm 1\in \mathbb Z$}. \markbox{5281}{Since} \markbox{5282}{this} \markbox{5283}{is} \markbox{5284}{not} possible, \fi
	\end{proof}
	
	
	
	\markbox{5285}{Denote} \markbox{5286}{the} \markbox{m5287}{$i$}-th \markbox{5288}{homology} \markbox{5289}{groups} 
	\markbox{m5290}{$H_i(\mathbb{C}G_{n,k}; \mathbb{Q}), \, H_i(\mathbb{S}^m; \mathbb{Q})$} \markbox{5291}{and} 
	\markbox{m5292}{$H_i(\mathbb{S}^m \times \mathbb{C}G_{n,k}; \mathbb{Q})$}, \markbox{5293}{by} \markbox{m5294}{$H_i^{\mathbb{C}G}, H_i^{\mathbb{S}}$} \markbox{5295}{and} \markbox{m5296}{$H_i^{\times}$}, respectively. \markbox{5297}{Let} \markbox{m5298}{$d$} \markbox{5299}{denote} \markbox{5300}{the} \markbox{5301}{complex} \markbox{5302}{dimension} \markbox{5303}{of} \markbox{m5304}{$\mathbb CG_{n,k}$}, \markbox{5305}{given} \markbox{5306}{by} \markbox{m5307}{$d = k(n - k)$}. \markbox{5308}{Then} \markbox{5309}{we} \markbox{5310}{have} \markbox{5311}{the} \markbox{5312}{following} proposition.
	\begin{proposition}\label{CP of CGnk}
		\markbox{5313}{Consider} \markbox{5314}{a} \markbox{5315}{complex} \markbox{5316}{Grassmannian} \markbox{m5317}{$\mathbb{C}G_{n,k}$} \markbox{5318}{such} \markbox{5319}{that} \markbox{5320}{the} \markbox{5321}{hypothesis} \eqref{Homer} \markbox{5322}{is} \markbox{5323}{satisfied} \markbox{5324}{and} \markbox{m5325}{$k(n-k)$} \markbox{5326}{is} even. \markbox{5327}{Let} \markbox{m5328}{$g $} \markbox{5329}{be} \markbox{5330}{a} \markbox{5331}{continuous} \markbox{5332}{map} \markbox{5333}{on} \markbox{m5334}{$\mathbb{C}G_{n,k}$} \markbox{5335}{with} \markbox{5336}{nonzero} \markbox{5337}{Brouwer} degree. \markbox{5338}{Then} \markbox{5339}{the} \markbox{5340}{pair} \markbox{m5341}{$(\mathbb{C}G_{n,k}, g)$} \markbox{5342}{has} \markbox{5343}{the} \markbox{5344}{coincidence} property.
	\end{proposition}
	
	
	\begin{proof}
	Self-maps \markbox{5345}{with} \markbox{5346}{nonzero} \markbox{5347}{Brouwer} \markbox{5348}{degree} \markbox{5349}{induces} \markbox{5350}{automorphisms} \markbox{5351}{in} \markbox{5352}{the} \markbox{5353}{rational} \markbox{5354}{cohomology} algebra. \markbox{5355}{Using} \markbox{5356}{Theorem} \ref{hom and hof} \markbox{5357}{part} \textit{(i)}, \markbox{5358}{there} \markbox{5359}{exist} \markbox{5360}{a} \markbox{5361}{nonzero} \markbox{5362}{rational} \markbox{m5363}{$\lambda$} \markbox{5364}{such} \markbox{5365}{that} \markbox{m5366}{$g^*(c_i)=\lambda^ic_i, \forall i\in I.$} \markbox{5367}{Let} \markbox{m5368}{$f$} \markbox{5369}{be} \markbox{5370}{a} \markbox{5371}{continuous} \markbox{5372}{map} \markbox{5373}{on} \markbox{m5374}{$\mathbb CG_{n,k}$} \markbox{5375}{and} \markbox{5376}{using} \thmref{hom and hof} \markbox{5377}{part} \textit{(i)}, \markbox{5378}{there} \markbox{5379}{exists} \markbox{m5380}{$\mu \in \mathbb Q$} \markbox{5381}{such} \markbox{5382}{that}
		\[
		f^*(c_i)=\mu^ic_i, \forall i\in I.
		\]
		\markbox{5383}{Then} \markbox{5384}{by} \markbox{5385}{the} \markbox{5386}{Universal} \markbox{5387}{Coefficient} Theorem, \markbox{m5388}{$ \hom_{\mathbb{Q}} (H_i^{\mathbb CG};\mathbb Q) \cong H^i_{\mathbb CG}$} non-canonically \markbox{5389}{which} \markbox{5390}{implies} \markbox{5391}{that} 
		\begin{align*}
			\varphi \circ f_* &= f^*(\varphi) , \, \forall
			\varphi \in \hom _{\mathbb Q}\big(H_{2i}^{\mathbb CG}, \mathbb Q\big)\cong H^{2i}_{\mathbb CG}.\\
			\varphi(f_*(x))&=(f^*(\varphi))(x)= \mu^i\varphi(x)=\varphi(\mu^ix),\; \forall x\in H_{2i}^{\mathbb CG}.
		\end{align*}
		\markbox{5392}{The} \markbox{5393}{last} \markbox{5394}{equation} \markbox{5395}{implies} \markbox{5396}{that} \markbox{m5397}{$f_*(x)=\mu^ix,\, \forall x\in  H_{2i}^{\mathbb CG}.$}
		%This implies for all $\varphi\in \hom_{\mathbb Q}(H_{2i}^{\mathbb CG},\mathbb Q)$ and $x\in H_{2i}^{\mathbb CG}$, we have
		Now \markbox{5398}{observe} \markbox{5399}{that} \markbox{m5400}{$D\circ g^*\circ D^{-1}\circ f_*: H_{2i}^{\mathbb CG}\to H_{2i}^{\mathbb CG}$} \markbox{5401}{is} \markbox{5402}{given} \markbox{5403}{by} 
		\[
		D\circ g^*\circ D^{-1}\circ f_*(x)=D\circ g^*\circ D^{-1}(\mu^ix)=\mu^iD\circ g^*(D^{-1}x)=\mu^iD(\lambda^{d-i}D^{-1}x)=\mu^i\lambda^{d-i}x.
		\]
		\markbox{5404}{Thus} \markbox{5405}{for} \markbox{m5406}{$x\in H_{2i}^{\mathbb{C}G}$}, \markbox{5407}{the} \markbox{5408}{Lefschetz} \markbox{5409}{coincidence} \markbox{5410}{number} \markbox{5411}{is} \markbox{5412}{given} \markbox{5413}{by}
		\[
		\begin{array}{ll}
			L(f,g) &= \sum_{i=0}^d (-1)^{2i}
			\mathrm{tr}(D\circ g^*\circ D^{-1}\circ f_*(x)) \\[6pt]
			&=  \sum_{i=0}^d d_{2i}\mu^i\lambda^{d-i}  \\[6pt]
			&= \lambda^d \sum_{i=0}^d d_{2i}(\mu/\lambda)^i \neq 0 \quad (\because \lambda\neq 0) 
		\end{array}
		\]
		\markbox{5414}{where} \markbox{m5415}{$d_{2i}$} \markbox{5416}{denotes} \markbox{m5417}{$\dim_{\mathbb Q}H^{2i}_{\mathbb CG}$} \markbox{5418}{and} \markbox{5419}{the} \markbox{5420}{last} \markbox{5421}{equation} \markbox{5422}{holds} \markbox{5423}{by} \markbox{5424}{using} \lemref{sum neq 0}. Therefore, \markbox{5425}{using} \thmref{LCT} \markbox{5426}{the} \markbox{5427}{pair} \markbox{m5428}{$(\mathbb{C}G_{n,k},g)$} \markbox{5429}{has} \markbox{5430}{the} \markbox{5431}{coincidence} property.
	\end{proof}
	
	\iffalse{}  \textcolor{teal}{
		In the case of a quaternionic Grassmannian $\mathbb H G_{n,k}$, an Adams operation of degree $-1$ cannot be induced by any self-map, since its action on $H^*(\mathbb H G_{n,k};\mathbb Z/3)$ does not commute with the reduced third power operation (see Theorem~2(2) of \cite{glover-homer}). Consequently, the value $\lambda=-1$ cannot occur in the sum $\sum_{i=0}^{d} d_{2i}\lambda^i$ from Lemma~\ref{sum neq 0}, so the sum is nonzero in all cases, regardless of the parity of $d=k(n-k)$. Proceeding as in Proposition~\ref{CP of CGnk}, one obtains the following extension of Theorem~2(2) in \cite{glover-homer}:
		\begin{proposition}\label{CP of HGnk}
			For any homotopy equivalence $g$ of a quaternionic Grassmannian $\mathbb H G_{n,k}$, the pair $(\mathbb H G_{n,k},g)$ has the coincidence property.
		\end{proposition}
	}
	\fi
	
	
	
	
	\subsection{} \markbox{5432}{Denote} \markbox{5433}{by} \markbox{m5434}{$H_*^\times = \bigoplus_{i\geq 0} H_i^{\times},\, H_*^{\mathbb{C}G} = \bigoplus_{i\geq 0} H_i^{\mathbb{C}G},\, H_*^{\mathbb{S}} = \bigoplus_{i\geq 0} H_i^{\mathbb{S}}$} \markbox{5435}{and} \markbox{m5436}{$\vartheta$} \markbox{5437}{the} \markbox{5438}{fundamental} \markbox{5439}{class} \markbox{m5440}{$[\mathbb S^m]\in H_m^{\mathbb S}$}. \markbox{5441}{Let} \markbox{m5442}{$\{v_q\}$} \markbox{5443}{be} \markbox{5444}{a} \markbox{5445}{homogeneous} \markbox{5446}{basis} \markbox{5447}{of} \markbox{m5448}{$H_*^{\mathbb{C}G}$}, \markbox{5449}{and} \markbox{5450}{let} \markbox{m5451}{$\{\delta_{v_q}\}$} \markbox{5452}{denote} \markbox{5453}{the} \markbox{5454}{corresponding} \markbox{5455}{dual} \markbox{5456}{basis} \markbox{5457}{of}  
	\markbox{m5458}{$\operatorname{Hom}(H_*^{\mathbb{C}G}, \mathbb Q) \cong H^*_{\mathbb{C}G}$}, \markbox{5459}{such} \markbox{5460}{that}  
	\markbox{m5461}{$ \delta_{v_q} (v_p) = \delta_{qp}$} \markbox{5462}{where} \markbox{m5463}{$\delta_{qp}$} \markbox{5464}{is} \markbox{5465}{the} \markbox{5466}{Kronecker} \markbox{5467}{delta} function. \markbox{5468}{Without} \markbox{5469}{loss} \markbox{5470}{of} generality, \markbox{5471}{assume} \markbox{5472}{that} \markbox{m5473}{$1=v_0 \in \{v_i\}$} \markbox{5474}{represents} \markbox{5475}{the} \markbox{5476}{generator} \markbox{5477}{of} \markbox{m5478}{$H_0^{\mathbb{C}G} \cong \mathbb Q$}.
	
	\markbox{5479}{Over} \markbox{m5480}{$\mathbb Q$}, \markbox{5481}{the} K\"unneth \markbox{5482}{Theorem} \markbox{5483}{yields} \markbox{5484}{the} \markbox{5485}{following} \markbox{5486}{decompositions} 
	\iffalse	Set \markbox{m5487}{$d = 2k(n - k)$}, \markbox{m5488}{$m=2s$} \markbox{5489}{and} \markbox{m5490}{$t={n \choose k}$}. 
	
	\markbox{5491}{By} Remark~\ref{lift}, \markbox{5492}{every} \markbox{5493}{continuous} \markbox{5494}{map} \markbox{m5495}{$f$} \markbox{5496}{on} \markbox{m5497}{$P(m,n,k)$} \markbox{5498}{admits} \markbox{5499}{a} \markbox{5500}{lift} 
	\markbox{m5501}{$\tilde f$} \markbox{5502}{on} \markbox{m5503}{$\mathbb{S}^m \times \mathbb{C}G_{n,k}$} \markbox{5504}{satisfying} 
	\markbox{m5505}{$f \circ \pi = \pi \circ \tilde f,$}
	\markbox{5506}{where} \markbox{m5507}{$\pi : \mathbb{S}^m \times \mathbb{C}G_{n,k} \to P(m,n,k)$} \markbox{5508}{is} \markbox{5509}{the} \markbox{5510}{double} \markbox{5511}{covering} map. 
	\markbox{5512}{We} \markbox{5513}{continue} \markbox{5514}{to} \markbox{5515}{denote} \markbox{5516}{such} \markbox{5517}{lifts} \markbox{5518}{by} \markbox{5519}{the} \markbox{5520}{corresponding} \markbox{5521}{maps} \markbox{5522}{on} \markbox{m5523}{$P(m,n,k)$} \markbox{5524}{with} \markbox{5525}{a} tilde.
	\markbox{5526}{By} Theorem~\ref{main thm}, \markbox{5527}{there} \markbox{5528}{exists}  \markbox{m5529}{$\lambda \in \mathbb{Q}$} \markbox{5530}{or} \markbox{m5531}{$P_i\in H^{2i-m}_{\mathbb CG} \;\forall i\in I$} \markbox{5532}{with} \markbox{5533}{some} \markbox{m5534}{$P_i\neq 0$}, \markbox{5535}{such} \markbox{5536}{that} \
	
	\begin{equation}
		\text{either (i) } 
		\tilde f^*(c_i) = \lambda^i c_i,\forall i\in I, \text{or (ii) } \tilde f^*(c_i)=uP_i, \  \forall i\in I , 
	\end{equation}
	\begin{equation}
		\text{and either (iii) }\tilde f^*(u) = \mu u \text{ for some }\mu\in \mathbb Q, \text{or (iv) }\tilde f^*(u) \in H^m_{\mathbb{C}G}.
	\end{equation}
	
	
	%We have the induced maps in homology and cohomology $\tilde f_*$ and $\tilde f^*$, respectively, related by the Kronecker pairing 
	%$\langle \;,\;\rangle:H^i_\times\times H_i^\times\to \mathbb Q$ defined by
	%\[
	%\langle \tilde f^* \varphi, x\rangle = \langle \varphi, \tilde f_* x\rangle 
	%\quad \text{for all } \varphi \in H^i_\times,\, x \in H_i^\times.
	%\]
	%and $u \in H^*_\times$ corresponds to the fundamental class 
	%of $[\mathbb S^m] \in H^{\mathbb S}_m \subseteq H^\times_m.$
	
	%Fix a basis $\{1, v_1, v_2, \ldots, v_t\}$ for $H_*^{\mathbb{C}G}$. 
	%The corresponding dual basis of $\hom(H_*^{\mathbb{C}G
		%}, \mathbb{Q}) \cong H^*_{\mathbb{C}G}$ is 
	%$\{\delta_1, \delta_{v_1}, \delta_{v_2}, \ldots, \delta_{v_t}\}$, 
	%where \ $\delta_{v_i}(x) = \langle \delta_{v_i}, x \rangle$ defined to be $1$ if and only if $x=v_i$, and $0$ otherwise, for all $v_i$.
	\fi  
	\begin{equation}\label{kunneth}
		H_i^\times \cong H_i^{\mathbb{C}G} \oplus (\vartheta \otimes H_{i-m}^{\mathbb{C}G}), 
		\qquad 
		H^i_\times \cong H^i_{\mathbb{C}G} \oplus uH^{i-m}_{\mathbb{C}G},
	\end{equation}
	\markbox{5537}{where}  \markbox{m5538}{$u \in H^m_{\times} \cong \operatorname{Hom}(H_m^{\times}, \mathbb Q)$} \markbox{5539}{corresponds} \markbox{5540}{to} \markbox{5541}{the} \markbox{5542}{element} \markbox{m5543}{$\delta_{\vartheta\otimes 1}$}.
	
	
	\markbox{5544}{Using} \eqref{kunneth}, \markbox{5545}{we} \markbox{5546}{can} \markbox{5547}{extend} \markbox{5548}{the} \markbox{5549}{chosen} \markbox{5550}{basis} \markbox{m5551}{$\{v_q\}$} \markbox{5552}{of} \markbox{m5553}{$H_*^{\mathbb{C}G}$} \markbox{5554}{to} \markbox{m5555}{$	\{v_q\} \cup \{\vartheta \otimes v_q\}$} \markbox{5556}{of} \markbox{m5557}{$H_*^\times$} \markbox{5558}{such} \markbox{5559}{that} \markbox{5560}{the} \markbox{5561}{corresponding} \markbox{5562}{dual} \markbox{5563}{basis} \markbox{5564}{can} \markbox{5565}{also} \markbox{5566}{be} \markbox{5567}{extended} \markbox{5568}{from} \markbox{m5569}{$\{\delta_{v_q}\}$} \markbox{5570}{of} \markbox{m5571}{$ \hom (H_*^{\mathbb{C}G};\mathbb{Q})$} \markbox{5572}{to} \markbox{m5573}{$\{\delta_{v_q}\} \cup \{\delta_{\vartheta \otimes v_q}\} $} \markbox{5574}{of} \markbox{m5575}{$\hom (H_*^{\times};\mathbb{Q})$} satisfying:
	%\[\{v_i\} \cup \{\vartheta \otimes v_i\} \subseteq H_*^\times, \qquad 	\{\delta_{v_i}\} \cup \{\delta_{\vartheta \otimes v_i}\} \subseteq H^*_\times.\]
	%Since $\delta_{U \otimes 1} = \delta_U = u \in H^m_{\mathbb S}$, we write $\delta_{U \otimes v_i}$ simply as $u\,\delta_{v_i}$.With respect to these bases, the Kronecker pairing  $\langle\,\cdot\,,\cdot\,\rangle : H^*_\times \times H_*^\times \longrightarrow \mathbb Q$satisfies 
	\begin{equation}\label{Kronecker relations}
		\delta_{v_q}( v_p) = \delta_{qp}, \quad
		\delta_{v_q}( \vartheta \otimes v_p)= 0, \quad
		\delta_{\vartheta\otimes v_q}(v_p) = 0, \quad
		\delta_{\vartheta\otimes v_q}(\vartheta \otimes v_p)= \delta_{qp}.
	\end{equation}
	%since $\langle u, \vartheta\rangle=1$ and $u$ (resp. $\vartheta$) kills the other basis elements.Thus the matrix of the pairing in these bases is the identity, hence the  map$\kappa_i: H^i_\times \to \operatorname{Hom}(H_i^\times,\mathbb Q)$, $\kappa_i(\varphi)(x)=\langle \varphi, x\rangle$, is an isomorphism for all $i$. 
	Let \markbox{m5576}{$f$} \markbox{5577}{be} \markbox{5578}{a} \markbox{5579}{continuous} \markbox{5580}{function} \markbox{5581}{on} \markbox{m5582}{$P(m,n,k)$}. \markbox{5583}{Using} \remref{lift} \markbox{5584}{and} \markbox{5585}{the} \markbox{5586}{Universal} \markbox{5587}{Coefficient} Theorem, \markbox{5588}{there} \markbox{5589}{exist} \markbox{5590}{a} \markbox{5591}{lift} \markbox{m5592}{$\tilde{f}$} \markbox{5593}{on} \markbox{m5594}{$\mathbb{S}^m\times \mathbb{C}G_{n,k}$} \markbox{5595}{satisfying} \begin{equation}\label{comm with phi}
		\varphi \circ \tilde{f}_* = \tilde{f}^*(\varphi) , \, \forall
		\varphi \in \hom _{\mathbb Q}\big(H_{2i}^{\mathbb CG}, \mathbb Q\big)\cong H^{2i}_{\mathbb CG}.
	\end{equation}
	
		Poincar\'e \markbox{5596}{duality} \markbox{5597}{on} \markbox{m5598}{$\mathbb S^{m}\times \mathbb C G_{n,k}$} \markbox{5599}{can} \markbox{5600}{be} \markbox{5601}{described} \markbox{5602}{in} \markbox{5603}{terms} \markbox{5604}{of} \markbox{5605}{the} \markbox{5606}{duality} \markbox{5607}{on} \markbox{5608}{the} \markbox{5609}{Grassmannian} factor. \markbox{5610}{Let} 
		\markbox{m5611}{$D_{\mathbb C G}\colon H^{i}_{\mathbb C G}\to H_{2d-i}^{\mathbb C G}$} 
		\markbox{5612}{be} \markbox{5613}{the} Poincar\'e \markbox{5614}{duality} \markbox{5615}{isomorphism} \markbox{5616}{defined} \markbox{5617}{in} \eqref{poinc dua} \markbox{5618}{for} \markbox{m5619}{$\mathbb C G_{n,k}$}, \markbox{5620}{where} \markbox{m5621}{$d=k(n-k)$}.  
		\markbox{5622}{The} Poincar\'e \markbox{5623}{duality} \markbox{5624}{isomorphism} \markbox{5625}{on} \markbox{5626}{the} \markbox{5627}{product}
		\markbox{5628}{is} \markbox{5629}{then} \markbox{5630}{determined} \markbox{5631}{on} \markbox{5632}{the} \markbox{5633}{basis} \markbox{5634}{elements} \markbox{5635}{by}  
		\begin{equation}
			D\colon H^{j}_{\times}\to H_{m+2d-j}^{\times}, \quad \delta_{v_i}\mapsto \vartheta\otimes D_{\mathbb C G}(\delta_{v_i})\quad\text{and}\quad \delta_{\vartheta\otimes v_i} \mapsto D_{\mathbb C G}(\delta_{v_i}).
		\end{equation}
		\markbox{5636}{We} \markbox{5637}{are} \markbox{5638}{now} \markbox{5639}{ready} \markbox{5640}{to} \markbox{5641}{establish} \markbox{5642}{the} \markbox{5643}{following} lemmas, \markbox{5644}{which} \markbox{5645}{will} \markbox{5646}{be} \markbox{5647}{useful} \markbox{5648}{in} \markbox{5649}{the} sequel.
	\begin{lemma}\label{image of homf}
		\markbox{5650}{Let} \markbox{m5651}{$f$} \markbox{5652}{be} \markbox{5653}{a} \markbox{5654}{continuous} \markbox{5655}{function} \markbox{5656}{on} \markbox{m5657}{$P(m,n,k)$} \markbox{5658}{and} \markbox{m5659}{$\tilde{f}$} \markbox{5660}{be} \markbox{5661}{the} \markbox{5662}{lift} \markbox{5663}{defined} \markbox{5664}{in} \remref{lift} \markbox{5665}{such} \markbox{5666}{that} \markbox{m5667}{$\tilde{f}^*(c_1)\neq au,\, a\in \mathbb{Q}$} \markbox{5668}{and} \markbox{m5669}{$k<n-k$}. \markbox{5670}{Then} \markbox{5671}{there} \markbox{5672}{exist} \markbox{m5673}{$\lambda \in \mathbb{Q}\backslash \{0\}$} \markbox{5674}{and} \markbox{m5675}{$\mu \in \mathbb{Q}$} \markbox{5676}{such} \markbox{5677}{that} \markbox{5678}{the} \markbox{5679}{induced} \markbox{5680}{map} \markbox{m5681}{$\tilde{f}_*$} \markbox{5682}{on} \markbox{m5683}{$H_*^{\times}$} \markbox{5684}{is} \markbox{5685}{of} \markbox{5686}{the} \markbox{5687}{following} form.
		\begin{enumerate}
			\item Either \markbox{m5688}{$\tilde{f}_*(\vartheta\otimes x) = \mu \lambda^i (\vartheta \otimes x),\, \forall x\in H_{2i}^{\mathbb{C}G}$} \markbox{5689}{or} \markbox{m5690}{$\tilde{f}_*(\vartheta\otimes x) \in H_*^{\mathbb{CG}},\, \forall x \in H_*^{\mathbb{C}G}$}.
			\item \markbox{m5691}{$\tilde{f}_*(x) = \lambda^i x + \vartheta\otimes y,$} \markbox{5692}{for} \markbox{5693}{some} \markbox{m5694}{$y \in H_{2i-m}^{\mathbb{C}G}, \, \forall x \in H_{2i}^{\mathbb{C}G}.$}
		\end{enumerate}
		Moreover, \markbox{m5695}{$y=0$} \markbox{5696}{in} \textit{(2)} \markbox{5697}{if} \markbox{m5698}{$\tilde{f}_*(\vartheta\otimes x) = \mu \lambda^i (\vartheta \otimes x),\, \forall x\in H_{2i}^{\mathbb{C}G}$}.
	\end{lemma}
	\begin{proof}
		\markbox{5699}{Using} \thmref{main thm}, \markbox{5700}{there} \markbox{5701}{exist} \markbox{m5702}{$\lambda \in \mathbb{Q}\backslash \{0\}$} \markbox{5703}{such} \markbox{5704}{that} \markbox{m5705}{$\tilde{f}^*(c_i) = \lambda^i c_i, \forall i \in I$} \markbox{5706}{and} \markbox{5707}{either} \markbox{m5708}{$\tilde{f}^*(u) = \mu u,\, \mu \in \mathbb{Q}$} \markbox{5709}{or} \markbox{m5710}{$\tilde{f}^*(u)\in H^*_{\mathbb{C}G}$}. \markbox{5711}{It} \markbox{5712}{is} \markbox{5713}{sufficient} \markbox{5714}{to} \markbox{5715}{prove} \markbox{5716}{the} \markbox{5717}{result} \markbox{5718}{for} \markbox{5719}{the} \markbox{5720}{chosen} \markbox{5721}{basis} \markbox{m5722}{$\{v_q\}\cup\{\vartheta\otimes v_q\}$} \markbox{5723}{of} \markbox{m5724}{$H_{*}^{\times}$}. 
		
		\markbox{5725}{Let} \markbox{5726}{us} \markbox{5727}{consider} \markbox{5728}{the} \markbox{5729}{first} \markbox{5730}{case} \markbox{5731}{where} \markbox{m5732}{$\tilde{f}^*(u) = \mu u$}. \markbox{5733}{Using} \markbox{m5734}{$H^*_{\mathbb{C}G}\cong \hom(H_*^{\mathbb{C}G},\mathbb{Q})$}, \markbox{5735}{we} \markbox{5736}{have} \begin{equation}\label{fstarco}
			\tilde{f}^*(\delta_{v_p}) = \lambda^i \delta_{v_{p}}, \, \forall v_{p}\in H_{2i}^{\mathbb{C}G}, \quad \tilde{f}^*(\delta_{\vartheta\otimes v_{p}})  = \mu \lambda^i (\delta_{\vartheta \otimes v_{p}}), \, \forall v_{p}\in H_{2i}^{\mathbb{C}G}.
		\end{equation}
		\markbox{5737}{If} \markbox{m5738}{$m$} \markbox{5739}{is} odd, \markbox{5740}{then} \markbox{5741}{the} \markbox{5742}{coefficient} \markbox{5743}{of} \markbox{5744}{any} \markbox{5745}{basis} \markbox{5746}{element} \markbox{m5747}{$v_p \in H_*^{\mathbb{C}G}$} \markbox{5748}{in} \markbox{m5749}{$\tilde{f}_*(\vartheta \otimes v_q)$} \markbox{5750}{and} \markbox{m5751}{$\vartheta \otimes v_p$} \markbox{5752}{in} \markbox{m5753}{$\tilde{f}_*(v_q)$} \markbox{5754}{is} \markbox{5755}{zero} \markbox{5756}{because} \markbox{m5757}{$\tilde{f}_*$} \markbox{5758}{is} \markbox{5759}{a} \markbox{5760}{graded} map. \markbox{5761}{Let} \markbox{5762}{us} \markbox{5763}{consider} \markbox{5764}{the} \markbox{5765}{case} \markbox{5766}{where} \markbox{m5767}{$m=2s$}.
		\markbox{5768}{By} \eqref{comm with phi} \markbox{5769}{and} \eqref{fstarco}, \markbox{5770}{the} \markbox{5771}{coefficient} \markbox{5772}{of} \markbox{5773}{a} \markbox{5774}{basis} \markbox{5775}{element} \markbox{m5776}{$v_p\in H_{2i+m}^{\mathbb{C}G}$} \markbox{5777}{in} \markbox{m5778}{$\tilde{f}_*(\vartheta \otimes v_q)$} \markbox{5779}{written} \markbox{5780}{as} \markbox{5781}{a} \markbox{m5782}{$\mathbb{Q}$}-linear \markbox{5783}{combination} \markbox{5784}{of} \markbox{5785}{the} \markbox{5786}{basis} \markbox{5787}{elements} \markbox{5788}{from} \markbox{m5789}{$\{v_q\}\cup\{\vartheta\otimes v_q\}$} \markbox{5790}{is} \markbox{5791}{the} following:
		$$	 \delta_{v_p}\circ \tilde f_*(\vartheta\otimes v_q)
		=  \tilde f^* (\delta_{v_p}) (\vartheta\otimes v_q)
		=  \lambda^{i+s} \delta_{v_p}(\vartheta\otimes v_q)
		= 0,\, \forall v_q \in H_{2i}^{\mathbb{C}G}$$
		\markbox{5792}{and} \markbox{5793}{the} \markbox{5794}{coefficient} \markbox{5795}{of} \markbox{5796}{a} \markbox{5797}{basis} \markbox{5798}{element} \markbox{m5799}{$\vartheta \otimes v_p\in \vartheta \otimes H_{2i}^{\mathbb{C}G}$} \markbox{5800}{in} \markbox{m5801}{$\tilde{f}_*(\vartheta \otimes v_q)$} \markbox{5802}{is}
		$$ \delta_{\vartheta\otimes v_p}\circ \tilde f_*(\vartheta\otimes v_q)
		=  \tilde f^*(\delta_{\vartheta\otimes v_p})(\vartheta\otimes v_q)
		=  \mu \lambda^{i}\;\delta_{\vartheta \otimes v_p}(\vartheta\otimes v_q)= \mu\lambda^{i}\delta_{pq},\, \forall v_q \in H^{\mathbb{C}G}_{2i}.$$
		\markbox{5803}{This} \markbox{5804}{implies} \markbox{5805}{that} $$\tilde{f}_*(\vartheta \otimes v_q) = \mu \lambda^i (\vartheta \otimes v_q), \, \forall v_q \in H_{2i}^{\mathbb{C}G}.$$ 
		\markbox{5806}{Using} \markbox{5807}{similar} \markbox{5808}{calculations} \markbox{5809}{given} above, \markbox{5810}{it} \markbox{5811}{is} \markbox{5812}{easy} \markbox{5813}{to} \markbox{5814}{show} \markbox{5815}{that} $$\delta_{v_p}\circ \tilde{f}_*(v_q) = \lambda^i \delta_{pq},\, \forall v_q \in H_{2i}^{\mathbb{C}G},\quad \delta_{\vartheta \otimes v_p}\circ \tilde{f}_*(v_q)=0,\, \forall v_q\in H_{2i}^{\mathbb{C}G}.$$ Therefore, \markbox{m5816}{$\tilde{f}_*(v_q) = \lambda^i v_q,\, \forall v_q \in H_{2i}^{\mathbb{C}G}$}.\\
		
		\markbox{5817}{If} \markbox{m5818}{$\tilde{f}^*(u)\in H^*_{\mathbb{C}G}$}. \markbox{5819}{Again} \markbox{5820}{using} \markbox{m5821}{$H^*_{\mathbb{C}G}\cong \hom(H_*^{\mathbb{C}G},\mathbb{Q})$}, \markbox{5822}{we} \markbox{5823}{have} \begin{equation}\label{second u}
			\tilde{f}^*(\delta_{v_p}) = \lambda^i \delta_{v_{p}}, \, \forall v_{p}\in H_{2i}^{\mathbb{C}G}, \quad \tilde{f}^*(\delta_{\vartheta\otimes v_{p}})  \in H^*_{\mathbb{C}G},\, \forall  v_{p}\in H_{2i}^{\mathbb{C}G}.
		\end{equation}
		\markbox{5824}{By} \eqref{comm with phi} \markbox{5825}{and} \eqref{second u}, \markbox{5826}{we} \markbox{5827}{get} \markbox{m5828}{$\delta_{v_p}\circ \tilde{f}_*(v_q) = \lambda^i \delta_{pq},\, \forall v_q \in H_{2i}^{\mathbb{C}G},$} \markbox{5829}{which} \markbox{5830}{implies} \markbox{5831}{that} \markbox{m5832}{$\tilde{f}_*(x) = \lambda^i x + \vartheta\otimes y,$} \markbox{5833}{for} \markbox{5834}{some} \markbox{m5835}{$y \in H_{2i-m}^{\mathbb{C}G}, \, \forall x \in H_{2i}^{\mathbb{C}G}.$} \markbox{5836}{Note} \markbox{5837}{that} \markbox{m5838}{$\tilde f^*(\delta_{\vartheta\otimes v_p})\in H_{\mathbb CG}^*$} \markbox{5839}{and} \markbox{5840}{equal} \markbox{5841}{to} \markbox{5842}{some} \markbox{m5843}{$\sum a_j\delta_{v_j}$}. \markbox{5844}{Then} 
		%coefficient of $\vartheta \otimes v_p $ in $\tilde f_*(\vartheta \otimes v_q)$ is 
		\begin{equation}\label{computation}
			\delta_{\vartheta\otimes v_p}\circ \tilde f_*(\vartheta\otimes v_q)= \tilde f^*(\delta_{\vartheta\otimes v_p})(\vartheta\otimes v_q)=\sum a_j\delta_{v_j}(\vartheta\otimes v_q)=0.
		\end{equation}
		Hence, \markbox{m5845}{$\tilde f_*(\vartheta \otimes v_q)\in H_*^{\mathbb CG}$} \markbox{5846}{for} \markbox{5847}{all} \markbox{m5848}{$\vartheta \otimes v_q\in\vartheta\otimes H_*^{\mathbb CG} $}.
		%	$ \text{ or } \tilde{f}^*(\vartheta\otimes v_{p}) \in H_*^{\mathbb{C}G},\, \forall v_{p}\in H_*^{\mathbb{C}G}$
	\end{proof}
\iffalse	\begin{remark}\label{lemrem}
		\markbox{5849}{For} \markbox{5850}{the} \markbox{5851}{case} \markbox{m5852}{$k=n-k$} \markbox{5853}{in} \lemref{image of homf}, \markbox{5854}{if} \markbox{m5855}{$\tilde{f}^*(c_i) = \lambda^i c_i$} \markbox{5856}{then} \markbox{5857}{we} \markbox{5858}{will} \markbox{5859}{get} \markbox{5860}{the} \markbox{5861}{same} \markbox{5862}{result} \markbox{5863}{otherwise} \markbox{5864}{when} \markbox{m5865}{$\tilde{f}^*(c_i) = (-\lambda)^i c_i$} \markbox{5866}{then} \markbox{m5867}{$\lambda$} \markbox{5868}{will} \markbox{5869}{be} \markbox{5870}{replaced} \markbox{5871}{by} \markbox{m5872}{$-\lambda$} \markbox{5873}{in} \lemref{image of homf}. 
	\end{remark}\fi
	\begin{lemma}\label{image of homf under hom}
		\markbox{5874}{Assume} \markbox{5875}{that} \markbox{5876}{the} \markbox{5877}{hypothesis} \eqref{Homer} \markbox{5878}{is} satisfied. 	\markbox{5879}{Let} \markbox{m5880}{$f$} \markbox{5881}{be} \markbox{5882}{a} \markbox{5883}{continuous} \markbox{5884}{function} \markbox{5885}{on} \markbox{m5886}{$P(m,n,k)$} \markbox{5887}{and} \markbox{m5888}{$\tilde{f}$} \markbox{5889}{be} \markbox{5890}{the} \markbox{5891}{lift} \markbox{5892}{defined} \markbox{5893}{in} \remref{lift} \markbox{5894}{such} \markbox{5895}{that} \markbox{m5896}{$\tilde{f}^*(c_1)= au,\, a\in \mathbb{Q}$}. \markbox{5897}{Then} \markbox{5898}{the} \markbox{5899}{induced} \markbox{5900}{map} \markbox{m5901}{$\tilde{f}_*$} \markbox{5902}{on} \markbox{m5903}{$H_{*}^{\times}$} \markbox{5904}{is} \markbox{5905}{of} \markbox{5906}{the} \markbox{5907}{following} form.
		\begin{enumerate}
			\item \markbox{m5908}{$\tilde f_* (x)\in \vartheta\otimes H_{2i-m}^{\mathbb CG},\, \forall x\in H_{2i}^{\mathbb CG},\, \forall i>0$}.
			\item \markbox{m5909}{$\tilde{f}_*(\vartheta\otimes 1) = \mu (\vartheta\otimes 1 )+y, \, y\in H_{m}^{\mathbb{C}G}, \quad\\ \tilde f_*(\vartheta \otimes x)\in H_{2i+m}^{\mathbb{C}G},\, \forall x \in H_{2i}^{\mathbb{C}G}, i>0\;$} \markbox{5910}{if} \markbox{m5911}{$\tilde{f}^*(u) = \mu u,\,  \mu \in \mathbb{Q}$} 
		\end{enumerate}
	\end{lemma}
	\begin{proof}
		\markbox{5912}{Using} \propref{main thm 2}, \markbox{5913}{we} \markbox{5914}{have} \markbox{m5915}{$\tilde{f}^*(c_i) = u P_i,$} \markbox{5916}{for} \markbox{5917}{some} \markbox{m5918}{$P_i\in H^*_{\mathbb{C}G}$} \markbox{5919}{and} \markbox{5920}{either} \markbox{m5921}{$\tilde{f}^*(u) = \mu u,\, \mu \in \mathbb{Q}$} \markbox{5922}{or} \markbox{m5923}{$\tilde{f}^*(u)\in H^*_{\mathbb{C}G}$}. \\
		\markbox{5924}{Let} \markbox{5925}{us} \markbox{5926}{consider} \markbox{5927}{the} \markbox{5928}{first} \markbox{5929}{case} \markbox{5930}{where} \markbox{m5931}{$\tilde{f}^*(u) = \mu u$}. \markbox{5932}{Using} \markbox{m5933}{$H^*_{\mathbb{C}G}\cong \hom(H_*^{\mathbb{C}G},\mathbb{Q})$}, \markbox{5934}{we} \markbox{5935}{have} \markbox{5936}{for} \markbox{m5937}{$i>0$} 
		\begin{equation}\label{fstarcom}
			\tilde{f}^*(\delta_{v_p}) = \sum a_{jp}\delta_{\vartheta \otimes v_j} , \, \forall v_{p}\in H_{2i}^{\mathbb{C}G}, \quad \tilde{f}^*(\delta_{\vartheta\otimes 1})  =\mu \delta_{\vartheta \otimes 1}, \quad \tilde{f}^*(\delta_{\vartheta\otimes v_{p}})  = 0, \, \forall v_{p}\in H_{2i}^{\mathbb{C}G}.
		\end{equation}
		\markbox{5938}{Using} \eqref{comm with phi}, \eqref{fstarcom} \markbox{5939}{and} \markbox{5940}{similar} \markbox{5941}{calculations} \markbox{5942}{given} \markbox{5943}{in} \markbox{5944}{the} \markbox{5945}{proof} \markbox{5946}{of} \lemref{image of homf}, \markbox{5947}{we} \markbox{5948}{have} \markbox{5949}{for} \markbox{m5950}{$ v_p\neq 1$}
		$$\delta_{v_p}\circ \tilde{f}_*(v_q) = 0,\, \forall v_q \in H_{2i}^{\mathbb{C}G},\quad \delta_{\vartheta\otimes v_p}\circ \tilde{f}_*(\vartheta\otimes v_q) =0,\, \forall v_q \in H_{2i}^{*}$$ \markbox{5951}{that} \markbox{5952}{concludes} \markbox{5953}{the} result.
		
		\markbox{5954}{When} \markbox{m5955}{$\tilde{f}^*(u)\in H^*_{\mathbb{C}G}$} \markbox{5956}{then} \markbox{5957}{also} \markbox{5958}{we} \markbox{5959}{have} \markbox{m5960}{$\tilde{f}^*(\delta_{v_p}) = \sum a_{jp}\delta_{\vartheta \otimes v_j} , \, \forall v_{p}\in H_{2i}^{\mathbb{C}G},\, \forall i>0$} \markbox{5961}{which} \markbox{5962}{implies} \markbox{5963}{that} \markbox{m5964}{$\delta_{v_p}\circ \tilde{f}_*(v_q) = 0,\, \forall v_q \in H_{2i}^{\mathbb{C}G},\,\forall i>0.$}   		
	\end{proof}
	\iffalse	Consequently, \markbox{5965}{for} \markbox{5966}{the} \markbox{5967}{lift} \markbox{m5968}{$\tilde f:\mathbb S^m\times \mathbb{C}G_{n,k}\to \mathbb S^m\times \mathbb{C}G_{n,k}$}, \markbox{5969}{the} \markbox{5970}{induced} \markbox{5971}{maps} 
	\markbox{m5972}{$\tilde f^*:H^i_\times\to H^i_\times$} \markbox{5973}{and} \markbox{m5974}{$\tilde f_*:H_i^\times\to H_i^\times$} \markbox{5975}{are} \markbox{5976}{adjoint} \markbox{5977}{with} \markbox{5978}{respect} \markbox{5979}{to} \markbox{5980}{this} \markbox{5981}{perfect} pairing, i.e.,   
	\[
	\langle \tilde f^* \varphi, x\rangle=\langle \varphi, \tilde f_* x\rangle \text{ for all }\varphi\in H^i_{\times},x\in H_i^\times.
	\]\fi
	
	%To determine $\tilde f_*(U)$ when (iii) holds, write 
	%$\tilde f_*(U) = aU + \sum b_i v_i \in H_m^{\mathbb{S}} \oplus H_m^{\mathbb{C}G}\cong H_m^\times$.  
	%Then 
	%$\langle u, \tilde f_* U \rangle = \langle u, aU \rangle + \langle u, \sum b_i v_i \rangle = a + 0 = a$.  
	%On the other hand, 
	%$\langle u, \tilde f_* U \rangle = \langle \tilde f^* u, U \rangle = \langle \mu u, U \rangle = \mu$, 
	%so $a = \mu$.  
	%Proceeding similarly replacing  $u=\delta_U$  by $\delta_{v_i}$ yields $b_i = 0$ for all $i$.  
	%Hence, $\tilde f_*(U) = \mu U$.
	
	\iffalse	\underline{When (i) and (iii) hold}, \markbox{5982}{for} \markbox{5983}{any} \markbox{5984}{basis} \markbox{5985}{element} \markbox{m5986}{$v_q \in H_{2i}^{\mathbb{C}G}$}, \markbox{5987}{the} \markbox{5988}{coefficient} \markbox{5989}{of} \markbox{m5990}{$v_p$} \markbox{5991}{in} \markbox{m5992}{$\tilde f_*(v_q)$}, , \markbox{5993}{is} 
	\begin{equation}\label{i,ii,1}
		\langle \delta_{v_p}, \tilde f_*v_q\rangle
		= \langle \tilde f^* \delta_{v_p}, v_q\rangle
		= \langle \lambda^i \delta_{v_p}, v_q\rangle
		= \lambda^i \delta_{pq},
	\end{equation}
	\markbox{5994}{and} \markbox{5995}{the} \markbox{5996}{coefficient} \markbox{5997}{of} \markbox{m5998}{$\vartheta\otimes v_p \in \vartheta\otimes H_{2i}^{\mathbb{C}G}$} \markbox{5999}{in} \markbox{m6000}{$\tilde f_*(v_q)$} \markbox{6001}{is} 
	\begin{equation}\label{1,iii,2}
		\langle \delta_{\vartheta\otimes v_p}, \tilde f_*v_q\rangle
		= \langle \tilde f^*(\delta_{\vartheta\otimes v_p}), v_q\rangle
		= \langle \mu\lambda^{i}\delta_{\vartheta\otimes v_p}, v_q\rangle
		= 0.
	\end{equation}
	Thus, \markbox{m6002}{$\tilde f_*(x) = \lambda^i x$}, \markbox{6003}{for} \markbox{6004}{all} \markbox{m6005}{$x \in H_{2i}^{\mathbb{C}G}$}. 
	Also, \markbox{6006}{the} \markbox{6007}{coefficient} \markbox{6008}{of} \markbox{m6009}{$v_p \in H_m^{\mathbb{C}G}$} \markbox{6010}{in} \markbox{m6011}{$\tilde f_*(\vartheta\otimes v_q)$} \markbox{6012}{is} 
	\markbox{6013}{t} \markbox{6014}{of} \markbox{m6015}{$\vartheta\otimes v_p \in \vartheta \otimes H_{2i}^{{\mathbb CG}}$} \markbox{6016}{in} \markbox{m6017}{$\tilde f_*(\vartheta\otimes v_q)$} \markbox{6018}{is} 
	\markbox{6019}{This} \markbox{6020}{shows} \markbox{6021}{that} \markbox{m6022}{$\tilde f_*(\vartheta \otimes x) = \mu \lambda^{i}\vartheta \otimes x$} \markbox{6023}{for} \markbox{6024}{all} \markbox{m6025}{$x\in \vartheta\otimes H_{2i}^{\mathbb CG}$}.\fi
	
	\iffalse	\underline{When (ii) and (iii) hold},  \markbox{6026}{the} \markbox{6027}{coefficient} \markbox{6028}{of} \markbox{m6029}{$v_p$} (\markbox{m6030}{$p\neq 0$}) \markbox{6031}{in} \markbox{m6032}{$\tilde f_*(v_q)$}, \markbox{6033}{if} \markbox{m6034}{$\tilde f^*\delta_{v_p}=\sum a_i\delta_{\vartheta \otimes v_i} $},  \markbox{6035}{is}
	\[
	\langle \delta_{v_p}, \tilde f_*v_q\rangle=\langle\tilde f^*\delta_{v_p},v_q\rangle=\langle \sum a_i\delta_{\vartheta \otimes v_i},v_q\rangle=\sum 
	a_i\langle  \delta_{\vartheta \otimes v_i},v_q\rangle=0,
	\]
	Thus, \markbox{m6036}{$\tilde f_* (x)\in \vartheta\otimes H_*^{\mathbb CG}$} \markbox{6037}{for} \markbox{6038}{all} \markbox{m6039}{$x\in H_*^{\mathbb CG}$}. Also, \markbox{6040}{the} \markbox{6041}{coefficient} \markbox{6042}{of} \markbox{m6043}{$\vartheta\otimes v_p$} \markbox{6044}{in} \markbox{m6045}{$\tilde f_*(\vartheta\otimes v_q)$} \markbox{6046}{is}
	\[
	\langle \delta_{\vartheta \otimes v_p},\tilde f_*(\vartheta\otimes v_q)\rangle=\langle\tilde f^*(\delta_{\vartheta \otimes v_p}), \vartheta \otimes v_q\rangle =
	\begin{cases}
		\langle \mu \delta_{\vartheta \otimes v_0}, \vartheta \otimes v_q\rangle=\mu\delta_{0q}& \text{if } p=0,\\
		\langle 0, \vartheta \otimes v_q\rangle =0& \text{if }p\neq 0.
	\end{cases}
	\]
	Therefore, \markbox{6047}{for} \markbox{6048}{all} \markbox{m6049}{$x\in H_{2i}^{\mathbb CG}$}, \markbox{m6050}{$\tilde f_*(\vartheta\otimes x)=\begin{cases}
		\mu \vartheta\otimes x +y \;\text{ for some }y\in H_{2i+m}^{\mathbb CG}  &\text{if } i=0 ,\\
		\quad \quad0 &\text{if }i>0.
	\end{cases}$} \fi
	
	\iffalse	\underline{When (i) and (iv) hold}, \markbox{6051}{the} \markbox{6052}{coefficient} \markbox{6053}{of} \markbox{m6054}{$v_p\in H_{2i}^{\mathbb CG}$} \markbox{6055}{in} \markbox{m6056}{$\tilde f_*(v_q)$} \markbox{6057}{is} 
	\[
	\langle \delta_{v_p},\tilde f_*(v_q)\rangle=\langle\tilde f^*\delta_{v_p},v_q\rangle=\langle \lambda^i\delta_{v_p},v_q\rangle=\lambda^i\delta_{pq}.
	\]
	\markbox{6058}{This} \markbox{6059}{implies} \markbox{6060}{that} \markbox{m6061}{$\tilde f_*(x)=\lambda^ix+\vartheta \otimes y$} \markbox{6062}{for} \markbox{6063}{some} \markbox{m6064}{$\vartheta \otimes y\in \vartheta \otimes H_{2i}^{\mathbb CG}$}, \markbox{6065}{for} \markbox{6066}{all} \markbox{m6067}{$x\in H_{2i}^{\mathbb CG}$}. \fi
	
	
	\iffalse	\underline{When (ii) and (iv) hold}, \markbox{6068}{we} \markbox{6069}{can} \markbox{6070}{write} \markbox{m6071}{$\tilde f^*(\delta_{v_p})=\sum a_j\delta_{\vartheta\otimes v_j}$} %and $\tilde f^*(\delta_{\vartheta\otimes v_p})=\sum b_j\delta_{\vartheta\otimes v_j}$
	for \markbox{6072}{some} \markbox{m6073}{$a_j\in \mathbb Q$} \markbox{6074}{and} \markbox{m6075}{$p\neq 0$}.
	\markbox{6076}{Now} \markbox{6077}{the} \markbox{6078}{coefficient} \markbox{6079}{of} \markbox{m6080}{$v_p$} \markbox{6081}{in} \markbox{m6082}{$\tilde f_*(v_q)$} \markbox{6083}{is}
	\[
	\langle \delta_{v_p},\tilde f_*v_q\rangle=\langle\tilde f^*\delta_{v_p}, v_q\rangle=\langle\sum a_j\delta_{\vartheta\otimes v_j}, v_q\rangle=0,
	\]
	\markbox{6084}{implying} \markbox{m6085}{$\tilde f_*(x)\in \vartheta\otimes H_*^{\mathbb CG}$} \markbox{6086}{for} \markbox{6087}{all} \markbox{m6088}{$x\in H_*^{\mathbb CG}$}. \fi
	\iffalse
	Also, \markbox{6089}{the} \markbox{6090}{coefficient} \markbox{6091}{of} \markbox{m6092}{$ v_p$} \markbox{6093}{in} \markbox{m6094}{$\tilde f_*(\vartheta\otimes v_q)$} \markbox{6095}{is}
	\[
	\langle \delta_{v_p},\tilde f_*(\vartheta\otimes v_q)\rangle=\langle\tilde f^*\delta_{v_p}, \vartheta\otimes v_q\rangle=\langle\sum b_j\delta_{v_j},\vartheta\otimes v_q\rangle=0.
	\]
	\markbox{6096}{This} \markbox{6097}{ensures} \markbox{6098}{that} \markbox{m6099}{$\tilde f_*(\vartheta\otimes x)\in H_*^{\mathbb CG}$} \markbox{6100}{for} \markbox{6101}{all} \markbox{m6102}{$\vartheta\otimes x\in \vartheta\otimes H_*^{\mathbb CG}$}.
	
	
	\markbox{6103}{We} \markbox{6104}{refer} \markbox{6105}{to} \markbox{6106}{the} \markbox{6107}{above} \markbox{6108}{observations} \markbox{6109}{from} \markbox{6110}{any} \markbox{6111}{combination} \markbox{6112}{of} \markbox{6113}{one} \markbox{6114}{condition} \markbox{6115}{from} 
	\markbox{m6116}{$(\mathrm{i}),(\mathrm{ii})$} \markbox{6117}{and} \markbox{6118}{one} \markbox{6119}{from} \markbox{m6120}{$(\mathrm{iii}),(\mathrm{iv})$}, \markbox{6121}{by} \markbox{6122}{the} \markbox{6123}{symbol} \markbox{m6124}{$\mathscr{X}$}.
	
	
	
	%Summarizing above, we have for all $x\in H_{2i}^{\mathbb CG}$,
	%\[
	%\tilde f_*(x)=
	%\begin{cases}
	%   \lambda^ix  &\text{ if (i) and (iii) hold},\\
	%    \lambda^ix +\vartheta\otimes y, \text{ for some }y\in H_*^{\mathbb CG}& \text{ if (i) and (iv) hold}.
	%^\end{cases}
%\]


\vspace{.2in}\fi

\iffalse
******

Thus, \markbox{6125}{for} \markbox{6126}{all} \markbox{m6127}{$x \in H^{\mathbb CG}_{2i}$}, \markbox{m6128}{$\varphi \in H^{2i}_{\mathbb CG}$}, \markbox{6129}{and} \markbox{m6130}{$y \in H^m_\times$}, \markbox{6131}{we} \markbox{6132}{have}
\begin{equation}\label{}
	\langle \varphi, \tilde f_* x\rangle =
	\begin{cases}
		\langle \varphi, \lambda_1^i x\rangle & \text{if (i) holds,} \\
		\langle u P_{\varphi}, x\rangle & \text{if (ii) holds},\\
	\end{cases}
	\quad \text{and} \quad
	\langle u, \tilde f_* y\rangle =
	\begin{cases}
		\langle u, \mu_1 y\rangle & \text{if (iii) holds,} \\
		& \text{if (iv) holds}.\\
	\end{cases}
\end{equation}

Therefore, \markbox{6133}{for} \markbox{6134}{all} \markbox{m6135}{$x\in H_{2i}^{\mathbb CG}$}, \markbox{m6136}{$\tilde f_*(x)=\lambda_1^ix$} \markbox{6137}{if} (i) holds, \markbox{m6138}{$\tilde f_*(x)\in [\mathbb S^m]\otimes H_{2i-m}^{\mathbb CG}$} \markbox{6139}{if} (ii) holds, \markbox{6140}{and} \markbox{m6141}{$\tilde f_*([\mathbb S^m])=\mu_1 [\mathbb S^m]$} \markbox{6142}{if} (iii) holds, \markbox{6143}{where} \markbox{m6144}{$[\mathbb S^m]\in H_m^{\mathbb S}$} \markbox{6145}{denotes} \markbox{6146}{the} \markbox{6147}{fundamental} \markbox{6148}{class} \markbox{6149}{of} \markbox{m6150}{$\mathbb S^m$} \markbox{6151}{satisfying} \markbox{m6152}{$\langle u,[\mathbb S^m]\rangle=1.$}
\fi


\subsection{} \markbox{6153}{The} \markbox{6154}{following} \markbox{6155}{theorems} \markbox{6156}{provide} \markbox{6157}{a} \markbox{6158}{criteria} \markbox{6159}{for} \markbox{6160}{the} \markbox{6161}{existence} \markbox{6162}{of} \markbox{6163}{coincidence} \markbox{6164}{points} \markbox{6165}{between} \markbox{6166}{a} \markbox{6167}{pair} \markbox{6168}{of} \markbox{6169}{continuous} \markbox{6170}{functions} \markbox{6171}{on} \markbox{m6172}{$P(m,n,k)$}.

\begin{theorem}\label{coincidence thm}
	\markbox{6173}{Let} \markbox{m6174}{$P(m,n,k)$} \markbox{6175}{be} \markbox{6176}{a} \markbox{6177}{generalized} \markbox{6178}{Dold} \markbox{6179}{manifold} \markbox{6180}{with} \markbox{m6181}{$k<n-k$} \markbox{6182}{and} \markbox{m6183}{$k(n-k)$} even. \markbox{6184}{Let} \markbox{m6185}{$f$} \markbox{6186}{and} \markbox{m6187}{$g$} \markbox{6188}{be} \markbox{6189}{two} \markbox{6190}{continuous} \markbox{6191}{maps} \markbox{6192}{on} \markbox{m6193}{$P(m,n,k)$} \markbox{6194}{and} \markbox{m6195}{$\tilde f, \tilde g$} \markbox{6196}{be} \markbox{6197}{their} \markbox{6198}{lifts} \markbox{6199}{as} \markbox{6200}{defined} \markbox{6201}{in}  \markbox{6202}{Remark} \ref{lift} \markbox{6203}{such} \markbox{6204}{that}
	%and suppose that the induced endomorphisms in cohomology satisfies:	
	\begin{enumerate}	
		\item \markbox{m6205}{$g^*$} \markbox{6206}{is} \markbox{6207}{an} \markbox{6208}{automorphism} \markbox{6209}{of} \markbox{m6210}{$H^*(P(m,n,k);\mathbb Q)$}. 
		\item \markbox{m6211}{$\tilde{f}^*(c_1) \neq au,\, a \in \mathbb{Q}$}.
		\item \markbox{m6212}{$\deg(p\circ g \circ s)\neq -\deg (p\circ f\circ s)$} \markbox{6213}{if} \markbox{m6214}{$m$} \markbox{6215}{is} odd.
	\end{enumerate}
	\markbox{6216}{where} \markbox{m6217}{$s$} \markbox{6218}{denotes} \markbox{6219}{a} \markbox{6220}{section} \markbox{6221}{of} \markbox{6222}{the} \markbox{m6223}{$X$}-bundle \markbox{6224}{projection} \markbox{m6225}{$p$} \markbox{6226}{defined} \markbox{6227}{in} \eqref{sectio} \markbox{6228}{and} \eqref{proj}.	Then, \markbox{6229}{there} \markbox{6230}{is} \markbox{6231}{a} \markbox{6232}{point} \markbox{6233}{of} \markbox{6234}{coincidence} \markbox{6235}{of} \markbox{m6236}{$f$} \markbox{6237}{and}  \markbox{m6238}{$g$}.
\end{theorem}
\begin{proof}
	\markbox{6239}{Using} \corref{automor}, \markbox{6240}{we} \markbox{6241}{have} \markbox{m6242}{$\tilde g^*$} \markbox{6243}{is} \markbox{6244}{an} \markbox{6245}{automorphism} \markbox{6246}{on} \markbox{m6247}{$H^*_{\times}$} \markbox{6248}{given} \markbox{6249}{by}
	\markbox{m6250}{$\tilde g^*(c_i) = \lambda_1^i c_i$}, \markbox{6251}{and} \markbox{m6252}{$\tilde g^*(u) = \mu_1 u$} \markbox{6253}{for} \markbox{6254}{some} \markbox{m6255}{$\lambda_1, \mu_1 \in \mathbb{Q}\backslash \{0\}$} \markbox{6256}{if} \markbox{m6257}{$k<n-k$}.
	
	\markbox{6258}{Using} \lemref{image of homf}, \markbox{6259}{there} \markbox{6260}{exist} \markbox{m6261}{$\lambda \in \mathbb{Q}\backslash \{0\}$} \markbox{6262}{and} \markbox{m6263}{$\mu \in \mathbb{Q}$} \markbox{6264}{such} \markbox{6265}{that} \markbox{m6266}{$\tilde{f}_*$} \markbox{6267}{is} \markbox{6268}{of} \markbox{6269}{the} \markbox{6270}{following} form, 
	\begin{equation}\label{flower star}
		\begin{split}
			\tilde f_*(x)=\lambda^ix+\vartheta\otimes y, \text{ for some }y\in H_{2i-m}^{\mathbb CG}, \, \forall x \in H_{2i}^{\mathbb{C}G}\\
			\tilde f_*(\vartheta\otimes x)=\mu\lambda^i (\vartheta\otimes x) , \text{ or } \tilde{f}_*(\vartheta \otimes x) = z, \text{ for some }z\in H_{2i+m}^{\mathbb CG},\, \forall x \in H_{2i}^{\mathbb{C}G}
		\end{split}
	\end{equation}
	\markbox{6271}{To} \markbox{6272}{prove} \markbox{6273}{that} \markbox{m6274}{$f$} \markbox{6275}{has} \markbox{6276}{a} \markbox{6277}{point} \markbox{6278}{of} \markbox{6279}{coincidence} \markbox{6280}{with} \markbox{m6281}{$g$}, \markbox{6282}{it} \markbox{6283}{is} \markbox{6284}{sufficient} \markbox{6285}{to} \markbox{6286}{prove} \markbox{6287}{that} \markbox{6288}{either} \markbox{m6289}{$\tilde{f}$} \markbox{6290}{or} \markbox{6291}{the} \markbox{6292}{composition} \markbox{m6293}{$\theta \circ \tilde{f} $} \markbox{6294}{has} \markbox{6295}{a} \markbox{6296}{point} \markbox{6297}{of} \markbox{6298}{coincidence} \markbox{6299}{with} \markbox{m6300}{$g$} \markbox{6301}{where} \markbox{m6302}{$\theta = \alpha \times \sigma$} \markbox{6303}{defined} \markbox{6304}{in} \secref{gds}. \markbox{6305}{By} \thmref{LCT}, \markbox{6306}{we} \markbox{6307}{need} \markbox{6308}{to} \markbox{6309}{compute} \markbox{m6310}{$L(\tilde f, \tilde g)$} \markbox{6311}{and} \markbox{m6312}{$L(\theta \circ \tilde f, \tilde g)$}.
	
	\markbox{6313}{For} \markbox{m6314}{$x\in H_{2i}^{\mathbb{C}G}$}, \markbox{6315}{we} \markbox{6316}{have} 
	\begin{equation*}\label{D cal}
		\begin{split}
			D\tilde g^* D^{-1} \tilde f_*(x)=\mu_1 \lambda^i\lambda_1^{d-i} x + \vartheta\otimes y'\text{ for some }y' \in H^{\mathbb CG}_{2i-m}\\
			D\tilde g^* D^{-1} \tilde f_*(\vartheta\otimes x)= \mu \lambda^i\lambda_1^{d-i}(\vartheta\otimes x)+z' \text{ for some }z'\in H_{2i+m}^{\mathbb CG}.
		\end{split}
	\end{equation*}
	\markbox{6317}{where} \markbox{m6318}{$z^{'} =0$} \markbox{6319}{or} \markbox{m6320}{$\mu =0$} \markbox{6321}{depending} \markbox{6322}{on} \markbox{6323}{the} \markbox{6324}{image} \markbox{6325}{of} \markbox{m6326}{$\tilde{f}_*(\vartheta \otimes x)$}.
	%Now we compute the Lefschetz number of the map $L(\tilde{f},\tilde g)$  and $L(\theta\circ \tilde f, \tilde g)$ under the consideration that $\lambda=0$ if (ii) holds and $\mu=0$ if (iv) holds. 
	Recall \markbox{6327}{that}  \markbox{m6328}{$d_{2i}$} \markbox{6329}{denote} \markbox{6330}{the} \markbox{6331}{dimension} \markbox{m6332}{$\dim H^{2i}_{\mathbb CG}$}. \markbox{6333}{The} \markbox{6334}{Lefschetz} \markbox{6335}{number}  \markbox{m6336}{$L(\tilde{f},\tilde g)$} \markbox{6337}{is}
	\begin{equation*}\label{L(f,g)}
		L(\tilde f,\tilde g) =(\mu_1 + \mu) \sum_{i=0}^{k(n-k)} d_{2i} \lambda^i\lambda_1^{d-i}.
	\end{equation*}
	\markbox{6338}{Using} \markbox{6339}{the} \lemref{sum neq 0} \markbox{6340}{and} \markbox{6341}{the} \markbox{6342}{fact} \markbox{6343}{that} \markbox{m6344}{$\lambda_1\neq 0$}, \markbox{6345}{the} \markbox{6346}{sum}
	\[
	\sum_{i=0}^{k(n-k)} d_{2i} \lambda^i\lambda_1^{d-i}=\lambda_1^d\sum_{i=0}^{k(n-k)} d_{2i} (\lambda/\lambda_1)^i\neq 0,
	\]
	\markbox{6347}{Since} \markbox{m6348}{$\tilde f\circ\theta=\theta\circ \tilde f$}, \markbox{6349}{it} \markbox{6350}{follows} \markbox{6351}{that} $$(\theta\circ \tilde  f )^*(c_i)= (-1)^i \tilde{f}^*(c_i), \forall i \in I, \quad (\theta \circ\tilde  f)^*(u)=\begin{cases}
		-\tilde  f^*(u), \text{ if } m \text{ is even,}\\
		\tilde  f^*(u), \text{ if } m \text{ is odd}.
	\end{cases}$$ 
%	If $m$ is odd, using $\deg(p\circ g \circ s)\neq -\deg (p\circ f\circ s)$ i.e. $\mu_1 \neq -\mu$, we have $L(\theta\circ\tilde f, \tilde g) = L(\tilde f, \tilde g)\neq 0$ and we have the result. \\
	If \markbox{m6352}{$m$} \markbox{6353}{is} even, \markbox{6354}{then}  
	\begin{equation*}\label{D with theta}
		\begin{split}
			D\tilde g^* D^{-1} (\theta\circ \tilde f)_*(x)=\mu_1(- \lambda)^i\lambda_1^{d-i} x + \vartheta\otimes y''\text{ for some }y'' \in H^{\mathbb CG}_{2i-m} \\
			D\tilde g^* D^{-1} (\theta\circ \tilde f)_*(\vartheta\otimes x)= -\mu (-\lambda)^i\lambda_1^{d-i}\vartheta\otimes x+z'' \text{ for some }z''\in H_{2i+m}^{\mathbb CG}.
		\end{split}
	\end{equation*}
	Thus,  \markbox{6355}{the} \markbox{6356}{Lefschetz} \markbox{6357}{number} \markbox{6358}{is}
	\begin{equation*}\label{L(theta f,g)}
		L(\theta \circ\tilde f,\tilde g) =(\mu_1 - \mu)\sum_{i=0}^{k(n-k)} d_{2i} (-\lambda)^i\lambda_1^{d-i}.
	\end{equation*}
	Also, \markbox{6359}{using}  \markbox{m6360}{$\mu_1\neq 0$} \markbox{6361}{and} \lemref{sum neq 0}  \markbox{6362}{it} \markbox{6363}{follows} \markbox{6364}{that} \markbox{6365}{that} \markbox{6366}{either} \markbox{m6367}{$L(\tilde f, \tilde g)$} \markbox{6368}{or} \markbox{m6369}{$L(\theta\circ \tilde f,\tilde g)$} \markbox{6370}{is} nonzero. 
	
	\markbox{6371}{If} \markbox{m6372}{$m$} \markbox{6373}{is} odd, \markbox{m6374}{$	L(\theta \circ\tilde f,\tilde g) =(\mu_1 + \mu)\sum_{i=0}^{k(n-k)} d_{2i} (-\lambda)^i\lambda_1^{d-i}.$} \markbox{6375}{Using}  \lemref{sum neq 0} \markbox{6376}{and} \markbox{m6377}{$\deg(p\circ g \circ s)\neq -\deg (p\circ f\circ s)$} \markbox{6378}{that} \markbox{6379}{is} \markbox{m6380}{$\mu_1 \neq -\mu$}, \markbox{6381}{we} \markbox{6382}{have} \markbox{6383}{both} \markbox{m6384}{$L(\tilde{f},\tilde{g})$} \markbox{6385}{and} \markbox{m6386}{$L(\theta \circ\tilde f,\tilde g)$} \markbox{6387}{are} nonzero. 
	\markbox{6388}{This} \markbox{6389}{ensures} \markbox{6390}{that} \markbox{6391}{there} \markbox{6392}{exist} \markbox{6393}{a} \markbox{6394}{point} \markbox{6395}{of} \markbox{6396}{conincidence} \markbox{6397}{between} \markbox{m6398}{$f $} \markbox{6399}{and} \markbox{m6400}{$g$}.
\end{proof}
%	Using  \remref{lemrem} in the case $k=n-k$, we need to replace $\lambda$ by $-\lambda$ in \eqref{flower star} if $\tilde{f}^*(c_i) = -\lambda^i c_i$. In the rest of the calculations, $\lambda$ will be replaced by $-\lambda$ and we get the same result.
\begin{theorem}\label{coincidence thm under hom}
	\markbox{6401}{Let} \markbox{m6402}{$P(m,n,k)$} \markbox{6403}{be} \markbox{6404}{a} \markbox{6405}{generalized} \markbox{6406}{Dold} \markbox{6407}{manifold} \markbox{6408}{with} \markbox{m6409}{$k(n-k)$} \markbox{6410}{even} \markbox{6411}{and} \markbox{6412}{assume} \markbox{6413}{that} \markbox{6414}{the} \markbox{6415}{hypothesis} \eqref{Homer} \markbox{6416}{is} satisfied. \markbox{6417}{Let} \markbox{m6418}{$g$} \markbox{6419}{and} \markbox{m6420}{$f$} \markbox{6421}{are} \markbox{6422}{two} \markbox{6423}{continuous} \markbox{6424}{maps} \markbox{6425}{on} \markbox{m6426}{$P(m,n,k)$} \markbox{6427}{and} \markbox{m6428}{$\tilde g, \tilde f$} \markbox{6429}{be} \markbox{6430}{their} \markbox{6431}{lifts} \markbox{6432}{as} \markbox{6433}{defined} \markbox{6434}{in}  \markbox{6435}{Remark} \ref{lift} \markbox{6436}{such} \markbox{6437}{that}
	%and suppose that the induced endomorphisms in cohomology satisfies:	
	\begin{enumerate}	
		\item \markbox{m6438}{$g^*$} \markbox{6439}{is} \markbox{6440}{an} \markbox{6441}{automorphism} \markbox{6442}{of} \markbox{m6443}{$H^*(P(m,n,k);\mathbb Q)$}. 
		\item \markbox{m6444}{$\tilde{f}^*(u) = \mu u,\, \mu \in \mathbb{Q}$} \markbox{6445}{if} \markbox{m6446}{$\tilde{f}^*(H^*_{\mathbb CG}) \nsubseteq H^*_{\mathbb CG}$} \markbox{6447}{and} \markbox{m6448}{$m$} \markbox{6449}{is} even.
		\item \markbox{m6450}{$\deg(p\circ g \circ s)\neq -\deg (p\circ f\circ s)$} \markbox{6451}{if} \markbox{m6452}{$m$} \markbox{6453}{is} odd.
	\end{enumerate}
	\markbox{m6454}{$s$} \markbox{6455}{denotes} \markbox{6456}{a} \markbox{6457}{section} \markbox{6458}{of} \markbox{6459}{the} \markbox{m6460}{$X$}-bundle \markbox{6461}{projection} \markbox{m6462}{$p$} \markbox{6463}{defined} \markbox{6464}{in} \eqref{sectio} \markbox{6465}{and} \eqref{proj}. Then, \markbox{6466}{there} \markbox{6467}{is} \markbox{6468}{a} \markbox{6469}{point} \markbox{6470}{of} \markbox{6471}{coincidence} \markbox{6472}{of} \markbox{m6473}{$f$} \markbox{6474}{and}  \markbox{m6475}{$g$}.
\end{theorem}
\begin{proof}
	\markbox{6476}{If} \markbox{m6477}{$\tilde{f}^*(c_1) \neq au, \, a\in \mathbb{Q}$} \markbox{6478}{then} \markbox{6479}{we} \markbox{6480}{have} \markbox{6481}{the} \markbox{6482}{result} \markbox{6483}{by} \thmref{coincidence thm}.\\ \markbox{6484}{Let} \markbox{6485}{us} \markbox{6486}{consider} \markbox{6487}{the} \markbox{6488}{other} \markbox{6489}{case} \markbox{6490}{when} \markbox{m6491}{$\tilde{f}^*(c_1) = au, \, a\in \mathbb{Q}$}, \markbox{6492}{using} \thmref{main thm 2} \markbox{6493}{we} \markbox{6494}{have} \markbox{m6495}{$\tilde{f}^*(c_i) = uP_i, \text{ for some } P_i\in H^{2i-m}_{\mathbb{C}G}.$} 
	
	\markbox{6496}{If} \markbox{m6497}{$P_i \neq 0$} \markbox{6498}{for} \markbox{6499}{some} \markbox{m6500}{$i$} \markbox{6501}{in} \markbox{m6502}{$I$} \markbox{6503}{then} \markbox{m6504}{$\tilde{f}^*(H^*_{\mathbb CG}) \nsubseteq H^*_{\mathbb CG}$}. \markbox{6505}{Since} \markbox{m6506}{$\tilde{f}^*$} \markbox{6507}{is} \markbox{6508}{graded} \markbox{6509}{and} \markbox{6510}{by} \textit{(2)} \markbox{6511}{we} \markbox{6512}{have} \markbox{m6513}{$\tilde{f}^*(u) = \mu u,\, \mu \in \mathbb{Q}$}. \markbox{6514}{Using} \lemref{image of homf under hom}, \markbox{m6515}{$\tilde{f}_*$} \markbox{6516}{is} \markbox{6517}{of} \markbox{6518}{the} \markbox{6519}{following} form, \begin{equation}\label{uin CG}
		\begin{split}
			\tilde f_*(x)= \vartheta\otimes y, \text{ for some }y\in H_{2i-m}^{\mathbb CG}, \, \forall x \in H_{2i}^{\mathbb{C}G}, \, i>0\\
			\tilde f_*(\vartheta\otimes x)=\mu(\vartheta\otimes x) +z, \text{ for some }z\in H_{2i+m}^{\mathbb CG},\, \forall x \in H_{2i}^{\mathbb{C}G}
		\end{split}
	\end{equation}
	\markbox{6520}{where} \markbox{m6521}{$\mu =0$} \markbox{6522}{if} \markbox{m6523}{$i>0$}. \markbox{6524}{By} \corref{automor}, \markbox{6525}{we} \markbox{6526}{have} \markbox{m6527}{$\tilde g^*$} \markbox{6528}{is} \markbox{6529}{an} \markbox{6530}{automorphism} \markbox{6531}{on} \markbox{m6532}{$H^*_{\times}$} \markbox{6533}{given} \markbox{6534}{by}
	\markbox{m6535}{$\tilde g^*(c_i) = \lambda_1^i c_i$}, \markbox{6536}{and} \markbox{m6537}{$\tilde g^*(u) = \mu_1 u$} \markbox{6538}{for} \markbox{6539}{some} \markbox{m6540}{$\lambda_1, \mu_1 \in \mathbb{Q}\backslash \{0\}$}. \markbox{6541}{Using} \thmref{LCT} \markbox{6542}{and} \markbox{6543}{the} \markbox{6544}{similar} \markbox{6545}{calculations} \markbox{6546}{as} \markbox{6547}{done} \markbox{6548}{in} \markbox{6549}{the} \markbox{6550}{proof} \markbox{6551}{of} \thmref{coincidence thm}, \markbox{6552}{we} \markbox{6553}{get} $$L(\tilde f,\tilde g) =(\mu_1 + \mu) d_0 \lambda_1^d, \quad L(\theta \circ\tilde f,\tilde g) =\begin{cases}
		(\mu_1 - \mu)d_{0} \lambda_1^{d},  \text{ if } m \text{ is even,}\\
		(\mu_1 +\mu)d_{0} \lambda_1^{d},  \text{ if } m \text{ is odd.}
	\end{cases}$$ \markbox{6554}{Using} \markbox{m6555}{$\lambda_1 \neq 0$} \markbox{6556}{and} \markbox{m6557}{$\mu_1 \neq 0$}, \markbox{6558}{either} \markbox{m6559}{$L(\tilde f,\tilde g) $} \markbox{6560}{or} \markbox{m6561}{$ L(\theta \circ\tilde f,\tilde g)$} \markbox{6562}{is} \markbox{6563}{non} \markbox{6564}{zero} \markbox{6565}{if} \markbox{m6566}{$m$} \markbox{6567}{is} even. \markbox{6568}{Using} \markbox{m6569}{$\deg(p\circ g \circ s)\neq -\deg (p\circ f\circ s)$} i.e. \markbox{m6570}{$\mu_1 \neq -\mu$} \markbox{6571}{we} \markbox{6572}{have} \markbox{m6573}{$L(\tilde f, \tilde g) = L(\theta \circ \tilde f, \tilde g) \neq 0$}.  Hence, \markbox{6574}{we} \markbox{6575}{get} \markbox{6576}{the} result.
	
	\markbox{6577}{Let} \markbox{6578}{us} \markbox{6579}{consider} \markbox{6580}{the} \markbox{6581}{case} \markbox{6582}{when} \markbox{m6583}{$P_i =0,\, \forall i \in I$}, \markbox{6584}{if} \markbox{m6585}{$\tilde{f}^*(u) = \mu u, \mu \in \mathbb{Q}$} \markbox{6586}{then} \markbox{6587}{the} \markbox{6588}{proof} \markbox{6589}{remains} \markbox{6590}{exactly} \markbox{6591}{the} \markbox{6592}{same} \markbox{6593}{as} \markbox{6594}{given} above. \markbox{6595}{We} \markbox{6596}{need} \markbox{6597}{to} \markbox{6598}{focus} \markbox{6599}{on} \markbox{6600}{the} \markbox{6601}{case} \markbox{6602}{when} \markbox{m6603}{$\tilde{f}^*(u)\in H^*_{\mathbb{C}G}$}. \markbox{6604}{Using} \lemref{image of homf under hom} \markbox{6605}{and} \eqref{computation}, \markbox{6606}{we} \markbox{6607}{have} $$\tilde{f}_*(x) = \vartheta \otimes y, \text{ for some }y\in H_{2i-m}^{\mathbb CG}, \, \forall x \in H_{2i}^{\mathbb{C}G}, \, i>0, \quad \tilde{f}_*(\vartheta \otimes x) \in H_*^{\mathbb{C}G}, \forall x\in  H_*^{\mathbb{C}G}.$$ \markbox{6608}{This} \markbox{6609}{is} \markbox{6610}{exactly} \markbox{6611}{the} \markbox{6612}{same} \markbox{6613}{if} \markbox{6614}{we} \markbox{6615}{take} \markbox{m6616}{$\mu =0$} \markbox{6617}{in} \eqref{uin CG}. \markbox{6618}{The} \markbox{6619}{rest} \markbox{6620}{of} \markbox{6621}{the} \markbox{6622}{calculations} \markbox{6623}{also} \markbox{6624}{remains} \markbox{6625}{the} \markbox{6626}{same} \markbox{6627}{and} \markbox{6628}{we} \markbox{6629}{get} \markbox{6630}{the} result.
\end{proof}



\begin{remark}
	\markbox{6631}{There} \markbox{6632}{are} \markbox{6633}{many} \markbox{6634}{situations} \markbox{6635}{when} \markbox{6636}{the} \markbox{6637}{map} \markbox{m6638}{$f$} \markbox{6639}{satisfies} \markbox{6640}{the} \markbox{6641}{required} \markbox{6642}{hypothesis} \textit{(2)} \markbox{6643}{considered} \markbox{6644}{in} \thmref{coincidence thm} \markbox{6645}{or} \thmref{coincidence thm under hom}. \markbox{6646}{Some} \markbox{6647}{of} \markbox{6648}{them} \markbox{6649}{are} \markbox{6650}{as} follows: 
	\begin{enumerate}
		\item The \markbox{6651}{lift} \markbox{m6652}{$\tilde f$} \markbox{6653}{stabilizes} \markbox{6654}{a} \markbox{6655}{copy} \markbox{6656}{of} Grassmannian, i.e., \markbox{m6657}{$\tilde f(\{x_0\}\times\mathbb CG_{n,k})\subseteq \{x_0\}\times\mathbb CG_{n,k}$} \markbox{6658}{for} \markbox{6659}{some} \markbox{m6660}{$x_0\in \mathbb S^m$}. 
		\item  The \markbox{6661}{map} \markbox{m6662}{$p_1\circ \tilde f^*\circ i_1: H^*_{\mathbb CG}\to H^*_{\mathbb C G}$} \markbox{6663}{is} \markbox{6664}{an} automorphism, equivalently, \markbox{m6665}{$f^*(c_1^2)=\lambda^2c_1^2,\, \lambda\in \mathbb Q\backslash \{0\},$} \markbox{6666}{where} \markbox{m6667}{$p_1$} \markbox{6668}{and} \markbox{m6669}{$i_1$} \markbox{6670}{are} \markbox{6671}{defined} \markbox{6672}{in} \eqref{comm diagram}.
		\item The \markbox{6673}{map} \markbox{m6674}{$p_2\circ \tilde f\circ i_1:\mathbb S^m\to \mathbb CG_{n,k}$} \markbox{6675}{is} \markbox{6676}{rationally} \markbox{6677}{null} homotopic, \markbox{6678}{where} \markbox{m6679}{$p_2$} \markbox{6680}{is} \markbox{6681}{the} \markbox{6682}{projection} \markbox{6683}{onto} \markbox{6684}{the} \markbox{6685}{second} \markbox{6686}{summand} \markbox{6687}{and} \markbox{m6688}{$i_1$} \markbox{6689}{is} \markbox{6690}{the} \markbox{6691}{inclusion} \markbox{6692}{into} \markbox{6693}{the} \markbox{6694}{first} summand. %$i_1:\mathbb S^m\hookrightarrow\mathbb S^m\times \mathbb CG_{n,k}$ be the inclusion into the first component and $p_2:\mathbb S^m\times \mathbb CG_{n,k}\to \mathbb CG_{n,k}$ be the projection onto the second component.
	\end{enumerate} 
\end{remark}



	\markbox{6695}{Under} \markbox{6696}{the} \markbox{6697}{assumption} \markbox{m6698}{$m>2k$}, \markbox{6699}{any} \markbox{6700}{continuous} \markbox{6701}{map} \markbox{m6702}{$f$} \markbox{6703}{on} \markbox{6704}{the} \markbox{6705}{generalized} \markbox{6706}{Dold} \markbox{6707}{space} \markbox{m6708}{$P(m,n,k)$}, \markbox{6709}{the} \markbox{6710}{lift} \markbox{m6711}{$\tilde f$} (from Remark~\ref{lift}) \markbox{6712}{satisfies} \markbox{m6713}{$\tilde f^{*}(c_i)=\lambda^ic_i$} \markbox{6714}{for} \markbox{6715}{all} \markbox{m6716}{$i\in I$}. \markbox{6717}{Hence} condition~\textit{(2)} \markbox{6718}{of} Theorem~\ref{coincidence thm under hom} \markbox{6719}{may} \markbox{6720}{be} omitted, \markbox{6721}{and} \markbox{6722}{one} \markbox{6723}{obtains} \markbox{6724}{the} \markbox{6725}{following} consequence.
	\begin{corollary}
		\markbox{6726}{Let} \markbox{m6727}{$P(m,n,k)$} \markbox{6728}{be} \markbox{6729}{a} \markbox{6730}{generalized} \markbox{6731}{Dold} \markbox{6732}{space} \markbox{6733}{with} \markbox{m6734}{$m$} \markbox{6735}{and} \markbox{m6736}{$k(n-k)$} \markbox{6737}{both} even. \markbox{6738}{Assume} \markbox{m6739}{$m>2k$}, \markbox{6740}{and} \markbox{6741}{the} \markbox{6742}{hypothesis} \eqref{Homer} \markbox{6743}{is} satisfied.
		Then, \markbox{6744}{for} \markbox{6745}{any} \markbox{6746}{continuous} \markbox{6747}{function} \markbox{m6748}{$g$} \markbox{6749}{on} \markbox{m6750}{$P(m,n,k)$} \markbox{6751}{that} \markbox{6752}{induces} \markbox{6753}{an} \markbox{6754}{automorphism} \markbox{6755}{on} \markbox{m6756}{$H^*(P(m,n,k);\mathbb{Q})$}, \markbox{6757}{the} \markbox{6758}{pair} \markbox{m6759}{$(P(m,n,k),g)$} \markbox{6760}{has} \markbox{6761}{the} \markbox{6762}{coincidence} property. \\ \markbox{6763}{In} particular, \markbox{6764}{for} \markbox{m6765}{$g=\mathrm{id}$}, \markbox{6766}{the} \markbox{6767}{space} \markbox{m6768}{$P(m,n,k)$} \markbox{6769}{has} \markbox{6770}{the} fixed-point property.
	\end{corollary}










\markbox{6771}{In} \thmref{coincidence thm under hom}, \markbox{6772}{the} \markbox{6773}{first} \markbox{6774}{assumption} \markbox{6775}{that} \markbox{m6776}{$g^*$} \markbox{6777}{is} \markbox{6778}{an} \markbox{6779}{automorphism} \markbox{6780}{of} \markbox{m6781}{$H^*(P(m,n,k); \mathbb{Q})$} \markbox{6782}{can} \markbox{6783}{be} \markbox{6784}{relaxed} \markbox{6785}{by} \markbox{6786}{assuming} \markbox{m6787}{$\mu$} \markbox{6788}{is} nonzero, \markbox{6789}{which} \markbox{6790}{leads} \markbox{6791}{to} \markbox{6792}{the} \markbox{6793}{following} proposition. 
\begin{proposition}
	\markbox{6794}{Let} \markbox{m6795}{$P(m,n,k)$} \markbox{6796}{be} \markbox{6797}{a} \markbox{6798}{generalized} \markbox{6799}{Dold} \markbox{6800}{manifold} \markbox{6801}{with} \markbox{m6802}{$k(n-k)$} \markbox{6803}{even} \markbox{6804}{and} \markbox{6805}{assume} \markbox{6806}{that} \markbox{6807}{the} \markbox{6808}{hypothesis} \eqref{Homer} \markbox{6809}{is} satisfied. \markbox{6810}{Let} \markbox{m6811}{$g$} \markbox{6812}{and} \markbox{m6813}{$f$} \markbox{6814}{are} \markbox{6815}{two} \markbox{6816}{continuous} \markbox{6817}{maps} \markbox{6818}{on} \markbox{m6819}{$P(m,n,k)$} \markbox{6820}{and} \markbox{m6821}{$\tilde g, \tilde f$} \markbox{6822}{be} \markbox{6823}{their} \markbox{6824}{lifts} \markbox{6825}{as} \markbox{6826}{defined} \markbox{6827}{in}  \markbox{6828}{Remark} \ref{lift} \markbox{6829}{such} \markbox{6830}{that}
	\begin{enumerate}
		\item \markbox{m6831}{$\tilde g^*(H^*_{\mathbb CG})= H^*_{\mathbb CG}$}.
		\item \markbox{m6832}{$\tilde f^*(u)=\mu u,\, \mu\in \mathbb Q\backslash \{0\}$}
	\end{enumerate}
Then, \markbox{6833}{there} \markbox{6834}{is} \markbox{6835}{a} \markbox{6836}{point} \markbox{6837}{of} \markbox{6838}{coincidence} \markbox{6839}{of} \markbox{m6840}{$f$} \markbox{6841}{and}  \markbox{m6842}{$g$}.
\end{proposition}
\markbox{6843}{The} \markbox{6844}{proof} \markbox{6845}{of} \markbox{6846}{the} \markbox{6847}{above} \markbox{6848}{proposition} \markbox{6849}{is} \markbox{6850}{similar} \markbox{6851}{to} \markbox{6852}{the} \markbox{6853}{proof} \markbox{6854}{of} \thmref{coincidence thm under hom}. Therefore, \markbox{6855}{we} \markbox{6856}{omit} \markbox{6857}{the} details.

\iffalse \begin{proposition}
	\markbox{6858}{Let} \markbox{m6859}{$P(m,n,k)$} \markbox{6860}{be} \markbox{6861}{a} \markbox{6862}{generalized} \markbox{6863}{Dold} manifold. \markbox{6864}{Consider} \markbox{6865}{two} \markbox{6866}{continuous} \markbox{6867}{maps} \markbox{m6868}{$f,g$} \markbox{6869}{on} \markbox{m6870}{$P(m,n,k)$} \markbox{6871}{and} \markbox{m6872}{$\tilde f, \tilde g$} \markbox{6873}{be} \markbox{6874}{their} \markbox{6875}{lifts} \markbox{6876}{as} \markbox{6877}{defined} \markbox{6878}{in}  \markbox{6879}{Remark} \ref{lift}.
	\markbox{6880}{Suppose} \markbox{6881}{that} \markbox{6882}{the} \markbox{6883}{induced} \markbox{6884}{endomorphisms} \markbox{6885}{in} \markbox{6886}{cohomology} satisfies:
	\begin{enumerate}
		\item \markbox{m6887}{$\tilde g^*(c_i)\in uH^*_{\mathbb CG},\forall i\in I$} \markbox{6888}{and} \markbox{m6889}{$\tilde g^*(u)=\mu_1u$} \markbox{6890}{for} \markbox{6891}{some} \markbox{m6892}{$\mu_1\in \mathbb Q$},
		\item \markbox{m6893}{$\tilde f$} \markbox{6894}{has} \markbox{6895}{nonzero} Brouwer-degree.
	\end{enumerate}
	Then, \markbox{6896}{there} \markbox{6897}{is} \markbox{6898}{a} \markbox{6899}{point} \markbox{6900}{of} \markbox{6901}{coincidence} \markbox{6902}{of} \markbox{m6903}{$f$} \markbox{6904}{and}  \markbox{m6905}{$g$}.
\end{proposition}
\fi




\iffalse If \markbox{6906}{we} \markbox{6907}{assume} \markbox{m6908}{$m> 2k$}, \markbox{6909}{then} \markbox{6910}{we} \markbox{6911}{have} \markbox{6912}{more} \markbox{6913}{information} \markbox{6914}{about} \markbox{m6915}{$\tilde f^*$} \markbox{6916}{and} \markbox{6917}{in} \markbox{6918}{that} \markbox{6919}{case} \markbox{6920}{we} don't \markbox{6921}{have} \markbox{6922}{to} \markbox{6923}{assume} \markbox{6924}{hypothesis} \textit{(2)}.

\markbox{6925}{We} \markbox{6926}{need} \markbox{6927}{to} \markbox{6928}{talk} \markbox{6929}{about} \markbox{6930}{the} \markbox{6931}{similar} \markbox{6932}{situations} \markbox{6933}{when} \markbox{6934}{we} \markbox{6935}{assume} \markbox{6936}{Hoffman} \markbox{6937}{instead} \markbox{6938}{of} Glover-Homer.

\markbox{6939}{We} \markbox{6940}{should} \markbox{6941}{be} \markbox{6942}{able} \markbox{6943}{to} \markbox{6944}{find} \markbox{6945}{some} \markbox{6946}{GDS} \markbox{m6947}{$P(m,n,k)$} \markbox{6948}{which} \markbox{6949}{has} FPP.

\markbox{6950}{What} \markbox{6951}{happens} \markbox{6952}{if} \markbox{6953}{we} \markbox{6954}{allow} \markbox{m6955}{$m$} \markbox{6956}{to} \markbox{6957}{be} \markbox{6958}{odd} \markbox{6959}{in} \markbox{m6960}{$P(m,n,k)$}?
\fi

\section*{Acknowledgements}
\markbox{6961}{Part} \markbox{6962}{of} \markbox{6963}{this} \markbox{6964}{work} \markbox{6965}{was} \markbox{6966}{carried} \markbox{6967}{out} \markbox{6968}{while} \markbox{6969}{the} \markbox{6970}{first} \markbox{6971}{author} \markbox{6972}{was} \markbox{6973}{a} \markbox{6974}{postdoctoral} \markbox{6975}{fellow} \markbox{6976}{at} \markbox{6977}{IISER} Berhampur, \markbox{6978}{which} \markbox{6979}{the} \markbox{6980}{author} \markbox{6981}{gratefully} acknowledges.







%\section*{References}

\begin{thebibliography}{99}
	
	%\bibitem[A]{adams} Adams, J. F. Vector fields on spheres.
	%Ann. of Math. {\bf 75} (1962), 603--632.
	%\bibitem[ABS]{ABS} Atiyah, M. F.; Bott, R., Shapiro, A. Clifford modules. Topology {\bf 3} (1964), no. suppl, 3--38.
	%\bibitem[AH]{atiyah-hirzebruch} Atiyah, M. F.; Hirzebruch, F.
	%Vector bundles and homogeneous spaces.Proc. Sympos. Pure Math., Vol. III, 7--38
	%American Mathematical Society, Providence, RI, 1961.
	
	%\bibitem[B]{borel} Borel, A. Sur la cohomologie des espaces fibr{\'e}s principaux et des espaces homogènes de groupes de Lie compacts. Ann. of Math, Vol. {\bf 57} (1953) 115--207.
	
	\bibitem[B]{brewster} S. Brewster, \textit{Automorphisms of the cohomology ring of finite Grassmann manifolds}, \markbox{6982}{Thesis} (Ph.D.)---The \markbox{6983}{Ohio} \markbox{6984}{State} University, \markbox{6985}{ProQuest} LLC, \markbox{6986}{Ann} Arbor, \markbox{6987}{MI} (1978), 102 pp. \url{http://gateway.proquest.com/openurl?url_ver=Z39.88-2004&rft_val_fmt=info:ofi/fmt:kev:mtx:dissertation&res_dat=xri:pqdiss&rft_dat=xri:pqdiss:7908118}
	
	\bibitem[BH]{brewster-homer} S. \markbox{6988}{Brewster} \markbox{6989}{and} W. Homer, \textit{Rational automorphisms of Grassmann manifolds}, Proc. Amer. Math. Soc. \textbf{88} (1983), no. 1, 181--183, \url{https://doi.org/10.2307/2045137}.
	
	%\bibitem[B2]{borel} Borel, A. Cohomologie mod $2$
	% de certains espaces homogènes.
	%Comment. Math. Helv. {\bf 27} (1953)165--197 (1953).
	%\bibitem[B3]{borel-lag} Borel, A. Linear algebraic groups. Springer-Verlag, New York.
	%\bibitem[CF]{conner-floyd} Conner, P. E.; Floyd, E. E. {\it Differentiable periodic maps.} Ergebnisse der Mathematik...
	%\bibitem[D]{davis} Davis, D. Projective product spaces. J. Topol. {\bf 3} (2010), 265--279.
	%\bibitem[DJ]{dj} Davis, M.; Januszkiewicz, T. Convex polytopes...
	
	\bibitem[Do]{dold} A. Dold, \textit{Erzeugende der Thomschen Algebra }\markbox{m6990}{$\mathfrak{N}$}, Math. Z. \textbf{65} (1956), 25--35, \url{https://doi.org/10.1007/BF01473868}.

    \bibitem[Du]{duan} H. Duan, \textit{Self-maps of the Grassmannian of complex structures}, \markbox{6991}{Compositio} Math. \textbf{132} (2002), no. 2, 159--175, \url{https://doi.org/10.1023/A:1015885227445}.
	
	\bibitem[DF]{duan-fang} H. \markbox{6992}{Duan} \markbox{6993}{and} L. Fang, \textit{Homology rigidity of Grassmannians}, \markbox{6994}{Acta} Math. Sci. Ser. \markbox{6995}{B} \textbf{29} (2009), no. 3, 697--704, \url{https://doi.org/10.1016/S0252-9602(09)60065-5}.

    \bibitem[DZ]{duan-zhao} H. \markbox{6996}{Duan} \markbox{6997}{and} X. Zhao, \textit{The classification of cohomology endomorphisms of certain flag manifolds}, \markbox{6998}{Pacific} J. Math. \textbf{192} (2000), no. 1, 93--102, \url{https://doi.org/10.2140/pjm.2000.192.93}.
	%\bibitem[F1]{fujii-66} Fujii, M. $K_U$-groups of Dold manifolds...
	%\bibitem[F2]{fujii-69} Fujii, M. Ring structures...
	
	%\bibitem[F]{fulton} Fulton, W. *Introduction to Toric Varieties*...
	%\bibitem[K]{karoubi} Karoubi, M. *K-theory*...
	%\bibitem[HM]{hm} Hattori, A.; Masuda, M...
	%\bibitem[H]{husemoller} Husemoller, D. *Fibre bundles*...
	
	\bibitem[GH1]{glover-homer} H. \markbox{6999}{Glover} \markbox{7000}{and} B. Homer, \textit{Endomorphisms of the cohomology ring of finite Grassmann manifolds}, \markbox{7001}{Geometric} \markbox{7002}{applications} \markbox{7003}{of} \markbox{7004}{homotopy} \markbox{7005}{theory} (Proc. Conf., Evanston, Ill., 1977), I, pp. 170--193.
	

    \bibitem[GH2]{glover-homer coin} H. \markbox{7006}{Glover} \markbox{7007}{and} W. Homer, \textit{Fixed points on flag manifolds}, \markbox{7008}{Pacific} J. Math. \textbf{101} (1982), no. 2, 303--306, \url{http://projecteuclid.org/euclid.pjm/1102724776}.
	
	\bibitem[GS]{goswami-sarkar} A. \markbox{7009}{Goswami} \markbox{7010}{and} S. Sarkar, \textit{Endomorphisms of the Cohomology Algebra of Certain Homogeneous Spaces}, \url{https://arxiv.org/abs/2509.09363}

	\bibitem[Ho1]{hoffman} M. Hoffman, \textit{Endomorphisms of the cohomology of complex Grassmannians}, Trans. Amer. Math. Soc. \textbf{281} (1984), no. 2, 745--760, \url{https://doi.org/10.2307/2000083}.
    
    \bibitem[Ho2]{hoffman-noncoin} M. Hoffman, \markbox{7011}{Noncoincidence} \markbox{7012}{index} \markbox{7013}{of} manifolds, \markbox{7014}{Pacific} J. Math. \textbf{115}(2) (1984), 373--383, \url{http://projecteuclid.org/euclid.pjm/1102708254}.

    \bibitem[HH]{hoffman-homer} M. \markbox{7015}{Hoffman} \markbox{7016}{and} W. Homer, \textit{On cohomology automorphisms of complex flag manifolds}, Proc. Amer. Math. Soc. \textbf{91} (1984), no. 4, 643--648, \url{https://doi.org/10.2307/2044817}.
	
	

 %   \bibitem[Kh]{khare} Khare, S.S. On Dold manifolds, Topology and its Applications, Volume 33, Issue 3, 1989, Pages 297-307.

  %  \bibitem[Ko]{korbas} Korbaš, J. On parallelizability and span of the Dold manifolds,  Proc. Amer. Math. Soc. \textbf{141} (2013), no.~8, 2933--2939.

    
	\bibitem[L]{lin} X. Z. Lin, \textit{Geometric realization of Adams maps}, \markbox{7017}{Acta} Math. Sin. \textbf{27} (2011), no. 5, 863--870, \url{https://doi.org/10.1007/s10114-011-0164-y}.

    \bibitem[M]{mandal} M. Mandal, \markbox{7018}{Cohomology} \markbox{7019}{of} \markbox{7020}{generalized} \markbox{7021}{Dold} manifolds, \markbox{7022}{Thesis} (Ph.D.), \markbox{7023}{Homi} \markbox{7024}{Bhabha} \markbox{7025}{National} Institute, \markbox{7026}{The} \markbox{7027}{Institute} \markbox{7028}{of} \markbox{7029}{Mathematical} \markbox{7030}{Sciences} (2024).

%    \bibitem[Mu]{mukerjee} Mukerjee, H.~K. Classification of homotopy Dold manifolds,  New York J. Math. \textbf{9} (2003), 271--293.
	
	\bibitem[MS1]{mandal-sankaran} M. \markbox{7031}{Mandal} \markbox{7032}{and} P. Sankaran, \textit{Cohomology of generalized Dold spaces}, \markbox{7033}{Topology} Appl. \textbf{310} (2022), \markbox{7034}{Paper} No. 108040, 16 pp, \url{https://doi.org/10.1016/j.topol.2022.108040}.
    
	\bibitem[MS2]{mandal-sankaran2} M. \markbox{7035}{Mandal} \markbox{7036}{and} P. Sankaran, \textit{Cohomology and K-theory of generalized Dold manifolds fibred by complex flag manifolds}, \url{https://arxiv.org/abs/2407.03932}.


	
	%\bibitem[MiSt]{milnor-stasheff} Milnor, J. W.; Stasheff, J. D. {\it Characteristic classes.} Annals of Mathematics Studies, {\bf 76}, Princeton University Press, Princeton, NJ., 1974.
	
	\bibitem[NS]{nath-sankaran} A. \markbox{7037}{Nath} \markbox{7038}{and} P. Sankaran, \textit{On generalized Dold manifolds}, \markbox{7039}{Osaka} J. Math. \textbf{56} (2019), no. 1, 75--90, Errata, \textbf{57} (2020), no. 2, 505--506, \url{https://projecteuclid.org/euclid.ojm/1547607627}.
	
	\bibitem[O]{O} L.S. O'Neill, \textit{On the fixed point property for Grassmann manifolds}, \markbox{7040}{Thesis} (Ph.D.)---The \markbox{7041}{Ohio} \markbox{7042}{State} \markbox{7043}{University}
\markbox{7044}{ProQuest} LLC, \markbox{7045}{Ann} Arbor, \markbox{7046}{MI} (1974). 52 pp, \url{http://gateway.proquest.com/openurl?url_ver=Z39.88-2004&rft_val_fmt=info:ofi/fmt:kev:mtx:dissertation&res_dat=xri:pqdiss&rft_dat=xri:pqdiss:7511411}.	

	\bibitem[P]{Papadima} S. Papadima, \textit{Rigidity properties of compact Lie groups modulo maximal tori}, Math. Ann. \textbf{275} (1986), no. 4, 637--652, \url{https://doi.org/10.1007/BF01459142}.

	
	%\bibitem[Sp]{spanier} Spanier, E. H. {\em Algebraic Topology}. McGraw Hill, 1966. Reprinted by Springer-Verlag, New York, 1975.
	
	\bibitem[ST1]{shiga-tezuka} H. \markbox{7047}{Shiga} \markbox{7048}{and} M. Tezuka, \textit{Rational fibrations, homogeneous spaces with positive Euler characteristics and Jacobians}, Ann. Inst. \markbox{7049}{Fourier} (Grenoble) \textbf{37} (1987), no. 1, 81--106, \url{https://doi.org/10.5802/aif.1078}.
	
	\bibitem[ST2]{shiga-tezuka2} H. \markbox{7050}{Shiga} \markbox{7051}{and} M. Tezuka, \textit{Cohomology automorphisms of some homogeneous spaces}, \markbox{7052}{Singapore} \markbox{7053}{topology} \markbox{7054}{conference} (Singapore, 1985), \markbox{7055}{Topology} Appl. \textbf{25} (1987), no. 2, 143--150, Errata, \textbf{34} (1990), no. 2, 207, \url{https://doi.org/10.1016/0166-8641(87)90007-1}.
	
%	\bibitem[U]{ucci} Ucci, J. J. Immersions and embeddings of Dold manifolds. {\it Topology} {\bf 4} (1965), 283--293.

    \bibitem[W]{wong} P. Wong, \textit{Fixed point theory for homogeneous spaces. II}, Fund. Math. \textbf{186} (2005), no. 2, 161--175, \url{https://doi.org/10.4064/fm186-2-4}.


	
\end{thebibliography}
\end{document}