\documentclass{amsart}
\usepackage{amsmath}
\usepackage{amssymb}
\usepackage[mathscr]{eucal}

% --- Other utilities ---
\usepackage[shortlabels]{enumitem}
\usepackage{graphicx}
\usepackage[all]{xy}
\usepackage[dvipsnames]{xcolor}
\usepackage{tikz-cd}

% --- Hyperref should be loaded last ---
\usepackage[colorlinks=true,citecolor=red,urlcolor=blue,linkcolor=blue]{hyperref}

% --- Theorem environments ---
\newtheorem{theorem}{Theorem}[section]
\newtheorem{proposition}[theorem]{Proposition}
\newtheorem{lemma}[theorem]{Lemma}
\newtheorem{remark}[theorem]{Remark}
\newtheorem{definition}[theorem]{Definition}
\newtheorem{example}[theorem]{Example}
\newtheorem{corollary}[theorem]{Corollary}

% --- Commands ---
\newcommand{\thmref}[1]{Theorem~\ref{#1}}
\newcommand{\thmmref}[1]{Theorem$^\prime$~\ref{#1}}
\newcommand{\secref}[1]{Section~\ref{#1}}
\newcommand{\lemref}[1]{Lemma~\ref{#1}}
\newcommand{\propref}[1]{Proposition~\ref{#1}}
\newcommand{\propnref}[1]{Proposition$'$~\ref{#1}}
\newcommand{\corref}[1]{Corollary~\ref{#1}}
\newcommand{\remref}[1]{Remark~\ref{#1}}
\newcommand{\defref}[1]{Definition~\ref{#1}}
\newcommand{\subsecref}[1]{Subsection~\ref{#1}}
\newcommand{\homeo}{\mathrm{Homeo}}
\newcommand{\gal}{\mathrm{Gal}}
\newcommand{\bz}{\mathbb{Q}}
\newcommand{\bn}{\mathbb{N}}
\newcommand{\bq}{\mathbb{Q}}
\newcommand{\br}{\mathbb{R}}
\newcommand{\bc}{\mathbb{C}}
\newcommand{\bp}{\mathbb{P}}
\newcommand{\bs}{\mathbb{S}}
\newcommand{\co}{\mathcal{O}}
\newcommand{\rank}{\mathrm{Rank}}
\newcommand{\cl}{\mathcal{L}}
\newcommand{\lr}{\longrightarrow}
\renewcommand{\hom}{\mathrm{Hom}}
\newcommand{\wt}{\widetilde}
\newcommand{\im}{\mathrm{Im}}
\newcommand{\re}{\mathrm{Re}}
\newcommand{\Span}{\mathrm{span}}
\newcommand{\tor}{\mathrm{Tor}}
\newcommand{\pic}{\mathrm{Pic}}
\newcommand{\flag}{\mathrm{Flag}}
\newcommand{\ul}{\underline}
\newcommand{\fix}{\mathrm{Fix}}

\title[Graded Endomorphisms of $H^*(\mathbb{S}^m \times \mathbb{C}G_{n,k};\mathbb Q)$]{Rational Cohomology Endomorphisms of product of Sphere with Grassmannian and coincidence theory}

\author[M. Mandal]{Manas Mandal}
\address{Indian Institute of Technology Kanpur, Kanpur 208016, India}
	\email{manasm.imsc@gmail.com}
	
	\author[D. Setia]{Divya Setia}
	\address{Institute of Mathematics, Polish Academy of Sciences, Krak{\'o}w 31-027, Poland}
	\email{divyasetia01@gmail.com}
	
	\subjclass[2020]{Primary: 55S37, 08A35,  55M20; Secondary: 14M15}
	\keywords{cohomology endomorphisms, complex Grassmann manifolds, generalized Dold spaces,  fixed point theory, coincidence theory}

\begin{document}
\begin{abstract}
We classified graded endomorphisms of the rational cohomology algebra of the product of a sphere and a complex Grassmannian, whose images are nonzero  in the second cohomology of the Grassmannian.

We also derive necessary conditions for the generalized Dold spaces to satisfy the coincidence property, in particular the fixed-point property. As an application of our results, we obtain several sufficient conditions for the existence of a point of coincidence between a pair of continuous functions on certain generalized Dold spaces. 
\end{abstract}
	
\maketitle
	
	\section{Introduction} \label{intro}
	The classification of endomorphisms of the rational cohomology algebra of formal spaces was greatly motivated by Sullivan's theory where it was  proved that rational homotopy class of self-maps are completely determined by the induced graded endomorphisms of their rational cohomology algebras. 
	
	In \cite{brewster}, the authors developed the foundational work by classifying automorphisms of the rational cohomology algebra of complex Grassmannian. Their results were generalized in \cite{hoffman}, where the author classified graded endomorphisms of the rational cohomology algebra of complex Grassmannian which are nonzero on dimension $2$. Further, he conjectured that every graded endomorphism  vanishing on dimension two is necessarily trivial. This conjecture was proved in \cite{glover-homer} for several cases.
	
	The cohomology endomorphisms are also studied for a variety of homogeneous spaces $G/H$, where $G$ is a compact connected Lie group and $H$ is a closed
	subgroup of maximal rank. This is a topic of interest since past fifty years and are studied in several papers \cite{shiga-tezuka2, brewster-homer, hoffman-homer, Papadima, duan, duan-fang, duan-zhao, lin, goswami-sarkar}. 
	
	%However, we are interested in product spaces because a little is known about how cohomology endomorphisms behave under products.
    However, the behavior of cohomology endomorphisms for the product spaces is comparatively less explored, and this provides a direction for study.
    We are mainly interested in the product of a sphere
	with complex Grassmannian $\mathbb S^m\times \mathbb CG_{n,k}$ because the sphere $\mathbb{S}^m$ has a simple, singly generated cohomology algebra, while the Grassmannian $\mathbb{C}G_{n,k}$ (of $k$-planes in $\mathbb C^n$)
	carries the rich structure arising from Schubert calculus. We have classified graded endomorphisms of the rational cohomology algebra $H^*(\mathbb{S}^m \times \mathbb{C}G_{n,k};\mathbb{Q})$, whose image has a nonzero component in $H^2(\mathbb CG_{n,k},\mathbb Q)$.  As an application we obtain useful results in coincidence theory and in particular, fixed-point theory.  
	
	The rational cohomology algebra of the product
	\[
	H^*(\mathbb S^m \times \mathbb CG_{n,k};\mathbb Q)\cong H^*(\mathbb S^m,\mathbb Q)\otimes H^*(\mathbb CG_{n,k};\mathbb Q)
	\]
	is generated by $u,c_1,c_2,\dots,c_k$, where $H^*(\mathbb{C}G_{n,k};\mathbb{Q})$ (resp. $H^*(\mathbb S^m;\mathbb Q)$) is generated by certain Chern classes $c_1, \dots, c_k$ (resp. $u$). Now, we are ready to state one of the main results of our paper which proves that the rigidity of $H^*(\mathbb CG_{n,k};\mathbb Q)$ persists in $H^*(\mathbb S^m \times \mathbb CG_{n,k};\mathbb Q)$ even in the presence of spherical cohomology classes.
	\begin{theorem}
		Let $\phi$ be a graded endomorphism of 
		$H^*(\mathbb S^{m}\times\mathbb C G_{n,k};\mathbb Q)$ 
		satisfying $\phi(c_1)\neq a u, \, a\in\mathbb Q$.  
		Then there exists a nonzero rational $\lambda$ such that the following holds.
		\begin{enumerate}\label{result main}
			\item If $k < n - k,$ 
			$$ \phi(c_i) = \lambda^i c_i, \forall i \in \{1,2,\dots,k\}$$
			If $k = n - k$, there is an additional possibility of $\phi$ that it is induced by the homeomorphism 
			\[
			\mathbb{C}G_{2k,k} \longrightarrow \mathbb{C}G_{2k,k}, 
			\quad L \longmapsto L^{\perp},
			\]
			where $L^{\perp}$ denotes the orthogonal complement of the $k$-plane $L$ in $\mathbb{C}^{2k}$.
			
			\item 
			The image of $H^*(\mathbb{S}^m;\mathbb{Q})$ under $\phi$ lies in 
			$H^*(\mathbb{S}^m;\mathbb{Q})$ or in $H^*(\mathbb{C}G_{n,k};\mathbb{Q})$ i.e. $$\phi(u) = \mu u,\, \mu \in \mathbb{Q}, \text{ or } \phi(u) \in H^*(\mathbb{C}G_{n,k};\mathbb{Q}), \text{ with } (\phi(u))^2 =0.$$
			
		\end{enumerate}
	\end{theorem}
	Unlike the case of the complex Grassmannian, we cannot expect a graded
	endomorphism of $H^*(\mathbb S^m \times \mathbb CG_{n,k};\mathbb Q)$ to be
	trivial merely because it vanishes in $H^2(\mathbb{C}G_{n,k}; \mathbb{Q})$. 
	In fact, we proved that for any choice of $P_i \in H^{2i-m}(\mathbb{C}G_{n,k};\mathbb{Q})$ and $Q\in \mathbb Qu\cup  H^*(\mathbb CG_{n,k};\mathbb Q)$ with $Q^2=0$, there exist a graded endomorphism $\phi$ on $H^*(\mathbb S^m \times \mathbb CG_{n,k};\mathbb Q)$ such that $\phi(c_i) = uP_i, \, \forall i$ and $\phi(u)=Q$.
	%In fact, we proved that for any choice of $P_i \in H^{2i-m}(\mathbb{C}G;\mathbb{Q})$ and $Q =au, a\in \mathbb{Q}$, there exist a graded endomorphism $\phi$ on $H^*(\mathbb S^m \times \mathbb CG_{n,k};\mathbb Q)$ such that $\phi(c_i) = uP_i, \, \forall i$ and $\phi(u)=Q$.
	We also proved that if a continuous function on $\mathbb S^m \times \mathbb CG_{n,k}$ stabilizes a copy of Grassmannian then the induced cohomology endomorphism stabilizes the subalgebra $H^*(\mathbb S^m ;\mathbb Q)$.
	
	Our study is also motivated by the theory of generalized Dold spaces because the product space $\mathbb S^m \times \mathbb CG_{n,k}$ is a double cover of certain  generalized Dold spaces (GDS). %denoted by $P(m,n,k)$.
	The classical Dold manifolds were introduced in \cite{dold} to construct odd-dimensional generators for Thom's unoriented cobordism ring. In this paper, we are interested in the GDS introduced in \cite{nath-sankaran} and defined as  %These spaces have been extensively studied from various aspects (see \cite{ucci, khare, korbas, mukerjee}). 
	\[
	P(m,n,k):=\mathbb S^m\times \mathbb CG_{n,k}/\!\!\sim, \text { where } (s,L)\sim (-s,\bar L).
	\]
	As an application of \thmref{result main}, we describe endomorphisms of $H^*(P(m,n,k);\mathbb{Q})$ induced by continuous functions on $P(m,n,k)$. Using this description, we prove that every automorphism of $H^*(P(m,n,k);\mathbb{Q})$ induced by a continuous function on $P(m,n,k)$ lifts to an automorphism of $H^*(\mathbb S^m\times \mathbb CG_{n,k}; \mathbb{Q})$ if $n>2$. 
	
	Our broader aim is to apply \thmref{result main} to obtain results in coincidence theory. Coincidence theory has been extensively studied in \cite{hoffman-noncoin, glover-homer coin, wong}. %and fixed point theory.
	A pair $(X,g)$ where $g$ is a continuous function on $X$ is said to have the coincidence property if $g$ has a point of coincidence with every continuous function on $X$. In particular, if $g$ is the identity map, then the coincidence property is same as the fixed-point property of $X$.
	%We are providing necessary conditions for certain generalized Dold spaces to satisfy coincidence property.
	
	We have generalized Theorem $2$ of \cite{glover-homer} to the setting of coincidence theory and proved that the pair $(\mathbb{C}G_{n,k},g)$ satisfies the coincidence property if $k(n-k)$ is even and $g$ has nonzero Brouwer degree.
	To conclude, using the Lefschetz Coincidence Theorem and \thmref{result main}, we obtained multiple situations when two continuous functions on the generalized Dold space $P(m,n,k)$ are guaranteed to have a point of coincidence, and found certain pairs $(P(m,n,k),g)$ satisfying the coincidence property.
	%Using Lefschetz Coincidence Theorem and \thmref{result main}, we succeed in obtaining sufficient conditions for the existence of a point of coincidence between any two continuous functions on $P(m,n,k)$ when $k(n-k)$ is even.
	%At last, we explored when two continuous functions on the generalized Dold space $P(m,n,k)$ has a point of coincidence. Using Lefschetz Coincidence Theorem and \thmref{result main}, we succeed in obtaining sufficient conditions for the existence of a point of coincidence between any two continuous functions on $P(m,n,k)$ when $k(n-k)$ is even.
	
	The paper is organized as follows: \\
	In \secref{section 2} we develop the necessary background and recall some relevant results. \secref{section 3} is devoted to the study of graded endomorphisms of the rational cohomology algebra of $\mathbb S^{m}\times \mathbb C G_{n,k}$, from which we extract several consequences. These are applied in \secref{section 4} to obtain the coincidence-theoretic results.
	
	\iffalse For formal spaces, Sullivan's theory shows that the rational homotopy class of a self-maps are completely determined by the induced graded endomorphisms of their rational cohomology algebras. This viewpoint has motivated extensive work on cohomology endomorphisms of various classical  spaces.
	%, particularly homogeneous spaces of compact Lie groups.
	%Initial results for complex Grassmannians were obtained by Brewster \cite{}, Glover-Homer \cite{}, and Hoffman \cite{}, followed by contributions of  Shiga-Tezuka \cite{}, Papadima \cite{}, and Duan et al. (\cite{},\cite{},\cite{}) for broader classes of homogeneous spaces.
	% $G/H$, where $G$ is complact connected Lie group and $H$ is a closed subgroup of maximal rank.
	
	
	Early foundational work on cohomology endomorphisms of complex Grassmannians
	was carried out in 1978 by Brewster, in his Ph.D.\ thesis \cite{brewster} on cohomology
	automorphisms, and by Glover and Homer \cite{glover-homer}.
	This was followed by Hoffman's 1984 work
	\cite{hoffman}, which classified all rational cohomology endomorphisms that are
	non-trivial in degree two. He showed that for the Grassmannians $\mathbb CG_{n,k}$ (of $k$-planes in $\mathbb{C}^n$), such endomorphisms are precisely \emph{Adams maps}: each cohomology class $x$ of degree $2i$ maps to a scalar multiple $\lambda^i x$ for some fixed nonzero $\lambda \in \mathbb{Q}$, except in special symmetric cases $n=2k$, where additional involutive automorphisms appear. Hoffman further conjectured that every endomorphism vanishing in degree two is necessarily trivial. The results of Glover and Homer (see \cite{glover-homer}) support this conjecture in several cases, under a certain hypothesis, say (GH).
	
	
	Later, cohomology endomorphisms were studied for a variety of homogeneous
	spaces $G/H$, where $G$ is a compact connected Lie group and $H$ is a closed
	subgroup of maximal rank. For examples,
	Shiga and Tezuka examined cohomology
	automorphisms of homogeneous spaces of simple Lie groups \cite{shiga-tezuka2}, and Papadima described all cohomology automorphisms when the subgroup is a maximal torus,
	working over both $\mathbb Q$ and $\mathbb R$ \cite{Papadima}. Hoffman and Homer also
	proposed a general classification for homogeneous spaces arising from unitary
	groups, together with partial results supporting their conjecture. Further
	progress includes the study of integral cohomology endomorphisms of the Grassmannian of complex structures
	$SO(2n)/U(n)$ \cite{duan}, and the work of Duan and Fang on the eight exceptional
	Grassmannians \cite{duan-fang}. Related developments appear in \cite{duan-zhao}, \cite{lin}, and \cite{goswami-sarkar}.
	
	
	Most of these results, however, concern specific classes of homogeneous spaces.
	Comparatively little is known about how cohomology endomorphisms behave for
	products. The product of a sphere
	with a complex Grassmannian $\mathbb S^m\times \mathbb CG_{n,k}$ offers a particularly useful setting for such questions: the
	sphere has a simple, singly generated cohomology algebra, while the Grassmannian
	carries the rich structure arising from Schubert calculus.  
	Although the garded endomorphisms of the rational cohomology algebra of the product, $\mathrm{End}\big(H^*(\mathbb S^m\times \mathbb CG_{n,k};\mathbb Q)\big)$, are interesting in their own right, our broader aim is to apply them to the \textit{coincidence theory} of certain \textit{generalized Dold space}s.
	Coincidence theory studies the set of points $x$ for which two maps 
	$f,g : X \to X$ satisfy $f(x)=g(x)$, and seeks criteria for the 
	existence or elimination of such points. 
	
	
	%For a continuous map $g$ on a space $X$, the pair $(X,g)$ is said to have the \emph{coincidence property} (abbreviated as CP) if every continuous map $f:X\to X$ has a coincidence point with $g$; that is, $f(x)=g(x)$ for some $x\in X$.
	The classical Dold manifolds $P(m,n):=\mathbb S^{m}\times\mathbb CP^{n}/\!\sim$, 
	where $(s,L)\sim(-s,\bar L)$, were introduced by Dold \cite{dold} to construct odd-dimensional generators for Thom's unoriented cobordism ring. These spaces have been extensively studied and generalized. Nath and Sankaran \cite{nath-sankaran} extended the construction by replacing $\mathbb CP^{n}$ with a Hausdorff space $X$ equipped with an involution $\sigma$ with nonempty fixed points. A further generalization in \cite{mandal-sankaran} replaces the sphere $\mathbb S^{m}$ by a space $S$ equipped with a free involution $\alpha$, leading to
	\[
	P(S,\alpha,X,\sigma):=S\times X/\!\sim,\text{ where } (s,x)\sim(\alpha(s),\sigma(x)),
	\]
	which is called a \emph{generalized Dold space} (GDS).
	Here the quotient map $\pi: S\times X\to P(S,\alpha,X,\sigma)$ is a double covering.
	%The space $P(S,\alpha,X,\sigma)$ has a $X$-bundle structure over $Y:=S/\!\sim,$ where $s\sim \alpha(s)$.
	
	
	The class of GDS which we are interested in is $P(m,n,k):=P(\mathbb S^m,\alpha,\mathbb CG_{n,k},\sigma)$, where $\alpha $ is the antipodal map on the sphere and the involution $\sigma$ on $\mathbb CG_{n,k}$ is induced from the standard complex conjugation on $\mathbb C^n$. One observes that for any map $f$ on $P(m,n,k)$, there exist a $\alpha\times \sigma$-equivariant map $\tilde f$ on $\mathbb S^m\times\mathbb CG_{n,k}$ such that $\pi\circ \tilde f=f\circ \pi.$
	
	
	Let us regard $H^*(\mathbb S^m;\mathbb Q)$ and $H^*(\mathbb CG_{n,k};\mathbb Q)$ as
	subalgebras of $H^*(\mathbb S^m \times \mathbb CG_{n,k};\mathbb Q)$.
	Since $H^*(\mathbb CG_{n,k};\mathbb Q)$ is generated by certain Chern classes
	$c_1,c_2,\dots,c_k$, it follows that 
	\[
	H^*(\mathbb S^m \times \mathbb CG_{n,k};\mathbb Q)\cong H^*(\mathbb S^m,\mathbb Q)\otimes H^*(\mathbb CG_{n,k};\mathbb Q)
	\]
	is generated by $u,c_1,c_2,\dots,c_k$, where $u$ generates $H^m(\mathbb S^m;\mathbb Q)$.
	We are now ready to present our main results; the following is one of them, which shows that the rigidity of
	$H^*(\mathbb CG_{n,k};\mathbb Q)$ persists in the cohomology of the  product: even in the presence
	of spherical cohomology classes, any graded endomorphism whose image has a nonzero component in $H^{2}(\mathbb C G_{n,k};\mathbb Q)$ forces  $H^*(\mathbb CG_{n,k};\mathbb Q)$ to behave exactly as it does on its own.
	More precisely, we obtain: %that any graded endomorphism which is nontrivial on $H^2(\mathbb CG_{n,k};\mathbb Q)$ necessarily preserves  $H^*(\mathbb CG_{n,k};\mathbb Q)$.  Furthermore, the image of $H^*(\mathbb S^m;\mathbb Q)$ under such an endomorphism lies either entirely in $H^*(\mathbb S^m;\mathbb Q)$ or entirely in $H^*(\mathbb CG_{n,k};\mathbb Q)$.
	\begin{theorem}
		Let $\phi$ be a graded endomorphism of 
		$H^*(\mathbb S^{m}\times\mathbb C G_{n,k};\mathbb Q)$ 
		satisfying $\phi(c_1)\neq a u$ for any $a\in\mathbb Q$.  
		Then the following statements hold:
		\begin{enumerate}
			\item If $k < n - k$, there exists a nonzero rational number $\lambda$ such that
			\[
			\phi(x) = \lambda^i x 
			\quad \text{for all } x \in H^{2i}(\mathbb{C}G_{n,k}; \mathbb{Q}).
			\]
			If $k = n - k$, there is an additional possibility of $\phi$ induced by the homeomorphism 
			\[
			\mathbb{C}G_{2k,k} \longrightarrow \mathbb{C}G_{2k,k}, 
			\quad L \longmapsto L^{\perp},
			\]
			where $L^{\perp}$ denotes the orthogonal complement of the $k$-plane $L$ in $\mathbb{C}^{2k}$.
			
			\item 
			The image of $H^*(\mathbb{S}^m;\mathbb{Q})$ under $\phi$ lies either in 
			$H^*(\mathbb{S}^m;\mathbb{Q})$ or in $H^*(\mathbb{C}G_{n,k};\mathbb{Q})$.
			
		\end{enumerate}
	\end{theorem}
	
	Unlike the case of the complex Grassmannian, we cannot expect a graded
	endomorphism of $H^*(\mathbb S^m \times \mathbb CG_{n,k};\mathbb Q)$ to be
	trivial merely because it vanishes in degree two. Indeed, for any choice of
	elements $P_i \in H^{2i-m}(\mathbb S^m \times \mathbb CG_{n,k};\mathbb Q)$,
	$i=1,\dots,k$, one obtains a well-defined graded endomorphism $\phi$ of
	$H^*(\mathbb S^m \times \mathbb CG_{n,k};\mathbb Q)$ by setting
	$\phi(c_i) = uP_i$ for $i=1,\dots,k$ and $\phi(u)=u$, when the hypothesis (GH) holds. Hence, even in the case $P_1 = 0$, a large family of nontrivial graded endomorphisms still exists. See Proposition \ref{main thm 2}.
	We show that if a map on $\mathbb S^{m}\times \mathbb C G_{n,k}$ stabilizes a Grassmannian factor, then the induced cohomology endomorphism stabilizes the subalgebra $H^*(\mathbb S^{m};\mathbb Q)$. See \thmref{ind from top}.
	
	%(see Theorem \ref{ind from top}). 
	%\textcolor{gray}{If the map on $\mathbb S^m \times \mathbb CG_{n,k}$ has nonzero  Brouwer degree, the induced map in cohomology splits as a tensor product of two maps: one acting on the cohomology of the sphere and the other on the cohomology of the Grassmannian. (see  \ref{}).}
	
	\iffalse
	Using the study of $\mathrm{End}(H^*(\mathbb S^{m}\times \mathbb C G_{n,k};\mathbb Q))$, we obtain several coincidence-theoretic consequences. The next result provides a necessary criterion for determining when the pair $\big(P(S,\alpha,X,\sigma),g\big)$ have the coincidence property, for certain maps $g$ on the GDS, expressed in terms of its fibre $X$ and its base $Y:=S/\!\sim_{\alpha}$.
	
	
	
	
	\begin{proposition}
		For  a continuous map $g$ on the generalized Dold space $P(S,\alpha, X,\sigma)$, the pair $\big(P(S,\alpha,X,\sigma),g\big)$ does not have the CP if one of the following hold:
		\begin{enumerate}
			\item The pair $\big(Y,p \circ g \circ s\big)$ does not have the CP, 
			where $s$ denotes a global section of the $X$-bundle  
			$p: P(S, X) \to Y$, and $g$ is a fiber bundle map.
			\item  There exists a $\sigma$-equivariant map $f$  on $X$ and a  $\alpha \times \sigma$-equivariant map $\tilde g$ on $S\times X$ inducing $g$ such that $\mathrm{id}_S \times f$ coincides with neither $\tilde{g}$ nor $(\alpha \times \sigma) \circ \tilde{g}$.
		\end{enumerate}
	\end{proposition}
	\fi
	
	Using the study of $\mathrm{End}\bigl(H^*(\mathbb S^{m}\times \mathbb C G_{n,k};\mathbb Q)\bigr),$
	we obtain several coincidence-theoretic consequences. First, we extend Theorem~2 of \cite{glover-homer} to the setting of coincidence theory and show that for a continuous self-map $g$ with nonzero Brouwer degree on $\mathbb C G_{n,k}$ with $k(n-k)$ even, any continuous map $f$ on $\mathbb CG_{n,k}$ has a coincidence with $g$. See Proposition~\ref{CP of CGnk}.
	
	We then investigate the coincidence behaviour of the generalized Dold spaces $P(m,n,k)$. For any continuous self-map $g$ with nonzero Brouwer degree on $P(m,n,k)$  with $k(n-k)$ even and $m$ even, every map $f$ satisfying $\tilde f^{*}(c_1)\neq a u$ for any $a\in\mathbb Q$ has a coincidence with $g$; when $m$ is odd, one additionally requires $\tilde f^{*}(u)\neq -\tilde g^{*}(u)$. See Theorem~\ref{coincidence thm}.
	
	Under hypothesis (GH), if $g$ is a homotopy equivalence on $P(m,n,k)$ with $k(n-k)$ even, then $g$ has a coincidence with any map $f$ on $P(m,n,k)$ such that: \\ 
	-- when $m$ is even,  $\tilde f^{*}(c_1)=a u, a\in \mathbb Q$ implies $\tilde f^{*}(u)=b u$ for some $b\in\mathbb Q$;  \\
	-- when $m$ is odd,  $\tilde f^{*}(u)\neq -\tilde g^{*}(u)$ holds. See Theorem~\ref{coincidence thm under hom}.  \\
	Moreover, if $m>2k$, no additional assumptions on $f$ are needed.
	
	
	
	
	The paper is organized as follows: In Section~2 we develop the necessary background and recall some relevant results. Section~3 is devoted to the study of graded endomorphisms of the rational cohomology algebra of $\mathbb S^{m}\times \mathbb C G_{n,k}$, from which we extract several consequences. These are applied in Section~4 to obtain the coincidence-theoretic results.\fi
	
	
	
	
	
	
	%%%%%%%%%%%%%%%%%%%%%%%%%%%%%%%
	\section{Preliminaries}  \label{section 2}
	
	In this section, we discuss some preliminaries and recall some results that will be required to proceed with our study.
	
	
	
	
	
	
	
	\subsection{Cohomology of complex Grassmannians}
	
	
	Let $\mathbb{C}G_{n,k}$ denote the complex Grassmannian consisting of complex $k$-planes in $\mathbb{C}^n$. 
	%As a homogeneous space, it is diffeomorphic to $U(n)/U(k) \times U(n-k)$, where $U(r)$ denotes the group of unitary matrices of order $r$. 
	Let $\gamma_{n,k}$ and $\beta_{n,k}$ denote the canonical complex $k$-plane and $(n-k)$-plane bundles, respectively, over $\mathbb{C}G_{n,k}$.
	Let the total Chern classes of the vector bundles $\gamma_{n,k}$ and $\beta_{n,k}$ be denoted by $c(\gamma_{n,k}) = c$ and $c(\beta_{n,k}) = \bar{c}$, respectively. Thus,
	$$c = 1 + c_1 + c_2 + \cdots + c_k, \quad \bar{c} = 1 + \bar{c}_1 + \bar{c}_2 + \cdots + \bar{c}_{n-k},$$
	where $c_i$ and $\bar{c}_i$ denote the $i$-th Chern classes of $\gamma_{n,k}$ and $\beta_{n,k}$, respectively.
	Since $\gamma_{n,k} \oplus \beta_{n,k} \cong \varepsilon_{\mathbb{C}}^n$, it follows that  $c \cdot \bar{c} = 1$.
	The  cohomology ring of the complex Grassmannian is well known and given by  
	$$
	H^*_{\mathbb{C}G}:=H^*(\mathbb{C}G_{n,k};\mathbb{Q}) \cong \mathbb{Q}[c_1, c_2, \dots, c_k, \bar{c}_1, \bar{c}_2, \dots, \bar{c}_{n-k}]/\langle h_r: 1\leq r\leq n\rangle,$$  
	where  the relations $h_r$ for $r = 1, 2, \dots, n$ are induced from the homogeneous parts of the equation $c\cdot \bar c=1$ and given by  
	\[
	h_r := \sum_{i+j=r} c_i \bar{c}_j.
	\]  
	Using the relations $h_r, r=1,2,...,n-k$, the generators $\bar{c}_i$ for $i = 1, 2, \dots, n-k$ can be expressed inductively in terms of $c_i$ for $i = 1, 2, \dots, k$.  Consequently, the relations $h_r$ for $r = n-k+1, \dots, n$ become homogeneous polynomials in $c_i$ of degree $2r$, where the degree of each $c_i$ is $2i$. Then the cohomology ring $H^*_{\mathbb{C}G}$ can be rewritten as  
	\begin{equation}\label{cohomo of grass}
		\mathbb{Q}[c_1, c_2, \dots, c_k]/\langle h_{n-k+1}, h_{n-k+2}, \dots, h_n \rangle.
	\end{equation}
	Since there are no relations among the generators $c_i$ for $i = 1, 2, \dots, k$ up to degree $2(n-k)$, the set of all monomials of degree $2r$ in terms of $c_1, c_2, \ldots, c_k$ forms a $\mathbb{Q}$-basis of $H^{2r}(\mathbb{C}G_{n,k};\mathbb{Q})$ for $r \leq n-k$.
	
	From now on, we denote the indexing set $\{1,2,\dots, k\}$ by $I$.
	\begin{remark}
		We can assume  $k\leq n-k$ for  $\mathbb{C}G_{n,k}$ as $\mathbb{C}G_{n,k}$ is homeomorphic to $\mathbb{C}G_{n,n-k}$ by using orthogonal complementation.
	\end{remark}
	The complex Grassmannian $\mathbb{C}G_{n,k}$ is a homogeneous space and can be represented as the quotient of the unitary group $U(n)$ by the stabilizer subgroup $U(k)\times U(n-k)$ that is 
	\begin{equation}\label{cgn as hom}
		\mathbb{C}G_{n,k} = U(n)/ (U(k)\times U(n-k)). 
	\end{equation} Now we recall a result given in \cite{shiga-tezuka}.    
	\begin{theorem}[\cite{shiga-tezuka}, Theorem \(A^{'}\)]\label{Tezuka}
		Let $D_i(H^*(G/H;\mathbb{Q}))$ be the $\mathbb{Q}$-vector space of $\mathbb{Q}$-derivations of $H^*(G/H;\mathbb{Q})$ which decreases the degree by $i>0$ where $G$ is a connected, compact Lie group and $H$ is a closed subgroup of maximal rank.
		%such that $\rank (H) = \rank (G)$.
		Then, for all $i$, $$D_i(H^*(G/H;\mathbb{Q})) =0.$$  
	\end{theorem}
	\subsection{Graded endomorphisms on $\mathbf{H^*_{\mathbb{C}G}}$}
	It was conjectured in \cite{O} that any graded endomorphism $\phi$ of the cohomology algebra $H^*_{\mathbb{C}G}$ is an\textit{ Adams }map when $k < n - k$; that is, there exists a rational $\lambda$ such that
	$\phi(c_i) = \lambda^i c_i$, for all $ i \in I.$ Glover and Homer (see \cite{glover-homer}) and Hoffman (see \cite{hoffman}) proved the conjecture under the following hypothesis respectively:
	\begin{align}
		\text{Either } k \leq 3 \text{ and } n > 2k \text{, or } k>3 \text{ and } n>2k^2 -1. \label{Homer}\\
		\text{ The graded endomorphism } \varphi  \text{ of } H^*_{\mathbb{C}G} \text{ satisfies } \varphi(c_1) = \lambda c_1, \lambda\neq 0.  \label{Hoff}
	\end{align}
	%In this context, Glover and Homer \ref{Hoffman} (see \cite{glover-homer}) proved the following result.
	Let us recall those results proved in \cite{glover-homer, hoffman} that will be used in the rest of this paper.  
	\begin{theorem}[\cite{glover-homer}, Theorem 1, \cite{hoffman}, Theorem 1.1]\label{hom and hof}
		(i) Assume that the hypothesis \eqref{Homer} is satisfied. Then for every graded endomorphism $\varphi$ on $ H^*(\mathbb{C}G_{n,k}; \mathbb{Q})$, there exists a rational $\lambda$ such that
		\[\varphi(c_i) = \lambda^i c_i,  \quad \forall i \in I.\]
		%where $c_i$ denotes the $i$-th Chern class of the canonical complex $k$-plane bundle $\gamma_{n,k}$ over the complex Grassmannian $\mathbb{C}G_{n,k}$.
		(ii) Assume that the hypothesis \eqref{Hoff} is satisfied. Then, we have
		$$\varphi(c_i) = \begin{cases}
			\lambda^i c_i,   \forall i \in I& \text{ if } k<n-k,\\
			\lambda^i c_i,  \forall i \in I \quad \text{ or } \quad(-\lambda)^i (c^{-1})_i,    \forall i \in I & \text{ if } k= n-k,
		\end{cases}$$
		where $ (c^{-1})_i $ is the $ 2i $-dimensional part of the inverse of $ c = 1 + c_1 + \cdots + c_k $ in $ H^*(\mathbb CG_{n,k}; \mathbb{Q}) $.
	\end{theorem}
	\iffalse Hoffman provided a classification of graded endomorphisms of $H^*(\mathbb{C}G_{n,k}; \mathbb{Q})$ that are nonvanishing on $H^2(\mathbb{C}G_{n,k}; \mathbb{Q})$ (see \cite{hoffman}). We recall his result in the following theorem.
	\begin{theorem}[]\label{hoffman}
		Let $ \varphi $ be an endomorphism of $ H^*(\mathbb CG_{n,k}; \mathbb{Q}) $ with $ \varphi(c_1) = \lambda c_1 $, $ \lambda\neq 0 $. Then if $ k < n-k $,
		\[
		\varphi(c_i) = \lambda^i c_i, \quad 1 \le i \le k.
		\]
		
		If $ k = n $, there is the additional possibility
		\[
		\varphi(c_i) = (-\lambda)^i (c^{-1})_i, \quad 1 \le i \le k,
		\]
		
		
	\end{theorem}\fi
	
	\subsection{Generalized Dold spaces}  \label{gen dold}
	In \cite{dold}, the author introduced the notion of \textit{classical Dold manifolds}  
	$P(m,n) := \mathbb{S}^m \times \mathbb{C}P^n / \!\! \sim$  
	where $ (s, L) \sim (-s, \bar{L}) $, where the involution $L\mapsto\bar L$ on $\mathbb CG_{n,k}$ is induced from the standard conjugation on $\mathbb C^n$, to construct generators in odd dimensions for Ren{\'e} Thom's unoriented cobordism ring. %These spaces are of great interest, and many generalizations and studies have been done on them in the literature.
	
	\iffalse (Put this in introduction)In \cite{nath-sankaran}, authors generalized the notion of Dold manifolds by replacing the complex projective space $\mathbb{C}P^n$ with an almost complex manifold $X$ equipped with a complex conjugation, i.e., an involution $\sigma: X \to X$ with nonempty fixed-point set such that the differential
	$d\sigma|_p: T_pX \to T_{\sigma(p)}X$
	satisfies
	$J_{\sigma(p)} \circ d\sigma_p = -d\sigma_p \circ J_p,$
	where $J$ denotes the almost complex structure on $X$, and called them \textit{generalized Dold manifolds} to investigate their manifold properties such as tangent bundles, Stiefel--Whitney classes, (stable) parallelizability, cobordism classes, and related aspects.
	\fi
	
	In \cite{nath-sankaran, mandal-sankaran}, the authors generalized the notion of classical Dold manifolds by replacing the sphere $ \mathbb{S}^m $ with an arbitrary topological space $ S $ equipped with a free involution $ \alpha $, analogous to the antipodal map on $ \mathbb{S}^m $, and $ \mathbb CP^n $ with an arbitrary topological space $ X $ with an involution $ \sigma: X \to X $ having a nonempty fixed-point set, analogously to complex conjugation on $\mathbb CP^n$. Then the quotient space  
	\begin{equation}\label{gen dold space}
		P(S, \alpha, X, \sigma) := S \times X / \!\! \sim, \quad \text{where } (s, x) \sim (\alpha(s), \sigma(x)), 
	\end{equation}  
	is called \textit{generalized Dold space} (in short GDS), often denoted simply as $ P(S, X) $. Moreover, the quotient map  
	$ S \times X \to P(S,X) $  
	is a double covering map. %The focus of their study were used to study cohomology and complex $K$-theory of various GDS.
	
	Let us fix a notation $ Y $ for $ S/\!\!\sim_{\alpha} $, where $ s\sim_{\alpha} \alpha(s), \forall s\in S $. Then, a GDS $ P(S,X) $ is the total space of a fiber bundle  
	$X\hookrightarrow P(S,X) \twoheadrightarrow Y $, where the fiber bundle projection is \begin{equation}\label{proj}
		p: P(S,X) \twoheadrightarrow Y, \quad [s,x]\mapsto [s].
	\end{equation}  Choosing a fixed-point of $\sigma$, say $x_0\in \text{Fix}(\sigma)\neq \emptyset$, we can construct a section of the fiber bundle \begin{equation}\label{sectio}
		s: Y \hookrightarrow P(S,X), \quad [s]\mapsto [s,x_0].
	\end{equation}    
	In fact, we have an embedding  
	$Y\times \text{Fix}(\sigma)\hookrightarrow P(S,X),$  
	where $ \text{Fix}(\sigma) \subseteq X $ has the subspace topology induced from $ X $.
	
	
	
	
	
	
	\subsection{Rational cohomology of $\mathbf{P(\mathbb S^m,\mathbb CG_{n,k})}$} \label{gds}
	%The rational cohomology ring of \textit{generalized Dold space} $ P(\mathbb{S}^m, \mathbb{C}G(\nu)) $, which are fibered by a partial complex flag manifold $ \mathbb{C}G(\nu) $ of type $ \nu = (\nu_1 \leq \nu_2 \leq \dots \leq \nu_s) $ over a real projective space $ \mathbb{R}P^m $, is computed in \cite{mandal-sankaran2}. In particular, and we recall some relevant facts here.
	
	
	The GDS $P(\mathbb S^m,\mathbb CG_{n,k})$ is defined as
	\[
	\mathbb S^m\times \mathbb CG_{n,k}/\!\!\sim, \text { where } (s,L)\sim (-s,\bar L),
	\]
	for which, $\mathbb S^m$ is equipped with the free action generated by the antipodal map $\alpha$ and the involution $\sigma: L \mapsto \bar L$ on $\mathbb CG_{n,k}$ is induced from the standard complex conjugation on $\mathbb C^n.$ We denote $P(\mathbb S^m,\mathbb CG_{n,k})$ simply by $P(m,n,k).$
	By the K\"unneth formula, we have
	\begin{equation}\label{Cohomology of H_times}
		H_{\times}^* := H^*(\mathbb{S}^m \times \mathbb{CG}_{n,k}; \mathbb{Q}) \cong H^*(\mathbb{S}^m; \mathbb{Q}) \otimes H^*(\mathbb{CG}_{n,k}; \mathbb{Q}) \cong \frac{\mathbb{Q}[u, c_1, \dots, c_k]}{\langle u^2, h_{n-k+1}, \dots, h_n \rangle}
	\end{equation}
	where $u \in H^m(\mathbb S^m; \mathbb Q)$ denotes the generator corresponding to the fundamental class of $\mathbb S^m$. Note that \begin{equation}\label{H in terms of u}
		H^*_{\times} \cong H^*_{\mathbb{C}G}[u]/\langle u^2 \rangle \cong H^*_{\mathbb{C}G} \oplus u H^*_{\mathbb{C}G},
	\end{equation} where the latter isomorphism is a $\mathbb Q$-module isomorphism. We have that $H^*_{\mathbb{C}G}$ is a subring of $H^*_\times$.
	%For simplicity, denote $H^*(\mathbb{S}^m \times \mathbb{CG}_{n,k}; \mathbb{Q}) $ by $H_{\times}^*$
	The product involution $\theta:= \alpha\times\sigma$ on $\mathbb S^m\times \mathbb CG_{n,k}$ induces an involution $\theta^*$ on $H^*_\times$ given by
	\begin{equation}\label{defn of theta*}
		\theta^*(c_i) = (-1)^i c_i,i\in I, \quad \theta^*(u) =  
		\begin{cases}  
			u, & \text{if } m \text{ is odd}, \\  
			-u, & \text{if } m \text{ is even}.  
		\end{cases}  
	\end{equation}
	%where $ H^*(\mathbb{S}^m; \mathbb{Q}) \cong \mathbb{Q}[u]/\langle u^2 \rangle $.
	%The induced involution $\theta^*$ is given by
	%t is proved in \cite{mandal-sankaran2} that $ H^*(P(m,n,k); \mathbb{Q}) $ is isomorphic to the fixed subring $ \text{Fix}(H^*(\theta; \mathbb{Q})) $ under $ \theta^* $ of $ H^*(\mathbb{S}^m \times \mathbb{C}G_{n,k}; \mathbb{Q}) $ (Proposition 3.13 of \cite{mandal-sankaran2}). We recall the result below.
	The cohomology ring $ H^*(P(m, n,k);\mathbb Q) $ was computed in \cite{mandal-sankaran2} and the following result was proved.
	\begin{theorem}[{\cite[Theorem 3.13]{mandal-sankaran2}}]\label{cohomology of P(m,n,k)}
		The cohomology algebra \( H^*(P(m,n,k); \mathbb{Q}) \) is isomorphic to the subalgebra
		$\mathrm{Fix}(\theta^*) \subseteq H^*(\mathbb{S}^m \times \mathbb{CG}_{n,k}; \mathbb{Q}),$
		generated by the following elements:
		\begin{align*}
			u \, c_{2p-1},\quad c_{2j},\quad c_{2p-1} \, c_{2q-1},\; \forall 2p-1, 2q-1, 2j \in I, \text{ if } m \text{ is even};\\
			u,\quad c_{2j},\quad c_{2p-1} \, c_{2q-1},\; \forall 2p-1, 2q-1, 2j \in I, \text{ if } m \text{ is odd}.
		\end{align*}
	\end{theorem}
	A description of the cohomology algebra $ H^*(P(m,n,k); \mathbb{Q}) $, as a quotient of a polynomial algebra, can be deduced as a particular case in Theorem 3.14 of \cite{mandal-sankaran2}.
	%Since the description of the cohomology  algebra $H^*(P(m,n,k);\mathbb Q)$ in abstarct variables, is quite complicated in notations, instead we shall present the $\fix (H^*_\times)$ as $\mathcal R/$
	
	
	
	
	
	
	
	
	
	%%%%%%%%%%%%%%%%%%%%%%%%%%%5%%%%%%
	
	\section{Graded endomorphisms of $H^*(\mathbb S^m\times \mathbb CG_{n,k};\mathbb Q)$} \label{section 3}
	
	In this section, we classify graded endomorphisms of the rational cohomology algebra $ H^*(\mathbb S^m\times \mathbb CG_{n,k}; \mathbb{Q}) $ whose images are nonzero in $H^2(\mathbb CG_{n,k};\mathbb Q)$. Our approach relies on the study of graded endomorphisms of $ H^*(\mathbb{C}G_{n,k}; \mathbb{Q}) $ from \cite{glover-homer} and \cite{hoffman}. Assume $m>0$ for the rest of this paper.
	\subsection{} The cohomology ring of the complex Grassmannian $ \mathbb{C}G_{n,k} $ is generated by the Chern classes $c_i,\forall i \in I $ as given in \eqref{cohomo of grass}. In \eqref{Cohomology of H_times}, we see that the cohomology ring of $S^m \times \mathbb{C}G_{n,k}$ is generated by $u, c_i, \forall i \in I$. Therefore, it is sufficient to describe the images of the generators to classify graded endomorphisms of $H_{\times}^*$. The following is the main result of this section.
	
	
	
	
	
	
	
	% We now state the following result concerning the classification of graded endomorphisms of $H^*_\times$ that are non-vanishing on  $H^2_{\mathbb{C}G} \subseteq H^*_\times$.
	\begin{theorem}\label{main thm}
		Let $\phi$ be a graded endomorphism of $H^*_{\times}$ satisfying $\phi(c_1) \neq \mu u,\, \mu \in \mathbb{Q}$.  
		Then the following holds, \begin{enumerate}
			\item Either $\phi(u)=au$ for some $a \in \mathbb{Q}$, or $\phi(u) \in H^*_{\mathbb{C}G} \subseteq H^*_{\times}$ with $\phi(u)^2=0$ in $H^*_{\times}$.
			\item There exists $\lambda \in \mathbb Q\backslash\{0\}$ such that
			$$\phi(c_i) = \begin{cases}
				\lambda^i c_i,   \forall i \in I& \text{ if } k<n-k,\\
				\lambda^i c_i,  \forall i \in I \quad \text{ or } \quad(-\lambda)^i (c^{-1})_i,    \forall i \in I & \text{ if } k= n-k,
			\end{cases}$$
		\end{enumerate} where $ (c^{-1})_i $ is the $ 2i $-dimensional part of the inverse of $ c = 1 + c_1 + \cdots + c_k $ in $ H^*_{ \mathbb CG} $.
	\end{theorem}
	
	% \textcolor{red}{Maybe we can write this theorem in full generality when $\mathbb CG_{n,k}$ is replaced by $G/H$ as in Theorem $A'$ in \cite{shiga-tezuka}, and deduce this result as corollary.}
	
	\begin{proof}
		%Since $\phi$ is graded and non-vanishing on $H^2_{\mathbb CG}$, the image $\phi(c_1) = \lambda c_1+uP_1$ for some nonzero $\lambda \in \mathbb{Q}$ and $P_1\in H^{2-m}_{\mathbb CG}$.
		From equation \eqref{Cohomology of H_times} and \eqref{H in terms of u}, we have $H^*_{\times}\cong \mathcal R/\mathcal I \cong H^*_{\mathbb{C}G} \oplus u H^*_{\mathbb{C}G}$, where $\mathcal R:=\mathbb Q[u,c_1,\ldots,c_k]$ and $\mathcal I:=\langle u^2, h_{n-k+1},\ldots,h_n\rangle$.
		%, we may regard $H^*_{\mathbb{C}G}$ as a subring of $H^*_{\times}$.
		%Therefore, the elements in $H^*_\times$ are of the $P+uQ$ where $P$ and $Q$ are in $H^*_{\mathbb{C}G}$.
        %that are not multiple of $u$, can be written purely in terms of $c_1,c_2,\ldots,c_k.$
		
		%We begin with \textit{(1)} for $k < n - k$.  
		Let $p_1: H^*_{\times} = H^*_{\mathbb{C}G} \oplus u H^*_{\mathbb{C}G} \to H^*_{\mathbb{C}G}$ be the projection onto the first summand and $i_1: H^*_{\mathbb{C}G} \hookrightarrow H^*_{\mathbb{C}G} \oplus u H^*_{\mathbb{C}G}$ be the inclusion into the first summand. The composite $\phi_1:= p_1 \circ \phi \circ i_1$ is a degree-preserving endomorphism of $H^*_{\mathbb{C}G}$. We have the following diagram:
		% https://q.uiver.app/#q=WzAsNCxbMCwwLCJIXipfe1xcbWF0aGJiIENHfVxcb3BsdXMgdSBIXipfe1xcbWF0aGJiIENHfSJdLFsxLDAsIkheKl97XFxtYXRoYmIgQ0d9IFxcb3BsdXMgdSBIXipfe1xcbWF0aGJiIENHfSJdLFswLDEsIkheKl97XFxtYXRoYmIgQ0d9Il0sWzEsMSwiSF4qX3tcXG1hdGhiYiBDR30iXSxbMCwxLCJcXHBoaSJdLFsyLDAsImlfMSIsMCx7InN0eWxlIjp7InRhaWwiOnsibmFtZSI6Imhvb2siLCJzaWRlIjoidG9wIn19fV0sWzEsMywicF8xIiwwLHsic3R5bGUiOnsiaGVhZCI6eyJuYW1lIjoiZXBpIn19fV0sWzIsMywiXFx0aWxkZVxccGhpIl1d
		\begin{equation}\label{comm diagram}
			\begin{tikzcd}
				{H^*_{\mathbb CG}\oplus u H^*_{\mathbb CG}} & {H^*_{\mathbb CG} \oplus u H^*_{\mathbb CG}} \\
				{H^*_{\mathbb CG}} & {H^*_{\mathbb CG}}
				\arrow["\phi", from=1-1, to=1-2]
				\arrow["{p_1}", two heads, from=1-2, to=2-2]
				\arrow["{i_1}", hook, from=2-1, to=1-1]
				\arrow["{\phi_1}", from=2-1, to=2-2]
			\end{tikzcd}
		\end{equation}
		Thus, for  $x \in H^*_{\mathbb{C}G} \subset H^*_\times$, one can write
		$\phi(x) = \phi_1(x) + u P_x$
		for some $P_x \in H^*_{\mathbb{C}G} \subset H^*_\times$ because the kernal of $p_1$, $\ker(p_1) = u H^*_{\mathbb{C}G}$.
		\iffalse we have
		\begin{equation} 
			\phi(x) = \phi_1(x) + u P_x, \quad \text{where  } P_x\in H^*_{\mathbb CG}.
		\end{equation}\fi
		This implies that \begin{equation}\label{defn of phi}
			\phi(c_i) = \phi_1(c_i) + u P_{c_i},\, \forall i \in I.
		\end{equation} For simplicity, denote $P_{c_i}$ by $P_i\in H^{2i-m}_{\mathbb CG}$ which is a polynomial in $c_1, \dots, c_k$ of degree $2i - m$ as $\deg c_i = 2i$ and $\deg u = m$.
		
		
		
		Since $\phi(c_1)\neq \mu u, \, \mu \in \mathbb{Q}$, that implies $\phi(c_1)$ is of the form $\lambda c_1+\mu u,\, \lambda,\mu \in \mathbb{Q}, \, \lambda \neq 0$. Then we have $\phi_1(c_1) = \lambda c_1,\, \lambda\neq 0$ on $H^*_{\mathbb{C}G}$.  By \thmref{hom and hof} part \textit{(ii)}, we have, 
		\begin{equation}\label{phi_1}
			\phi_1(c_i) = \begin{cases}
				\lambda^i c_i,   \forall i \in I& \text{ if } k<n-k,\\
				\lambda^i c_i,  \forall i \in I \quad \text{ or } \quad(-\lambda)^i (c^{-1})_i,    \forall i \in I & \text{ if } k= n-k,
			\end{cases}
		\end{equation} where $ (c^{-1})_i $ is the $ 2i $-dimensional part of the inverse of $ c = 1 + c_1 + \cdots + c_k $ in $ H^*_{ CG} $. Using the observations given above it is convenient to prove part \textit{(2)} first. \\
		
		
		\textit{proof of part (2):} Using \eqref{defn of phi} and \eqref{phi_1}, it is sufficient to prove that $P_i =0, \, \forall i \in I$. By \eqref{phi_1}, we have that $\phi_1$ is an automorphism of $H^*_{\mathbb CG}$. Using the invertibility of $\phi_1$ and \eqref{defn of phi}, let $D: H^*_{\mathbb{C}G} \rightarrow H^*_{\mathbb{C}G}$ be defined by $$D(x) = P_{\phi_1^{-1}(x)},\, \forall x \in H^*_{\mathbb{C}G}.$$
		%By degree comparison, $P_i =0, \, \forall i \in I$ if $m$ is odd or $m > 2k$. Hence, we assume that $m$ is even, i.e. $m = 2s$ with $s\in I$.
		%whenever $\phi(x) = \phi_1(x) + uP_x$.  
		Equivalently, we have $D(\phi_1(x)) = P_x$. Now, we prove that $D$ is a $\mathbb Q$-linear transformation and satisfies the Leibniz rule.
		%Linearity over $\mathbb{Q}$ is immediate, since for $t \in \mathbb{Q}$ one has
		\begin{equation}\label{D is linear}
			\begin{split}
				uP_{tx} &= \phi(tx) - \phi_1(tx) = t(\phi(x) - \phi_1(x)) = utP_{x},\, \forall t \in \mathbb{Q},\\
				uP_{x + y} &= \phi(x+y) - \phi_1(x+y) = \phi(x) - \phi_1(x)+ \phi(y) - \phi_1(y)\\ &= u(P_{x}+P_{y}),\\
				uP_{x y} &= \phi(x y) - \phi_1(xy) = \phi(x)\phi(y)-\phi_1(x)\phi_1(y) \\
				&= (\phi_1(x)+uP_{x})(\phi_1(y)+uP_y)- \phi_1(x)\phi_1(y)\\
				&= u(P_x\phi_1(y)+\phi_1(x)P_y).
			\end{split}
		\end{equation}
		Using \eqref{H in terms of u} and \eqref{D is linear}, we get  \begin{align*}
			& D(t\phi_1(x)) = tD(\phi_1(x)),\quad D(\phi_1(x)+\phi_1(y)) = D(\phi_1(x))+D(\phi_1(y)),\\
			&D(\phi_1(x)\phi_1(y)) = D(\phi_1(x))\phi_1(y)+\phi_1(x)D(\phi_1(y)).
		\end{align*}
		\iffalse \textcolor{red}{(Do we need to emphasise that $D(h_i)\in \langle h_{n-k+1},\ldots,h_n\rangle$, or it is obvious from linearlity?)}
		
		To verify that $D$ satisfies the Leibniz condition $D(xy)=D(x)y+xD(y)$,
		it suffices to check that
		$D(\phi_1(x)\phi_1(y))=D(\phi_1(x))\phi_1(y)+\phi_1(x)D(\phi_1(y))$
		for all $x,y \in H^*_{\mathbb{C}G}$, since $\phi_1$ is an automorphism of $H^*_{\mathbb{C}G}$.
		Indeed, from
		$\phi(xy)=\phi(x)\phi(y)=(\phi_1(x)+uP_x)(\phi_1(y)+uP_y)$ where $x,y\in H^*_{\mathbb CG}$, we obtain
		$\phi(xy)=\phi_1(xy)+u(P_x\phi_1(y)+\phi_1(x)P_y)$.
		Comparing with $\phi(xy)=\phi_1(xy)+uP_{xy}$ gives
		$P_{xy}=P_x\phi_1(y)+\phi_1(x)P_y$, and therefore
		$D(\phi_1(x)\phi_1(y))=D(\phi_1(x))\phi_1(y)+\phi_1(x)D(\phi_1(y))$. \fi
		This proves that $D$ is a derivation. For $x \in H^i_{\mathbb{C}G}$, we have $D(x) \in H^{i-m}_{\mathbb{C}G}$ which implies that the derivation $D$ decreases the degree by $\deg(u)=m>0$. By \eqref{cgn as hom} and \thmref{Tezuka}, we get that $D$ is a zero derivation. In particular $$D(\phi_1(c_i))=P_i=0, \, \forall i \in I.$$
		%This completes the proof of \textit{(1)} in both cases $k<n-k$ and $k=n-k$.
		
		\textit{proof of part (1):} Since $\phi$ is a graded endomorphism on $H^*_{\times}$, therefore $$\phi(u) = a u + P, \, a \in \mathbb{Q}, \text{ satisfying } (a u + P)^2 =0,$$ where $P$ is a homogeneous polynomial in $c_1, \dots, c_k$ of degree $m$. We have $P^2 + 2 a u P =0$ in $H^*_{\times}$. Using \eqref{H in terms of u}, we get that $2aP =0$ in $H^*_{\times}=\mathcal{R}/\mathcal{I}$. Hence, either $a=0$ or $P\in \mathcal{I}$.
		%for some $f, f_i\in \mathcal R$ and generators $h_i$ of $\mathcal{I}\subset \mathcal R$. Write $f_i = g_i + u l_i$ with $g_i, l_i$ in $\mathbb{Q}[c_1, \dots, c_k]$, then comparing terms not in $u^2$ gives:\[P^2 + 2 a u P = \sum g_i h_i + u \sum l_i h_i.\]Thus, $2 a u P = u \sum l_i h_i \in \mathcal{I}$, so if $a \neq 0$, then $P \in \mathcal{I}$ and hence $\phi(u) = a u$ in $H^*_{\times}$. If $a = 0$, then $\phi(u) = P(c_1, \dots, c_k)$ with $P^2 \in \mathcal{I}$.This completes the proof.
	\end{proof}
	
	
		\begin{remark}
			\thmref{main thm} classifies all graded endomorphisms $\phi$ of $H^*_\times$ whose image is nonzero in $H^2_{\mathbb CG}$ if $n>2$. In fact, $n>2$ implies $c_1^2\neq 0$ and $\phi(u) \neq ac_1,\, a \in \mathbb{Q}\setminus\{0\}$ as $\phi(u)^2=0$. Therefore, the only remaining possibility is $\phi(c_1)\neq \mu u,\, \mu \in \mathbb Q.$ 
			
			On the other hand, when $n=2,$ $\mathbb CG_{n,k}$ is either a point or $\mathbb S^2$ and the classification of graded endomorphisms of $H^*_\times$ is easy.
		\end{remark}
	
	
	
	\subsection{} In \thmref{main thm}, we assume that $\phi(c_1) \neq \mu u$. Let us try to look at the other case where $\phi(c_1) = \mu u$. To address this, we use part (i) of \thmref{hom and hof} which leads to the following proposition.
	
	
	\begin{proposition}\label{main thm 2}
		Assume that hypothesis \eqref{Homer} is satisfied.
		Let $\phi$ be a graded endomorphism such that $\phi(c_1)=\mu u,\, \mu \in \mathbb{Q}$ in $H^*_{\times}$. Then
		\begin{enumerate}
			\item Either $\phi(u)=a u$ for some $a \in \mathbb{Q}$, or $\phi(u) \in H^*_{\mathbb{C}G} \subseteq H^*_{\times}$ with $\phi(u)^2=0$ in $H^*_{\times}$.
			\item  $\phi(c_i) = uP_i, \, \forall i >1,$ where $P_i \in H^{2i-m}_{\mathbb CG}\subseteq H^*_\times$. 
		\end{enumerate}
	\end{proposition}
	\begin{proof} \textit{(1):} The proof of part \textit{(1)} is exactly the same as the proof of part \textit{(1)} of \thmref{main thm}. Therefore, we omit the details.
		
		\textit{(2):} Using \eqref{comm diagram}, we have that the map $\phi_1$ is a graded endomorphism on $H^*_{\mathbb{C}G}$ such that $\phi_1(c_1) =0$. By \thmref{hom and hof}, $\phi_1(c_i) =0, \, \forall i\in I$, then by \eqref{defn of phi}, we get $\phi(c_i) = uP_i$ for some $P_i \in H^*_{\mathbb{C}G}$, with $\deg(P_i) = 2i - m$. \end{proof}
	%  When $\lambda \neq 0$, the proof follows directly from Theorem~\ref{main thm}.
	\begin{remark}
		In \thmref{main thm} and \propref{main thm 2}, if we assume $2m \leq n-k$  then $\phi(u)=0$ whenever $\phi(u) \in H^*_{\mathbb{C}G}$. This is because $H^*_{\mathbb{C}G}$ has no nontrivial relations up to degree $2(n-k)$ and $u^2=0$ implies that $\phi(u)^2=0$ forcing $\phi(u)=0$.
		
	\end{remark}
	%Now suppose $\lambda = 0$, i.e., $\phi(c_1) = 0$ in $H^*_\times$.
	%Unlike the case of Grassmannian, the following proposition guarantees the existence of non-trivial graded endomorphisms $\phi$ on $H^*_\times$ even if $\phi(c_1) =0$.
   A graded endomorphism of $H^*_{\mathbb CG}$ that vanishes on
   $H^2_{\mathbb CG}$ is expected to be trivial, in view of Hoffman's conjecture \cite{hoffman}. However, unlike the case of the complex Grassmannian, there exist many non-trivial graded endomorphisms of $H^*_\times$ that vanish on $H^2_{\mathbb CG}$. The following proposition provides   such examples when $m$ is even and   $1\le m \le 2k$.



	
	
	\begin{proposition}
		For each $i\in I$, choose $P_i \in H^{2i-m}_{\mathbb C G} \subseteq H^*_\times$ and either $Q = au,\, a\in \mathbb Q$, or  $Q\in H^*_{\mathbb CG}\subseteq H^*_{\times}$ with $Q^2=0$  in $ H^*_{\times}$. Then there exist a graded endomorphism $\phi$ on $H^*_\times$ such that 
		\[
		\phi(c_i)=uP_i, \; \forall i\in I, \text{ and  } \quad \phi(u)=Q.
		\]
	\end{proposition}
	\begin{proof}
		Define $\phi$ on $H^*_{\times} = \mathcal{R}/\mathcal{I}$ by $\phi(c_i)=uP_i, \; \forall i\in I, \text{ and } \phi(u)=Q.$ It is sufficient to prove that $\phi$ is well defined, that is, $\mathcal{I}\subseteq \ker (\phi)$. Observe that $u^2 =0$ in $H^*_{\times}$ which implies that \begin{equation}\label{ideal cicj}
			\phi(c_i c_j) = \phi(c_i) \phi(c_j) = uP_i \cdot uP_j = u^2 P_i P_j = 0.
		\end{equation} Using \eqref{ideal cicj} and $\phi(u^2)=Q^2 =0$, we have $\mathcal{I}\subseteq \langle u^2, c_i c_j \,|\, i,j \in I \rangle \subseteq \ker(\phi).$ 
	\end{proof}
	
	
	\subsection{} In this subsection, we derive some immediate applications of \thmref{main thm}.
	\begin{corollary}
		%It is natural to consider a more general situation with several spheres instead of a single one.  
		Let us consider $X = \mathbb{S}^{2m_1} \times \cdots \times \mathbb{S}^{2m_r} \times \mathbb{C}G_{n,k}$ and denote by $u_j$ the generator of $H^{2m_j}(\mathbb{S}^{2m_j}; \mathbb{Q})$ corresponding to the fundamental class of $\mathbb S^{2m_j}$ for all $1\leq j \leq r.$ Define  
		\[
		H^*_{\mathbf{m}, \mathbb{C}G} := H^*(\mathbb{S}^{2m_1} \times \cdots \times \mathbb{S}^{2m_r} \times \mathbb{C}G_{n,k}; \mathbb{Q})
		\;\cong\; H^*_{\mathbb{C}G}[u_1,\ldots,u_r] \big/ \langle u_1^2,\ldots,u_r^2 \rangle,
		\] where $\mathbf{m} = (m_1, \ldots, m_r)$.
		%where each $u_j$ denotes the degree-$m_j$ generator of $H^*(\mathbb{S}^{m_j}; \mathbb{Q})$.  
		Suppose $\phi: H^*_{\mathbf{m}, \mathbb{C}G} \to H^*_{\mathbf{m}, \mathbb{C}G}$ is a graded endomorphism satisfying $\phi(c_1)=\lambda c_1,\, \lambda \neq 0$. Then $$\phi(c_i) = \begin{cases}
			\lambda^i c_i,   \forall i \in I& \text{ if } k<n-k,\\
			\lambda^i c_i,  \forall i \in I \quad \text{ or } \quad(-\lambda)^i (c^{-1})_i,    \forall i \in I & \text{ if } k= n-k,
		\end{cases}$$
		where $ (c^{-1})_i $ is the $ 2i $-dimensional part of the inverse of $ c = 1 + c_1 + \cdots + c_k $ in $ H^*_{ \mathbb CG} $. 
	\end{corollary}
	\begin{proof}
		The proof of this corollary is similar to the proof of part \textit{2} of \thmref{main thm}. Apply induction on $r$ and replace $\mathbb{C}G_{n,k}$ with $\hat{X} := \mathbb{S}^{2m_1}\times\cdots\times \mathbb{S}^{2m_{i-1}}\times \mathbb{S}^{2m_{i+1}}\times\cdots\times \mathbb{S}^{2m_r}\times \mathbb{C}G_{n,k},$ and the sphere $\mathbb{S}^m$ with $\mathbb{S}^{2m_i}$ in \thmref{main thm}. Since \begin{equation}\label{Sm as hom}
			\mathbb S^{2m_j}=SO(2m_j+1)/SO(2m_j)
		\end{equation}
		where the orthogonal groups $SO(2m_j+1)$ and $SO(2m_j)$ have the same rank $m_j$. Using \eqref{Sm as hom} and \eqref{cgn as hom}, $\hat{X}$ satisfies the hypothesis of \thmref{Tezuka}. Therefore, every $\mathbb{Q}$-linear derivation of $H^*(\hat{X};\mathbb{Q})$ that decreases the degree by $2m_i$ is trivial.
		%Since the $m_i$ are even, the    and $\mathbb CG_{n,k}=U(n)/U(k)\times U(n-k)$, the space $X$ can be realized as a homogeneous space $G/H$, where $G$ is a connected compact Lie group and $H$ a closed subgroup of maximal rank.  This satisfies the hypotheses of Shiga--Tezuka's Theorem~$A'$ \cite{sigha-tezuka}, which then implies that  The argument then proceeds as in Theorem~\ref{main thm}, which completes the proof.  
	\end{proof}
	Let us turn our attention to the generalized Dold spaces $P(m,n,k)$ defined in \subsecref{gds}. The following remark helps us to describe endomorphisms of $H^*(P(m,n,k);\mathbb{Q})$ induced by continuous functions on $P(m,n,k)$. These observations will be used in \secref{section 4}.
	%to understand how continuous self-maps of $P(m,n,k)$ induce homomorphisms on the cohomology ring $H^*(P(m,n,k);\mathbb{Q})$. 
	%Let us record a few useful observations.
	\begin{remark}\label{lift}
		For a continuous map $f$ on $P(m,n,k)$, we have  
		\begin{equation} \label{lift of f}
			f_*\circ \pi_*\big(\pi_1(\mathbb{S}^m\times \mathbb{C}G_{n,k})\big)
			\subseteq \pi_*\big(\pi_1(\mathbb{S}^m\times \mathbb{C}G_{n,k})\big),
		\end{equation}
		where $\pi_1(X)$ denotes the fundamental group of a topological space $X$. Hence, the composite $f\circ \pi$ admits a lift $\tilde f$ on
		$\mathbb{S}^m\times \mathbb{C}G_{n,k}$ for the double covering  
		$\pi:\mathbb{S}^m\times \mathbb{C}G_{n,k}\to P(m,n,k)$.
	\end{remark}
	Using \remref{lift}, we get the following commutative diagram,
	
	\begin{equation}\label{comm diag on H}
		\begin{tikzcd}
			{H^*(P(m,n,k);\mathbb Q)}  & {H^*(\mathbb S^m\times \mathbb CG_{n,k};\mathbb Q)} \\
			{H^*(P(m,n,k);\mathbb Q)} & {H^*(\mathbb S^m\times \mathbb CG_{n,k};\mathbb Q).}
			\arrow["{\pi^*}", from=1-1, to=1-2]
			\arrow["{f^*}"', from=1-1, to=2-1]
			%\arrow[hook, from=1-2, to=1-3]
			\arrow["{\bar f^*}", from=1-2, to=2-2]
			%\arrow["{\bar f^*}", from=1-3, to=2-3]
			\arrow["{\pi^*}", from=2-1, to=2-2]
			%\arrow[hook, from=2-2, to=2-3]
		\end{tikzcd}
	\end{equation}
	where $\pi^*$ is an injective map. Using \thmref{cohomology of P(m,n,k)} and \eqref{comm diag on H} we obtain the following two corollaries. 
	%as an immediate application of respectively. 	
	\begin{corollary}\label{cor3}
		Let $f^*$ be an endomorphism of $H^*(P(m,n,k); \mathbb{Q})$ induced by a continuous function $f$ on $P(m,n,k)$ satisfying $f^*(c_1^2) \ne 0$. Then
		$f^*$ is the restriction of a graded endomorphism $\tilde{f}^*$ on  $H^*_\times$ satisfying $\tilde{f}^*(c_1) = \lambda c_1, \lambda \neq 0$,  to the fixed subring $\mathrm{Fix}(\theta^*)$ of $H^*_\times$ where $\theta = \alpha\times \sigma$.
	\end{corollary}
	\begin{corollary}\label{cor4}
		Let $f^*$ be an endomorphism of $H^*(P(m,n,k); \mathbb{Q})$ induced by a continuous function $f$ on $P(m,n,k)$ satisfying $f^*(c_1^2) =0$ and $n>2$. Then
		$f^*$ is the restriction of a graded endomorphism $\tilde{f}^*$ on  $H^*_\times$ satisfying $\tilde{f}^*(c_1) = au, a \in \mathbb{Q}$,  to the fixed subring $\mathrm{Fix}(\theta^*)$ of $H^*_\times$ where $\theta = \alpha\times \sigma$.
	\end{corollary}
	Using \thmref{main thm} in \corref{cor3}, and \propref{main thm 2} in \corref{cor4} along with hypothesis \eqref{Homer}, we can determine $f^*$.
    %if we add the assumption that the hypothesis \eqref{Homer} is satisfied in \corref{cor4}.
	
	Moreover, there exist graded endomorphisms of $H^*(P(m,n,k))$ that are not induced by any continuous self-map of $P(m,n,k)$, and cannot be realized as restrictions of graded endomorphisms of $H^*_{\times}$. Let us see an example of such graded endomorphism.
	
	
	%Also, there exist graded endomorphisms of $H^*(P(m,n,k); \mathbb{Q})$ which are not restriction of any graded endomorphism of $H^*_\times$, if the graded endomorphism is not induced from a continuous function on $P(m,n,k)$. Let us see an example of such graded endomorphism of $H^*(P(m,n,k);\mathbb{Q})$.
	\begin{example}
		If $m$ odd, $n>2$ and $k = 1$, then $P(m,n,1)$ is fibered by the complex projective space  $\mathbb{C} P^{n-1}$ over the real projective space $\mathbb{R} P^m$. In this case,  $H^*_\times\cong \mathbb Q[u,c_1]/\langle u^2, c_1^n\rangle$ and using \eqref{defn of theta*}\ and \thmref{cohomology of P(m,n,k)}, the rational cohomology ring $$H^*(P(m,n,1); \mathbb{Q}) \cong \mathbb{Q}[u, b] / \langle u^2, b^{\lfloor (n+1)/2 \rfloor} \rangle,$$ where $u$ is a generator of $H^m(\mathbb{R}P^m;\mathbb Q)$ and $b$ restricts to $c_1^2\in H^2(\mathbb CP^{n-1};\mathbb Q)$ under the fiber inclusion. 
		%Here, $\lfloor x\rfloor$ denotes the greatest integer less than or equal to $x$.
		
		Consider the endomorphism
		\[
		\phi \colon H^*(P(m,n,1); \mathbb{Q}) \to H^*(P(m,n,1); \mathbb{Q}), \quad \text{defined by }\quad u \mapsto u,  b \mapsto -b.
		\]
		Then $\phi$ is a well-defined graded endomorphism but it cannot be a restriction of a graded endomorphism of $H^*_\times$ because any such map induces $c_1^2 \mapsto \lambda^2 c_1^2$ for some $\lambda \in \mathbb{Q}$, and $\lambda^2 \neq -1$.
	\end{example}
	
	
	\iffalse
	\begin{corollary}
		Assume that $P(m,n,k)$ be an orientable manifold and $n>2$. Let $f$ be a continuous map on $P(m,n,k)$ with nonzero Brouwer degree. Then  
		the induced map $f^*$ is an automorphism on $H^*(P(m,n,k);\mathbb Q)$.
	\end{corollary}
	\begin{proof}
		Using \remref{lift}, there exists a lift $\tilde f: \mathbb{S}^m\times \mathbb{C}G_{n,k}\to \mathbb{S}^m\times \mathbb{C}G_{n,k}$ 
		satisfying $\pi\circ \tilde f = f\circ \pi$. Using \corref{cor3} and \corref{cor4}, we have $f^*$ is the restriction of a graded endomorphism $\tilde{f}^*$ on  $H^*_\times$ satisfying either $$\tilde{f}^*(c_1) = \lambda c_1, \lambda \neq 0, \text{ if } f^*(c_1^2) \neq 0, \text{ or } \tilde f^*(c_1) =au, a\in \mathbb{Q}, \text{ if } f^*(c_1^2) =0,$$  to the fixed subring $\mathrm{Fix}(\theta^*)$ of $H^*_\times$.
		
		Let $d=k(n-k)$.  The top cohomology $H^{2d+m}_\times\cong\mathbb{Q}$ is generated by 
		$uc_1^d$.  The nonzero Brouwer degree of $f$ implies nonzero Brouwer degree of $\tilde f$ i.e. \begin{equation}\label{brouwer}
			\tilde f^*(uc_1^d)=\nu\,uc_1^d \in uH^{*}_{\mathbb{C}G}, \, \nu \neq 0.
		\end{equation}
		
		Let us consider the first case where $f^*(c_1^2) \neq 0$ then $\tilde f^*(c_1) = \lambda c_1, \lambda \neq 0$. Using \thmref{main thm}, we have $\tilde f^*(c_i) = \lambda^i c_i$. Also, $$\tilde f^*(u) = \mu u, \mu \in \mathbb{Q}, \text{ or } \tilde f^*(u)\in H^*_{\mathbb{C}G}.$$ If $\tilde f^*(u)\in H^*_{\mathbb{C}G}$, then $\tilde f^*(uc_1^d) \in H^*_{\mathbb{C}G}$
		which is a contradiction to \eqref{brouwer}. Therefore, $\tilde f^*(u) = \mu u $ where $\mu \neq 0$ because $\nu \neq 0$. So, $\tilde f^*$ is an automorphism.
		
		Let us consider the other case when $f^*(c_1^2) = 0$ then $\tilde f^*(c_1) = au, a \in \mathbb{Q}$. Since, $u^2 =0$ and $d\geq 2$ we have $$\tilde f^*(uc_1^d) = \tilde f^*(u) (\tilde f^*(c_1))^d = \tilde f^*(u) a^d u^d =0$$ which is a contradiction to \eqref{brouwer}. Hence, this case would not arise. 
	\end{proof}
	\fi
	
	The following corollary helps us to understand the relationship between the automorphisms of $H^*(P(m,n,k))$ with the automorphisms of $H^*_{\times}$.
	\begin{corollary}\label{automor}
		Let $f^*$ be an automorphism of $H^*(P(m,n,k);\mathbb{Q})$ induced by a continuous function $f$ on $P(m,n,k)$ and assume that $n> 2$. Then $\tilde{f}^*$ is an automorphism of $H^*_{\times}$, where $\tilde{f}$ is as in \remref{lift}. \\ Moreover there exist $\lambda, \mu \in \mathbb{Q}\backslash \{0\}$ such that $\tilde{f}^*(u) = \mu u$ and $\tilde{f}^*(c_i)$ is of the form given in \textit{(2)} of \thmref{main thm}.
	\end{corollary}
	\begin{proof}
		Using \remref{lift}, we have $\tilde{f}^*$ is a graded endomorphism of $H^*_{\times}$. When $n>2$, we have $c_1^2\neq 0$ in $\fix (\theta^*)\subseteq H^*_\times$, where Fix$(\theta^*)$ is the fixed subring under $\theta^*$ defined in \eqref{defn of theta*}. Since $f^*$ is an automorphism, we have $f^*(c_1^2) \neq 0$. Using \corref{cor3}, there exist $\lambda \in \mathbb{Q}$ such that $\tilde{f}^*(c_1) = \lambda c_1, \lambda \neq 0$.\\
		By \thmref{main thm}, $\tilde{f}^*(c_i)$ is of the form given in \textit{(2)} of \thmref{main thm}. Also, $$\tilde{f}^*(u) = \mu u, \, \mu \in \mathbb{Q} \quad \text{ or } \quad \tilde{f}^*(u) = Q$$ where $Q$ is a polynomial of degree $m$ in $H^*_{\mathbb{C}G}$ with $Q^2 =0$. To conclude the result, we need to prove that $\tilde{f}^*(u) = \mu u$ where $\mu \neq 0$. \\
		Suppose that $\tilde{f}^*(u) = Q$, then the image set $\im \tilde{f}^*\subseteq H^*_{\mathbb{C}G}$. Using \corref{cor3}, we get $$\im f^*  \cong \im \tilde{f^*}|_{Fix (\theta^*)}\subseteq H^*_{\mathbb{C}G}.$$ This is a contradiction to the assumption that $f^*$ is an automorphism because using \thmref{cohomology of P(m,n,k)}, either $u$ or $uc_{1}$ (depending on the parity of $m$) is in $\im f^*=\fix (\theta^*) \cong H^*(P(m,n,k);\mathbb{Q})$. Therefore, $\tilde{f}^*(u) = \mu u, \mu \in \mathbb{Q}$ and $\mu \neq 0$ because $f^*$ is an automorphism.
	\end{proof}
	\subsection{} The following theorem provides a criterion for the image of the spherical cohomology class mapped to a scaler multiple of itself under the graded endomorphism on
	$H^*(\mathbb{S}^m \times \mathbb{C}G_{n,k};\mathbb Z)$ induced from a continuous map.
	
	
	
	\begin{theorem}\label{ind from top}
		Let $f$ be a continuous map on $\mathbb S^m\times \mathbb CG_{n,k}$ such that it stabilizes a copy of Grassmannian $\{{x_0}\}\times \mathbb CG_{n,k}$ for some $x_0\in \mathbb S^m.$ Then the induced endomorphism in cohomology satisfies $f^*(u) = \mu u$ for some $\mu \in \mathbb Z$.
	\end{theorem}
	
	
	%\begin{theorem}\label{ind from top}
	%Let $x_0 \in \mathbb{S}^m$ and consider a graded endomorphism $f^*$ on $H^*_{\times}$  induced from a topological map $f$ on $\mathbb S^m\times \mathbb CG_{n,k}$ such that $f (\{x_0 \}\times\mathbb CG_{n,k})\subseteq \{x_0\}\times \mathbb{C}G_{n,k}$.  Then $f^*(u)=\mu u$ for some $\mu\in \mathbb Q.$
	%\end{theorem}
	
	
	\begin{proof}
		Let $\mathbb T^m$ be the torus $(\mathbb S^1)^m$ and
		$q:\mathbb T^m\to \mathbb S^m$ be the quotient map that collapses the complement $C$ of an open disk $D\subset \mathbb T^m$ to the point $x_0$ in $\mathbb{S}^m$.  Denote $p_i$ the $i$-th projection map on $\mathbb S^m\times \mathbb CG_{n,k}$ for $i=1,2$ and $s:\mathbb S^m\setminus\{x_0\}\to D$ is the inverse of the restriction $q|_D$.
		Since $f$ stabilizes $\{x_0\}\times \mathbb{C}G_{n,k}$, define continuous maps
		$g:\mathbb{C}G_{n,k}\to \mathbb{C}G_{n,k}$ by $(x_0,g(y)) = f(x_0,y)$ and $\tilde f:\mathbb T^m\times \mathbb{C}G_{n,k}\to \mathbb T^m\times \mathbb{C}G_{n,k}$ by
		\[
		\tilde f(x,y)=
		\begin{cases}
			\big(s\circ p_1\circ f(q(x),y),\,\,p_2\circ f(q(x),y)\big), & x\in D,\\[2pt]
			\big(x,g(y)\big), & x\in C.
		\end{cases}
		\]
		%where and $p_i$ denotes the $i$-th projection.  
		Then it is easy to check that the following diagram commutes:
		\[ \begin{tikzcd}
			{\mathbb T^m\times \mathbb{C}G_{n,k}} & {\mathbb T^m\times \mathbb{C}G_{n,k}} \\
			{\mathbb S^m\times \mathbb{C}G_{n,k}} & {\mathbb S^m\times \mathbb{C}G_{n,k}}
			\arrow["{\tilde f}", from=1-1, to=1-2]
			\arrow["{q\times \mathrm{id}}"', two heads, from=1-1, to=2-1]
			\arrow["{q\times \mathrm{id}}", two heads, from=1-2, to=2-2]
			\arrow["f", from=2-1, to=2-2]
		\end{tikzcd}
		\]
		
		Since, the quotient map $q$ has Brouwer degree 1, the induced map on rational cohomology $q^*: H^*(\mathbb{S}^m; \mathbb{Z}) \rightarrow H^*(\mathbb{T}^m;\mathbb{Z})$ sends $u\mapsto 1\cdot u_1 u_2 \dots u_m$ where $u_i$ denote the one dimensional cohomology class corresponding to the fundamental class of the $i$-th circle factor of $\mathbb T^m$ for $i\in \{1,2,\ldots,m\}$ with appropriate orientation. 
		%Let $u$ be a generator of $H^m(\mathbb S^m;\mathbb Z)$ so that $q^*(u)=u_1u_2\cdots u_m$.  
		Since $H^{\mathrm{odd}}(\mathbb{C}G_{n,k};\mathbb Z)=0$, the induced map $\tilde f^*$ sends each $u_i$ to a polynomial $P_i(u_1,\dots,u_m)$.  We slightly abuse notation by using the same symbols for the cohomology classes of $H^*(\mathbb S^m;\mathbb Z)$ and $H^*(\mathbb{C}G_{n,k};\mathbb Z)$ when viewed in $H^*(\mathbb S^m\times \mathbb CG_{n,k};\mathbb Z)$.
		The induced diagram in cohomology implies the following commutative diagram.
		\[
		\begin{tikzcd}
			{\prod_{i=1}^m u_i} & {\prod_{i=1}^m P_i(u_1,\ldots,u_m)} \\
			u & {f^*(u)}
			\arrow["{\tilde f^*}", maps to, from=1-1, to=1-2]
			\arrow["{(q\times \mathrm{id})^*}"', maps to, from=2-1, to=1-1]
			\arrow["{f^*}", maps to, from=2-1, to=2-2]
			\arrow["{(q\times \mathrm{id})^*}"', maps to, from=2-2, to=1-2]
		\end{tikzcd}
		\]
		This implies that $f^*(u)$ does not contain any nonzero element from $H^*(\mathbb{C}G_{n,k};\mathbb Z)$.  
		Thus, $f^*(u)=\mu u$ for some $\mu\in \mathbb Z$.
	\end{proof}
	%\begin{remark}
		%To remain consistent with notations, we have written the proof of \thmref{ind from top} over $\mathbb{Q}$, but the same proof also works over $\mathbb{Z}$.
	%\end{remark}
	
	
	
	
	
	
	
	
	
	
	
	
	\section{Coincidence theory of $P(m,n,k)$} \label{section 4}
	In this section, we study the \textit{coincidence theory} of generalized Dold spaces $P(m,n,k)$ defined in \subsecref{gds}. We establish the necessary conditions for a generalized Dold space $P(S,X)$ defined in \eqref{gen dold space} to satisfy the coincidence property. 
	%Our study builds upon previous results on the \emph{graded endomorphisms} of the rational cohomology ring $H^*_\times$ in section~\ref{graded endomorphism} and the \emph{rational cohomology ring of generalized Dold spaces} as established in \cite{mandal-sankaran2}.
	
	\subsection{} Let us recall certain definitions that will be required in the rest of this section.
	
	\begin{definition}
		Let $(X,g)$ be a pair, where $g$ is a continuous map on a topological space $X$. The pair $(X,g)$ is said to have the \textbf{coincidence property} (in short, CP) if, for every continuous map $f : X \to X$, there exists a point $x \in X$ such that $f(x) = g(x)$.
	\end{definition}
	
	If we consider $g$ to be the identity map on $X$, then the notion of coincidence reduces to that of a fixed point, resulting in the following definition.
	
	
	
	\begin{definition}
		A topological space $X$ is said to have \textbf{fixed-point property} (FPP) if every continuous map $f : X \to X$ admits a fixed-point; that is, there exists $x \in X$ such that $f(x) = x$.
	\end{definition}
	
	
	%Our aim is to understand the situations when two continuous maps on the generalized Dold spaces $P(S,X)$ defined in \subsecref{gen dold}, have coincidence.
	 The following proposition provides a criteria in terms of the fiber $X$ and the base space $Y := S/\!\!\sim_\alpha$, allowing one to infer the coincidence properties of the total space $P(S,X)$.
	
	%As a first step, we establish the following proposition, which provides necessary conditions for a generalized Dold space to exhibit certain coincidence properties. 
	
	
	
	
	\iffalse
	\begin{proposition}
		\  A generalized Dold space $P(S,\alpha,X,\sigma)$ does not have fixed point property if any of the following holds:\\
		(i) $Y=S/\!\!\sim _{\alpha}$ does not have the fixed point property.\\
		(ii) There exists a map $f:X\to X$ having no fixed-point and $f\circ \sigma=\sigma \circ f$.
	\end{proposition}
	\begin{proof}
		(i) Let $g:Y\to Y$ be fixed point free. Then $\phi:=s\circ g\circ p$ on $P(S,X)$ also has no fixed points, where $p$ is the $X$-bundle projection and $s$ is a section.  
		(ii).   Suppose that $f:X\to X$ has no fixed points and that $f\circ \sigma=\sigma \circ f$.
		Now define $\psi:P(S,X)\to P(S,X)$ as $[s,x]\mapsto [s,f(x)]$. The well-definedness follows because $\psi([\alpha(s),\sigma(x)])=[\alpha(s),f\circ\sigma(x)]=[\alpha(s),\sigma \circ f(x)]=[s,f(x)]=\psi ([s,x])$. Clearly, $\psi$ has no fixed points. This completes the proof.
	\end{proof}
	\fi
	
	
	%Recall that $P(S,X)$ has the fiber bundle structure $X\hookrightarrow P(SX)\twoheadrightarrow Y:=S/\!\! \sim_\alpha$, where $p$ is the $X$-bundle projection and $s$ is a section.  
	\begin{proposition}\label{necessary condition}
		Let $(P(S,X),g)$ be a pair, where $g$ is a continuous map on the generalized Dold space $P(S,X)$. Then $(P(S,X),g)$ does not have the CP if one of the following hold:
		\begin{enumerate}
			\item The continuous map $g$ is a fiber bundle map and the pair $(Y,p \circ g \circ s)$ does not have the CP, 
			where $Y = S/\!\!\sim_\alpha$ and $s$ denotes a section of the $X$-bundle projection $p$ defined in \eqref{sectio} and \eqref{proj}.
			
			\item  
			There exists a $\sigma$-equivariant map $f$ (i.e. $f\circ\sigma =\sigma\circ f$) on $X$ and a $\alpha \times \sigma$-equivariant map $\tilde g$ on $S\times X$ inducing $g$ such that
			$\mathrm{id}_S \times f$ coincides with neither $\tilde{g}$ nor
			$(\alpha \times \sigma) \circ \tilde{g}$.
		\end{enumerate}
		
		
		
	\end{proposition}
	\begin{proof} \textit{(1)}
		Suppose that the pair $(Y,p \circ g \circ s)$ does not have the CP. 
		Then there exists a continuous map $f : Y \to Y$ such that \begin{equation} \label{pogos}
			f(x) \neq p \circ g \circ s(x), \, \forall x \in Y.
		\end{equation}
		We are given that $g$ is a fiber bundle map, which implies that there exist  $g_1:Y\to Y$, satisfying  $p \circ g = g_1 \circ p.$
		Consider $p\circ g\circ s=g_1\circ p\circ s=g_1 $.
		Thus, $p \circ g = g_1 \circ p$ implies
		\[
		p \circ g(x) = p \circ g \circ s \circ p(x),\, \forall x \in P(S, X).
		\]
		Define the map $\phi := s \circ f \circ p$ on $P(S, X)$. We claim that $\phi(y)\neq g(y), \, \forall y \in P(S,X)$. \\ Suppose there exist $y \in P(S, X)$ such that $\phi(y) = g(y)$, then 
		\[
		p \circ g \circ s(p(y)) = p \circ g(y)
		= p \circ s \circ f \circ p (y) 
		= f(p(y)),
		\]
		which contradicts \eqref{pogos}.
		
		
		\textit{(2)} %Let $G$ be a group of order $2$ generated by $\alpha \times \sigma$ acting on the topological space $S\times X$ by composition.
		Let $G$ denote the group of deck transformations of the double covering $\pi : S \times X \to P(S, X)$, generated by the free involution $\alpha \times \sigma.$
		The proof then follows from a general observation that if for two $G$-equivariant maps $\tilde\phi, \tilde\psi$ on $S\times X$, the maps $\tilde \phi$ and $t \cdot \tilde\psi$ have no point of coincidence, for any $t \in G$; then the maps they induce on the orbit space $P(S,X)$, namely $\phi, \psi$, are also coincidence-free.
		%$id_{S}\times f$ coincides with neither $\tilde g$ nor $(\alpha \times \sigma)\circ \tilde g$ then the two $G$-equivariant maps $\tilde\phi, \tilde\psi: M \to M$ on a topological space $M$, the maps $\tilde \phi$ and $t \cdot \tilde\psi$ have no point of coincidence, for any $t \in G$; then the maps they induce on the orbit space $M/G$, namely $\phi, \psi: M/G \to M/G$, are coincidence-free.
	\end{proof}
	
	\iffalse
	\begin{proposition}
		
		
		Let $g : P(S,X) \to P(S,X)$ be a map induced by an $\alpha \times \sigma$-equivariant map $\tilde{g} : S \times X \to S \times X$. 
		Suppose there exists a $\sigma$-equivariant map $f : X \to X$ such that both the pairs $id_S\times f, \tilde g$ and $id_S\times f,\alpha\times\sigma\circ \tilde g$ dont have any point of coincidence.
		Then the space $P(S,X)$ does not have the $g$-coincidence property.
	\end{proposition}
	\begin{proof}
		Define $\phi : P(S,X) \to P(S,X)$ by $\phi([s,x]) = [s, f(x)]$. 
		This map is well defined because
		$\phi([\alpha(s), \sigma(x)]) = [\alpha(s), f(\sigma(x))] = [\alpha(s), \sigma(f(x))] = [s, f(x)]$, using the fact that $f\circ \sigma=\sigma\circ f$.
		
		Similarly, define $\psi : P(S,X) \to P(S,X)$ by $\psi([s,x]) = [g_1(s), g_2(x)]$, 
		where $g_j = p_j \circ \tilde{g} \circ i_j$, and $p_j$ and $i_j$ denote the projection and inclusion maps on the $j$-th factor ($j=1,2$). 
		The map $\psi$ is well defined since $\tilde{g}$ is $\alpha \times \sigma$-equivariant.
		
		Since, the maps $f$ and $p_2 \circ \tilde{g} \circ i_2 = g_2$ have no coincidence points. 
		Hence, for any $[s,x] \in P(S,X)$, the second coordinates in $\phi([s,x])$ and $\psi([s,x])$ are different. 
		Therefore, $\phi$ and $\psi$ have no points of coincidence, and $P(S,X)$ does not possess the $g$-CP.
	\end{proof}
	\fi
	
		In particular if we take $g$ to be the identity map in \propref{necessary condition}, we recover Proposition~7.2.1 of \cite{mandal}, which proves that if the base $Y$ does not have the FPP, or there exist a $\sigma$-equivarinat map on the fibre $X$ with no fixed point then $P(S,X)$ does not have the FPP. As a consequence, $P(m,n)$ does not have the FPP if either $m$ or $n$ is odd.
	\iffalse
	\begin{remark}\label{necessary criterion for FFP}
		A generalized Dold space $P(S,X)$ does not have the FPP if any of the following holds:
		
		\begin{enumerate}
			\item $Y=S/\!\!\sim _{\alpha}$ does not have the FPP.
			\item There exists a $\sigma$-equivariant map $f$ on $X$ that has no fixed point.
		\end{enumerate}
	\end{remark}
	
	
	
	
	
	
	
	
	
	
	We have the following observation as an immediate consequence of Remark \ref{necessary criterion for FFP}.
	
	\begin{remark}
		Let $P(m,n)$ be a classical Dold manifold such that either $m$ or $n$ is odd. Then  
		%and the existence of a fixed point free map on $\mathbb CP^n$ which commutes with the conjugation, 
		$P(m,n)$ does not have the FPP.
	\end{remark}
	\begin{proof}
		When $m$ is odd, the proof follows from part \textit{(1)} of \remref{necessary criterion for FFP}. \\ When $n$ is odd, we have a continuous $\sigma$-equivariant map  $f:\mathbb CP^n\to \mathbb CP^n$ defined by 
		\[
		[x_1:x_2:\cdots:x_n:x_{n+1}]\mapsto [ x_2:- x_1:\cdots: x_{n+1}:- x_{n}]
		\] which does not have any fixed point.
		%Clearly, $f\circ \sigma=\sigma\circ f$, where  
		Thus, using part \textit{(2)} of \remref{necessary criterion for FFP}, $P(m,n)$ does not have fixed-point property.
	\end{proof}
	%\textcolor{red}{Converse of the above remark seems to be true.}
	\fi
	
	
	\iffalse
	We need the following theorems in the sequal.
	\begin{theorem}[Lefschetz Fixed-Point Theorem]\label{Lefschetz theorem}
		Let $X$ be a finite simplicial complex, and let $f : X \to X$ be a continuous map.  
		Define the number
		\[
		\tau(f) = \sum_{n} (-1)^n \operatorname{tr}\!\big(f_* : H_n(X) \to H_n(X)\big),
		\]
		called the \emph{Lefschetz number} of $f$.  
		If $\tau(f) \ne 0$, then $f$ has at least one fixed point.
	\end{theorem}
	\fi
	
	
	%\begin{theorem}[\cite{glover-homer}, Theorem 2]\label{GH2} For $k,n$ as in Theorem \ref{GH1}, the complex Grassmannian $\mathbb CG_{n,k}$ has the fixed-point property if and only if $k(n-k)$ is even. \end{theorem}
	
	
	\iffalse
	In (Theorem 2, \cite{glover-homer}), it is proved that $\mathbb CG_{n,k}$ has fixed-point property if and only if $k(n-k)$ is even, provided either (i) $k\leq 3$ and $n>2k$ or (ii) $k>3$ and $n> 2k^2-1$.
	It is well known that there exists a fixed-point free map   $ L \mapsto L^\perp   $ on   $ \mathbb{C}G_{n,k}   $ when   $ k = n - k   $.
	
	The following remark follows easily from Theorem \ref{hoffman} \cite{hoffman} and Theorem \ref{GH2} \cite{glover-homer}.
	\begin{remark}Let $f$ be continuous self-map on $\mathbb CG_{n,k}$ such that
		$ f^*(c_1) = \lambda c_1 $, $ \lambda\neq 0 $. Assume that $k(n-k)$ is even and $k\neq n-k$. Then $f$ has a fixed point. 
	\end{remark}
	\begin{proof}
		By Theorem~\ref{hoffman} \cite{hoffman}, the map $f^*$ is an Adams map. Using the computation of the Lefschetz number $L(f)$ from the proof of Theorem~\ref{GH1}, we observe that $L(f) \ne 0$ because $k(n-k)$ is even. Thus, $f$ must have a fixed point. This completes the proof.
	\end{proof}
	\fi
	
	\subsection{} Let us recall a well known result in coincidence theory, the Lefschetz Coincidence Theorem, which will be used to prove results in the rest of this paper.\\
	
	For a closed oriented manifold $M$ of dimension $n$, let $[M]\in H^n(M;\mathbb Q)$ denote a chosen fundamental class. Then we have the Poincar\'e  duality isomorphism $D_M:H^k(M;\mathbb Q)\to H_{n-k}(M;\mathbb Q)$, defined by 
	\begin{equation}\label{poinc dua}
		D_M(\alpha)=[M]\frown \alpha, \, \forall \alpha\in H^k(M;\mathbb Z).
	\end{equation}

	
	\begin{theorem}[Lefschetz Coincidence Theorem]\label{LCT}
		Let $f,g$ be two continuous maps on a compact, connected, oriented manifold $M$ of  dimension $n$. The Lefschetz coincidence number is defined as 
		$$L(f,g):= \sum_{i=0}^{n} (-1)^i \mathrm{tr}\big(D_M \circ g^* \circ D_M^{-1} \circ f_* \; : \; H_i(M;\mathbb{Q}) \longrightarrow H_i(M;\mathbb{Q})\big).
		$$ If $L(f,g) \neq 0$, then there exists $x \in M$ such that $f(x) = g(x)$.
	\end{theorem}
	When  $g=\operatorname{id}_M$, the theorem reduces to the Lefschetz Fixed-Point Theorem for $M$.\\
	
	To study the coincidence theory of generalized Dold spaces fibred by complex Grassmannians over real projective spaces, it is helpful to first understand the coincidence theory of complex Grassmannians, a topic of independent interest. We now prove the following lemma (cf. Theorem 2, \cite{glover-homer}) to prove \propref{CP of CGnk}.
	\begin{lemma}\label{sum neq 0}
		Let $d_{2i}$ be the $2i$-th Betti number of a complex Grassmannian $\mathbb CG_{n,k}$ with $d=k(n-k)$ even. Then the sum \(\sum _{i=0}^d d_{2i}\lambda^i\neq 0,\, \forall \lambda \in \mathbb Q.\)
	\end{lemma}
	\begin{proof}
		Let us consider the sum $\sum_{i=0}^d d_{2i}\lambda^i$, when $\lambda$ is an integer. 
		Clearly, $$\sum_{i=0}^d d_{2i}\lambda^i \equiv 1 \pmod{\lambda}.$$ %(see Theorem 2 in \cite{glover-homer}). 
		Hence, $\sum_{i=0}^d d_{2i}\lambda^i \neq 0$, if $\lambda \neq \pm 1$. When $\lambda = 1$, the sum is also positive and therefore nonzero. It remains to consider the case where $\lambda = -1$.
		Let $\chi(\mathbb{R}G_{n,k})$ denote the Euler-Poincar{\'e} characteristic of $\mathbb{R}G_{n,k}$ and be defined by $$\chi(X):=\sum_{i\ge0}\dim H^i(\mathbb{R}G_{n,k};\mathbb{Z}_2)$$ where $\mathbb{R}G_{n,k}$ denotes the Grassmannian of real $k$-planes in $\mathbb{R}^n$.
		Now we observe that 
		$\sum_{i=0}^d d_{2i}(-1)^i = \chi(\mathbb{R}G_{n,k})$ where $d_{2i} = \operatorname{dim} H^{2i}(\mathbb{C}G_{n,k}; \mathbb{Q}) = \dim H^{i}(\mathbb{R}G_{n,k}; \mathbb{Z}_2)$. 
		It is a well known fact that $\chi(\mathbb{R}G_{n,k}) \neq 0$ if $k(n-k)$ is even. 
		
		Let us move to the other case where $\lambda\in \mathbb{Q}\backslash \mathbb{Z}$. Suppose $\sum_{i=0}^{d} d_{2i}\lambda^i =0$ for some $\lambda = \frac{p}{q}$ where $p$ and $q$ are coprime integers. Since $d_0 =d_d =1$, using the rational root theorem $p|1$ and $q|1$. Hence, $\lambda = \pm 1$, which is a contradiction. Therefore, we conclude that $\sum_{i=0}^{d} d_{2i}\lambda^i \neq 0$ for all $\lambda\in \mathbb Q$. 
		%Thus, the sum  $\sum _{i=1}^d d_{2i}\lambda^i$ is nonzero for every integer value of $\lambda.$ 
		\iffalse	Now, suppose for contradiction that there exists a rational number $\lambda = a/b$  written in lowest terms (i.e., $\gcd(a,b)=1$) such that the sum vanishes. Note that in the polynomial $\sum_{i=0}^{d} d_{2i}\lambda^i$, both the leading coefficient $d_d$ and the constant term $d_0$ are equal to $1$. By the Rational Root Theorem, the rational root $\lambda = a/b$ must satisfy $a \mid 1$ and $b \mid 1$, hence $\lambda = \pm 1\in \mathbb Z$. Since this is not possible, \fi
	\end{proof}
	
	
	
	Denote the $i$-th homology groups 
	$H_i(\mathbb{C}G_{n,k}; \mathbb{Q}), \, H_i(\mathbb{S}^m; \mathbb{Q})$ and 
	$H_i(\mathbb{S}^m \times \mathbb{C}G_{n,k}; \mathbb{Q})$, by $H_i^{\mathbb{C}G}, H_i^{\mathbb{S}}$ and $H_i^{\times}$, respectively. Let $d$ denote the complex dimension of $\mathbb CG_{n,k}$, given by $d = k(n - k)$. Then we have the following proposition.
	\begin{proposition}\label{CP of CGnk}
		Consider a complex Grassmannian $\mathbb{C}G_{n,k}$ such that the hypothesis \eqref{Homer} is satisfied and $k(n-k)$ is even. Let $g $ be a continuous map on $\mathbb{C}G_{n,k}$ with nonzero Brouwer degree. Then the pair $(\mathbb{C}G_{n,k}, g)$ has the coincidence property.
	\end{proposition}
	
	
	\begin{proof}
	Self-maps with nonzero Brouwer degree induces automorphisms in the rational cohomology algebra. Using Theorem \ref{hom and hof} part \textit{(i)}, there exist a nonzero rational $\lambda$ such that $g^*(c_i)=\lambda^ic_i, \forall i\in I.$ Let $f$ be a continuous map on $\mathbb CG_{n,k}$ and using \thmref{hom and hof} part \textit{(i)}, there exists $\mu \in \mathbb Q$ such that
		\[
		f^*(c_i)=\mu^ic_i, \forall i\in I.
		\]
		Then by the Universal Coefficient Theorem, $ \hom_{\mathbb{Q}} (H_i^{\mathbb CG};\mathbb Q) \cong H^i_{\mathbb CG}$ non-canonically which implies that 
		\begin{align*}
			\varphi \circ f_* &= f^*(\varphi) , \, \forall
			\varphi \in \hom _{\mathbb Q}\big(H_{2i}^{\mathbb CG}, \mathbb Q\big)\cong H^{2i}_{\mathbb CG}.\\
			\varphi(f_*(x))&=(f^*(\varphi))(x)= \mu^i\varphi(x)=\varphi(\mu^ix),\; \forall x\in H_{2i}^{\mathbb CG}.
		\end{align*}
		The last equation implies that $f_*(x)=\mu^ix,\, \forall x\in  H_{2i}^{\mathbb CG}.$
		%This implies for all $\varphi\in \hom_{\mathbb Q}(H_{2i}^{\mathbb CG},\mathbb Q)$ and $x\in H_{2i}^{\mathbb CG}$, we have
		Now observe that $D\circ g^*\circ D^{-1}\circ f_*: H_{2i}^{\mathbb CG}\to H_{2i}^{\mathbb CG}$ is given by 
		\[
		D\circ g^*\circ D^{-1}\circ f_*(x)=D\circ g^*\circ D^{-1}(\mu^ix)=\mu^iD\circ g^*(D^{-1}x)=\mu^iD(\lambda^{d-i}D^{-1}x)=\mu^i\lambda^{d-i}x.
		\]
		Thus for $x\in H_{2i}^{\mathbb{C}G}$, the Lefschetz coincidence number is given by
		\[
		\begin{array}{ll}
			L(f,g) &= \sum_{i=0}^d (-1)^{2i}
			\mathrm{tr}(D\circ g^*\circ D^{-1}\circ f_*(x)) \\[6pt]
			&=  \sum_{i=0}^d d_{2i}\mu^i\lambda^{d-i}  \\[6pt]
			&= \lambda^d \sum_{i=0}^d d_{2i}(\mu/\lambda)^i \neq 0 \quad (\because \lambda\neq 0) 
		\end{array}
		\]
		where $d_{2i}$ denotes $\dim_{\mathbb Q}H^{2i}_{\mathbb CG}$ and the last equation holds by using \lemref{sum neq 0}. Therefore, using \thmref{LCT} the pair $(\mathbb{C}G_{n,k},g)$ has the coincidence property.
	\end{proof}
	
	\iffalse{}  \textcolor{teal}{
		In the case of a quaternionic Grassmannian $\mathbb H G_{n,k}$, an Adams operation of degree $-1$ cannot be induced by any self-map, since its action on $H^*(\mathbb H G_{n,k};\mathbb Z/3)$ does not commute with the reduced third power operation (see Theorem~2(2) of \cite{glover-homer}). Consequently, the value $\lambda=-1$ cannot occur in the sum $\sum_{i=0}^{d} d_{2i}\lambda^i$ from Lemma~\ref{sum neq 0}, so the sum is nonzero in all cases, regardless of the parity of $d=k(n-k)$. Proceeding as in Proposition~\ref{CP of CGnk}, one obtains the following extension of Theorem~2(2) in \cite{glover-homer}:
		\begin{proposition}\label{CP of HGnk}
			For any homotopy equivalence $g$ of a quaternionic Grassmannian $\mathbb H G_{n,k}$, the pair $(\mathbb H G_{n,k},g)$ has the coincidence property.
		\end{proposition}
	}
	\fi
	
	
	
	
	\subsection{} Denote by $H_*^\times = \bigoplus_{i\geq 0} H_i^{\times},\, H_*^{\mathbb{C}G} = \bigoplus_{i\geq 0} H_i^{\mathbb{C}G},\, H_*^{\mathbb{S}} = \bigoplus_{i\geq 0} H_i^{\mathbb{S}}$ and $\vartheta$ the fundamental class $[\mathbb S^m]\in H_m^{\mathbb S}$. Let $\{v_q\}$ be a homogeneous basis of $H_*^{\mathbb{C}G}$, and let $\{\delta_{v_q}\}$ denote the corresponding dual basis of  
	$\operatorname{Hom}(H_*^{\mathbb{C}G}, \mathbb Q) \cong H^*_{\mathbb{C}G}$, such that  
	$ \delta_{v_q} (v_p) = \delta_{qp}$ where $\delta_{qp}$ is the Kronecker delta function. Without loss of generality, assume that $1=v_0 \in \{v_i\}$ represents the generator of $H_0^{\mathbb{C}G} \cong \mathbb Q$.
	
	Over $\mathbb Q$, the K\"unneth Theorem yields the following decompositions 
	\iffalse	Set $d = 2k(n - k)$, $m=2s$ and $t={n \choose k}$. 
	
	By Remark~\ref{lift}, every continuous map $f$ on $P(m,n,k)$ admits a lift 
	$\tilde f$ on $\mathbb{S}^m \times \mathbb{C}G_{n,k}$ satisfying 
	$f \circ \pi = \pi \circ \tilde f,$
	where $\pi : \mathbb{S}^m \times \mathbb{C}G_{n,k} \to P(m,n,k)$ is the double covering map. 
	We continue to denote such lifts by the corresponding maps on $P(m,n,k)$ with a tilde.
	By Theorem~\ref{main thm}, there exists  $\lambda \in \mathbb{Q}$ or $P_i\in H^{2i-m}_{\mathbb CG} \;\forall i\in I$ with some $P_i\neq 0$, such that \
	
	\begin{equation}
		\text{either (i) } 
		\tilde f^*(c_i) = \lambda^i c_i,\forall i\in I, \text{or (ii) } \tilde f^*(c_i)=uP_i, \  \forall i\in I , 
	\end{equation}
	\begin{equation}
		\text{and either (iii) }\tilde f^*(u) = \mu u \text{ for some }\mu\in \mathbb Q, \text{or (iv) }\tilde f^*(u) \in H^m_{\mathbb{C}G}.
	\end{equation}
	
	
	%We have the induced maps in homology and cohomology $\tilde f_*$ and $\tilde f^*$, respectively, related by the Kronecker pairing 
	%$\langle \;,\;\rangle:H^i_\times\times H_i^\times\to \mathbb Q$ defined by
	%\[
	%\langle \tilde f^* \varphi, x\rangle = \langle \varphi, \tilde f_* x\rangle 
	%\quad \text{for all } \varphi \in H^i_\times,\, x \in H_i^\times.
	%\]
	%and $u \in H^*_\times$ corresponds to the fundamental class 
	%of $[\mathbb S^m] \in H^{\mathbb S}_m \subseteq H^\times_m.$
	
	%Fix a basis $\{1, v_1, v_2, \ldots, v_t\}$ for $H_*^{\mathbb{C}G}$. 
	%The corresponding dual basis of $\hom(H_*^{\mathbb{C}G
		%}, \mathbb{Q}) \cong H^*_{\mathbb{C}G}$ is 
	%$\{\delta_1, \delta_{v_1}, \delta_{v_2}, \ldots, \delta_{v_t}\}$, 
	%where \ $\delta_{v_i}(x) = \langle \delta_{v_i}, x \rangle$ defined to be $1$ if and only if $x=v_i$, and $0$ otherwise, for all $v_i$.
	\fi  
	\begin{equation}\label{kunneth}
		H_i^\times \cong H_i^{\mathbb{C}G} \oplus (\vartheta \otimes H_{i-m}^{\mathbb{C}G}), 
		\qquad 
		H^i_\times \cong H^i_{\mathbb{C}G} \oplus uH^{i-m}_{\mathbb{C}G},
	\end{equation}
	where  $u \in H^m_{\times} \cong \operatorname{Hom}(H_m^{\times}, \mathbb Q)$ corresponds to the element $\delta_{\vartheta\otimes 1}$.
	
	
	Using \eqref{kunneth}, we can extend the chosen basis $\{v_q\}$ of $H_*^{\mathbb{C}G}$ to $	\{v_q\} \cup \{\vartheta \otimes v_q\}$ of $H_*^\times$ such that the corresponding dual basis can also be extended from $\{\delta_{v_q}\}$ of $ \hom (H_*^{\mathbb{C}G};\mathbb{Q})$ to $\{\delta_{v_q}\} \cup \{\delta_{\vartheta \otimes v_q}\} $ of $\hom (H_*^{\times};\mathbb{Q})$ satisfying:
	%\[\{v_i\} \cup \{\vartheta \otimes v_i\} \subseteq H_*^\times, \qquad 	\{\delta_{v_i}\} \cup \{\delta_{\vartheta \otimes v_i}\} \subseteq H^*_\times.\]
	%Since $\delta_{U \otimes 1} = \delta_U = u \in H^m_{\mathbb S}$, we write $\delta_{U \otimes v_i}$ simply as $u\,\delta_{v_i}$.With respect to these bases, the Kronecker pairing  $\langle\,\cdot\,,\cdot\,\rangle : H^*_\times \times H_*^\times \longrightarrow \mathbb Q$satisfies 
	\begin{equation}\label{Kronecker relations}
		\delta_{v_q}( v_p) = \delta_{qp}, \quad
		\delta_{v_q}( \vartheta \otimes v_p)= 0, \quad
		\delta_{\vartheta\otimes v_q}(v_p) = 0, \quad
		\delta_{\vartheta\otimes v_q}(\vartheta \otimes v_p)= \delta_{qp}.
	\end{equation}
	%since $\langle u, \vartheta\rangle=1$ and $u$ (resp. $\vartheta$) kills the other basis elements.Thus the matrix of the pairing in these bases is the identity, hence the  map$\kappa_i: H^i_\times \to \operatorname{Hom}(H_i^\times,\mathbb Q)$, $\kappa_i(\varphi)(x)=\langle \varphi, x\rangle$, is an isomorphism for all $i$. 
	Let $f$ be a continuous function on $P(m,n,k)$. Using \remref{lift} and the Universal Coefficient Theorem, there exist a lift $\tilde{f}$ on $\mathbb{S}^m\times \mathbb{C}G_{n,k}$ satisfying \begin{equation}\label{comm with phi}
		\varphi \circ \tilde{f}_* = \tilde{f}^*(\varphi) , \, \forall
		\varphi \in \hom _{\mathbb Q}\big(H_{2i}^{\mathbb CG}, \mathbb Q\big)\cong H^{2i}_{\mathbb CG}.
	\end{equation}
	
		Poincar\'e duality on $\mathbb S^{m}\times \mathbb C G_{n,k}$ can be described in terms of the duality on the Grassmannian factor. Let 
		$D_{\mathbb C G}\colon H^{i}_{\mathbb C G}\to H_{2d-i}^{\mathbb C G}$ 
		be the Poincar\'e duality isomorphism defined in \eqref{poinc dua} for $\mathbb C G_{n,k}$, where $d=k(n-k)$.  
		The Poincar\'e duality isomorphism on the product
		is then determined on the basis elements by  
		\begin{equation}
			D\colon H^{j}_{\times}\to H_{m+2d-j}^{\times}, \quad \delta_{v_i}\mapsto \vartheta\otimes D_{\mathbb C G}(\delta_{v_i})\quad\text{and}\quad \delta_{\vartheta\otimes v_i} \mapsto D_{\mathbb C G}(\delta_{v_i}).
		\end{equation}
		We are now ready to establish the following lemmas, which will be useful in the sequel.
	\begin{lemma}\label{image of homf}
		Let $f$ be a continuous function on $P(m,n,k)$ and $\tilde{f}$ be the lift defined in \remref{lift} such that $\tilde{f}^*(c_1)\neq au,\, a\in \mathbb{Q}$ and $k<n-k$. Then there exist $\lambda \in \mathbb{Q}\backslash \{0\}$ and $\mu \in \mathbb{Q}$ such that the induced map $\tilde{f}_*$ on $H_*^{\times}$ is of the following form.
		\begin{enumerate}
			\item Either $\tilde{f}_*(\vartheta\otimes x) = \mu \lambda^i (\vartheta \otimes x),\, \forall x\in H_{2i}^{\mathbb{C}G}$ or $\tilde{f}_*(\vartheta\otimes x) \in H_*^{\mathbb{CG}},\, \forall x \in H_*^{\mathbb{C}G}$.
			\item $\tilde{f}_*(x) = \lambda^i x + \vartheta\otimes y,$ for some $y \in H_{2i-m}^{\mathbb{C}G}, \, \forall x \in H_{2i}^{\mathbb{C}G}.$
		\end{enumerate}
		Moreover, $y=0$ in \textit{(2)} if $\tilde{f}_*(\vartheta\otimes x) = \mu \lambda^i (\vartheta \otimes x),\, \forall x\in H_{2i}^{\mathbb{C}G}$.
	\end{lemma}
	\begin{proof}
		Using \thmref{main thm}, there exist $\lambda \in \mathbb{Q}\backslash \{0\}$ such that $\tilde{f}^*(c_i) = \lambda^i c_i, \forall i \in I$ and either $\tilde{f}^*(u) = \mu u,\, \mu \in \mathbb{Q}$ or $\tilde{f}^*(u)\in H^*_{\mathbb{C}G}$. It is sufficient to prove the result for the chosen basis $\{v_q\}\cup\{\vartheta\otimes v_q\}$ of $H_{*}^{\times}$. 
		
		Let us consider the first case where $\tilde{f}^*(u) = \mu u$. Using $H^*_{\mathbb{C}G}\cong \hom(H_*^{\mathbb{C}G},\mathbb{Q})$, we have \begin{equation}\label{fstarco}
			\tilde{f}^*(\delta_{v_p}) = \lambda^i \delta_{v_{p}}, \, \forall v_{p}\in H_{2i}^{\mathbb{C}G}, \quad \tilde{f}^*(\delta_{\vartheta\otimes v_{p}})  = \mu \lambda^i (\delta_{\vartheta \otimes v_{p}}), \, \forall v_{p}\in H_{2i}^{\mathbb{C}G}.
		\end{equation}
		If $m$ is odd, then the coefficient of any basis element $v_p \in H_*^{\mathbb{C}G}$ in $\tilde{f}_*(\vartheta \otimes v_q)$ and $\vartheta \otimes v_p$ in $\tilde{f}_*(v_q)$ is zero because $\tilde{f}_*$ is a graded map. Let us consider the case where $m=2s$.
		By \eqref{comm with phi} and \eqref{fstarco}, the coefficient of a basis element $v_p\in H_{2i+m}^{\mathbb{C}G}$ in $\tilde{f}_*(\vartheta \otimes v_q)$ written as a $\mathbb{Q}$-linear combination of the basis elements from $\{v_q\}\cup\{\vartheta\otimes v_q\}$ is the following:
		$$	 \delta_{v_p}\circ \tilde f_*(\vartheta\otimes v_q)
		=  \tilde f^* (\delta_{v_p}) (\vartheta\otimes v_q)
		=  \lambda^{i+s} \delta_{v_p}(\vartheta\otimes v_q)
		= 0,\, \forall v_q \in H_{2i}^{\mathbb{C}G}$$
		and the coefficient of a basis element $\vartheta \otimes v_p\in \vartheta \otimes H_{2i}^{\mathbb{C}G}$ in $\tilde{f}_*(\vartheta \otimes v_q)$ is
		$$ \delta_{\vartheta\otimes v_p}\circ \tilde f_*(\vartheta\otimes v_q)
		=  \tilde f^*(\delta_{\vartheta\otimes v_p})(\vartheta\otimes v_q)
		=  \mu \lambda^{i}\;\delta_{\vartheta \otimes v_p}(\vartheta\otimes v_q)= \mu\lambda^{i}\delta_{pq},\, \forall v_q \in H^{\mathbb{C}G}_{2i}.$$
		This implies that $$\tilde{f}_*(\vartheta \otimes v_q) = \mu \lambda^i (\vartheta \otimes v_q), \, \forall v_q \in H_{2i}^{\mathbb{C}G}.$$ 
		Using similar calculations given above, it is easy to show that $$\delta_{v_p}\circ \tilde{f}_*(v_q) = \lambda^i \delta_{pq},\, \forall v_q \in H_{2i}^{\mathbb{C}G},\quad \delta_{\vartheta \otimes v_p}\circ \tilde{f}_*(v_q)=0,\, \forall v_q\in H_{2i}^{\mathbb{C}G}.$$ Therefore, $\tilde{f}_*(v_q) = \lambda^i v_q,\, \forall v_q \in H_{2i}^{\mathbb{C}G}$.\\
		
		If $\tilde{f}^*(u)\in H^*_{\mathbb{C}G}$. Again using $H^*_{\mathbb{C}G}\cong \hom(H_*^{\mathbb{C}G},\mathbb{Q})$, we have \begin{equation}\label{second u}
			\tilde{f}^*(\delta_{v_p}) = \lambda^i \delta_{v_{p}}, \, \forall v_{p}\in H_{2i}^{\mathbb{C}G}, \quad \tilde{f}^*(\delta_{\vartheta\otimes v_{p}})  \in H^*_{\mathbb{C}G},\, \forall  v_{p}\in H_{2i}^{\mathbb{C}G}.
		\end{equation}
		By \eqref{comm with phi} and \eqref{second u}, we get $\delta_{v_p}\circ \tilde{f}_*(v_q) = \lambda^i \delta_{pq},\, \forall v_q \in H_{2i}^{\mathbb{C}G},$ which implies that $\tilde{f}_*(x) = \lambda^i x + \vartheta\otimes y,$ for some $y \in H_{2i-m}^{\mathbb{C}G}, \, \forall x \in H_{2i}^{\mathbb{C}G}.$ Note that $\tilde f^*(\delta_{\vartheta\otimes v_p})\in H_{\mathbb CG}^*$ and equal to some $\sum a_j\delta_{v_j}$. Then 
		%coefficient of $\vartheta \otimes v_p $ in $\tilde f_*(\vartheta \otimes v_q)$ is 
		\begin{equation}\label{computation}
			\delta_{\vartheta\otimes v_p}\circ \tilde f_*(\vartheta\otimes v_q)= \tilde f^*(\delta_{\vartheta\otimes v_p})(\vartheta\otimes v_q)=\sum a_j\delta_{v_j}(\vartheta\otimes v_q)=0.
		\end{equation}
		Hence, $\tilde f_*(\vartheta \otimes v_q)\in H_*^{\mathbb CG}$ for all $\vartheta \otimes v_q\in\vartheta\otimes H_*^{\mathbb CG} $.
		%	$ \text{ or } \tilde{f}^*(\vartheta\otimes v_{p}) \in H_*^{\mathbb{C}G},\, \forall v_{p}\in H_*^{\mathbb{C}G}$
	\end{proof}
\iffalse	\begin{remark}\label{lemrem}
		For the case $k=n-k$ in \lemref{image of homf}, if $\tilde{f}^*(c_i) = \lambda^i c_i$ then we will get the same result otherwise when $\tilde{f}^*(c_i) = (-\lambda)^i c_i$ then $\lambda$ will be replaced by $-\lambda$ in \lemref{image of homf}. 
	\end{remark}\fi
	\begin{lemma}\label{image of homf under hom}
		Assume that the hypothesis \eqref{Homer} is satisfied. 	Let $f$ be a continuous function on $P(m,n,k)$ and $\tilde{f}$ be the lift defined in \remref{lift} such that $\tilde{f}^*(c_1)= au,\, a\in \mathbb{Q}$. Then the induced map $\tilde{f}_*$ on $H_{*}^{\times}$ is of the following form.
		\begin{enumerate}
			\item $\tilde f_* (x)\in \vartheta\otimes H_{2i-m}^{\mathbb CG},\, \forall x\in H_{2i}^{\mathbb CG},\, \forall i>0$.
			\item $\tilde{f}_*(\vartheta\otimes 1) = \mu (\vartheta\otimes 1 )+y, \, y\in H_{m}^{\mathbb{C}G}, \quad\\ \tilde f_*(\vartheta \otimes x)\in H_{2i+m}^{\mathbb{C}G},\, \forall x \in H_{2i}^{\mathbb{C}G}, i>0\;$ if $\tilde{f}^*(u) = \mu u,\,  \mu \in \mathbb{Q}$ 
		\end{enumerate}
	\end{lemma}
	\begin{proof}
		Using \propref{main thm 2}, we have $\tilde{f}^*(c_i) = u P_i,$ for some $P_i\in H^*_{\mathbb{C}G}$ and either $\tilde{f}^*(u) = \mu u,\, \mu \in \mathbb{Q}$ or $\tilde{f}^*(u)\in H^*_{\mathbb{C}G}$. \\
		Let us consider the first case where $\tilde{f}^*(u) = \mu u$. Using $H^*_{\mathbb{C}G}\cong \hom(H_*^{\mathbb{C}G},\mathbb{Q})$, we have for $i>0$ 
		\begin{equation}\label{fstarcom}
			\tilde{f}^*(\delta_{v_p}) = \sum a_{jp}\delta_{\vartheta \otimes v_j} , \, \forall v_{p}\in H_{2i}^{\mathbb{C}G}, \quad \tilde{f}^*(\delta_{\vartheta\otimes 1})  =\mu \delta_{\vartheta \otimes 1}, \quad \tilde{f}^*(\delta_{\vartheta\otimes v_{p}})  = 0, \, \forall v_{p}\in H_{2i}^{\mathbb{C}G}.
		\end{equation}
		Using \eqref{comm with phi}, \eqref{fstarcom} and similar calculations given in the proof of \lemref{image of homf}, we have for $ v_p\neq 1$
		$$\delta_{v_p}\circ \tilde{f}_*(v_q) = 0,\, \forall v_q \in H_{2i}^{\mathbb{C}G},\quad \delta_{\vartheta\otimes v_p}\circ \tilde{f}_*(\vartheta\otimes v_q) =0,\, \forall v_q \in H_{2i}^{*}$$ that concludes the result.
		
		When $\tilde{f}^*(u)\in H^*_{\mathbb{C}G}$ then also we have $\tilde{f}^*(\delta_{v_p}) = \sum a_{jp}\delta_{\vartheta \otimes v_j} , \, \forall v_{p}\in H_{2i}^{\mathbb{C}G},\, \forall i>0$ which implies that $\delta_{v_p}\circ \tilde{f}_*(v_q) = 0,\, \forall v_q \in H_{2i}^{\mathbb{C}G},\,\forall i>0.$   		
	\end{proof}
	\iffalse	Consequently, for the lift $\tilde f:\mathbb S^m\times \mathbb{C}G_{n,k}\to \mathbb S^m\times \mathbb{C}G_{n,k}$, the induced maps 
	$\tilde f^*:H^i_\times\to H^i_\times$ and $\tilde f_*:H_i^\times\to H_i^\times$ are adjoint with respect to this perfect pairing, i.e.,   
	\[
	\langle \tilde f^* \varphi, x\rangle=\langle \varphi, \tilde f_* x\rangle \text{ for all }\varphi\in H^i_{\times},x\in H_i^\times.
	\]\fi
	
	%To determine $\tilde f_*(U)$ when (iii) holds, write 
	%$\tilde f_*(U) = aU + \sum b_i v_i \in H_m^{\mathbb{S}} \oplus H_m^{\mathbb{C}G}\cong H_m^\times$.  
	%Then 
	%$\langle u, \tilde f_* U \rangle = \langle u, aU \rangle + \langle u, \sum b_i v_i \rangle = a + 0 = a$.  
	%On the other hand, 
	%$\langle u, \tilde f_* U \rangle = \langle \tilde f^* u, U \rangle = \langle \mu u, U \rangle = \mu$, 
	%so $a = \mu$.  
	%Proceeding similarly replacing  $u=\delta_U$  by $\delta_{v_i}$ yields $b_i = 0$ for all $i$.  
	%Hence, $\tilde f_*(U) = \mu U$.
	
	\iffalse	\underline{When (i) and (iii) hold}, for any basis element $v_q \in H_{2i}^{\mathbb{C}G}$, the coefficient of $v_p$ in $\tilde f_*(v_q)$, , is 
	\begin{equation}\label{i,ii,1}
		\langle \delta_{v_p}, \tilde f_*v_q\rangle
		= \langle \tilde f^* \delta_{v_p}, v_q\rangle
		= \langle \lambda^i \delta_{v_p}, v_q\rangle
		= \lambda^i \delta_{pq},
	\end{equation}
	and the coefficient of $\vartheta\otimes v_p \in \vartheta\otimes H_{2i}^{\mathbb{C}G}$ in $\tilde f_*(v_q)$ is 
	\begin{equation}\label{1,iii,2}
		\langle \delta_{\vartheta\otimes v_p}, \tilde f_*v_q\rangle
		= \langle \tilde f^*(\delta_{\vartheta\otimes v_p}), v_q\rangle
		= \langle \mu\lambda^{i}\delta_{\vartheta\otimes v_p}, v_q\rangle
		= 0.
	\end{equation}
	Thus, $\tilde f_*(x) = \lambda^i x$, for all $x \in H_{2i}^{\mathbb{C}G}$. 
	Also, the coefficient of $v_p \in H_m^{\mathbb{C}G}$ in $\tilde f_*(\vartheta\otimes v_q)$ is 
	t of $\vartheta\otimes v_p \in \vartheta \otimes H_{2i}^{{\mathbb CG}}$ in $\tilde f_*(\vartheta\otimes v_q)$ is 
	This shows that $\tilde f_*(\vartheta \otimes x) = \mu \lambda^{i}\vartheta \otimes x$ for all $x\in \vartheta\otimes H_{2i}^{\mathbb CG}$.\fi
	
	\iffalse	\underline{When (ii) and (iii) hold},  the coefficient of $v_p$ ($p\neq 0$) in $\tilde f_*(v_q)$, if $\tilde f^*\delta_{v_p}=\sum a_i\delta_{\vartheta \otimes v_i} $,  is
	\[
	\langle \delta_{v_p}, \tilde f_*v_q\rangle=\langle\tilde f^*\delta_{v_p},v_q\rangle=\langle \sum a_i\delta_{\vartheta \otimes v_i},v_q\rangle=\sum 
	a_i\langle  \delta_{\vartheta \otimes v_i},v_q\rangle=0,
	\]
	Thus, $\tilde f_* (x)\in \vartheta\otimes H_*^{\mathbb CG}$ for all $x\in H_*^{\mathbb CG}$. Also, the coefficient of $\vartheta\otimes v_p$ in $\tilde f_*(\vartheta\otimes v_q)$ is
	\[
	\langle \delta_{\vartheta \otimes v_p},\tilde f_*(\vartheta\otimes v_q)\rangle=\langle\tilde f^*(\delta_{\vartheta \otimes v_p}), \vartheta \otimes v_q\rangle =
	\begin{cases}
		\langle \mu \delta_{\vartheta \otimes v_0}, \vartheta \otimes v_q\rangle=\mu\delta_{0q}& \text{if } p=0,\\
		\langle 0, \vartheta \otimes v_q\rangle =0& \text{if }p\neq 0.
	\end{cases}
	\]
	Therefore, for all $x\in H_{2i}^{\mathbb CG}$, $\tilde f_*(\vartheta\otimes x)=\begin{cases}
		\mu \vartheta\otimes x +y \;\text{ for some }y\in H_{2i+m}^{\mathbb CG}  &\text{if } i=0 ,\\
		\quad \quad0 &\text{if }i>0.
	\end{cases}$ \fi
	
	\iffalse	\underline{When (i) and (iv) hold}, the coefficient of $v_p\in H_{2i}^{\mathbb CG}$ in $\tilde f_*(v_q)$ is 
	\[
	\langle \delta_{v_p},\tilde f_*(v_q)\rangle=\langle\tilde f^*\delta_{v_p},v_q\rangle=\langle \lambda^i\delta_{v_p},v_q\rangle=\lambda^i\delta_{pq}.
	\]
	This implies that $\tilde f_*(x)=\lambda^ix+\vartheta \otimes y$ for some $\vartheta \otimes y\in \vartheta \otimes H_{2i}^{\mathbb CG}$, for all $x\in H_{2i}^{\mathbb CG}$. \fi
	
	
	\iffalse	\underline{When (ii) and (iv) hold}, we can write $\tilde f^*(\delta_{v_p})=\sum a_j\delta_{\vartheta\otimes v_j}$ %and $\tilde f^*(\delta_{\vartheta\otimes v_p})=\sum b_j\delta_{\vartheta\otimes v_j}$
	for some $a_j\in \mathbb Q$ and $p\neq 0$.
	Now the coefficient of $v_p$ in $\tilde f_*(v_q)$ is
	\[
	\langle \delta_{v_p},\tilde f_*v_q\rangle=\langle\tilde f^*\delta_{v_p}, v_q\rangle=\langle\sum a_j\delta_{\vartheta\otimes v_j}, v_q\rangle=0,
	\]
	implying $\tilde f_*(x)\in \vartheta\otimes H_*^{\mathbb CG}$ for all $x\in H_*^{\mathbb CG}$. \fi
	\iffalse
	Also, the coefficient of $ v_p$ in $\tilde f_*(\vartheta\otimes v_q)$ is
	\[
	\langle \delta_{v_p},\tilde f_*(\vartheta\otimes v_q)\rangle=\langle\tilde f^*\delta_{v_p}, \vartheta\otimes v_q\rangle=\langle\sum b_j\delta_{v_j},\vartheta\otimes v_q\rangle=0.
	\]
	This ensures that $\tilde f_*(\vartheta\otimes x)\in H_*^{\mathbb CG}$ for all $\vartheta\otimes x\in \vartheta\otimes H_*^{\mathbb CG}$.
	
	
	We refer to the above observations from any combination of one condition from 
	$(\mathrm{i}),(\mathrm{ii})$ and one from $(\mathrm{iii}),(\mathrm{iv})$, by the symbol $\mathscr{X}$.
	
	
	
	%Summarizing above, we have for all $x\in H_{2i}^{\mathbb CG}$,
	%\[
	%\tilde f_*(x)=
	%\begin{cases}
	%   \lambda^ix  &\text{ if (i) and (iii) hold},\\
	%    \lambda^ix +\vartheta\otimes y, \text{ for some }y\in H_*^{\mathbb CG}& \text{ if (i) and (iv) hold}.
	%^\end{cases}
%\]


\vspace{.2in}\fi

\iffalse
******

Thus, for all $x \in H^{\mathbb CG}_{2i}$, $\varphi \in H^{2i}_{\mathbb CG}$, and $y \in H^m_\times$, we have
\begin{equation}\label{}
	\langle \varphi, \tilde f_* x\rangle =
	\begin{cases}
		\langle \varphi, \lambda_1^i x\rangle & \text{if (i) holds,} \\
		\langle u P_{\varphi}, x\rangle & \text{if (ii) holds},\\
	\end{cases}
	\quad \text{and} \quad
	\langle u, \tilde f_* y\rangle =
	\begin{cases}
		\langle u, \mu_1 y\rangle & \text{if (iii) holds,} \\
		& \text{if (iv) holds}.\\
	\end{cases}
\end{equation}

Therefore, for all $x\in H_{2i}^{\mathbb CG}$, $\tilde f_*(x)=\lambda_1^ix$ if (i) holds, $\tilde f_*(x)\in [\mathbb S^m]\otimes H_{2i-m}^{\mathbb CG}$ if (ii) holds, and $\tilde f_*([\mathbb S^m])=\mu_1 [\mathbb S^m]$ if (iii) holds, where $[\mathbb S^m]\in H_m^{\mathbb S}$ denotes the fundamental class of $\mathbb S^m$ satisfying $\langle u,[\mathbb S^m]\rangle=1.$
\fi


\subsection{} The following theorems provide a criteria for the existence of coincidence points between a pair of continuous functions on $P(m,n,k)$.

\begin{theorem}\label{coincidence thm}
	Let $P(m,n,k)$ be a generalized Dold manifold with $k<n-k$ and $k(n-k)$ even. Let $f$ and $g$ be two continuous maps on $P(m,n,k)$ and $\tilde f, \tilde g$ be their lifts as defined in  Remark \ref{lift} such that
	%and suppose that the induced endomorphisms in cohomology satisfies:	
	\begin{enumerate}	
		\item $g^*$ is an automorphism of $H^*(P(m,n,k);\mathbb Q)$. 
		\item $\tilde{f}^*(c_1) \neq au,\, a \in \mathbb{Q}$.
		\item $\deg(p\circ g \circ s)\neq -\deg (p\circ f\circ s)$ if $m$ is odd.
	\end{enumerate}
	where $s$ denotes a section of the $X$-bundle projection $p$ defined in \eqref{sectio} and \eqref{proj}.	Then, there is a point of coincidence of $f$ and  $g$.
\end{theorem}
\begin{proof}
	Using \corref{automor}, we have $\tilde g^*$ is an automorphism on $H^*_{\times}$ given by
	$\tilde g^*(c_i) = \lambda_1^i c_i$, and $\tilde g^*(u) = \mu_1 u$ for some $\lambda_1, \mu_1 \in \mathbb{Q}\backslash \{0\}$ if $k<n-k$.
	
	Using \lemref{image of homf}, there exist $\lambda \in \mathbb{Q}\backslash \{0\}$ and $\mu \in \mathbb{Q}$ such that $\tilde{f}_*$ is of the following form, 
	\begin{equation}\label{flower star}
		\begin{split}
			\tilde f_*(x)=\lambda^ix+\vartheta\otimes y, \text{ for some }y\in H_{2i-m}^{\mathbb CG}, \, \forall x \in H_{2i}^{\mathbb{C}G}\\
			\tilde f_*(\vartheta\otimes x)=\mu\lambda^i (\vartheta\otimes x) , \text{ or } \tilde{f}_*(\vartheta \otimes x) = z, \text{ for some }z\in H_{2i+m}^{\mathbb CG},\, \forall x \in H_{2i}^{\mathbb{C}G}
		\end{split}
	\end{equation}
	To prove that $f$ has a point of coincidence with $g$, it is sufficient to prove that either $\tilde{f}$ or the composition $\theta \circ \tilde{f} $ has a point of coincidence with $g$ where $\theta = \alpha \times \sigma$ defined in \secref{gds}. By \thmref{LCT}, we need to compute $L(\tilde f, \tilde g)$ and $L(\theta \circ \tilde f, \tilde g)$.
	
	For $x\in H_{2i}^{\mathbb{C}G}$, we have 
	\begin{equation*}\label{D cal}
		\begin{split}
			D\tilde g^* D^{-1} \tilde f_*(x)=\mu_1 \lambda^i\lambda_1^{d-i} x + \vartheta\otimes y'\text{ for some }y' \in H^{\mathbb CG}_{2i-m}\\
			D\tilde g^* D^{-1} \tilde f_*(\vartheta\otimes x)= \mu \lambda^i\lambda_1^{d-i}(\vartheta\otimes x)+z' \text{ for some }z'\in H_{2i+m}^{\mathbb CG}.
		\end{split}
	\end{equation*}
	where $z^{'} =0$ or $\mu =0$ depending on the image of $\tilde{f}_*(\vartheta \otimes x)$.
	%Now we compute the Lefschetz number of the map $L(\tilde{f},\tilde g)$  and $L(\theta\circ \tilde f, \tilde g)$ under the consideration that $\lambda=0$ if (ii) holds and $\mu=0$ if (iv) holds. 
	Recall that  $d_{2i}$ denote the dimension $\dim H^{2i}_{\mathbb CG}$. The Lefschetz number  $L(\tilde{f},\tilde g)$ is
	\begin{equation*}\label{L(f,g)}
		L(\tilde f,\tilde g) =(\mu_1 + \mu) \sum_{i=0}^{k(n-k)} d_{2i} \lambda^i\lambda_1^{d-i}.
	\end{equation*}
	Using the \lemref{sum neq 0} and the fact that $\lambda_1\neq 0$, the sum
	\[
	\sum_{i=0}^{k(n-k)} d_{2i} \lambda^i\lambda_1^{d-i}=\lambda_1^d\sum_{i=0}^{k(n-k)} d_{2i} (\lambda/\lambda_1)^i\neq 0,
	\]
	Since $\tilde f\circ\theta=\theta\circ \tilde f$, it follows that $$(\theta\circ \tilde  f )^*(c_i)= (-1)^i \tilde{f}^*(c_i), \forall i \in I, \quad (\theta \circ\tilde  f)^*(u)=\begin{cases}
		-\tilde  f^*(u), \text{ if } m \text{ is even,}\\
		\tilde  f^*(u), \text{ if } m \text{ is odd}.
	\end{cases}$$ 
%	If $m$ is odd, using $\deg(p\circ g \circ s)\neq -\deg (p\circ f\circ s)$ i.e. $\mu_1 \neq -\mu$, we have $L(\theta\circ\tilde f, \tilde g) = L(\tilde f, \tilde g)\neq 0$ and we have the result. \\
	If $m$ is even, then  
	\begin{equation*}\label{D with theta}
		\begin{split}
			D\tilde g^* D^{-1} (\theta\circ \tilde f)_*(x)=\mu_1(- \lambda)^i\lambda_1^{d-i} x + \vartheta\otimes y''\text{ for some }y'' \in H^{\mathbb CG}_{2i-m} \\
			D\tilde g^* D^{-1} (\theta\circ \tilde f)_*(\vartheta\otimes x)= -\mu (-\lambda)^i\lambda_1^{d-i}\vartheta\otimes x+z'' \text{ for some }z''\in H_{2i+m}^{\mathbb CG}.
		\end{split}
	\end{equation*}
	Thus,  the Lefschetz number is
	\begin{equation*}\label{L(theta f,g)}
		L(\theta \circ\tilde f,\tilde g) =(\mu_1 - \mu)\sum_{i=0}^{k(n-k)} d_{2i} (-\lambda)^i\lambda_1^{d-i}.
	\end{equation*}
	Also, using  $\mu_1\neq 0$ and \lemref{sum neq 0}  it follows that that either $L(\tilde f, \tilde g)$ or $L(\theta\circ \tilde f,\tilde g)$ is nonzero. 
	
	If $m$ is odd, $	L(\theta \circ\tilde f,\tilde g) =(\mu_1 + \mu)\sum_{i=0}^{k(n-k)} d_{2i} (-\lambda)^i\lambda_1^{d-i}.$ Using  \lemref{sum neq 0} and $\deg(p\circ g \circ s)\neq -\deg (p\circ f\circ s)$ that is $\mu_1 \neq -\mu$, we have both $L(\tilde{f},\tilde{g})$ and $L(\theta \circ\tilde f,\tilde g)$ are nonzero. 
	This ensures that there exist a point of conincidence between $f $ and $g$.
\end{proof}
%	Using  \remref{lemrem} in the case $k=n-k$, we need to replace $\lambda$ by $-\lambda$ in \eqref{flower star} if $\tilde{f}^*(c_i) = -\lambda^i c_i$. In the rest of the calculations, $\lambda$ will be replaced by $-\lambda$ and we get the same result.
\begin{theorem}\label{coincidence thm under hom}
	Let $P(m,n,k)$ be a generalized Dold manifold with $k(n-k)$ even and assume that the hypothesis \eqref{Homer} is satisfied. Let $g$ and $f$ are two continuous maps on $P(m,n,k)$ and $\tilde g, \tilde f$ be their lifts as defined in  Remark \ref{lift} such that
	%and suppose that the induced endomorphisms in cohomology satisfies:	
	\begin{enumerate}	
		\item $g^*$ is an automorphism of $H^*(P(m,n,k);\mathbb Q)$. 
		\item $\tilde{f}^*(u) = \mu u,\, \mu \in \mathbb{Q}$ if $\tilde{f}^*(H^*_{\mathbb CG}) \nsubseteq H^*_{\mathbb CG}$ and $m$ is even.
		\item $\deg(p\circ g \circ s)\neq -\deg (p\circ f\circ s)$ if $m$ is odd.
	\end{enumerate}
	$s$ denotes a section of the $X$-bundle projection $p$ defined in \eqref{sectio} and \eqref{proj}. Then, there is a point of coincidence of $f$ and  $g$.
\end{theorem}
\begin{proof}
	If $\tilde{f}^*(c_1) \neq au, \, a\in \mathbb{Q}$ then we have the result by \thmref{coincidence thm}.\\ Let us consider the other case when $\tilde{f}^*(c_1) = au, \, a\in \mathbb{Q}$, using \thmref{main thm 2} we have $\tilde{f}^*(c_i) = uP_i, \text{ for some } P_i\in H^{2i-m}_{\mathbb{C}G}.$ 
	
	If $P_i \neq 0$ for some $i$ in $I$ then $\tilde{f}^*(H^*_{\mathbb CG}) \nsubseteq H^*_{\mathbb CG}$. Since $\tilde{f}^*$ is graded and by \textit{(2)} we have $\tilde{f}^*(u) = \mu u,\, \mu \in \mathbb{Q}$. Using \lemref{image of homf under hom}, $\tilde{f}_*$ is of the following form, \begin{equation}\label{uin CG}
		\begin{split}
			\tilde f_*(x)= \vartheta\otimes y, \text{ for some }y\in H_{2i-m}^{\mathbb CG}, \, \forall x \in H_{2i}^{\mathbb{C}G}, \, i>0\\
			\tilde f_*(\vartheta\otimes x)=\mu(\vartheta\otimes x) +z, \text{ for some }z\in H_{2i+m}^{\mathbb CG},\, \forall x \in H_{2i}^{\mathbb{C}G}
		\end{split}
	\end{equation}
	where $\mu =0$ if $i>0$. By \corref{automor}, we have $\tilde g^*$ is an automorphism on $H^*_{\times}$ given by
	$\tilde g^*(c_i) = \lambda_1^i c_i$, and $\tilde g^*(u) = \mu_1 u$ for some $\lambda_1, \mu_1 \in \mathbb{Q}\backslash \{0\}$. Using \thmref{LCT} and the similar calculations as done in the proof of \thmref{coincidence thm}, we get $$L(\tilde f,\tilde g) =(\mu_1 + \mu) d_0 \lambda_1^d, \quad L(\theta \circ\tilde f,\tilde g) =\begin{cases}
		(\mu_1 - \mu)d_{0} \lambda_1^{d},  \text{ if } m \text{ is even,}\\
		(\mu_1 +\mu)d_{0} \lambda_1^{d},  \text{ if } m \text{ is odd.}
	\end{cases}$$ Using $\lambda_1 \neq 0$ and $\mu_1 \neq 0$, either $L(\tilde f,\tilde g) $ or $ L(\theta \circ\tilde f,\tilde g)$ is non zero if $m$ is even. Using $\deg(p\circ g \circ s)\neq -\deg (p\circ f\circ s)$ i.e. $\mu_1 \neq -\mu$ we have $L(\tilde f, \tilde g) = L(\theta \circ \tilde f, \tilde g) \neq 0$.  Hence, we get the result.
	
	Let us consider the case when $P_i =0,\, \forall i \in I$, if $\tilde{f}^*(u) = \mu u, \mu \in \mathbb{Q}$ then the proof remains exactly the same as given above. We need to focus on the case when $\tilde{f}^*(u)\in H^*_{\mathbb{C}G}$. Using \lemref{image of homf under hom} and \eqref{computation}, we have $$\tilde{f}_*(x) = \vartheta \otimes y, \text{ for some }y\in H_{2i-m}^{\mathbb CG}, \, \forall x \in H_{2i}^{\mathbb{C}G}, \, i>0, \quad \tilde{f}_*(\vartheta \otimes x) \in H_*^{\mathbb{C}G}, \forall x\in  H_*^{\mathbb{C}G}.$$ This is exactly the same if we take $\mu =0$ in \eqref{uin CG}. The rest of the calculations also remains the same and we get the result.
\end{proof}



\begin{remark}
	There are many situations when the map $f$ satisfies the required hypothesis \textit{(2)} considered in \thmref{coincidence thm} or \thmref{coincidence thm under hom}. Some of them are as follows: 
	\begin{enumerate}
		\item The lift $\tilde f$ stabilizes a copy of Grassmannian, i.e., $\tilde f(\{x_0\}\times\mathbb CG_{n,k})\subseteq \{x_0\}\times\mathbb CG_{n,k}$ for some $x_0\in \mathbb S^m$. 
		\item  The map $p_1\circ \tilde f^*\circ i_1: H^*_{\mathbb CG}\to H^*_{\mathbb C G}$ is an automorphism, equivalently, $f^*(c_1^2)=\lambda^2c_1^2,\, \lambda\in \mathbb Q\backslash \{0\},$ where $p_1$ and $i_1$ are defined in \eqref{comm diagram}.
		\item The map $p_2\circ \tilde f\circ i_1:\mathbb S^m\to \mathbb CG_{n,k}$ is rationally null homotopic, where $p_2$ is the projection onto the second summand and $i_1$ is the inclusion into the first summand. %$i_1:\mathbb S^m\hookrightarrow\mathbb S^m\times \mathbb CG_{n,k}$ be the inclusion into the first component and $p_2:\mathbb S^m\times \mathbb CG_{n,k}\to \mathbb CG_{n,k}$ be the projection onto the second component.
	\end{enumerate} 
\end{remark}



	Under the assumption $m>2k$, any continuous map $f$ on the generalized Dold space $P(m,n,k)$, the lift $\tilde f$ (from Remark~\ref{lift}) satisfies $\tilde f^{*}(c_i)=\lambda^ic_i$ for all $i\in I$. Hence condition~\textit{(2)} of Theorem~\ref{coincidence thm under hom} may be omitted, and one obtains the following consequence.
	\begin{corollary}
		Let $P(m,n,k)$ be a generalized Dold space with $m$ and $k(n-k)$ both even. Assume $m>2k$, and the hypothesis \eqref{Homer} is satisfied.
		Then, for any continuous function $g$ on $P(m,n,k)$ that induces an automorphism on $H^*(P(m,n,k);\mathbb{Q})$, the pair $(P(m,n,k),g)$ has the coincidence property. \\ In particular, for $g=\mathrm{id}$, the space $P(m,n,k)$ has the fixed-point property.
	\end{corollary}










In \thmref{coincidence thm under hom}, the first assumption that $g^*$ is an automorphism of $H^*(P(m,n,k); \mathbb{Q})$ can be relaxed by assuming $\mu$ is nonzero, which leads to the following proposition. 
\begin{proposition}
	Let $P(m,n,k)$ be a generalized Dold manifold with $k(n-k)$ even and assume that the hypothesis \eqref{Homer} is satisfied. Let $g$ and $f$ are two continuous maps on $P(m,n,k)$ and $\tilde g, \tilde f$ be their lifts as defined in  Remark \ref{lift} such that
	\begin{enumerate}
		\item $\tilde g^*(H^*_{\mathbb CG})= H^*_{\mathbb CG}$.
		\item $\tilde f^*(u)=\mu u,\, \mu\in \mathbb Q\backslash \{0\}$
	\end{enumerate}
Then, there is a point of coincidence of $f$ and  $g$.
\end{proposition}
The proof of the above proposition is similar to the proof of \thmref{coincidence thm under hom}. Therefore, we omit the details.

\iffalse \begin{proposition}
	Let $P(m,n,k)$ be a generalized Dold manifold. Consider two continuous maps $f,g$ on $P(m,n,k)$ and $\tilde f, \tilde g$ be their lifts as defined in  Remark \ref{lift}.
	Suppose that the induced endomorphisms in cohomology satisfies:
	\begin{enumerate}
		\item $\tilde g^*(c_i)\in uH^*_{\mathbb CG},\forall i\in I$ and $\tilde g^*(u)=\mu_1u$ for some $\mu_1\in \mathbb Q$,
		\item $\tilde f$ has nonzero Brouwer-degree.
	\end{enumerate}
	Then, there is a point of coincidence of $f$ and  $g$.
\end{proposition}
\fi




\iffalse If we assume $m> 2k$, then we have more information about $\tilde f^*$ and in that case we don't have to assume hypothesis \textit{(2)}.

We need to talk about the similar situations when we assume Hoffman instead of Glover-Homer.

We should be able to find some GDS $P(m,n,k)$ which has FPP.

What happens if we allow $m$ to be odd in $P(m,n,k)$?
\fi

\section*{Acknowledgements}
Part of this work was carried out while the first author was a postdoctoral fellow at IISER Berhampur, which the author gratefully acknowledges.







%\section*{References}

\begin{thebibliography}{99}
	
	%\bibitem[A]{adams} Adams, J. F. Vector fields on spheres.
	%Ann. of Math. {\bf 75} (1962), 603--632.
	%\bibitem[ABS]{ABS} Atiyah, M. F.; Bott, R., Shapiro, A. Clifford modules. Topology {\bf 3} (1964), no. suppl, 3--38.
	%\bibitem[AH]{atiyah-hirzebruch} Atiyah, M. F.; Hirzebruch, F.
	%Vector bundles and homogeneous spaces.Proc. Sympos. Pure Math., Vol. III, 7--38
	%American Mathematical Society, Providence, RI, 1961.
	
	%\bibitem[B]{borel} Borel, A. Sur la cohomologie des espaces fibr{\'e}s principaux et des espaces homogènes de groupes de Lie compacts. Ann. of Math, Vol. {\bf 57} (1953) 115--207.
	
	\bibitem[B]{brewster} S. Brewster, \textit{Automorphisms of the cohomology ring of finite Grassmann manifolds}, Thesis (Ph.D.)---The Ohio State University, ProQuest LLC, Ann Arbor, MI (1978), 102 pp. \url{http://gateway.proquest.com/openurl?url_ver=Z39.88-2004&rft_val_fmt=info:ofi/fmt:kev:mtx:dissertation&res_dat=xri:pqdiss&rft_dat=xri:pqdiss:7908118}
	
	\bibitem[BH]{brewster-homer} S. Brewster and W. Homer, \textit{Rational automorphisms of Grassmann manifolds}, Proc. Amer. Math. Soc. \textbf{88} (1983), no. 1, 181--183, \url{https://doi.org/10.2307/2045137}.
	
	%\bibitem[B2]{borel} Borel, A. Cohomologie mod $2$
	% de certains espaces homogènes.
	%Comment. Math. Helv. {\bf 27} (1953)165--197 (1953).
	%\bibitem[B3]{borel-lag} Borel, A. Linear algebraic groups. Springer-Verlag, New York.
	%\bibitem[CF]{conner-floyd} Conner, P. E.; Floyd, E. E. {\it Differentiable periodic maps.} Ergebnisse der Mathematik...
	%\bibitem[D]{davis} Davis, D. Projective product spaces. J. Topol. {\bf 3} (2010), 265--279.
	%\bibitem[DJ]{dj} Davis, M.; Januszkiewicz, T. Convex polytopes...
	
	\bibitem[Do]{dold} A. Dold, \textit{Erzeugende der Thomschen Algebra }$\mathfrak{N}$, Math. Z. \textbf{65} (1956), 25--35, \url{https://doi.org/10.1007/BF01473868}.

    \bibitem[Du]{duan} H. Duan, \textit{Self-maps of the Grassmannian of complex structures}, Compositio Math. \textbf{132} (2002), no. 2, 159--175, \url{https://doi.org/10.1023/A:1015885227445}.
	
	\bibitem[DF]{duan-fang} H. Duan and L. Fang, \textit{Homology rigidity of Grassmannians}, Acta Math. Sci. Ser. B \textbf{29} (2009), no. 3, 697--704, \url{https://doi.org/10.1016/S0252-9602(09)60065-5}.

    \bibitem[DZ]{duan-zhao} H. Duan and X. Zhao, \textit{The classification of cohomology endomorphisms of certain flag manifolds}, Pacific J. Math. \textbf{192} (2000), no. 1, 93--102, \url{https://doi.org/10.2140/pjm.2000.192.93}.
	%\bibitem[F1]{fujii-66} Fujii, M. $K_U$-groups of Dold manifolds...
	%\bibitem[F2]{fujii-69} Fujii, M. Ring structures...
	
	%\bibitem[F]{fulton} Fulton, W. *Introduction to Toric Varieties*...
	%\bibitem[K]{karoubi} Karoubi, M. *K-theory*...
	%\bibitem[HM]{hm} Hattori, A.; Masuda, M...
	%\bibitem[H]{husemoller} Husemoller, D. *Fibre bundles*...
	
	\bibitem[GH1]{glover-homer} H. Glover and B. Homer, \textit{Endomorphisms of the cohomology ring of finite Grassmann manifolds}, Geometric applications of homotopy theory (Proc. Conf., Evanston, Ill., 1977), I, pp. 170--193.
	

    \bibitem[GH2]{glover-homer coin} H. Glover and W. Homer, \textit{Fixed points on flag manifolds}, Pacific J. Math. \textbf{101} (1982), no. 2, 303--306, \url{http://projecteuclid.org/euclid.pjm/1102724776}.
	
	\bibitem[GS]{goswami-sarkar} A. Goswami and S. Sarkar, \textit{Endomorphisms of the Cohomology Algebra of Certain Homogeneous Spaces}, \url{https://arxiv.org/abs/2509.09363}

	\bibitem[Ho1]{hoffman} M. Hoffman, \textit{Endomorphisms of the cohomology of complex Grassmannians}, Trans. Amer. Math. Soc. \textbf{281} (1984), no. 2, 745--760, \url{https://doi.org/10.2307/2000083}.
    
    \bibitem[Ho2]{hoffman-noncoin} M. Hoffman, Noncoincidence index of manifolds, Pacific J. Math. \textbf{115}(2) (1984), 373--383, \url{http://projecteuclid.org/euclid.pjm/1102708254}.

    \bibitem[HH]{hoffman-homer} M. Hoffman and W. Homer, \textit{On cohomology automorphisms of complex flag manifolds}, Proc. Amer. Math. Soc. \textbf{91} (1984), no. 4, 643--648, \url{https://doi.org/10.2307/2044817}.
	
	

 %   \bibitem[Kh]{khare} Khare, S.S. On Dold manifolds, Topology and its Applications, Volume 33, Issue 3, 1989, Pages 297-307.

  %  \bibitem[Ko]{korbas} Korbaš, J. On parallelizability and span of the Dold manifolds,  Proc. Amer. Math. Soc. \textbf{141} (2013), no.~8, 2933--2939.

    
	\bibitem[L]{lin} X. Z. Lin, \textit{Geometric realization of Adams maps}, Acta Math. Sin. \textbf{27} (2011), no. 5, 863--870, \url{https://doi.org/10.1007/s10114-011-0164-y}.

    \bibitem[M]{mandal} M. Mandal, Cohomology of generalized Dold manifolds, Thesis (Ph.D.), Homi Bhabha National Institute, The Institute of Mathematical Sciences (2024).

%    \bibitem[Mu]{mukerjee} Mukerjee, H.~K. Classification of homotopy Dold manifolds,  New York J. Math. \textbf{9} (2003), 271--293.
	
	\bibitem[MS1]{mandal-sankaran} M. Mandal and P. Sankaran, \textit{Cohomology of generalized Dold spaces}, Topology Appl. \textbf{310} (2022), Paper No. 108040, 16 pp, \url{https://doi.org/10.1016/j.topol.2022.108040}.
    
	\bibitem[MS2]{mandal-sankaran2} M. Mandal and P. Sankaran, \textit{Cohomology and K-theory of generalized Dold manifolds fibred by complex flag manifolds}, \url{https://arxiv.org/abs/2407.03932}.


	
	%\bibitem[MiSt]{milnor-stasheff} Milnor, J. W.; Stasheff, J. D. {\it Characteristic classes.} Annals of Mathematics Studies, {\bf 76}, Princeton University Press, Princeton, NJ., 1974.
	
	\bibitem[NS]{nath-sankaran} A. Nath and P. Sankaran, \textit{On generalized Dold manifolds}, Osaka J. Math. \textbf{56} (2019), no. 1, 75--90, Errata, \textbf{57} (2020), no. 2, 505--506, \url{https://projecteuclid.org/euclid.ojm/1547607627}.
	
	\bibitem[O]{O} L.S. O'Neill, \textit{On the fixed point property for Grassmann manifolds}, Thesis (Ph.D.)---The Ohio State University
ProQuest LLC, Ann Arbor, MI (1974). 52 pp, \url{http://gateway.proquest.com/openurl?url_ver=Z39.88-2004&rft_val_fmt=info:ofi/fmt:kev:mtx:dissertation&res_dat=xri:pqdiss&rft_dat=xri:pqdiss:7511411}.	

	\bibitem[P]{Papadima} S. Papadima, \textit{Rigidity properties of compact Lie groups modulo maximal tori}, Math. Ann. \textbf{275} (1986), no. 4, 637--652, \url{https://doi.org/10.1007/BF01459142}.

	
	%\bibitem[Sp]{spanier} Spanier, E. H. {\em Algebraic Topology}. McGraw Hill, 1966. Reprinted by Springer-Verlag, New York, 1975.
	
	\bibitem[ST1]{shiga-tezuka} H. Shiga and M. Tezuka, \textit{Rational fibrations, homogeneous spaces with positive Euler characteristics and Jacobians}, Ann. Inst. Fourier (Grenoble) \textbf{37} (1987), no. 1, 81--106, \url{https://doi.org/10.5802/aif.1078}.
	
	\bibitem[ST2]{shiga-tezuka2} H. Shiga and M. Tezuka, \textit{Cohomology automorphisms of some homogeneous spaces}, Singapore topology conference (Singapore, 1985), Topology Appl. \textbf{25} (1987), no. 2, 143--150, Errata, \textbf{34} (1990), no. 2, 207, \url{https://doi.org/10.1016/0166-8641(87)90007-1}.
	
%	\bibitem[U]{ucci} Ucci, J. J. Immersions and embeddings of Dold manifolds. {\it Topology} {\bf 4} (1965), 283--293.

    \bibitem[W]{wong} P. Wong, \textit{Fixed point theory for homogeneous spaces. II}, Fund. Math. \textbf{186} (2005), no. 2, 161--175, \url{https://doi.org/10.4064/fm186-2-4}.


	
\end{thebibliography}
\end{document}
